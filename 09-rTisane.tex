Researchers in the case studies using \tisane made us aware of confusing
keywords and language constructs in \tisane's \SDSL. This motivated us to probe
more deeply into what challenges statistical novices face when expressing their
domain knowledge. We started with a lab study using \tisane to elicit
statistical non-experts' implicit definitions and assumptions about \SDSL
keywords. The study also helped us identify opportunities to refine \tisane's
interactivity. Based on study findings, we designed and evaluated \rTisane, a
system to assist novices in formalizing their conceptual knowledge to author
statistical models. 

\def\unit{\texttt{Unit}\xspace}
\def\measure{\texttt{Measure}\xspace}
\def\setup{\texttt{SetUp}\xspace}

\section{Elicitation lab study} \label{sec:exploratoryStudy}
% Motivation
% Our motivating question was, ``What implicit semantics do analysts use to construct conceptual models of how variables relate?''. 
\polish{rTisane should be (re-)introduced before this mention. Maybe note in 5.5
the goal of creating a revised Tisane version for R?: For the lab study, we
re-implemented \tisane (originally in Python) in R since (i) R is a widely used
programming language for data science and (ii) the research methods course
taught and used R.} Our aim was to increase the expressivity of \rTisane to
represent analysts' implicit domain knowledge. We used the first release of
\tisane~\cite{jun2022tisane} to probe analysts' internal processes and derive
design goals~\autoref{sec:rtisane_design_implications} for re-designing \tisane.

\subsection{Method}
%Participants
We recruited participants through a graduate-level quantitative research methods
course as a convenience sample to control recent exposure to
statistical concepts. Five computer science PhD students participated.
% All participants self-reported that they
% conducted research in programming languages or software engineering and planned
% to perform statistical analyses in future research projects. 
% \ej{Add: Approaches, mindsets of participants as they were completing the study}

% On a scale from
% one (not at all relatable) to five
% (extremely relatable), four participants indicated that the paper was relatable
% with a score of 3 or higher.
The study consisted of two parts: (i) a take-home assignment and (ii) an in-lab
session. The take-home assignment asked participants to read a recently
published CHI paper~\cite{winters2021heartbeat}\footnote{We chose the specific
paper because we believed its topic (i.e., empathetic biosignals) would be
broadly approachable and the statistical methods used (i.e., generalized linear
models) would be familiar with students enrolled in the research methods
course.} and describe the paper's research questions and hypotheses, the
authors' conceptual models, the study's design, and ways to analyze the data to
answer the research questions. The assignment was designed to ensure that
participants engaged with the paper's key ideas before coming into the lab. The
researcher reviewed each submission to prepare participant-specific questions
for a semi-structured, think-a-loud lab session. 

% used Tisane~\cite{jun2022tisane}, an open-source package for authoring
% generalized linear models with or without mixed effects, as a probe to
% understand how participants thought about and wanted to express their conceptual
% models and study designs. We 

At the start of the lab session, participants reviewed their homework submission
to remind themselves of the paper. The paper and participants' homework
responses remained available for reference throughout the study. Then,
participants completed three tasks: (i) declaring variables, (ii) expressing
conceptual models, and (iii) specifying study designs. For each task,
participants started with \tisane's language constructs to express their intent
and discussed their confusions, how they understood each presented construct,
and what they wanted to specify but could not (if applicable). The researcher repeatedly reminded
participants that the constructs presented were prototype possibilities and that
expressing their intentions was more important than using the constructs or
getting the syntax correct. Throughout, the researcher paid particular attention
to where \tisane broke down for participants and asked follow-up questions to
probe deeper into why. The researcher considered such breakdowns as openings
into semantic mismatches between the end-user and the DSL.

% Analysis 
We iteratively coded homework submissions, audio transcripts from the lab study
sessions, and lab study artifacts. We also consulted the
researcher's detailed notes from the lab sessions. 

% Based on an iteraWe conducted a content analysis of the produced code suggestions (output
% artifacts) along with a thematic analysis of the audio transcripts from each lab
% session. 

\subsection{Key Observations}
All participants demonstrated a working knowledge of the assigned paper's
motivating research questions, study design, and general study procedure. 
% Among
% the submitted conceptual models, four were a list of conceptual relationships,
% and one was a diagram. 
We made four key observations about what and how statistical non-experts wanted
to express their conceptual models: using varying degrees of specificity,
separating moderation from bivariate relationships, distinguishing between known
and hypothesized relationships, and considering alternative conceptual models.
Participants also suggested syntactic sugar options to improve the DSL's
usability. Based on these observations, we derived design goals for re-designing
\tisane (see~\autoref{sec:rTisane}).

\subsubsection{Participants express conceptual knowledge with varying details.}
Contrary to the popular belief that higher levels of abstraction are better for
end-users, we found that statistical non-experts want to
move up and down the ladder of abstraction when expressing conceptual models. 

When defining ``causes,'' P2 described ``[Causes] is...like when we teach
logic...it's like implication, right?....So I'm saying if we are observing an
emotion and...emotion observed can lead to a change in emotional perspective.''
P0, P1, and P3 contrasted a bidirectional relationship between variables,
formerly encapsulated in the \texttt{associates\_with} construct in \tisane, to
their implicit understanding of ``causes.'' For instance, P1 stated ``the most
like, utilitarian definition by if A causes B, then by changing A, I can change
B whereas \texttt{associates\_with} means that...if I can turn dial A, B might
not change.'' In addition to differentiating between causal and associative
relationships, three participants [P0, P1, P3] provided statements of
\textit{specifically how} a variable influenced another in the conceptual models
submitted as homework. For example, P0 wrote, ``Hearing a heartbeat that seems
to be aligned with visual cues makes someone feel \textit{more} strongly what
another person is feeling'' (emphasis added), specifying a positive influence of
``hearting a heartbeat'' on empathy. These observations suggest that analysts
have an intuitive understanding of causality but bluntly stating that a variable causes
another does not capture the richness or nuance of their implicit domain knowledge. 
Additional annotations about how a variable influences another are necessary.

%   in addition to vaguer statements without a clear
% positive/negative direction, such as ``Visual cues directly influence a person's
% perception of another's emotions'' [P0]. 
% The homework assignment and lab study materials may have primed participants to
% think about ``direct'' and ``indirect'' relationships and commutativity, so
% analysts' descriptions of causal relationships as ``unidirectional'' and
% associative relationships as ``bidirectional'' were not surprising [P0, P2, P3].

% how a bidirectional relationship between two variables 
% indicative of proxies measuring the same latent construct (``...for two things
% to be bi-directional and for it to be a really, really direct relationship. Like
% that just never happens in the real world without them turning out to be like
% proxies for the same exact thing.'') [P3]; and that 

\subsubsection{Participants find moderation difficult to separate from bivariate relationships.}
Participants consistently found \tisane's \texttt{moderates}
construct difficult to understand [P0, P1, P2, P3]. Participants expressed
confusion about what moderation implied about the relationship between
two variables. For example, P3 grappled with if ``moderates'' was shorthand for
expressing associative relationships between each independent variable and the
dependent variable, how moderation implies causal relationships, and if
statistical and conceptual definitions of moderation differed from each other:
``[L]et's say there's two independent variables and one dependent variable. And
each of the [independent] variables individually is not correlated with the
outcome. But if you put them together, then the correlation appears....I mean,
it's sort of a philosophical question of whether, like each of the ones
individually causes [the dependent variable] in that case. But thinking from
a...statistical perspective, I think that's a situation where you might be able
to express...language and experience level together cause lines of code but
individually they don't because no individual correlation would appear there.''
Therefore, a clear delineation between bivariate relationships and partial statistical specifications of interaction terms is necessary. 
% that (i) the semantics of moderation is unclear in
% Tisane and (ii) moderation may be easier to reason about when decomposed into
% what happens to the dependent variable when two or more independent variables
% have specific values. 
% \paragraph{Design implication: Too high level --> Remove moderates}

\subsubsection{Participants distinguish between known and suspected relationships.}
% [P0, P1, P2, P3, P4]
Participants described relationships established in prior work as
``assumptions'' or ``assertions'' to check separately from the key research
questions that tested ``suspected'' relationships. P0 described how ``maybe we
have to differentiate as to like the \textit{known} [relationships] are kind of
the things you're \textit{assuming} there's relationships between these things
whereas the \textit{suspected}...[are] the things kind of like your research
questions are saying like, `We \textit{think} there's this relationship
but...it's what we're testing for'' (emphasis added). Similarly, P4 suggested
that Tisane should warn end-users when assumptions about known relationships are
violated in a given data set: ``I would also say that it would be very handy to
be able to say, kind of \textit{assert} that language has no effect on the line
of code. And be warned if it's not the case, like if your \textit{assertion} is
not...verified automatically with the DSL, but warned...that while your
\textit{assumption} is not holding there is actually an effect, which could be
very handy on your study.'' (emphasis added). The inability to indicate
relationships that are either known or suspected in \tisane may explain why
analysts repeatedly preferred less technical verbs, such as ``influences'' [P0]
or ``leads to'' [P3]. For instance, P0 explained how she preferred
``influences'' over ``causes'' because ``I guess it's like \textit{a level of
sureness} in it in which, like, `cause' feels more confident in your answers
than `influences''' (emphasis added). Providing a way to label conceptual
relationships as assumptions or the focus of the present analysis could make
\texttt{causes} and \texttt{associates\_with}, the bivariate relationships in
\tisane, more approachable. 
% \paragraph{Design implication: Provide constructs for distinguishing betwen known and suspected relationships -->  assume/hypothesize}

% However, Tisane does not differentiate between known
% and suspected relationships, and as P4 noted, ``I don't know if everything that
% is not either specified to moderate, associate, or causes is by default,
% asserting that there is no effect,'' suggesting that the Tisane's DSL semantics
% could be more self-evident. In fact, 

\subsubsection{Participants consider alternative conceptual structures in the face of ambiguity.}
% In addition to detailing how variables influence one another, participants also
Participants grappled with what specific structures in a conceptual model meant. P1 and P3
described how a bidirectional relationship between two variables were really due
to hidden, confounding variables causing both variables. P3 described how ``in
the real world...when these bidirectional things happen, it means there's
sort of this middleman complex system. Or some like underlying process of which
[two variables are] both components...'' Another participant, P2, wondered aloud
about how even what appears like a direct relationship, may actually be a chain
of indirect or mediated relationships at a lower granularity: ``It's like Google
Maps. If you zoom out enough, that arrow becomes a direct arrow.'' These
observations suggest that while participants can deeply reflect on what could be
happening between variables conceptually, they need help exploring and
figuring out which of these structures matches their implicit understanding.
% P3's definition of associative relationships
% is consistent with an associative structure called a ``fork'' in causal DAGs.
% \paragraph{Need help exploring and picking among possible structures.}


\begin{comment}
\subsubsection{More expressivity for specifying study designs/experimental design}
**keep short**
Additional observations about expressing study design
More future work for how to express study designs

TODO: Participant as a separate construct
\end{comment}

\subsubsection{Participants expected more syntactic sugar for specifying data collection details.}
While our focus was on improving the support for conceptual modeling, we made a
few observations about challenges analysts faced when specifying data collection
details. First, analysts expected \textit{experimental conditions to be
standalone concepts}. In Tisane, experimental conditions can be specified as a
\measure of a \unit. Instead, P0 and P4 had separate conceptual categories for
conditions and measures in their mental models of study designs. P4 preferred a
separate condition data type currently unavailable in Tisane because the term
``Measure'' did not create a ``bucket'' appropriate for conditions. Second,
participants were interested in specifying trials, stimuli, and responses
elicited during each trial alongside participants: ``I want to have a trial
unit that is nested within trials, which is nested within or maybe I could just
have trial nested within Participant, but I'm not seeing a way to clearly
delineate or like to denote that'' [P1]. Future work should more closely examine
and iterate on language constructs and idioms for representing data collection
procedures. 

% In Tisane, nesting a trial within a
% participant would mean that multiple trials exist in each participant (e.g.,
% akin to how multiple children exist in a family), which even in natural
% language, does not capture what P1 wanted to convey, which was that each
% participant sees and provides responses for each trial. 
% \paragraph{Participant construct - syntactic sugar for Unit}

% P4 resorted to declaring
% condition as a \setup variable, violating the intended semantics of \setup
% variables in Tisane: ``Like it doesn't feel like the condition are measured.
% Right? I'm measuring the proxy. I'm not measuring the condition. So to me,
% that's why I'm putting the visual on the four condition as a...\setup
% variable.'' Similar to P4, when P0 saw the ``SetUp'' class, she thought it
% described conditions: ``[I]f I heard the word `setup,' I would think of more
% like conditions like, like, this person uses this IDE or this person is given
% this IDE.'' 

\subsection{Domain-Specific Language Re-design Goals} \label{sec:rtisane_design_implications} 
\polish{ It might be worth mentioning that, even if Tisane does not use some of the more nuanced annotations for model suggestion, these are valuable for externalizing analysts' conceptual models for documentation and communication (potentially including preregistration).}

Based on our lab study observations, we derived four design goals for
re-designing Tisane's DSL to more accurately capture analysts' implicit
conceptual models: 
%  about how
% participants internally represent conceptual models and study designs have three
% implications for improving the design of Tisane.

\def\optionalSpecificity{\textbf{DG1 - Optional specificity}\xspace}
\def\interactionAsPartialSpec{\textbf{DG2 - Interactions as partial specifications}\xspace}
\def\considerPossibilities{\textbf{DG3 - Consideration of possibilities}\xspace}
\def\assumeHypothesize{\textbf{DG4 - Distinction between assumed and hypothesized}\xspace}
\begin{itemize}
    \item \optionalSpecificity: Analysts should be able to provide optional
    details about how variables change in relation to each other (e.g., positive
    or negative changes in values) when describing conceptual relationships.
    \item \interactionAsPartialSpec: Analysts should annotate conceptual models with interaction terms they want to include in an output statistical model. 
    \item \considerPossibilities: When expressing ambiguous relationships, analysts should have support
    in considering and picking among multiple possible conceptual structures.
    \item \assumeHypothesize: Analysts should be able to distinguish between assumed and hypothesized relationships in their conceptual models. 
\end{itemize}

In the second release of \tisane, we addressed these goals through new language
constructs. We also supported syntactic sugar to more accurately capture
study design details. 

\polish{"We also supported syntactic sugar to more accurately capture study design details."
> Not clear to me why syntactic sugar more "accurately" captures details as opposed to more "efficiently"/"concisely".
> Also, what exactly was the sugar? If you stated this, it did not stick in my memory...}

\begin{comment}
\textbf{Deconstruct statistical constructs for clarity.}
Unsurprisingly, we found that class and function names such as ``Measure,''
``SetUp,'' and ``moderates'' confused participants. These names either suggested
informal conceptual categories or were too close to statistical jargon. As a
result, connecting a Tisane program's semantics to a mental model of study
design was challenging. Particularly insightful was participants' consternation
with the function ``moderates.'' Participants could reason about the details of
moderation relationships but did not know what the term meant or how to use it
to clearly communicate their conceptual models in detail. Using
specific,granular language constructs (e.g., pairwise relationships) and
allowing for system-gudiance to consider more complex statistical structures
(e.g., interaction effects for moderation) may help analysts more accurately
understand and use Tisane.

\textbf{Allow for conceptual ambiguity, make specificity optional.}
Despite being statistical non-experts, participants were noticeably focused on
low-level details about conceptual relationships. All participants
differentiated between known and suspected relationships and generally agreed
that known relationships should be checked for prior to assessing the suspected,
or ambiguous, ones. At the same time, when participants were confident about
some relationships, they were specific about if the relationships were
positive/negative. Welcoming ambiguity and specificity could help analysts write
and use conceptual models throughout data analysis and interpretation.

Both design implications suggest the need to re-structure Tisane's programming
and interaction models. First, Tisane's specification language needs to be more
specific and low-level. This finding is \textit{counterintuitive} because a
well-documented strategy for making programming tasks easier for non-experts has
been to raise the level of abstraction for a programming
domain~\cite{chasins2018rousillon,satyanarayan2017vega}. Yet, we have evidence
that suggests the opposite for specifying a conceptual model and study design.
Second, Tisane's specification process could be more tiered, and disambiguation
could leverage ambiguity in analysts' specification as opportunities for more
numerous intelligent suggestions and guidance. 

\todo{Mention that the focus needs to be on conceptual level, but within conceptual level, there should be opportunity to move up and down the ladder of conceptual abstraction}

%hand-offs to a reasoning engine to suggest possible analysis paths, for example. 

% the conceptual modeling and study design specification process to be more tiered

\end{comment}

\clearpage % to force the listing to start at the top of the next page
\rTisaneProgram
\section{System Design and Implementation} \label{sec:rTisane}

\rTisane consists of (i) a DSL for analysts to express their conceptual
models and (ii) interactive disambiguation steps to compile this high-level specification into a
script fitting a statistical model. 

So far, we have implemented \rTisane for GLMs. Given the breadth of findings
from the elicitation lab study, we narrowed the scope from \tisane in order to
really focus on designing and testing a set of language constructs core to
conceptual modeling. 
% Given the breadth of findings from the elicitation lab study, we decided to
% really focus on a set of language constructs core to conceptual modeling. 
% a DSL for expressing conceptual models
There are two key challenges to designing \rTisane: (i) ensuring the DSL's
constructs can express analysts' implicit conceptual models accurately and (ii) %identifying DSL primitives that 
balancing usability with rigor, allowing
analysts to express their often ``fuzzy'' conceptual assumptions without losing
precision to derive a statistical model.

% \rTisane supports a two-step process for analysts to specify conceptual models
% that are both expressive and precise. 
\def\Participant{\texttt{Participant}\xspace}
\def\Unit{\texttt{Unit}\xspace}
\def\Condition{\texttt{Condition}\xspace}
\def\Conditions{\texttt{Condition}s\xspace}

\subsection{\rTisanes Domain-Specific Language}
Like \tisane, analysts express variables, a conceptual model, and a query for a
statistical model. \rTisanes DSL prioritizes expressivity and usability 

\subsubsection{Declaring variables}
Analysts can express two types of variables: Units and Measures. Units represent
observational or experimental units from which analysts collect data (see line 5 in~\autoref{lst:rTisaneProgram}). 
A common unit is a participant in a study, so \rTisane provides syntactic sugar for
constructing a \Participant unit directly. \Participant is implemented as a wrapper for
declaring a \Unit.

Measures are attributes of Units collected in a \dataSet, so they are declared
through a Unit. Measures can be one of four
types: continuous, unordered categories (i.e., nominal), ordered categories
(i.e., ordinal), and counts (see lines 6-18 in~\autoref{lst:rTisaneProgram}). \rTisane provides syntactic sugar for declaring
\Conditions as either unordered or ordered categories. Analysts declare
unordered and ordered categories through the \texttt{categories} function.
Analysts can specify a variable is ordered by passing a list to the
\texttt{order} parameter. Otherwise, the variable is considered unordered.
Analysts can use \texttt{continuous} and \texttt{count} functions to declare
continuous and count Measures. 
% We
% chose this design to reduce the number of unique functions and better match
% semantic similarity. 


\begin{comment}
Units
syntactic sugar: `Participant'

Measures
syntactic sugar: `condition'
\end{comment}


\def\causes{\texttt{causes}\xspace}
\def\relates{\texttt{relates}\xspace}
\def\when{\texttt{when}\xspace}
\def\then{\texttt{then}\xspace}
\def\assume{\texttt{assume}\xspace}
\def\hypothesize{\texttt{hypothesize}\xspace}

\subsubsection{Expressing a conceptual model}
Once analysts have constructed variables, they can specify how these variables
relate conceptually. To do so, they construct a \texttt{ConceptualModel} and add
variable relationships to it (lines 20-31 in~\autoref{lst:rTisaneProgram}). The conceptual model %\texttt{ConceptualModel}
is represented as a graph where the variables are nodes and the relationships
are edges. 

There are two types of relationships: \causes and \relates. \causes indicates a
unidirectional influence from a cause to an effect. \causes
introduces a directed edge from the cause node to the effect node. \relates
indicates that two variables are related but exactly how is ambiguous because
the analyst is uncertain about the direction of influence. \relates introduces a
bi-directional edge  between two variables. During a disambiguation step,
\rTisane will walk analysts through possible graphical structures that a
bi-directional edge could represent (\considerPossibilities). To derive a
statistical model, \rTisane requires an analyst to assume a direction of
influence.


Towards the design goal of \optionalSpecificity, \rTisane allows analysts to
optionally specify \when and \then parameters in the \causes and \relates
functions. There are four comparisons analysts can specify in
\when and \then: \texttt{increases} (for continuous, ordered categories,
counts), \texttt{decreases} (for continuous, ordered categories, counts),
\texttt{equals} (for any measure type), and \texttt{notEquals} (for any measure
type). Supporting optional specificity is designed to (i) make the \rTisane
program an accurate document of analysts' implicit assumptions and (ii) suggest
ways to resolve conceptual ambiguity during disambiguation
(\considerPossibilities). 
% used when suggesting ways to resolve
% ambiguity in the input program during disambiguation
% (\autoref{subsec:conceptualModelDisambig}).

To add relationships to the conceptual model, analysts must assume or
hypothesize a relationship. This distinction supports how analysts distinguish
between assumed, or strongly held, and hypothesized, or more uncertain,
relationships. \rTisane requires analysts to make these explicit distinctions
(\assumeHypothesize) when adding conceptual relationships to a conceptual model.
In addition to specifying a relationship type, analysts must either \assume or
\hypothesize a relationship. 

% While analysts are thinking through and specifying \causes and \relates
% relationships, 

Analysts can also specify interactions between two or more variables by
declaring \texttt{interacts}. Interactions are annotations to conceptual models
and are added to the graph without \assume or \hypothesize. Interactions provide
additional information about existing relationships in the conceptual model
(\interactionAsPartialSpec). 
% Causes / relates (types of relationships)
% Optional specificity: when, then annotations
% Assume / Hypothesize (label relationships)
% Interactions as annotations 
% - default semantics: if labeled, interactions considered. Otherwise, not

\def\query{\texttt{query}\xspace}
\subsubsection{Querying for a statistical model}
Analysts \query \rTisane for a statistical model based on the input conceptual
model (lines 33-34 in~\autoref{lst:rTisaneProgram}). The query asks for a statistical model to accurately
estimate the average causal effect (ACE) of the independent variable on the
dependent variable. The querying process initiates the interactive compilation
process and results in an \texttt{R} script specifying and fitting a generalized
linear model. During interactive compilation, analysts engage in two loops to
disambiguate their (i) conceptual model and (ii) output statistical model. 

\subsection{Two-step Interactive Compilation}
There are two phases to interactively compiling a conceptual model to a
statistical model: (i) conceptual model disambiguation and (ii) statistical
model disambiguation. We added conceptual model disambiguation to address the
need to explore possible conceptual structures for resolving ambiguities
introduced by \relates (\considerPossibilities).

\subsubsection{Conceptual Model Disambiguation} \label{subsec:conceptualModelDisambig} 
\conceptualModelDisambiguation
The goal of conceptual model disambiguation is to make analysts' expressed
conceptual models precise enough to derive a statistical model, achieving
usability and rigor. Conceptual model disambiguation involves breaking cycles in
the conceptual model by (i) picking a direction for any \relates relationships
and/or (ii) removing edges. Cycles are necessary to break because they imply
multiple different data generating processes that could lead to different
statistical models. In this way, conceptual model disambiguation can help analysts
reflect on and clarify their implicit assumptions. 

To disambiguate conceptual models, \rTisane uses a GUI.~\autoref{fig:figureConceptualModelsDisambiguation} shows the conceptual model disambiguation interface for the input program in~\autoref{lst:rTisaneProgram}. The GUI shows a graph
representing analysts' conceptual models. If there are any \relates
relationships, \rTisane suggests ways analysts could assume a direction of
influence. Additionally, \rTisane suggests ways to break any cycles in the
conceptual model. As analysts make changes, the visible graph updates. The GUI
also explains why both these steps are necessary to derive a statistical model. 

Once analysts have disambiguated their conceptual models, \rTisane updates the
internal graph representation and derives a space of possible statistical
models. To narrow this space of possible statistical models down to one output
statistical model, \rTisane asks additional follow-up disambiguating questions. 

\begin{comment}
This problem is actually challenging because detecting all cycles in a graph is
an NP-hard problem. We adapt a version of Johnson's(?) algorithm.
\end{comment}

\subsubsection{Statistical model derivation and disambiguation}
\statisticalModelDisambiguation
To formulate possible statistical models, \rTisane considers potential
covariates to control for confounding, interactions, and family and link
functions.

To determine confounders, \rTisane uses more recent recommendations from
Cinelli, Forney, and Pearl~\cite{cinelli2020controls}\footnote{\tisane relied on
Vanderweele's recommendations for confounder
selection~\cite{vanderweele2019modifiedDisjunctiveCriterion}, but in \rTisane we
opted for more recent recommendations}. Cinelli et al.'s recommendations are
based on a meta-analysis of studies examining the impact of confounder selection
based on graphical structures on statistical modeling accuracy. By following
Cinelli et al.'s recommendations, \rTisane includes confounders that help assess
the average causal effect of the query's independent variable on the dependent
variable as accurately as possible. 

\rTisane searches for interactions analysts annotated in their conceptual models
and suggests any involving the query's dependent variable. Otherwise, \rTisane
does not consider any interactions. 

\rTisane determines family and link functions based on the query's dependent
variable data type. For queries involving continuous dependent variables,
\rTisane considers Gaussian, Inverse Gaussian, and Gamma families. For counts,
\rTisane considers Poisson and Negative Binomial families. For ordered
categories, \rTisane considers Binomial, Multinomial, Gaussian, Inverse
Gaussian, and Gamma family functions. For unordered categories, \rTisane
considers Binomial and Multinomial family functions. \rTisane outputs
statistical models fit using the \lme package in \texttt{R}, so \rTisane
considers any family and link function combinations supported in \lme.

In the GUI, analysts have the option to remove any confounders or interactions
based on their domain knowledge. Based on prior experience or domain
recommendations, analysts can also pick a family and link function pair if
multiple possibilities could apply. 

\section{Summative Evaluation: Controlled lab study} \label{sec:summativeEval}
% \highlight{We are in the process of running this lab study and collecting data.}

Two research questions motivated our evaluation of \rTisane:

\begin{itemize}
    \item \evalConceptualModels What is the impact of \rTisane on conceptual
    modeling?
    \item \evalStatisticalModels How does \rTisane impact the statistical models
    analysts implement? Specifically, how well do the statistical models
    analysts author on their own vs. with \rTisane fit the data? How are their
    formulations similar or different?
    % \item \evalLearning Do analysts learn about their discipline or data
    % analysis as a result of using rTisane?
\end{itemize}

\subsection{Study design}
% \todo{Include a diagram summarizing study design?}
We conducted a within-subjects (Tool support: \rTisane vs. none) think-a-loud
lab study that consisted of four phases. We designed the study based on the
assumption that conceptual modeling is a helpful strategy when specifying
statistical models. As a result, all participants completed the phases in the
following order.

\begin{itemize}
    \item \textbf{Phase 1: Warm up.} We presented participants with the
    following open-ended research question: ``What aspects of an adult's
    background and demographics are associated with income?'' We asked
    participants to specify a conceptual model including variables they thought
    influenced income. This warm-up exercise helped to externalize and keep
    track of participants' pre-conceived notions and assumptions prior to seeing
    a more restricted data schema.
    \item \textbf{Phase 2: Express conceptual models} We presented participants
    with a data schema describing a dataset from the U.S. Census Bureau. We then
    asked participants to specify a conceptual model using only the available
    variables. At the end, we asked participants about their
    experiences specifying their conceptual models in a brief survey and semi-structured interview.
    \item \textbf{Phase 3: Implement statistical models} We asked participants
    to implement ``a statistical model that assesses the influence of variables
    [they] believe to be important (in the context of additional potentially
    influential factors) on income,'' relying on only their conceptual model. We
    then asked participants about their experiences implementing statistical
    models through a brief survey and semi-structured interview. 
    \item \textbf{Phase 4: Exit interview.} The study concluded with a survey
    and semi-structured interview where we asked participants about their
    experience in the study, reactions to using \rTisane, and connecting
    conceptual models to statistical models.
\end{itemize} 

In order to assess the effect of tooling on conceptual models and the quality of
statistical models, we counterbalanced the order of tool support, or if
participants completed each task with or without rTisane first. The order of
tool use was the same for Phase 2 and Phase 3. Within each of Phase 2 and Phase,
half the participants completed the task on their own (without \rTisane) then
with \rTisane. The other half started with \rTisane and then did the task on
their own. Prior to using \rTisane in Phases 2 and 3, participants followed a
tutorial introducing the relevant language
constructs.~\autoref{appendix:summativeEvaluation} contains all the study
materials.

\noindent \paragraph{Participants} We recruited 13 data analysts on Upwork. We
screened for participants who reported having experience with authoring
generalized linear models and using R at a three or higher on a five-point
scale.~\autoref{tab:summativeEvaluationParticipants} summarizes the
participants' backgrounds. All studies were conducted over Zoom. Participants used \rTisane
on a remote controlled computer, so they did not have to install it on their
own. Each study lasted between two and three hours. Participant was compensated
\$25 per hour. We recorded participants' screens, video, and audio throughout
the study. We then transcribed the audio and used detailed researcher notes for
qualitative analyses.

\tableSummativeEvalParticipants
% One dataset was on demographic factors and income in the U.S. in 2018 from the
% U.S. Census Bureau (\datasetIncome). The other dataset was about demographic
% factors and health conditions in the U.S. (\datasetHealth) from \ej{FILL IN}.
% More information on how the datasets were created are found in the
% supplemental material.

\subsection{Analysis Approach}
Our analysis procedure consisted of two parts: (i) a thematic analysis of lab
notes, transcripts, and open-ended survey questions and (ii) an artifact
analysis of conceptual models and statistical models analysts authored with and
without \rTisane. For the conceptual models, we compared their form and content
between tool support conditions. For the statistical models, we compared the
overall statistical approach, specific statistical model formulation, and
rationale for analysis decisions and conclusions. We also compared two goodness
of fit measures between statistical models: AIC and BIC. We iterated on the
thematic analysis and artifact analysis separately at first and then interpreted
emergent observations across the two analyses. 

One of the 13 participants dropped out part way through the study due to
discomfort with programming in front of the researchers. We analyzed the data we
were able to collect from them. 
% The artifact analysis helped to
% ground and explain aspects of the thematic analysis.

\begin{comment}
We collected and analyzed quantitative and qualitative measures to answer our
research questions. For each research question, we describe our analysis
approach and results below. 
% We qualitatively analyzed conceptual models, statistical models, audio
% recordings, and open-ended survey responses. Whenever possible, we also
% quantified the frequency of characteristics in conceptual models and
% statistical models across participants. For the statistical models
% participants authored, we compared the AIC and BIC scores for matched pairs of
% the independent variable of interest. 


\subsubsection{\evalConceptualModels}
** How does rTisane compare to on own? **
Main effect of rTisane: Initial CM is very ambiguous and not formal, rTisane is making it more formal. 
- distribute IR
- use IR to target other things (e.g., science diagrams that a Participant said)

On own, participants were not sure how to structure a conceptual model. 
Diversity of conceptual models [from conceptual model analysis] -- form, complexity (number of relationships levels of relationships)


API structure + ... 
START HERE: 
- Group API clusters into smaller clusters
- Write one sentence about each cluster

<Decouple conceptual model from data?>

In general, people able to express conceptual model with rTisane. No major missing concepts, minor syntactic sugar + more advanced constructs desirable;
Critical to mention: **"no perceived influence"** -- confirmed by no statistical differences in NASA-TLX scores between conditions
Takeaway: rTisane provides process but not suggest relationships (control up to end-user) - no agenda for how end-users build model

% Consistency within vs. across participants 
We qualitatively analyzed how consistent participants' conceptual models were
between conditions. We noted common challenges translating free-form conceptual
models into rTisane programs. We also thematically analyzed participant
transcripts and survey responses describing the influence of rTisane on their
conceptual modeling processes. 

% \subsubsection{\evalConceptualModelAuthoring}
% % To understand rTisane's influence on the conceptual modeling process, we
% % compared the ratings and thematically analyzed survey responses to questions
% % asking how participants decided what conceptual relationships to specify.
% At the end of Phases 2 and 3, we asked participants to rate and describe their
% experience specifying conceptual models without and with rTisane, respectively.
% In addition, we kept track of the conceptual modeling challenges participants
% vocalized and noted how participants overcame these challenges in-situ. We
% compared the survey responses, observed challenges, and observed approaches
% between Phases 2 and 3 to assess how rTisane influenced the conceptual modeling
% process. 

\subsubsection{\evalStatisticalModels}
We used AIC, BIC, and R-squared values to assess how well statistical models
authored with vs. without rTisane fit the data. We used rTisane to statistically
model and assess the influence of rTisane on AIC, BIC, and R-squared values.

We also thematically analyzed participants' reactions to the similarities,
differences, and surprises between statistical models. 

% \todo{Add statistical models executed}
% \todo{Add rTisane analysis script to supplemental material}
\end{comment}

\subsection{Findings}
\subsubsection{RQ1: \rTisane's Impact on Conceptual Models}
\conceptualModelsScaffold

\textbf{Key takeaway: \rTisane scaffolded and productively constrained how analysts expressed
their conceptual models. As a result, analysts reflected on implicit domain
assumptions more deeply, considered new relationships, and felt they
accurately externalized their implicit assumptions.}

The conceptual models analysts expressed on their own were diverse in form,
meaning/content, and complexity. The majority [P2, P4, P5, P8, P11, P13] invoked
a graph-like structure. [P2, P4, P8 used \rTisane second; P5, P11, P13 used
\rTisane first].~\autoref{fig:figureConceptualModelsScaffold} illustrates four example conceptual
models from participants.\footnote{An example conceptual model given in the task instructions may
have biased analysts towards a graphical structure.} Participants also described
their conceptual models verbally [P10], in natural language text [P6, P9], and
as a timeline [P12]. P7, who used \rTisane first, even jumped to expressing
their conceptual model in a statistical model. P12's conceptual model was
particularly creative. His timeline featured variables ordered starting on the
left by how much an individual could intervene upon them (\autoref{fig:figureConceptualModelsScaffold}). P12's conceptual model reiterates our finding from the exploratory lab
study that analysts want to capture nuanced meaning in a conceptual model. 

Ten participants involved all five independent variables from the \dataSet in
their conceptual models [P2, P3, P4, P5, P7, P8, P9, P11, P12, P13]. Two
participants [P7, P13] also included interactions between variables in their
conceptual models. For instance, P13 specified a complex conceptual model
(\autoref{fig:figureConceptualModelsScaffold}) where age, race, and sex interacted to cause an
interaction between education and employment, which then causes income.

\theme{Without \rTisane, analysts found it difficult to express conceptual nuances.}
In a survey and interview about their conceptual modeling experiences,
participants shared that they found it difficult to author conceptual models
without tool support due to doubts about how to communicate nuances in
relationships [P3, 13] and concerns about mis-specifying relationships beyond
their domain knowledge [P5, P10]. P13 explained how they wanted to ``[i]dentify
how I may weigh certain variables based on my general awareness and knowledge
and overall weights of each variable of how one may affect income more or less
in various circumstances.'' Similarly, P8 described specifying their conceptual
model as a general ``struggle'' because ``When doing it myself, there are so
many possibilities [of expression].'' While \rTisane is not designed to prevent
mis-specifications due to limited domain knowledge, we found that \rTisane's
formalism removed the need for analysts to come up with how to express their
domain knowledge. They could focus on expressing what they knew. 

% With rTisane 
\theme{\rTisane encouraged analysts to think about their domains more deeply.}
\rTisane’s DSL deepened participants’ thinking [P3, P4, P7, P8, P10, P12, P13],
giving them, as P12 described, a structure to explore the ``boundaries of their
domain knowledge.'' P3 explained how even after specifying conceptual models on
her own, \rTisane’s four composable relationships (\assume/\hypothesize $x$
\causes, \relates) facilitated a deeper consideration of each relationship and
what she knew about each: ``Having to think about specifics like 'Do we know the
direction of the relationship' or 'What happens when a category
increases/decreases' actually helped me put my thoughts out more clearly. I was
able to think about more possible scenarios that could conflict with my current
assumption, which I was probably not doing [before]...In conclusion, I want to
say that looking at four possible ways to write a relationship made me think
more about each one of them.'' [P3] Similar to P3, P10 explained, ``My thinking
was that before I didn’t have much idea about how can I link my variable with
the output [variable], and how this can interact. And so it may need some trial
and error... using this API, there are predefined functions, they are translated
in R language, cause or relates, it made my task easier. This translation was
not on me anymore.'' Furthermore, P4 explained how the DSL’s support for
optional specificity ``encouraged [them] to think about the directionality of my
hypothesized relationships and for categorical variables to think about the
effect of each individual category.'' 
% Additionally, \rTisane’s language
% constructs helped participants more easily make explicit what they had in mind. 

\theme{\rTisane provided structure to express and inspect conceptual models.}
Participants appreciated how \rTisane structured their conceptual modeling
process [P2, P4, P9, P10, P11, P12, P13]. Participants found the \rTisane DSL
particularly helpful. P9 explained how rTisane ``led [him] to think about the
relationships first, and then whether they were what [he] was hypothesizing''
and how this process was the ``reverse of the way [he] would think about it
normally.'' Similarly, P4 explained how using the rTisane language constructs
required them to think through how different values of a variable (e.g.,
different categories) could change income. They observed that their conceptual
model with \rTisane was ``more specific'' than without tool support. P4 further
explained how \rTisane’s DSL ``encouraged [them] to think about interactions,
which [they] hadn't thought about before using rTisane.'' Four participants said
that \rTisane generally made it easier for them to specify their conceptual
models [P4, P8, P10, P12]. P4 and P10 even believed that rTisane’s ``formal
structure made [conceptual modeling] more rigorous'' [P4] and ``more
disciplined'' [P10].

Participants relied on the conceptual disambiguation step to verify that what
they expressed in code accurately represented their implicit assumptions [P2,
P8, P12]. P2, who had drawn a conceptual model as a graph on his own prior to
using \rTisane, said, ``The interactive process was a good
way to check that the graph came out the same way I was picturing it. It was
helpful because it is easier to look at than code'' [P2]. 
% said ``The idea of a graph made me think much differently about the
% model. I am not sure if this improved my model but it definitely made me
% approach it differently.'' Interestingly, P2 had drawn a conceptual model as a
% graph on his own prior to using rTisane. However, it seems that moving back and
% forth between code and a graph helped him to think about his assumptions in
% different ways because he also expressed, `

\theme{\rTisane is expressive enough to capture analysts' conceptual models accurately.}
Importantly, rTisane scaffolded the conceptual modeling process without
compromising expressivity. Five participants reported that rTisane had no
perceived influence on their conceptual models [P3, P4, P5, P6, P11]. Indeed,
three participants expressed identical conceptual models with and without
\rTisane [P9, P11, P12]. Interestingly, for six participants, the
conceptual models they authored with \rTisane were subgraphs of conceptual models
authored without \rTisane[P2, P3, P4, P5, P7, P8]. 
% Three participants included different variables [P13,
% P6] and relationships [P10] between conditions. 
For P2, P3, P4, and P8, who used \rTisane second, \rTisane appeared to help
focus them on a set of variables and relationships to analyze. P3 explained,
``As I started working with \rTisane, my first instinct was still to go back to
the canvas and do a brainstorming. The process of listing down the categories
and the generic relationship between the variables (which was biased to my
personal opinion) was still the same (with or without rTisane).'' For P5 and P7,
who used \rTisane first, \rTisane provided a starting conceptual model expand
upon on their own. For example, P7 authored a statistical model involving an
interaction between variables in their \rTisane conceptual model when asked to
specify a conceptual model on their own. It seems that just conceptual modeling
with \rTisane helped P7 translate a conceptual model to a statistical model on
his own. Taking these observations together, it seems that \rTisane's DSL can
support both convergent and divergent creative thinking about analysts' domain
knowledge. 
% Both these scenarios suggest that rTisane’s DSL for expressing
% conceptual models can support both convergent and divergent creative thinking
% about a domain, even to the point of helping an analyst, like P7, arrive at a
% statistical model on their own. 

% Key takeaway: rTisane was expressive, as seen in how particpiants felt free and
% able to express their diverse conceptual models using rTisane, rTisane did not
% impose the content. Rather, the key benefit of rTisane was in providing
% structure to / facilitating this process more.

\subsubsection{RQ2: \rTisane's Impact on Statistical Models}
\textbf{Key takeaway: With the exception of picking family and link functions,
\rTisane focused participants on their analysis goals over low-level details
that bogged them down without tool support. As a result, \rTisane improved the
statistical model authoring process, output statistical models, and
communication about statistical analyses.
% In the future, rTisane could improve explanations for
% analysis decisions and alternatives. 
}

On their own, three participants were not able to author a statistical model due
to unfamiliarity with statistical methods [P3], lack of time [P5], and reliance
on visual analyses (ie.g., heatmaps, scatterplots) [P12]. Of the remaining nine
participants who completed the study, six participants successfully authored
linear regression models [P2, P4, P7, P8, P9, P10]. A seventh participant, P6,
started to author a logistic regression model with Race and Income but stopped
before binarizing either variable. Two participants, both of whom had just
finished authoring statistical models with rTisane, implemented GLMs [P11, P13].
Despite task instructions, P11 started from the \rTisane output model script to
author their own. After observing the model's ``AIC is large, the residual is
large'' P11 determined ``I don't think this [\rTisane output model] is the right
fit.'' So, they log-transformed the income variable and fit a new statistical model. P11's experience mirrors how we anticipate analysts to build
upon \rTisane output statistical models in the future. 

\theme{Statistical models authored with \rTisane fit the data just as well or better than statistical models without \rTisane.}
Of the eight participants who successfully authored linear regression or
generalized linear models on their own, three implemented identical models with
or without \rTisane [P7, P9, P13]. Notably, all three had authored the
statistical model with \rTisane first, suggesting that \rTisane may have biased
their own modeling process. For another three participants [P4, P8, P10], their
statistical models with \rTisane had lower AIC and BIC scores than the
statistical models without \rTisane. In other words, \rTisane models fit the data
better or equally well for six out of eight participants. For P11, the
statistical model they authored without \rTisane dropped some observations, so
the models are not directly comparable. For P2, the \rTisane statistical model fit
worse than his own statistical model in part due to an observed change in his
motivation for analysis, discussed below. 

\theme{Without \rTisane, analysts change their analysis intent during statistical modeling.}
Without \rTisane, participants [P2, P5, P6, P8, P10], adopted a more exploratory
or ``data-driven'' approach, changing their analysis goals while authoring
statistical models. This theme is best illustrated by P2, who started with a
hypothesis that Occupation, or Employment, influenced Income. His conceptual
model in \rTisane had the variables Education, Age, Race, and Sex causing
Occupation, which in turn, causes Income (\autoref{fig:figureConceptualModelsScaffold}). 

He started authoring statistical models with the intent to assess this
hypothesis. On his own, he first authored an ANOVA with Occupation as the IV and
Income as the DV. Once he saw that Occupation had a statistically significant
influence on Income, he changed his analysis goal to assessing if the variables
causing Occupation would ``be able to predict which occupation...And then...the
income from the occupation just because that’s how I like structured it [in the
conceptual model] initially.'' However, P2 got stuck on how to author a model
with Occupation as the outcome variable because it was categorical, saying,
``But the way I structured it in like the diagram. I'm not sure exactly how to
do that, because Occupation's like categorical. Um, so I'm not sure like
exactly...how to model that.'' This roadblock led P2 to consider an alternative
``regression model with Income as like the output and then...all [the IVs] as
terms and then just include the interactions between Occupation and the terms
that were pointing into it, and that would just be one model.'' In other words,
P2 tried to author a single statistical model to assess if there was evidence
for his conceptual model. However, he was unaware of three key things. First,
given his conceptual model, he did not need to account for the other variables
to estimate the influence of Occupation on Income and assess his hypothesis.
Second, adding interaction terms would not capture the dependencies in the
conceptual model. Third, what P2 likely needs to assess all the relationships in
his conceptual model is a structural equation model.

While it is well documented that statistical analysis is an iterative
process~\cite{grolemund2014cognitive, jun2022hypoForm} and we saw evidence of
this among participants [P5, P6, P10, P11, P12], what P2's experience
exemplifies is how creative participants can be in convincing themselves that the
statistical model they authored not only assessed a particular hypothesis but
could also arbitrate if their conceptual models were supported by data.
Furthermore, this suggests an opportunity for \rTisane to support a more
iterative analysis process and help analysts author multiple models to assess an
entire conceptual model, not just the influence of a single independent variable
on a dependent variable. 

% First, rTisane keeps analysts
% focused on considering a set of variables based on their domain knowledge
% whereas without, analysts either struggle to author statistical models, try to
% justify analysis decisions based on results, or author poor fitting statistical
% models. 
% Second, to prevent faulty interpretation of statistical modeling results as
% support for an entire conceptual model, rTisane should support analysts in
% expressing a query to author SEMs to assess a conceptual model. 

% Trying out different models and seeing how each one assessed their
% entire conceptual models. From one perspective, analysts subtly changed their
% implicit analysis goal from assessing the evidence of the influence a variable
% or set of variables on Income to the seeing how the data supported their
% implicit assumptions. While statistical models can suggest evidence (or lack
% thereof) for supporting a conceptual model, to rigorously assess an entire
% conceptual model, analysts would really need to author structural equation
% models, which we did not observe at all. Therefore, we observe two things:

% Participants thought their statistical models gave support for their conceptual
% models based on the statistical significance of variables in the authored
% models.

\theme{Without \rTisane, analysts find statistical model formulation challenging.}
Participants reported formulating and evaluating statistical models [P2, P3, P5,
P8, P12], programming [P6, P13], and preparing data [P7] as the major challenges
to authoring statistical models without rTisane. For example, P3 explained how
``There are a number of statistical tests and it gets confusing if I don't
practice it frequently. This is what happened today, I haven't worked on a
hypothesis testing problem recently and while I knew what libraries to go to, I
was not sure which test to implement.'' Similarly, discussing the details of
which covariates to include in a statistical model given a conceptual model, P8
explained how he was uncertain about which ``upstream relationships,'' or
indirect causes, to include in a statistical model. Without \rTisane, he
described statistical model authoring as ``It immediately feels harder doing it
directly [without rTisane] like this'' [P8].
% P5 explained how ``A good understanding of all stats models to
% choose from is required.'' 

\theme{\rTisane focused analysts on their motivation for analysis.}
% rTisane’s main impact was that it turned analysts’ focus away from details that
% can distract from formulating a statistical model towards their motivation for
% analysis. As a result, 
In contrast, participants reported that \rTisane guided them to think about
their domains more [P2, P12], lightened their burden in authoring statistical
models [P10], and even promoted research transparency [P5] and reproducibility
[P4]. Furthermore, rTisane reinforced prior knowledge about statistical methods
[P6, P11] and helped participants learn more about GLMs [P4, P6, P7, P13]. P6,
who had tried to author a logistic regression model on her own, explained how
she could apply what she learned from using \rTisane to future analyses: ``I
like that a multivariate linear regression was used because this will inform
any future data analysis...''

\begin{comment}
Nevertheless, participants expected \rTisane to do more automatically.
Participants were expected rTisane’s output statistical model to include more
IVs [P2, P5], have interaction terms [P5, P6], and have coefficient values that
were similar to ones from their own statistical models without rTisane [P10].
Yet, when asked if anything surprised them about rTisane’s output statistical
model, six participants said the models were as they expected [P4, 6, 10, 11,
12, 13]. 
\end{comment}

% When using rTisane, P2 got as output a GLM with Occupation as the only IV and Income as the DV. While P2 wanted more of an explanation for why additional variables were not needed, he found that using rTisane made him think ``more about how things were related...more about relationships, what’s proven beforehand and what you’re looking to prove’’ [P2]. 
% This suggesting opportunities to not only explain the result of an analysis but also why the tool did the SM. 

% Given that asking about family and link functions is a low-level statistical
% implementation detail, the contrast between rTisane’s conceptual modeling
% abstractions and the family and link selection options was particularly stark.
% For example, 
% In the future, it may be more usable for rTisane to suggest a specific pair and
% explain its suggestion rather than require the analyst to pick.

\theme{Analysts want to use \rTisane for scientific communication, not just statistical authoring.}
When asked how they might imagine using \rTisane, participants identified two
groups who would benefit: analysts regardless of experience and less technical
team members. First, participants described how experienced and novice analysts
alike would benefit from using \rTisane [P2, P4, P9, P10, P12]. Second,
participants mentioned how conceptual models written using \rTisane could be
used as boundary objects~\cite{star1989boundaryObjects} in collaboration with less technical
stakeholders [P8, P9]. P8 detailed how a conceptual model written using rTisane
could be a communication tool, saying how the ``visual representation would play
a role in a dialogue with the PI.'' P8 went on to imagine how he would like to
use \rTisane’s conceptual model to generate process diagrams in scientific
papers. In other words, how \rTisane’s conceptual model could serve as an
intermediate representation for multiple kinds of outputs, not just statistical
models. 


\begin{comment}
\theme{Analysts want to use \rTisane iteratively.}
Participants described their typical analysis approach without rTisane as
iterative. During the study, this looked like visualizing the data [P6, 10,
11, 12], assessing correlations between variables [P6, P10] to pick variables
for an initial statistical model, or starting with ``a full model first and
then trim down and compare'' [P11]. In the scaffold condition, analysts engaged
with the connection between their conceptual and statistical models [P2, 5,
12]. For example, P5 refined her conceptual model prior to implementing a
statistical model: ``Conceptual model matters because it gives something to
start with...how all of them have impacts...after have done analysis, look at
R-sq and p-value to interpret significance of these predictors.'' P5 also
grappled with how to interpret the statistical modeling results in light of her
conceptual model, especially since she PICK Background or Prediction as reason.
viewed the purpose of analysis to be able to predict income: ``Really don’t see
how statistical analysis helps us with why…'' She also explained how even though
she had a conceptual model, she did not feel she had the appropriate
background:``[if she had] more of a social science...background knowledge that
we have to dig deeper.'' While P5 could have interpreted her statistical models
in light of the conceptual model she authored, her observation about background
is nevertheless indicative of….
\end{comment}


\subsection{Discussion}
\rTisane benefits analysts' conceptual models and statistical models. \rTisane's
DSL is expressive to capture analysts' diverse, nuanced conceptual models. In
addition, Importantly, the DSL's language constructs served as a starting point
for statistical analysts to reflect on their domain knowledge. A consequence of
\rTisane's DSL and interactive compilation process is that some participants
were able to author statistical analyses that they were not able to author on
their own. Others could author statistical models that fit the data better than
their own statistical models. These results highlight three key insights in
\rTisane: the benefits of a formalism, balancing usability and rigor, and the
potential for re-purposing the intermediate representation. 

% Benefit of formalism: Usability - communication tension Formalism as scaffolding
% and reflection While rTisane does not address the challenges of domain
% knowledge, it does focus analysts on aspects they can have control over.
% [Address how rTisane does not address all the problems users face] 

While unbounded expression in natural language, especially in the era of
ChatGPT~\cite{brown2020language}, is enticing, we found that participants found
the prospect of expressing their conceptual models using any means daunting. A
key benefit of \rTisane is that it introduces a formalism that productively
reduces the potentially infinite space of how to express conceptual
relationships into a finite set that is expressible in the API. Furthermore,
based on feedback from participants in the summative evaluation, it seems that
the DSL is effective because it is not only expressive but also usable, which we
attribute to our iterative language design process involving end-users.
Moreover, learning to use \rTisane's formalism required participants to reflect
on their domain knowledge. This highlights how a DSL structures the
specification and can turn the process of specification into a reflective
activity. In this regard, the conceptual disambiguation step was critical. The
graph visualization in the GUI helped analysts reflect on what they expressed
and how to resolve any ambiguities present. 

% The API is good enough to capture what people want to
% say. The formalism has the added benefit that it provides constraints/scaffolds
% for end-users to explore the boundaries of their domain knowledge. A map for
% reflection. 

A key challenge in designing \rTisane was balancing usability and rigor. On one
hand, we wanted to make it easy for analysts to express their conceptual models
(usability), but we also wanted to ensure that the conceptual models they
expressed were amenable to formal causal reasoning to derive statistical models
(rigor). We were able to achieve both in \rTisane by designing usable language
constructs in the DSL and increasing precision for rigor during disambiguation.

Finally, participants discussed the potential for using conceptual models to
communicate with less technical collaborators. Implicit in this recommendation
is an acknowledgement of the usefulness of a conceptual model as an intermediate
representation. While \rTisane is focused on using conceptual models to derive
statistical models, there may be additional ``backends'' to target, such as
scientific model diagramming or planning study procedures. 
% While a concern for adoption is that analysts may not want to express conceptual
% models if they do not see additional benefit
% Participants emphasized using these conceptual models as boundary
% objects with collaborators, especially those with less statistical expertise.
% One participant, P8, even equated conceptual models to the process diagrams in
% scientific papers. Put another way, conceptual models are useful intermediate
% representation for not just statistical analysis but also communication, a
% direction we are excited to pursue in the future. 

% The CM disambiguation interface enabled participants to ensure that their
% expressed conceptual model is what they intended and think more deeply about
% their domain. Participants emphasized using these conceptual models as boundary
% objects with collaborators, especially those with less statistical expertise.
% One participant, P8, even equated conceptual models to the process diagrams in
% scientific papers. Put another way, conceptual models are useful intermediate
% representation for not just statistical analysis but also communication, a
% direction we are excited to pursue in the future. 

% **important**Interpreted using the theory of hypothesis formalization, we find
% that analysts author statistical models following a pattern similar to
% hypothesis formalization. rTisane embodies hypothesis formalization, a core
% activity in statistical authoring + skill in developing statistical expertise.
% Therefore, we gain evidence of the benefits of supporting hypothesis
% formalization in tools. 
% Larger takeaway: ** Not about just the scaffolded steps but about the tool
% support for executing each of these steps**

% \subsubsection{Key takeaways}
% \textbf{Larger takeaway: ** Not about just the scaffolded steps but about the tool support for executing each of these steps**}

\subsection{Limitations and Future Work on \rTisane}
While participants found \rTisane helpful, they suggested four areas of
improvement: (i) family and link function selection, (ii) statistical model
interpretation, (iii) iterative model revision, and (iv) general system
usability. 

% \theme{\rTisane needs to provide more support for selecting family and link functions.}
Despite benefiting from rTisane, many participants had difficulty picking family
and link functions in rTisane [P2, P4, P5, P9, P10, P11]. P4 explained, ``I
didn't understand the benefit or tradeoffs between different specifications. It
wasn't obvious to me how to create a linear OLS regression, or why I would want
to use a specification besides linear OLS.'' Given how frequently participants
described \rTisane as facilitating higher-level of thinking, we attribute the
difficulty of selecting family and link functions to the stark contrast between
\rTisane’s relatively high-level conceptual modeling abstractions and the
low-level nature of selecting family and link functions. In the future, it will
likely be more usable for \rTisane to suggest a specific pair and explain its
suggestion rather than require the analyst to pick.

Once analysts execute the output statistical model from \rTisane, they find the
output results too low-level. Because \rTisane uses lme4 under the hood, the
outputs are the default model outputs from lme4. However, given that \rTisane's
input language is at the conceptual level, analysts expected the outputs to at
least relate back to the conceptual model they input. In other words, the input
and output levels of abstraction should be commensurate. This support would
facilitate what analysts already try to do with statistical analyses they author
on their own without \rTisane. P8 found the output from lme4 overwhelming,
saying, ``Looking at the summary() in R was too much to look at.'' He suggested a
simple way to tie the results back to his input conceptual model: ``Would be nice
if you could have the same visual representation with p-values/coefficients!''

Furthermore, while participants could iterate on their conceptual models by
adding or removing variables and relationships, they could not engage in a
larger iteration loop with their output statistical model from \rTisane.
Improving statistical result interpretation would help with model iteration. In
addition, participants also sought more direct support. For instance, P11
described the \rTisane output statistical model as ``an initial or baseline model
but follow-up evaluation of the model is needed.'' They wanted to ``go back and
tweak things a bit'' about their statistical model. This kind of model iteration
is not only typical of the participants' workflows but also even a best practice
recommendation from the statistics community~\cite{gelman2020modelExpansion}.
Supporting novice and more experienced analysts revise models will likely
require different levels of abstraction and automation. 

Finally, participants found going back and forth between code and an interface
outside their IDE complicated and ``clunky.'' While part of this may have been
in part due to the fact that participants were using \rTisane on a remote
desktop, embedding \rTisane in a notebook seems likely to reduce major usability
issues. Additionally, participants gave suggestions for syntactic sugar for
specifying conceptual models. For example, instead of specifying multiple causes
and relates statements, they wished they could batch specify and add them to the
conceptual model. Ways to reduce the specification burden for analysts by
providing syntactic sugar or even removing the need to program at all are
interesting avenues to explore. 


\section{Discussion, Limitations, and Future Work}
The exploratory lab study suggested the need to allow analysts to express their
conceptual models using more granular, low-level functions. Although obvious in
hindsight, this finding was \textit{counterintuitive} at the time. A widely held
belief, especially within the HCI community, is that the higher the level of
abstraction for a task, the better for end-users. However, we saw the opposite.
Statistical non-experts engaged deeply with conceptual models about their domain
and wanted to be more detailed and specific. In other words, while the focus on
the abstraction should be at the conceptual level, within that, analysts want to
move up and down the ladder of abstraction. More generally, our iterative
language design work with \tisane and \rTisane suggests that as long as
abstractions match the content-focus of end-users, there should be opportunities
to get low-level within those abstractions. This gives end-users the agency to
express themselves more fully, transforming the programming task from strictly a
means to an end to specification as a meaningful activity in itself. 

% Second, Tisane's specification process could be more tiered, and disambiguation
% could leverage ambiguity in analysts' specification as opportunities for more
% numerous intelligent suggestions and guidance. 
% \todo{Mention that the focus needs to be on conceptual level, but within conceptual level, there should be opportunity to move up and down the ladder of conceptual abstraction}

In designing \rTisane, a key challenge was in finding the right point to bring
in lower-level statistical modeling details. Concretely, in \rTisane analysts at some point must grapple with graphical and
mathematical representations in the disambiguation phases. This is because it
was not possible to remove all complexity from statistical modeling without the
risk of losing the analyst's sense of control or understanding. Thus, our focus
has been to strip away unnecessary complexity and help analysts navigate through
necessary complexity by designing informative abstraction lowering
disambiguating steps. It may be possible to avoid any interactive disambiguation
by executing all possible statistical models given an input, likely ambiguous,
conceptual model. Although this approach would accomplish a different objective
than our goal of compiling a specific conceptual model into a specific
statistical model, this approach may give greater insight into if analysts
really want, need, or benefit from disambiguation. 

\begin{comment}

Despite the goal to lower the
barriers to statistical specification, at some point, 

Another approach to explore in the
future may be to eliminate the need to engage analysts in disambiguation and
instead execute all possible statistical models given an input, likely
ambiguous, conceptual model. Although this approach would accomplish a different
objective than the goal here of compiling a specific conceptual model into a
specific statistical model, this approach may give greater insight into the need 

% Our answer was to have informative conceptual and
% statistical model disambiguation phases. 
robustnes of a particular effect in light of many possible conceptual models and
explanations.

% push on the idea that analysts care about conceptual ramifications and avoid asking them to 

In the future, there could be additional exploration into authoring a multiverse
of all possible statistical models given a specific ambiguous conceptual model.
This would accomplish a different objective than the goal of \tisane (and
\rTisane), which is to compile a specific conceptual model into a specific
statistical model for/with the end-user. The multiverse would help assess the
robustnes of a particular effect in light of many possible conceptual models and
explanations.

Concretely, it
took us several iterations to answer the question: What is the right point to
introduce cycle breaking and modeling to the end-user? 

**not remove all complexity but rather focus end-users on necessary complexity
and guide their thinking/help them navigate that complexity. 

\end{comment}

\section{Summary of Contributions}

% Reminder: Thesis statement
% Domain-specific languages that provide abstractions for expressing conceptual
% knowledge, data collection procedures, and analysis intents instead of specific
% statistical modeling decisions coupled with automated reasoning to compile
% conceptual specifications into statistical analysis code help statistical
% non-experts more readily author valid analyses. 

\rTisane provides a DSL with language constructs for expressing conceptual
models (\thesisChallengeExplicit) and integrates a two-phase interactive
disambiguation process for compiling conceptual knowledge into statistical
analysis code (\thesisChallengeRep). In a controlled lab study of \rTisane, we
found that the DSL is expressive enough to capture analysts' conceptual models
accurately, eases the burden of making their implicit assumptions explicit, and
pushes analysts to think about their domains more deeply. Using \rTisane,
analysts were able to author statistical models that fit the data just as well
as if not better than statistical models authored on their own. \rTisane even
helped analysts who were not able to author statistical models on their own get
to an output statistical model. Analysts also reported that through the process,
they learned about GLMs (\thesisChallengeUnderstanding). Together, these results
demonstrate how DSLs and automated reasoning together in fact do help
statistical non-experts author valid statistical analyses that they would not be
able to author otherwise.
% evidence for how
% connecting conceptual modeling to statistical modeling increases the statistical
% conclusion and external validity of analyses~\cite{shadish2010campbell}.  

% The conceptual model disambiguation process in \rTisane
% also facilitates reflection on implicit knowledge. 

% we refined what the programming and interaction model
% for expressing conceptual models and connecting them to statistical models
% should be. Most notably, the second release of \tisane, as \rTisane, provides
% more explicit support for conceptual model specification and disambiguation. 

% \tisane and \rTisane are in stark contrast to the current ecosystem of
% statistical analysis software designed to give analysts maximal mathematical and
% computational control at the cost of support for connecting conceptual and
% statistical models. 
% The pending lab study results will demonstrate the impact of
% \rTisane on (i) the conceptual models analysts specify and their reflection
% process, (ii) (output) statistical model quality, and (iii) awareness and
% learned insights analysts takeaway about their domain and data analysis process.

% The summative evaluation study concretizes the impact of \tisane. Analysts
% report being more reflective and systematic in their thinking about implicit
% conceptual assumptions due to \rTisanes DSL and conceptual model disambiguation
% process. The statistical models analysts produce also more accurately estimate
% the true relationships in the data, lower AIC/BIC and higher R-squared. 


% Prior publications
\textit{The exploratory study, rTisane design and implementation, and the
summative evaluation are in collaboration with Edward Misback, Jeffrey Heer, and
\reneJust. The corresponding paper~\cite{jun2023rTisane} is under submission and has not yet been
published.}
