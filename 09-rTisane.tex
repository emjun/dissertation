The previous chapter's case studies (\autoref{sec:tisane_case_studies})
highlighted \tisane's potential as a tool for statistical non-experts to author
generalized linear models with or without mixed effects. To build on \tisane's
strengths, we sought to understand the nuances of how statistical novices wish
to articulate their implicit domain knowledge and the challenges they confront
along the way. We started with a lab study using \tisane to elicit statistical
non-experts' implicit definitions and assumptions about \SDSL keywords
(\autoref{sec:exploratoryStudy}). The study also helped us identify
opportunities to refine \tisane's interactivity. Based on study findings, we
designed and evaluated \textbf{\rTisane, a system to assist novices in
formalizing their conceptual knowledge to author statistical models}. We
implemented \rTisane as an open-source R library based on suggestions from
participants and external research collaborators over the course of the previous
projects. Analysts suggested that tools like \tisane could benefit a wider
audience of novice and expert data analysts if there was an R implementation.
% \rTisane's implementation in R allowed us to reach a potentially wider audience of novice and expert data analysts using R for linear modeling.
An additional advantage of the R implementation was that it allowed us to directly
compare \rTisane to a scaffolded workflow using widely used linear modeling
libraries, including \texttt{lme4}, in R (\autoref{sec:summativeEval}). 


\section{Exploratory study}

Reflection: As progress through PhD research, got and grappled with how to get closer to users and to statistical theory in tandem.

\section{Second Release: rTisane}
{\color{orange} Update this section to match rTisane paper.}

\subsection{Updated DSL}
\subsection{Conceptual Model Disambiguation}
\subsection{Statistical model inference}

\section{Summative Evaluation: Controlled lab study} \label{sec:summativeEval}
% \highlight{We are in the process of running this lab study and collecting data.}

We had three research questions:

\begin{itemize}
    \item \evalConceptualModels What is the influence of rTisane on conceptual
    modeling?
    \item \evalStatisticalModels How does rTisane impact the statistical models
    analysts implement? Specifically, do the covariates, family functions, and
    link functions analysts include/exclude differ when implementing statistical
    models on their own vs. using rTisane? How well do statistical models
    analysts author on their own vs. using rTisane fit the data?  
    \item \evalLearning Do analysts learn about their discipline or data
    analysis as a result of using rTisane?
\end{itemize}

\subsection{Study design}
% \todo{Include a diagram summarizing study design?}
We conducted a within-subjects think-a-loud lab study that consisted of four phases: 

\begin{itemize}
    \item \textbf{Phase 1: Warm up.} We presented participants with the
    following open-ended research question: ``What aspects of an adult's
    background and demographics are associated with income?'' We asked
    participants to specify a conceptual model including variables they thought
    influenced income. This warm-up exercise helped to externalize and keep
    track of participants' pre-conceived notions and assumptions prior to seeing
    a more restricted data schema.
    \item \textbf{Phase 2: Express conceptual models} We presented participants
    with a data schema describing a dataset from the U.S. Census Bureau. We then
    asked participants to specify a conceptual model using only the available
    variables. At the end, we asked participants about their
    experiences specifying their conceptual models in a brief survey and semi-structured interview.
    \item \textbf{Phase 3: Implement statistical models} We asked participants
    to implement ``a statistical model that assesses the influence of variables
    [they] believe to be important (in the context of additional potentially
    influential factors) on income,'' relying on only their conceptual model. We
    then asked participants about their experiences implementing statistical
    models through a brief survey and semi-structured interview. 
    \item \textbf{Phase 4: Exit interview.} The study concluded with a survey
    and semi-structured interview where we asked participants to reflect on the
    process of explicitly connecting conceptual models to statistical models.
\end{itemize} 

We designed the study based on the assumption that conceptual modeling is a
helpful strategy when specifying statistical models. As a result, all
participants completed the phases in the above order. In order to assess the
effect of tooling on conceptual models and the quality of statistical models, we
counterbalanced the order in which participants specified conceptual and
statistical models. Half the participants specified their conceptual and
statistical models on their own (without rTisane) first. The other started with
rTisane.

\noindent \paragraph{Participants} We recruited 24 data analysts on Upwork. We
screened for participants who reported having experience with authoring
generalized linear models and using R. \highlight{Update based on recruitment:On
average, participants reported having conducted N projects using generalized
linear models. Participants were familiar with a range of software tools,
including...} All studies were conducted over Zoom. Each participant was
compensated \$50 for 120 minutes of their time. We recorded participants'
screens, video, and audio throughout the study. We then transcribed the audio
for qualitative analyses.

% One dataset was on demographic factors and income in the U.S. in 2018 from the
% U.S. Census Bureau (\datasetIncome). The other dataset was about demographic
% factors and health conditions in the U.S. (\datasetHealth) from \ej{FILL IN}.
% More information on how the datasets were created are found in the
% supplemental material.

\subsection{Analysis and Results}
\highlight{Results will be added once we have collected and analyzed the data over Summer 2023.}
We collected and analyzed quantitative and qualitative measures to answer our
research questions. For each research question, we describe our analysis
approach and results below. 
% We qualitatively analyzed conceptual models, statistical models, audio
% recordings, and open-ended survey responses. Whenever possible, we also
% quantified the frequency of characteristics in conceptual models and
% statistical models across participants. For the statistical models
% participants authored, we compared the AIC and BIC scores for matched pairs of
% the independent variable of interest. 


\subsubsection{\evalConceptualModels}
** How does rTisane compare to on own? **
Main effect of rTisane: Initial CM is very ambiguous and not formal, rTisane is making it more formal. 
- distribute IR
- use IR to target other things (e.g., science diagrams that a Participant said)

On own, participants were not sure how to structure a conceptual model. 
Diversity of conceptual models [from conceptual model analysis] -- form, complexity (number of relationships levels of relationships)


API structure + ... 
START HERE: 
- Group API clusters into smaller clusters
- Write one sentence about each cluster

<Decouple conceptual model from data?>

In general, people able to express conceptual model with rTisane. No major missing concepts, minor syntactic sugar + more advanced constructs desirable;
Critical to mention: **"no perceived influence"** -- confirmed by no statistical differences in NASA-TLX scores between conditions
Takeaway: rTisane provides process but not suggest relationships (control up to end-user) - no agenda for how end-users build model

% Consistency within vs. across participants 
We qualitatively analyzed how consistent participants' conceptual models were
between conditions. We noted common challenges translating free-form conceptual
models into rTisane programs. We also thematically analyzed participant
transcripts and survey responses describing the influence of rTisane on their
conceptual modeling processes. 

% \subsubsection{\evalConceptualModelAuthoring}
% % To understand rTisane's influence on the conceptual modeling process, we
% % compared the ratings and thematically analyzed survey responses to questions
% % asking how participants decided what conceptual relationships to specify.
% At the end of Phases 2 and 3, we asked participants to rate and describe their
% experience specifying conceptual models without and with rTisane, respectively.
% In addition, we kept track of the conceptual modeling challenges participants
% vocalized and noted how participants overcame these challenges in-situ. We
% compared the survey responses, observed challenges, and observed approaches
% between Phases 2 and 3 to assess how rTisane influenced the conceptual modeling
% process. 

\subsubsection{\evalStatisticalModels}
We used AIC, BIC, and R-squared values to assess how well statistical models
authored with vs. without rTisane fit the data. We used rTisane to statistically
model and assess the influence of rTisane on AIC, BIC, and R-squared values.

We also thematically analyzed participants' reactions to the similarities,
differences, and surprises between statistical models. 

% \todo{Add statistical models executed}
% \todo{Add rTisane analysis script to supplemental material}

\subsubsection{\evalLearning}
Finally, we gauged participants' learning based on a thematic analysis of
open-ended survey interview answers at the end of the study. 


\subsubsection{Key takeaways}
\textbf{Larger takeaway: ** Not about just the scaffolded steps but about the tool support for executing each of these steps**}

\subsection{Limitations}
To limit the number of language constructs in \rTisane introduced, we only
assessed language constructs for specifying a GLM. Given that the summative
evaluation was really focused on the core of \rTisane--the impact of conceptual
modeling on statistical modeling--we expect the results to apply. In fact, for
more complex data collection procedures that require mixed-effects, \rTisane may
have an even larger effect on statistical models and learning. 

\section{Discussion, Limitations, and Future Work}
The exploratory lab study suggested the need to allow analysts to express their
conceptual models using more granular, low-level functions. Although obvious in
hindsight, this finding was \textit{counterintuitive} at the time. A widely held
belief, especially within the HCI community, is that the higher the level of
abstraction for a task, the better for end-users. However, we saw the opposite.
Statistical non-experts engaged deeply with conceptual models about their domain
and wanted to be more detailed and specific. In other words, while the focus on
the abstraction should be at the conceptual level, within that, analysts want to
move up and down the ladder of abstraction. More generally, our iterative
language design work with \tisane and \rTisane suggests that as long as
abstractions match the content-focus of end-users, there should be opportunities
to get low-level within those abstractions. This gives end-users the agency to
express themselves more fully, transforming the programming task from strictly a
means to an end to specification as a meaningful activity in itself. 

% Second, Tisane's specification process could be more tiered, and disambiguation
% could leverage ambiguity in analysts' specification as opportunities for more
% numerous intelligent suggestions and guidance. 
% \todo{Mention that the focus needs to be on conceptual level, but within conceptual level, there should be opportunity to move up and down the ladder of conceptual abstraction}

In designing \rTisane, a key challenge was in finding the right point to bring
in lower-level statistical modeling details. Concretely, in \rTisane analysts at some point must grapple with graphical and
mathematical representations in the disambiguation phases. This is because it
was not possible to remove all complexity from statistical modeling without the
risk of losing the analyst's sense of control or understanding. Thus, our focus
has been to strip away unnecessary complexity and help analysts navigate through
necessary complexity by designing informative abstraction lowering
disambiguating steps. It may be possible to avoid any interactive disambiguation
by executing all possible statistical models given an input, likely ambiguous,
conceptual model. Although this approach would accomplish a different objective
than our goal of compiling a specific conceptual model into a specific
statistical model, this approach may give greater insight into if analysts
really want, need, or benefit from disambiguation. 

\polish{Design pattern for balancing usability and rigor}
Our approach in \rTisane is to
prioritize expressivity in the DSL and precision during interactive compilation. 

\polish{Future work: Suggest unobserved variables, more complex causal structures, perhaps based on similarities from a field/other analysts/other sources}

\polish{Discussion of rTisane results}
Novel for non-expert audience: formalizing a porcess that is innate among statistical experts

\begin{comment}

Despite the goal to lower the
barriers to statistical specification, at some point, 

Another approach to explore in the
future may be to eliminate the need to engage analysts in disambiguation and
instead execute all possible statistical models given an input, likely
ambiguous, conceptual model. Although this approach would accomplish a different
objective than the goal here of compiling a specific conceptual model into a
specific statistical model, this approach may give greater insight into the need 

% Our answer was to have informative conceptual and
% statistical model disambiguation phases. 
robustnes of a particular effect in light of many possible conceptual models and
explanations.

% push on the idea that analysts care about conceptual ramifications and avoid asking them to 

In the future, there could be additional exploration into authoring a multiverse
of all possible statistical models given a specific ambiguous conceptual model.
This would accomplish a different objective than the goal of \tisane (and
\rTisane), which is to compile a specific conceptual model into a specific
statistical model for/with the end-user. The multiverse would help assess the
robustnes of a particular effect in light of many possible conceptual models and
explanations.

Concretely, it
took us several iterations to answer the question: What is the right point to
introduce cycle breaking and modeling to the end-user? 

**not remove all complexity but rather focus end-users on necessary complexity
and guide their thinking/help them navigate that complexity. 

\end{comment}

\section{Summary of Contributions}

% Reminder: Thesis statement
% Domain-specific languages that provide abstractions for expressing conceptual
% knowledge, data collection procedures, and analysis intents instead of specific
% statistical modeling decisions coupled with automated reasoning to compile
% conceptual specifications into statistical analysis code help statistical
% non-experts more readily author valid analyses. 

\rTisane provides a DSL with language constructs for expressing conceptual
models (\thesisChallengeExplicit) and integrates a two-phase interactive
disambiguation process for compiling conceptual knowledge into statistical
analysis code (\thesisChallengeRep). In a controlled lab study of \rTisane, we
found that the DSL is expressive enough to capture analysts' conceptual models
accurately, eases the burden of making their implicit assumptions explicit, and
pushes analysts to think about their domains more deeply. Using \rTisane,
analysts were able to author statistical models that fit the data just as well
as if not better than statistical models authored on their own. \rTisane even
helped analysts who were not able to author statistical models on their own get
to an output statistical model. Analysts also reported that through the process,
they learned about GLMs (\thesisChallengeUnderstanding). Together, these results
demonstrate how DSLs and automated reasoning together in fact do help
statistical non-experts author valid statistical analyses that they would not be
able to author otherwise.
% evidence for how
% connecting conceptual modeling to statistical modeling increases the statistical
% conclusion and external validity of analyses~\cite{shadish2010campbell}.  

% The conceptual model disambiguation process in \rTisane
% also facilitates reflection on implicit knowledge. 

% we refined what the programming and interaction model
% for expressing conceptual models and connecting them to statistical models
% should be. Most notably, the second release of \tisane, as \rTisane, provides
% more explicit support for conceptual model specification and disambiguation. 

% \tisane and \rTisane are in stark contrast to the current ecosystem of
% statistical analysis software designed to give analysts maximal mathematical and
% computational control at the cost of support for connecting conceptual and
% statistical models. 
% The pending lab study results will demonstrate the impact of
% \rTisane on (i) the conceptual models analysts specify and their reflection
% process, (ii) (output) statistical model quality, and (iii) awareness and
% learned insights analysts takeaway about their domain and data analysis process.

% The summative evaluation study concretizes the impact of \tisane. Analysts
% report being more reflective and systematic in their thinking about implicit
% conceptual assumptions due to \rTisanes DSL and conceptual model disambiguation
% process. The statistical models analysts produce also more accurately estimate
% the true relationships in the data, lower AIC/BIC and higher R-squared. 


% Prior publications
\textit{The exploratory study, rTisane design and implementation, and the
summative evaluation are in collaboration with Edward Misback, Jeffrey Heer, and
\reneJust. The corresponding paper~\cite{jun2023rTisane} is under submission and has not yet been
published.}
