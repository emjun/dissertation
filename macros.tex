%%%% Fill out all these macros
\newcommand{\thesisTitle}{Data analysis tools for statistical non-experts}
\newcommand{\authorName}{Eunice Jun}
\newcommand{\advisor}{Jeffrey Heer}
\newcommand{\advisorTitle}{Jerre D. Noe Endowed Professor} % assistant, associate prof?
\newcommand{\advisorDepartment}{Paul G. Allen School of Computer Science \& Engineering}
% if you have two advisors, uncomment these two
\def\rene{Ren\'{e}\xspace}
\def\reneJust{Ren\'{e} Just\xspace}
\newcommand{\secondAdvisor}{\reneJust}
\newcommand{\secondAdvisorTitle}{Associate Professor}
\newcommand{\secondAdvisorDepartment}{Paul G. Allen School of Computer Science \& Engineering}

\newcommand{\readingCommitteeOne}{Tyler H. McCormick}
\newcommand{\graduationYear}{2023}

%\usepackage{} -- add any other packages you want
%\usepackage{proceed2e}
\usepackage[hyphens]{url}
\usepackage{amsfonts}
\usepackage{multirow,tabularx}
\usepackage{listings}
\usepackage{makecell}
\usepackage[linesnumbered]{algorithm2e}
\usepackage{amssymb}
\usepackage{graphicx}
\usepackage{amsmath}
\usepackage{float}
\usepackage{fullpage}
\usepackage{mathrsfs}
\usepackage{subcaption}
\usepackage{hyperref}% http://ctan.org/pkg/hyperref
\hypersetup{%
  %colorlinks = true,
  linkcolor  = black
}
\usepackage{mdwlist}
\usepackage{xspace}
\usepackage{setspace}
% \usepackage{times} % Set the typeface to Times Roman
\usepackage[scaled]{helvet} % ss
\usepackage{courier} % tt
\normalfont
\usepackage[T1]{fontenc}
\usepackage[utf8]{inputenc}
% \usepackage[cp1251]{inputenc}


%\usepackage[showframe=true]{geometry}
\usepackage{changepage}
\usepackage[labelfont=bf]{caption} % boldface caption title for floats
\usepackage[square]{natbib}
\usepackage{enumitem}
\usepackage{natbib}
\setcitestyle{numbers}
\usepackage{array,multirow}
\usepackage{fixltx2e}
\usepackage{booktabs}
\usepackage{bbm}
\usepackage{soul}
\usepackage{relsize}
\usepackage{eso-pic}
\usepackage{xspace}
\usepackage{epsfig}
\usepackage{graphicx}
\usepackage{amsmath}
\usepackage{amssymb}
\usepackage{float}
\usepackage{multirow}
\usepackage{rotating}
\usepackage{balance}
\usepackage{wrapfig}
% \usepackage{enumerate}
\usepackage{caption}
\usepackage{framed}
\usepackage{enumitem}
\usepackage{multirow}
\usepackage{graphicx}
\usepackage{color}
\usepackage{fixltx2e}
% \usepackage{tkz-graph}
\usepackage{caption}
\usepackage{subcaption}
\usepackage{tikz}
\usepackage{mathtools}
\usepackage{pifont}
\usepackage{scrextend}
\usepackage{sidecap}
\usepackage{graphicx}
\usepackage{comment}
\usepackage{longtable} % for long tables
% For uniform capitalization of section titles
% \usepackage{sectsty}
\usepackage{titlecaps}
% \sectionfont{\titlecap}
\let\oldsection\section
\renewcommand{\section}[1]{\oldsection{\titlecap{#1}}}
\Addlcwords{and or of as on in with rTisane for} % Keep these words lowercase
% \let\oldsubsection\subsection
% \renewcommand{\subsection}[1]{\oldsubsection{\titlecap{#1}}}


% For code snippets
\usepackage{upquote}
\usepackage{listings}
\usepackage{xcolor}
\usepackage{float}

\newcommand\upquote[1]{\textquotesingle#1\textquotesingle}

\definecolor{codegreen}{rgb}{0,0.6,0}
\definecolor{codegray}{rgb}{0.5,0.5,0.5}
\definecolor{codepurple}{rgb}{0.58,0,0.82}
\definecolor{mauve}{rgb}{0.58, 0, 0.82}
\definecolor{magenta}{RGB}{255,0,255}
\definecolor{lightgray}{rgb}{0.83, 0.83, 0.83}
% \definecolor{backcolour}{rgb}{0.95,0.95,0.92}

\lstdefinestyle{mystyle}{
    backgroundcolor=\color{white},
    commentstyle=\color{codegreen},
    keywordstyle=\color{magenta},
    numberstyle=\tiny\color{codegray},
    stringstyle=\color{codepurple},
    basicstyle=\ttfamily\footnotesize,
    breakatwhitespace=false,
    breaklines=true,
    captionpos=b,
    keepspaces=true,
    numbers=left,
    numbersep=5pt,
    showspaces=false,
    showstringspaces=false,
    showtabs=false,
    tabsize=2,
    deletekeywords={input, print, id},
    alsoletter={_},
    morekeywords={data, define_variables, tea.define_study_design, assume, hypothesize}
    % here are the additional keywords
%     emph={data, define_variables, define_study_design, assume, hypothesize, <, >, =, !},
    % they are underlines
%     emphstyle={\textbf}
}

\definecolor{light-gray}{gray}{0.95}
% \lstset{basicstyle=\linespread{1.1}\ttfamily\footnotesize,
%     backgroundcolor=\color{light-gray}, xleftmargin=0.7cm,
%     frame=tlbr, framesep=0.2cm, framerule=0pt,
% }

% \lstset{style=mystyle}


% make a custom style that looks good and can highlight some additional keywords
% \lstdefinestyle{tea}{
%   basicstyle=\ttfamily,
%   deletekeywords={input, print, id},
%   language=Python,
%   % here are the additional keywords
%   emph={data, define_variables, define_study_design, assume, hypothesize, <, >, =, !},
%   % they are underlines
%   emphstyle={\bf},
% }
% \lstset{
%   % use this style by default
%   style=mystyle,
%   % look better
%   columns=flexible,
%   showstringspaces=false,
%   % spacing, size, numbers, etc.
%   numbers=left,
%   xleftmargin=2em,
%   numberstyle=\tiny,
%   escapechar=|,
% }

% For Tikz graphics
\usepackage{tikz}
\usetikzlibrary{positioning,arrows.meta,graphs,backgrounds,fit,calc,quotes,shapes.multipart}
\usepackage{pgfplots}
\usepackage{pgfplotstable}

\usepackage{todo} % for todo list
\newcommand{\ej}[1]{\colorbox{blue!30}{#1}}
\newcommand{\highlight}[1]{\colorbox{yellow!30}{\textit{#1}}}
\newcommand{\polish}[1]{\colorbox{yellow!30}{#1}}
\newcommand{\revisit}[1]{\textcolor{orange}{Revisit: #1}}

% For changing size of text in Acknowledgements
% \newenvironment{localsize}[1]
% {%
%   \clearpage
%   \let\orignewcommand\newcommand
%   \let\newcommand\renewcommand
%   \makeatletter
%   \input{bk#1.clo}%
%   \makeatother
%   \let\newcommand\orignewcommand
% }
% {%
%   \clearpage
% }


% For appendices
\usepackage{pdfpages}
\usepackage{fancyhdr}

%%%%%% For consistent terminology 
% Tea 
\def\tea{Tea\xspace}
\def\TeaPL{Tea's programming language\xspace}
\def\TeaRS{Tea's runtime system\xspace}

\def\dataSet{dataset\xspace}
\def\dataSets{datasets\xspace}

% Hypothesis formalization 
%% Quotes
\newcommand{\shortquote}[1]{``\emph{#1}''}
\newcommand{\longquote}[1]{\vspace{-1pt}\begin{quote}``\emph{#1}''\end{quote}}
\newcommand{\theme}[1]{\vspace{-1pt}\subsubsection{Theme: #1} }
\newcommand{\groupingtheme}[1]{\subsubsection{\textbf{#1}} }
%% Research Questions
\def\rqSteps{\textbf{RQ1 - Steps}\xspace}
\def\rqProcess{\textbf{RQ2 - Process}\xspace}
\def\rqTools{\textbf{RQ3 - Tools}\xspace}

% Tool Implications
\def\higherLevel{\textit{DI1 - Raise level of abstracion}\xspace}
\def\connectConceptualStats{\textit{DI2 - Connect conceptual and statistical models}\xspace}
\def\relateStats{\textit{DI3 - Relate statistical methods}\xspace}

% Tisane terminology
\def\tisane{Tisane\xspace}
\def\rTisane{rTisane\xspace}
\def\rTisanes{rTisane's\xspace}
\def\Disambiguation{Disambiguation\xspace}
\def\disambiguation{disambiguation\xspace}
\def\SDSLlong{DSL\xspace}
\def\SDSL{DSL\xspace}

% rTisane summative eval 
%% Evaluation Research Questions
\def\evalConceptualModels{\textbf{RQ1 - Conceptual models}\xspace}
\def\evalStatisticalModels{\textbf{RQ2 - Statistical models}\xspace}
\def\evalLearning{\textbf{RQ3 - Learning}\xspace}

% Hypothesis formalization 
\def\hypoForm{\textit{hypothesis formalization}\xspace}
\def\HypoForm{\textit{Hypothesis formalization}\xspace}

%% Evaluation Research Questions
\def\rqWorkflow{\textbf{RQ1 - Workflow}\xspace}
\def\rqCognitive{\textbf{RQ2 - Cognitive fixation}\xspace}
\def\rqFuture{\textbf{RQ3 - Future possibilities}\xspace}

%% Design considerations
\def\dcConceptualKnowledge{\textit{DG1 - Conceptual knowledge}\xspace}
\def\dcConceptualKnowledgeLong{\textit{DG1 - Prioritize conceptual knowledge.}\xspace}
\def\dcValidity{\textit{DG2 - Validity}\xspace}
\def\dcValidityLong{\textit{DG2 - Prioritize the validity of models.}\xspace}
\def\dcGuidance{\textit{DG3 - Guidance and control}\xspace}
\def\dcGuidanceLong{\textit{DG3 - Give analysts guidance and control.}\xspace}
\def\dcStatisticalPlanning{\textit{DG4 - Statistical planning}\xspace}
\def\dcStatisticalPlanningLong{\textit{DG4 - Facilitate statistical planning without data.}\xspace}

% Tool Implications
\def\higherLevel{\textit{DI1 - Raise level of abstracion}\xspace}
\def\connectConceptualStats{\textit{DI2 - Connect conceptual and statistical models}\xspace}
\def\relateStats{\textit{DI3 - Relate statistical methods}\xspace}

%% Other
\def\lme{\texttt{lme4}\xspace}
\def\statsmodels{\texttt{statsmodels}\xspace}
\def\pymer{\texttt{pymer4}\xspace}
% \newcommand\upquote[1]{\textquotesingle#1\textquotesingle}
%% Referencing sections
\def\sectionautorefname{Section}
%% Referencing subsections
\def\subsectionautorefname{Subsection}
%% Referencing chapters 
\renewcommand{\chapterautorefname}{Chapter}
%% Other useful macros and abbreviations
\def\etal{et al.\xspace}
%% Qualitative analysis
\def\codebook{codebook\xspace}


%%%%%% For publications
\newcommand{\uistConf}[1]{\textit{ACM UIST #1}}
\newcommand{\chiConf}[1]{\textit{ACM CHI #1}}
\newcommand{\tochi}[1]{\textit{ACM Transactions of Computer-Human Interaction (TOCHI) #1}}


%$%%%% Macros for consistent typesetting of tables.

% Minimizing spacing for tables
\usepackage{setspace}


%% Fix spacing in non-right-aligned columns
\newcommand{\z}{\phantom{0}}

%% Shortcut for plus/minus sign
\newcommand{\p}{$\scriptsize{\pm}$}

%% Always center and bold face column headers
\newcommand\colH[1]{\multicolumn{1}{l}{\textbf{#1}}}
\newcommand\colHR[1]{\multicolumn{1}{r}{\textbf{#1}}}
\newcommand\colsH[2]{\multicolumn{#1}{c}{\textbf{#2}}}

%% New column type to "hide" a column
%\newcolumntype{H}{>{\setbox0=\hbox\bgroup}c<{\egroup}@{}}

%% Typeset an assumption
\newcommand{\assume}[1]{\tikz[baseline]\node[circle,inner xsep=2pt,inner ysep=0pt,
        draw=black,fill=white,scale=0.8,anchor=base] {#1};}
%% Typeset a valid assumption
\newcommand{\valid}[1]{\tikz[baseline]\node[circle,inner xsep=2pt,inner ysep=0pt,
        draw=black,fill=gray!50!white,scale=0.8,anchor=base] {#1};}
%% Typeset an invalid assumption
\newcommand{\invalid}[1]{\tikz[baseline]\node[circle,inner xsep=2pt,inner ysep=0pt,
        draw=black,fill=white!30!white,scale=0.8,anchor=base] {#1};}
\def\yes{$\checkmark$}
\def\no{---}

%%%%%%%%%%%%%%%%%%%%%%%%%%%%%%%%%%%%%%%%%%%%%%%%%%%%%%%%%%%%%%%%%%%%%%%%%%%%%%%%
%% Use \<name> or \codeid{name} for source-code related identifiers
\def\<#1>{\codeid{#1}}
\newcommand{\codeid}[1]{\ifmmode{\mbox{\small\ttfamily{#1}}}\else{\small\ttfamily #1}\fi}
\newcommand{\codeidsmall}[1]{\ifmmode{\mbox{\smaller\ttfamily{#1}}}\else{\smaller\ttfamily #1}\fi}


\makeatother
