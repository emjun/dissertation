% This chapter typically includes: 
% \begin{itemize}
%     \item a brief overview
%     \item a challenges section
%     \item a section about your approach
%     \item a section (or subsection in the approach section) giving a dissertation outline (a roadmap of the rest of the thesis)
% \end{itemize}

% Data analysis is ipmortant for a number of domains. <Examples> 
Statistical analysis plays a critical role in how people evaluate data and make
decisions. Policy makers rely on models to track disease, inform health
recommendations, and allocate resources. Scientists develop, evaluate, and
compare theories based on statistical results. Journalists report on new
findings in science, which individuals use to make decisions that impact their
nutrition, finances, and other aspects of their lives. Faulty statistical models can lead to spurious estimations of disease spread,
findings that do not generalize or reproduce, and a misinformed public. 
% As data science and statistical analysis has and continues to become more
% prevalent, the number of analysis authors who are not statistical experts will
% increase. 

Despite the prevalence of statistical analyses and their central importance to a
number of disciplines, they remain challenging to author accurately. The key
challenge in developing accurate statistical models lies not in a lack of access
to mathematical tools, of which there are many (e.g., R~\cite{team2013r},
Python~\cite{sanner1999python}, SPSS~\cite{spss}, and SAS~\cite{sas}), but in
accurately applying them in conjunction with domain theory, data collection,
statistical knowledge, and programming ability~\cite{mcelreath2020statistical}.
Analysts must translate their implicit domain knowledge into statistical models
that they can then implement and execute in code. However, this process---which
requires disciplinary, statistical, and programming expertise---is out of reach
for statistical non-experts who depend on accurate analyses, including many
researchers. 
% This process requires
% expertise in a discipline, statistics, and programming.
% Analysts must integrate multiple knowledge sources to specify statistical
% analyses. Yet, this integrative process is out of reach for statistical
% non-experts who depend on accurate analyses, including many researchers. 


\addcontentsline{toc}{section}{Approach}
\section*{Approach}
This dissertation asks if separating the above concerns and incorporating
automated reasoning in statistical software could benefit statistical
non-experts. Towards this goal, I combine techniques from human-computer
interaction, programming languages/software engineering, and statistics to (i)
characterize the cognitive and operational steps to author statistical analyses
and (ii) develop novel interactive systems that enable statistical non-experts
to author valid analyses. As detailed below, I not only move between systems
building and empirical studies but use each to deepen and enhance the other.

The work described in the dissertation demonstrates the following:% thesis statement: 
\addcontentsline{toc}{section}{Thesis statement}
\paragraph{Thesis statement}
% Centered on the insight that domain experts using statistical methods are
% focused on their domain 

%%% COULD MAKE THIS STRONGER?
Domain-specific languages that provide abstractions for expressing conceptual
knowledge, data collection procedures, and analysis intents instead of specific
statistical modeling decisions coupled with automated reasoning to compile
conceptual specifications into statistical analysis code help statistical
non-experts more readily author valid analyses. 

Three challenges fall out of this thesis statement: 

\addcontentsline{toc}{section}{Challenge 1}
\section*{Challenge 1: How to make implicit domain knowledge explicit.} %- domain knowledge
Designing abstractions focused on conceptual knowledge requires identifying what
domain knowledge analysts want and can express and balancing these constraints
with what automated reasoning approaches may require. What is easy to express
and what is easy to assume for the sake of automation may be at odds, especially
when analysts provide ambiguous specifications that could be compiled into
multiple statistical analyses. The challenge therefore, is to design language
constructs that are usable for analysts and useful for automated reasoning and
support interactive program specification as necessary.

% Finding: interactive disambiguation not just necessary for refinement and automated reasoning but *useful* to analysts for reflection

% Shifting focus onto the goal/motivation of analysis and less on the details that can overwhelm and restrict analysts

% **Not just higher levels of abstraction but appropriate abstractions that allow analysts to dig deeper into the appropriate parts

\addcontentsline{toc}{section}{Challenge 2}
\section*{Challenge 2: Represent and reason about key statistical analysis decisions} %- programming
A central idea in this thesis is that software systems should take on the
responsibility of translating conceptual knowledge into statistical analyses.
This is akin to a compilation process that requires representing the conceptual
knowledge analysts express and reasoning over it to derive statistical
analyses that respect statistical best practices and rules. A
major challenge is in picking representations so that the reasoning is straightforward. 
% not only straightforward but also beneficial in someway (expressivity, extensibility for both Tea and Tisane)

\addcontentsline{toc}{section}{Challenge 3}
\section*{Challenge 3: Increase analysts' statistical knowledge/understanding} %- statistics
% connect to Bellotti on user control?
While automating statistical analysis can be helpful, analysts relying on data
to make high-impact decisions (e.g., policy, scientific discovery) often need to
understand why an analysis approach is appropriate and what the implications of
the results are to their domain. Furthermore, software can inform how users
approach future analyses. Therefore, educating analysts about the applicability
and impact of statistical decisions and guiding their interpretation of results
are important.

\addcontentsline{toc}{section}{Summary of Contributions}
\section*{Summary of Contributions}
\ej{Add overview figure from reserach statement}
% Contributions: systems, empirical, and the beginnings of methodological innovation.
This dissertation makes systems and empirical contributions. Additionally, the
process of designing and developing domain-specific languages for end-users
shows the first steps towards developing methods for user-centric language
design. 
% we also explored new methods for eliciting and integrating user
% feedback throughout programming language design.

Specifically, I designed and implemented two systems, Tea~\cite{jun2019tea} and
Tisane~\cite{jun2022tisane}, that leverage \textbf{domain-specific languages}
(DSLs) to capture analysts' implicit assumptions and conceptual knowledge. Users
\textbf{interactively compile} these high-level specifications into low-level
code. To infer valid statistical analyses, the systems \textbf{programmatically
represent and reason about core statistical authoring challenges} as constraints
and graphs (\autoref{fig:tools}).
% As a result, my systems prevent common analysis
% mistakes~\cite{jun2019tea,jun2022tisane}. 

In summary, this dissertation's key contributions are
% The contributions of this research are as follows: 
\begin{itemize} 
    % Whereas hypothesis formalization has remained implicit in prior
    % descriptions of data analysis, we explicate this specific process. 
    \item a \textbf{formal constraint-based model} to specify and select among
    common Null Hypothesis Statistical Tests in Tea (see~\autoref{chapter:tea}); 
    \item empirical findings of how authoring analyses requires integrating
    conceptual, data, statistical, and programming expertise, which we summarize
    in our \textbf{theory of hypothesis formalization} (see~\autoref{chapter:hypoForm}); 
    \item an analysis of how the current statistical software ecosystem does not
    explicitly support and may even hinder hypothesis formalization, suggesting
    new \textbf{design opportunities and implications} (see~\autoref{chapter:hypoForm});
    \item a \textbf{mixed-initiative approach} for ``interactively compiling''
    linear models from conceptual and data relationships in Tisane; 
    \item empirical \textbf{findings on researchers' implicit semantics of
    conceptual models} (see~\autoref{chapter:tisane});
    \item \textbf{new language constructs and interaction methods} for
    reflecting on and refining conceptual models in a second version of Tisane,
    which we call rTisane (see~\autoref{chapter:tisane}); and
    \item qualitative and quantitative \textbf{results showing the benefit of
    recording conceptual models and compiling them into statistical models} in
    rTisane over a scaffolded workflow (see~\autoref{chapter:tisane}).
\end{itemize}

\begin{comment}
\section{Thesis outline}
% System --> Empirical --> System --> Empirical --> Empirical --> System --> Empirical 
This dissertation contributes new domain-specific languages (DSLs) for authoring
statistical analyses and a new theory describing the cognitive and operational
steps involved in authoring statistical analyses. In the process of designing
the second DSL, we also explored new methods for eliciting and integrating user
feedback throughout programming language design. The content of thesis is as
follows. 

\todo{Fill in this outline}

\section*{How to approach this dissertation} \todo{Decide if want to keep}

\section{Prior Publication and Authorship} \todo{fill in}
\end{comment}