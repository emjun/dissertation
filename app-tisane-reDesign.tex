Materials for the elicitation lab study described in
\autoref{sec:exploratoryStudy} follow. All materials were presented in editable
Markdown files. Each section below was in a separate file.

\section{Task 1: Variables}

You identified the following constructs and proxies.

For Conceptual Hypothesis 1 (CH1): <Fill in based on homework response construct: proxy>

For CH2:  <Fill in based on homework response construct: proxy>

For CH3: <Fill in based on homework response construct: proxy>

For CH4: <Fill in based on homework response construct: proxy>

\subsection*{Current idea}
\emph{The language constructs presented here and for the remainder of
the study are an \textbf{initial set} of ideas we have. We are actively
seeking to \textbf{add, remove, and adapt the language.}}

Variables can be Units, Measures, or SetUp variables.

\subsubsection{1. Units}

\begin{quote}
A data variable that can have attributes. For example, if you have
programmers in your dataset. Each programmer has a a level of
experience, programming language they write, an IDE they were assigned
(e.g., a condition), etc. Then, programmer variable can be a Unit. In
statistics, a Unit can represent either an
\href{https://en.wikipedia.org/wiki/Statistical_unit}{observational or
experimental unit}.
\end{quote}


A Unit has the following properties:
\begin{itemize}
    \item \textbf{name (str)}: The name of the variable. If you have data, this should correspond to the column's name. The dataset must be in long format.
    \item \textbf{cardinality (int)}: The number of unique values of the variable. cardinality is optional only if you have a data set. If specified, the language will check that the cardinality is correct if you include data in the design. If left unspecified, and data is available, the language will try to calculate the cardinality.
\end{itemize}

\begin{verbatim}
For example:
# To specify that there are 40 unique participants
# Without data
programmer <- Unit("participant", 40)
# With data
programmer <- Unit("participant") # cardinality is optional
\end{verbatim}

\hypertarget{measures-nominal-ordinal-and-numeric}{%
\subsubsection{2. Measures: Nominal, Ordinal, and
Numeric}\label{measures-nominal-ordinal-and-numeric}}

\begin{quote}
Measures are variables that describe a Unit. Measures can be nominal,
ordinal, or numeric. For example, if you have people in your dataset,
and each programmer has an eye color, height, age, experience level in
school, etc., then eye color, height, age, and experience level are
measures of the programmer unit.
\end{quote}

\subsubsection{2a. Nominal Measure}

\begin{quote}
Represents a categorical variable whose categories are not ordered. You
must specify a nominal measure through the Unit the Measure belongs to.
\end{quote}

A Nominal Measure has the following properties:
\begin{itemize}
    \item \textbf{unit (Unit)}: The unit that the measure is of.
    \item \textbf{name (str)}: The name of the categorical variable. If you have data, this should correspond to the column’s name in the data.
    \item \textbf{cardinality (int, optional)}: The number of unique values for the variable.
\end{itemize}

\begin{verbatim}
For example:
# To specify that IDE is measured for each programmer unit.
# Without data, the cardinality must be specified
ide <- nominal(programmer, "ide", 5)
# With data, the cardinality is inferred
ide <- nominal(programmer, "ide")
\end{verbatim}

\subsubsection{2b. Ordinal Measure}

\begin{quote}
Represents a categorical variable whose categories are ordered. You must
specify an ordinal measure through the Unit the Measure belongs to.
\end{quote}

An Ordinal Measure has the following properties:
\begin{itemize}
    \item \textbf{unit (Unit)}: The unit that the measure is of.
    \item \textbf{name (str)}: The name of the ordinal variable. If you have data, this should be the column name in the data.
    \item \textbf{order (list)}: The ordering of the categories of the variable.
\end{itemize}

\begin{verbatim}
For example:
# To specify that experience level is measured for each programmer unit.
experience_level <- ordinal(programmer, "experience level", 
                            ["low", "medium", "high"])
\end{verbatim}

\subsubsection{2c. Numeric Measure}\label{c.-numeric-measure}

\begin{quote}
Represents quantitative variables. Numeric variables take on values that
are integers and floats. You must specify a numeric measure through the
Unit the Measure belongs to.
\end{quote}

A Numeric Measure has the following properties:
\begin{itemize}
    \item \textbf{unit (Unit)}: The unit that the measure is of.
    \item \textbf{name (str)}: The name of the numeric variable. If you have data, this should correspond to the column’s name in the data.
\end{itemize}

\begin{verbatim}
For example:
# To specify that lines of code written in one hour 
    is measured for each programmer unit.
loc <- numeric(programmer, "lines of code")
\end{verbatim}

\subsubsection{3. SetUp}\label{setup}

\begin{quote}
SetUp variables are variables that describe the data collection setting.
SetUp variables are neither Units nor Measures. This can represent time,
year, etc. describing the data collection process. For example, if you
collected height data from a programmer over the course of four weeks,
the week number is neither a Unit (programmer) nor a Measure (height) of
a Unit.
\end{quote}

A SetUp variable has the following properties:
\begin{itemize}
    \item \textbf{name (str)}: The name of the variable. If you have data, this should correspond to the column’s name. The dataset must be in long format.
    \item \textbf{order (list, optional)}: Use a specific ordering of the values of environment settings.
    \item \textbf{cardinality (int, optional)}: The number of unique values of the variable.
\end{itemize}

\begin{verbatim}
For example:
week <- SetUp("time", [1, 2, 3, 4])
\end{verbatim}

\subsection*{Your input!}

Starting from these language ideas, specify any variables from the
paper.

Feel free to adapt or introduce new syntax if you feel it's missing or
there is a more intuitive way to express what you want to say. Also,
feel free to introduce new types of variables, language constructs, etc!
\emph{Really try to fully communicate what you want to express about the
variables.}
\clearpage



\section{Task 2: Study design and data collection}

You summarized the study design as follows: \textgreater{} FILL IN

\hypertarget{current-idea-data-measurement-relationships}{%
\subsection*{Current idea: Data measurement
relationships}\label{current-idea-data-measurement-relationships}}

\emph{As before, the language constructs presented here and for the
remainder of the study are an \textbf{initial set} of ideas we have. We
are actively seeking to \textbf{add, remove, and adapt the language.}}

We are thinking about two ways of specifying how data were collected.

\hypertarget{number-of-instances}{%
\subsubsection{Number of instances}\label{number-of-instances}}

We want to add a \texttt{number\_of\_instances} to how each Measure is
specified to indicate how many times a Unit provided an observation of
the Measure.

\begin{verbatim}
For example:
# There are 40 programmers in our study
programmer <- Unit("participant", 40)
# Each programmer is assigned to exactly one IDE condition
ide <- nominal(programmer, "ide", number_of_instances=1)
# Each programmer has an experience level
experience_level <- ordinal(programmer, "experience level", ["low", "medium", "high"], number_of_instances=1)
# Each programmer writes lines of code for the study
loc <- numeric(programmer, "lines of code", number_of_instances=1)

# Each programmer writes lines of code, once per week
week <- SetUp("time", [1, 2, 3, 4])
loc <- numeric(programmer, "lines of code", number_of_instances=week)
# The above is equivalento the below: 
loc <- numeric(programmer, "lines of code", number_of_instances=Per(Exactly(1), week))

# Each programmer writes lines of code, exactly three times per week (multiple measures)
loc <- numeric(programmer, "lines of code", number_of_instances=Per(Exactly(3), week))

# Each programmer writes lines of code, at most three times per week if a programmer can have 0, 1, 2, or 3 measures of lines of code 
loc <- numeric(programmer, "lines of code", number_of_instances=Per(AtMost(1), week))
\end{verbatim}

\hypertarget{your-input}{%
\subsection*{Your input!}\label{your-input}}

Specify how the variables were collected in the paper.

\begin{verbatim}
\end{verbatim}

\hypertarget{data-measurement-relationships}{%
\subsubsection{Data measurement
relationships}\label{data-measurement-relationships}}

Previously, you could specify the number of instances or observations
for each measure. In addition to specifying the number of
instances/observations a Unit provides for each Measure, we think it
might be important to specify that a Unit is nested within another Unit.

\begin{quote}
\emph{Nests within} means the base Unit (e.g., programmer) is nested
within a group Unit (e.g., team)
\end{quote}

\begin{verbatim}
For example:
# Programmers
programmer <- Unit("participant", 40)
# Teams
team <- Unit("team", 10)
# To specify that programmers work in teams at their company
programmer.nests_within(team)
\end{verbatim}

\hypertarget{your-input-1}{%
\subsection*{Your input!}\label{your-input-1}}

What types of data measurement relationships would you like to specify
for this paper? Specify any data measurement relationships between
variables.

Feel free to adapt or introduce new syntax if you feel it's missing.
Also, feel free to introduce new relationships, language constructs,
etc! \emph{Really try to fully communicate what you want to express
about the data measurement relationships.}

\begin{verbatim}
\end{verbatim}
\clearpage



\hypertarget{conceptual-model}{%
\section{Task 3: Conceptual model}\label{conceptual-model}}

You summarized the authors' conceptual model as below: \textgreater{}
FILL IN

\hypertarget{current-idea-conceptual-relationships}{%
\subsection*{Current idea: Conceptual
relationships}\label{current-idea-conceptual-relationships}}

\emph{As before, the language constructs presented here and for the
remainder of the study are an \textbf{initial set} of ideas we have. We
are actively seeking to \textbf{add, remove, and adapt the language.}}

We currently are thinking about providing three current language
constructs. We think there may be more or additional ways to fully
express/capture your conceptual model. We are curious what is most
natural, or intuitive, for you to express conceptual models.

\hypertarget{causes}{%
\subsubsection{1. Causes}\label{causes}}

\begin{quote}
\emph{Causes} means you either know or suspect that a variable causes
another.
\end{quote}

\begin{verbatim}
For example:
# Specifying all the variables
programmer <- Unit("participant", 40)
language <- nominal(programmer, "programming language")
ide <- nominal(programmer, "ide")
experience_level <- ordinal(programmer, "experience level", ["low", "medium", "high"])
loc <- numeric(programmer, "lines of code")

# IDE causes LOC
causes(ide, loc)
\end{verbatim}

\hypertarget{associates-with}{%
\subsubsection{2. Associates with}\label{associates-with}}

\begin{quote}
\emph{Associates with} means you know or suspect a relationship between
variables, but you are unsure if a variable causes another. Note,
\emph{associates with} is commutative.
\end{quote}

\begin{verbatim}
For example:
# Experience level and lines of code e are associated with another.
associates_with(experience_level, loc)
# Order does not matter:
associates_with(loc, experience_level)
\end{verbatim}

\hypertarget{moderates}{%
\subsubsection{3. Moderates}\label{moderates}}

\begin{quote}
\emph{Moderates} means you either know or suspect that a variable's
effect on another variable is moderated by one or more variables.
\end{quote}

\begin{verbatim}
For example:
```
# Experience level moderates the effect of programming language on lines of code
moderates(c(experience_level, language), loc)
# In other words, programming language moderates the effect of experience level on lines of code
# Order of moderating variables doesn't matter
moderates(c(language, experience_level), loc)

# Note: Moderates implies experience and programming language cause or are associated with lines of code (if not already specified)
\end{verbatim}

\hypertarget{your-input}{%
\subsection*{Your input!}\label{your-input}}

Starting from these language ideas, specify your conceptual model
relating the variables.

Feel free to adapt or introduce new syntax if you feel it's missing.
Also, feel free to introduce new relationships, language constructs,
etc! \emph{Really try to fully communicate what you want to express
about the data measurement relationships.}
\clearpage

\begin{comment}
\hypertarget{statistical-model}{%
\section{Statistical model}\label{statistical-model}}

You specified statistical models to assess each conceptual hypothesis. I
want you to imagine that you had the help of a system to guide you in
specifying the statistical model.

What was difficult about authoring your own analyses?

\begin{verbatim}
# Write your response here
\end{verbatim}

\hypertarget{current-idea-query-for-a-statistical-model}{%
\subsection*{Current idea: Query for a statistical
model}\label{current-idea-query-for-a-statistical-model}}

\emph{As before, the language constructs presented here and for the
remainder of the study are an \textbf{initial set} of ideas we have. We
are actively seeking to \textbf{add, remove, and adapt the language.}}

After specifying variables and relationshis, we can query for a
statistical model. In a query, we specify what variables we are
interested in and allow a system to suggest a statistical model, asking
for your input if necessary. \textgreater{} A query consists of a
dependent variable and a set of independent variables.

\begin{verbatim}
For example
# To query for a statistical model to answer our research question about how IDEs and experience impact lines of code a programmer writes:
infer_model(ivs=c(IDE, experience), dv=loc)
\end{verbatim}

\hypertarget{your-input}{%
\subsection*{Your input!}\label{your-input}}

Starting from these language ideas, specify each of the following.

Feel free to adapt or introduce new syntax if you feel it's missing.
Also, feel free to introduce new relationships, language constructs,
etc! )\emph{Really try to fully communicate what you want to express
about the data measurement relationships.}

Conceptual Hypothesis 1 (CH1): How would you specify a query to assess
CH1?

\begin{verbatim}
\end{verbatim}

Conceptual Hypothesis 2 (CH2): How would you specify a query to assess
CH2?

\begin{verbatim}
\end{verbatim}

Conceptual Hypothesis 3 (CH3): How would you specify a query to assess
CH3?

\begin{verbatim}
\end{verbatim}

Conceptual Hypothesis 4 (CH4): How would you specify a query to assess
CH4?

\begin{verbatim}
\end{verbatim}
\clearpage
\end{comment}

\begin{comment}
\hypertarget{anything-else}{%
\subsection*{Anything else?}\label{anything-else}}

Taking a step back from the program you've written with this new syntax,
is there anything you would like to add or change?

What else do you want to express in your program but currently feel
limited or unable to?

\begin{verbatim}
# Want to express...

# I would like to express it in this way...
\end{verbatim}
\end{comment}