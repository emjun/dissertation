\section{Analysis of Software Tools} \label{sec:toolsAnalysis}

To understand how the design of statistical computing tools may support or
hinder hypothesis formalization (\rqTools), we analyzed widely used software
packages and suites. Throughout, we use the term ``package'' to refer to a set
of programs that must be invoked through code, such as \texttt{lme4},
\texttt{scipy}, and \texttt{statsmodels}. We use the term ``suite'' to refer to
a collection of packages that end-users can access either through code or
graphical user interfaces (GUIs), such as SPSS, SAS, and JMP. We use the term
``tool'' to refer to both. Software packages were a unit of analysis because
they are necessary for model implementation regardless of medium (e.g.,
computational notebook, CoLab, RStudio). As such, our findings apply to tools
that provide wrappers around packages included in our sample.

\vspace{-2mm}
\subsection{Method}

\textbf{Sample:} Our sampling procedure involved two phases: (i)
identifying software packages and suites for model implementation (not visual analysis
tools like Tableau) mentioned more than once across the content analysis and lab
study and (ii) adding recommended packages and suites from online data science communities
our lab participants mentioned or used (e.g., \textit{Towards Data Science}). To
identify these additional tools, we consulted online data analysis
fora~\cite{grolemund2019:recommendedR, bobriakov2017:top15Python,
bobriakov2018:top20Python, prabhu2019:topPython}. The final sample included 20
statistical tools: 14 packages (R: 10, Python: 4); three suites that support
in-tool programming; and three suites that do not support programming.
Table~\ref{tableAnalysisOfTools} contains an overview of our sample and results.

\noindent\textbf{Analysis:} Four specific questions guided our analysis:
\begin{itemize}
    \item \textbf{Specialization:} Data workers in the lab study eagerly named
    specific statistical tools they would use and looked up tool documentation
    during the tasks. This prompted us to ask, \textit{How specialized are the
    tools, and how might specialization (or lack thereof) affect how end-users
    discover and use them to formalize hypotheses?}
    \item \textbf{Statistical Taxonomies:} Data workers in the lab study tried to
    mold their analyses to prior experiences and their taxonomies of statistical
    methods. We wondered what role tools play in this: \textit{How do tools
    organize and group statistical models? How might tool organization and
    end-users' taxonomies interplay during hypothesis formalization?}
    \item \textbf{Model Expression:} Data workers in the lab study jumped to model
    implementation throughout the tasks. Only half provided names of statistical
    methods. We wondered if this was due to how tools enable end-users to express
    their models: \textit{What notation must end-users use to express models in
    the tools?}
    \item \textbf{Computational Issues:} Data workers in the lab study described
    their statistical models using specific function calls. Similarly, although
    it was uncommon for researchers in the content analysis to specify the
    software tools they used, when they did, researchers specified the
    functions, parameters, and settings used. This prompted us to wonder about
    the importance of computational settings: \textit{What specific kinds of
    computational control do tools provide end-users and how might that impact
    hypothesis formalization?}
\end{itemize}

To answer the four questions for each statistical tool, the first author read
and took notes on published articles about tools' designs and implementations,
API documentation and reference manuals, and available source code; followed
online tutorials; consulted question-and-answer sites (e.g., StackExchange) when
necessary; and analyzed sample data with the tools. The first author paid
particular attention to tool organization, programming idioms, functions and
their parameters, and tool failure cases. Table~\ref{tableAnalysisOfTools}
contains citations for resources consulted in the analysis. The iterative
analysis process involved discussions among the co-authors about how to evaluate
the properties of tools from our perspectives as both tool designers/maintainers
and end-users. Here, we focus on end-user (hereafter referred to as analyst)
perspectives informed by our lab study and make callouts to details relevant for
tool designers.

\subsection{Findings and Discussion}
\tableSoftwareAnalysis
We discuss our findings in light of our characterization of hypothesis
formalization in Figure~\ref{figure:overview}. We refer to specific steps and
transitions in Figure~\ref{figure:overview} in \textbf{boldface}.

% The ecosystem of available statistical tools, in theory, support a wide range of
% statistical models and therefore conceptual hypotheses. However, the lack of
% support for navigating specialized libraries, low-level interfaces, and
% computational differences add barriers to the hypothesis formalization process
% and ultimately push analysts towards relying on their own statistical knowledge,
% which may be limited, and understanding low-level computational details, which
% may not be necessary for non-experts.

% \theme{Tool specialization pushes computational concerns higher up the hypothesis formalization process.}
\theme{Specialization.}
Half the tools [T2, T3, T4, T5, T6, T7, T8, T9, T11, T12] in our sample are
specialized in the scope of statistical analysis methods they support (e.g.,
\texttt{brms} supports Bayesian generalized linear multilevel modeling).
\texttt{edgeR} [T3] provides multiple modeling methods but is specialized to the
context of biological count data. Such specialized tools are vital to creating a
widely adopted statistical computing ecosystem, such as R. 

Despite its importance, tool specialization pushes computational concerns higher
up the hypothesis formalization process. Specialized tools require
analysts to consider computational settings while picking a statistical tool
and, possibly, even while mathematically relating their variables. They fuse the
last two steps of hypothesis formalization (\textbf{Statistical Specification}
and \textbf{Model Implementation}). Ultimately, specialization requires analysts
to have more (i) computational knowledge and (ii) foresight about their model
implementations at the cost of focusing on conceptual or data-related concerns
early in hypothesis formalization. 

% For example, to pick between \texttt{MCMCglmm} and
% \texttt{glmmTMB}, which both support generalized linear mixed-effects modeling, analysts would have to know that the two 
% employ different model fitting procedures, namely MCMC routines and maximum likelihood
% estimation using the `Template Model Builder' library, respectively. 

% Analysts must
% not only compare and contrast which optimizers tools support prematurely but
% also consider the ramification these computational choices have on their
% mathematical and conceptual representations of their hypothesis.

One way tool designers minimize the requisite computational knowledge and
foresight while providing the benefits of specialized packages --- which may be
optimal for specific statistical models or data analysis tasks --- is to provide
micro-ecosystems of packages. For example, R's
\texttt{tidymodels}~\cite{tidymodels} and \texttt{tidyverse}~\cite{tidyverse}
create micro-ecosystems that use consistent API syntax and semantics across
interoperable packages. They also push analysts towards what the tool designers
believe to be best practices, such as the use of the tidy data format~\cite{wickham2014tidy}. Tools
that aim to support hypothesis formalization may consider fitting into or
creating micro-ecosystems that provide tool support all along the process,
focusing analysts on concepts, data, or model implementation at various points. 

% \theme{Tool taxonomies introduce challenges that detract from hypothesis
% formalization.} 
\theme{Statistical taxonomies.}
A consequence of tool specialization is the fragmented view of statistical
approaches. For example, we observed analysts in the lab study who viewed the
analysis as a classification task gravitate towards machine learning-focused
libraries, such as \texttt{RandomForest} [T9], \texttt{Keras} [T11], and
\texttt{scikit-learn} [T12]. Because classification can be implemented as
logistic regression, any tool that supports logistic regression, such as the
core \texttt{stats} library in R [T10], provides equally valid, alternative
perspectives on the same analysis and hypothesis. However, tools obfuscate these
connections and do not aid analysts in considering reasonable statistical models
that may be unfamiliar or outside their personal taxonomy. This may explain why
analysts adhered to their personal taxonomies during the lab study.

This problem carries over to tools that support numerous statistical methods.
Ten tools in our sample intend to provide more comprehensive statistical support
[T1, T10, T13, T14, T15, T16, T17, T18, T19, T20]. These tools group statistical
approaches using brittle and inconsistent taxonomies based on data types [T17];
analysis classes that are both highly specific (e.g., ``Item
Response Theory'') and vague (e.g., ``Multivariate analyses'') [T15, T16, T17,
T18, T19, T20]; and disciplines or applications (e.g., ``Epidemiology and
related,'' ``Direct Marketing'') [T16, T17, T20]. Although well-intended to
simplify statistical method selection, tools' taxonomies are at times
misleading. For instance, JMP combines various linear models into a ``Fit
Model'' option that is separate from ``Predictive Modeling'' and ``Specialized
Modeling,'' which are also distinct from the more general ``Multivariate
Methods.'' Once analysts select the ``Fit Model'' option, they can specify the
``Personality'' of their model as ``Generalized Regression,'' ``Generalized
Linear Model,'' or ``Partial Least Squares,'' among many others. This JMP menu
structure implies that (i) a Partial Least Squares model
is distinct from a regression model when it is in fact a type of regression
model and (ii) regression is not useful for prediction, which is not the case. 

In these ways, tools add a ``Navigate taxonomies'' step before the
\textbf{Statistical Specification} step, requiring analysts to match their
conceptual hypotheses with the tools' taxonomies, which may misalign with their personal taxonomies. One reason for this issue may be that tools do
not leverage analysts' intermediate artifacts or understanding during hypothesis
formalization. By the time analysts transition to \textbf{Statistical
Specification}, they have refined their conceptual hypotheses, developed causal
models, and made observations about data. However, tools' taxonomies require
analysts to set these aside and consider another set of decisions imposed by
tool-specific groupings of statistical methods. In this way, tool taxonomies may introduce challenges that detract from hypothesis
formalization.

% \theme{Syntactic and semantic mismatches create a rift between model implementations and conceptual hypotheses.}
\theme{Model expression: Syntax and semantics}
% ~\footnote{Tools also provided syntactic idiosyncrancies for including and
% exclduing intercepts and defining crossed and nested effects structures. Tanaka
% et al.~\cite{tanaka2019symbolic} compare in detail the syntactical trade-offs
% between \texttt{lme4} [T6] and asmerl, a tool not in our sample.}. 

Fifteen tools in our sample provide analysts with interfaces that use
mathematical notation to express statistical models [T1, T2, T3, T4, T6, T7, T8,
T9, T10, T14, T16, T17, T18, T19, T20]. R and Python packages use symbolic
mathematical syntax, and SPSS and Stata use natural language-like syntax.
Expressing a linear model with Sex, Race, and their interaction as predictors of
Annual Income involves the formula \texttt{AnnualIncome $\sim$ Sex + Race +
Sex*Race} in \texttt{lme4} and \texttt{AnnualIncome BY Sex Race Sex*Race} in
SPSS.  In a linear execution of steps involved in hypothesis formalization where
analysts relate variables mathematically (\textbf{Mathematical Equation}) before
specifying and implementing models using tools (\textbf{Statistical
Specification}, \textbf{Model Implementation}), the mathematical interfaces
match analysts' progression. However, in the lab study, analysts did not specify
their models mathematically even when given the opportunity, suggesting that
mathematical syntax may not adequately capture analysts' conceptual or
statistical considerations. 

% and opportunities for higher-level libraries that do
% not require mathematics, which we discuss in the next section.

Syntactic similarity between packages may lower the barrier to trying and
adopting new statistical approaches that more directly test hypotheses and therefore benefit hypothesis formalization. At the same time, syntactic similarity may also introduce unmet expectations of
semantic similarity. For example, \texttt{brms} [T2] uses the same formula
syntax as \texttt{lme4} [T6], smoothing the transition between linear modeling
and Bayesian linear modeling for analysts. However, based on syntactic
similarity, analysts may incorrectly assume statistical equivalence in computed
model values. For example, in \texttt{brms}, the model intercept is the mean of
the posterior when all the independent variables are at \textit{their means},
but in \texttt{lme4}, the intercept is the mean of the model when all the
independent variables are at \textit{zero}. 

% This mismatch between syntax and
% statistical meaning may be especially significant for hypotheses

Conversely, tools introduce syntactic differences between statistical approaches
that are for the most part semantically equivalent, which may lead to additional
challenges in hypothesis formalization. For instance, an ANOVA with repeated
measures and a linear mixed effects model are similar in intent but require two
different function calls, one without a formula (e.g., \texttt{AnovaRM} in
\texttt{statsmodels} [T14]) and another with (e.g., \texttt{mixedlm} in
\texttt{statsmodels} [T14]). Even when considering only ANOVA, tools may provide
similar syntax but implement different sums of squares procedures for
partitioning variance (i.e., Type I, Type II, or Type III).\footnote{Type I is
(a) sensitive to the order in which independent variables are specified because
it assigns variance sequentially and (b) allows interaction terms. Type II (a)
does not assign variance sequentially and (b) does not allow interaction terms.
Type III (a) does not assign variance sequentially and (b) allows interaction
terms. For an easy-to-understand blog post, see~\cite{sumsofsquaresBlog}.} By
default, R's \texttt{stats} core package [T10] uses Type I, \texttt{statsmodels}
[T14] uses Type II, and \texttt{SPSS} [T16] uses Type III. The three different
sum of squares procedures lead to different F-statistics and p-values, which may
lead analysts to different conclusions. More importantly, the procedures encode
different conceptual hypotheses. If analysts have theoretical knowledge or
conceptual hypotheses about the order of independent variables, tools defaulting
to Type I (e.g., R's \texttt{stats} core library) align the model implementation
with the conceptual hypotheses. However, if analysts do not have such conceptual
hypotheses, tools' default behavior would execute (without error) and silently
respond to a conceptual hypothesis different from the one the analyst seeks to
test. In this way, syntactic and semantic mismatches can create a rift between
model implementations and conceptual hypotheses. Furthermore, the impact of tools' ``invisible'' model implementation choices
reinforces the interplay between conceptual and model implementation concerns
during hypothesis formalization. 


\begin{comment}
Tools should highlight the conceptual
assumptions and consequences of modeling choices beyond listing ways to change
defaults in their documentation manuals or Q\&A sites. Doing this would give
analysts more control over and insight into their analysis. Analysts could
revise and refine their hypotheses in response to statistical modeling
constraints or change the statistical models and tools they use to address their
hypotheses. 
\end{comment}

\vspace{-2mm}
% \theme{Fine-grained computational control may require conceptual hypothesis revisions.}
\theme{Computational issues.}
Tools provide end-users with options for optimizers and solvers used to fit
statistical models [T1, T2, T4, T6, T7, T8, T10, T11, T13, T16, T18],
convergence criteria used for fitting
models [T3, T6, T16, T18], and memory and CPU allocation [T2, T5, T12, T15], among more specific
customizations. % (e.g., for matrix multiplication).
For instance, \texttt{lme4} [T6] allows analysts to specify the nonlinear
optimizer and its settings (e.g., the number of iterations, convergence
criteria, etc.) used to fit models. In \texttt{brms} [T2], analysts can also
specify the number of CPUs to dedicate to fitting their models. Some
computational settings are akin to performance optimizations, affecting computer
utilization but not the results. However, not all computational changes are so
well-isolated.

For example, the failure of a model's inference algorithm to converge (in \textbf{Model Implementation})
may prompt mathematical re-formulation (\textbf{Mathematical Equation}), which
may cast \textbf{Observations about Data} in a new light, prompting
\textbf{Causal Model} and \textbf{Conceptual Hypothesis} revision. In other
words, computational failures and decisions may bubble up to conceptual hypothesis
revision and refinement, which may then trickle back down to model
implementation iteration, and so on. In this way, computational
control can be another entry into the dual-search process of hypothesis formalization. 
% Fine-grained computational control may require conceptual hypothesis revisions.
% In
% fact, in the lab study, analysts described their ``data-driven'' analyses in a
% similar light.

In theory this low-level control could help analysts formalize
nuanced conceptual hypotheses in diverse computational environments. However, we found
that tools do not currently provide feedback on the ramifications of these
computational changes, introducing a gulf of evaluation~\cite{norman1986cognitive}. Analysts can
easily change parameters to fine-tune their computational settings, but how they
should interpret their model implementations and revisions conceptually is
unaddressed, suggesting opportunities for future tools to bridge the conceptual
and model implementation gap. 

\vspace{-2mm}
\subsection{Takeaways from the Analysis of Tools}
% Jeff: I think the ultimate question is not just how to more seamlessly transition between approaches 
% (though this could be quite valuable for multiverse specification), but 
% how to help people determine what are a "reasonable" set of appropriate approaches.

% Jeff:  how does the end user express their needs in a way that helps translate from conceptual or partially-formalized concerns to implementation choices? 
% I think our larger argument and research trajectory is that 
% ultimately this is missing and needed, 
% so it would be good to see that reflected more strongly in the tools section. 
% Perhaps this is a missing fifth category among our tool analysis questions?

Taken together, our analysis shows that tools can support a wide range of
statistical models but expect analysts to have more statistical expertise than
may be realistic. They provide limited guidance for analysts (i) to express and
translate their conceptual and partially-formalized concerns and (ii) identify
reasonable models. Tools also provide little-to-no feedback on the conceptual
ramifications of model implementation iterations. These gaps reveal a misalignment
between analysts' hypothesis formalization processes and tools' expectations and
design. Possible reasons for this mismatch may be that tools do not scaffold or
embody the dual-search nature of hypothesis formalization or leverage all the
intermediate artifacts analysts may create (e.g., refined conceptual hypotheses,
causal models, data observations, partial specifications, etc.) throughout the
process.

% JH: this can wait for the next section
% Focusing on
% current tools' designs as opportunities for improved alignment between analysts
% and tools, we derive three concrete implications for future tools.


\begin{comment}

\theme{Syntactic differences that elide semantic similarities between statistical models hinder model implementation search.}
Fifteen tools in our sample provided analysts with interfaces that used
mathematical notation to express statistical models [T1, T2, T3, T4, T6, T7, T8,
T9, T10, T14, T16, T17, T18, T19, T20]. R and Python packages used symbolic
mathematical syntax, and SPSS and Stata used natural language-like syntax.
Expressing a linear model with Sex, Race, and their interaction as predictors of
Annual Income would involve the formula \texttt{AnnualIncome ~ Sex + Race +
Sex*Race} in \texttt{lme4} and \texttt{AnnualIncome BY Sex Race Sex*Race} in
SPSS. Packages use idiosyncratic syntax for including and excluding intercepts
and defining crossed and nested effects structures. For instance, in R packages,
the intercept for linear models is included by default even if end-users do not
write it. Tanaka et al.~\cite{tanaka2019symbolic} compare in detail the
syntactical trade-offs \texttt{lme4} [T6] and asmerl, a tool not in our sample,
make.

Despite using mathematical notation, tools elide when different statistical
approaches and functions are equivalent in mathematical principles and
implementations. For instance, an ANOVA with repeated measures and a linear
mixed effects model are mathematically equivalent but require two different
function calls, one without a formula (ANOVA) and another with (linear model).
Knowing that the linear model is a mathematical representation of the ANOVA, could
encourage analysts to compare and use both approaches in the future. 

% tools' syntax elide mathematical details that affect model implementation. For
% example, when end-users include categorical variables in their models in any
% programmatic tool in our sample, they write one \textit{variable} name even
% though tools include one fewer than the number of \textit{categories} in a
% categorical variable to the computed statistical model. 

In a linear execution of steps involved in hypothesis formalization where
analysts relate the relationships between variables mathematically
(\textbf{Mathematical Equation}) before specifying and implementing models using
tools (\textbf{Statistical Specification}, \textbf{Model Implementation}), the
mathematical interfaces match analysts' progression in hypothesis formalization.
However, we observed in the lab study that analysts do not specify their models
mathematically even when given the opportunity, which may be due to syntactic
inconsistency across tools or the missed opportunity to use mathematical
notation to highlight semantic differences and similarities between statistical model implementations. 


\theme{Syntactic similarities that elide semantic differences create a rift between model implementations and conceptual hypotheses.}

Even for the same common statistical method and function, ANOVA, tools implement
different sums of squares procedures for partitioning variance (i.e., Type I,
Type II, or Type III)~\footnote{Type I is (a) sensitive to the order in which
independent variables are specified because it assigns variance sequentially and
(b) allows interaction terms. Type II (a) does not assign variance sequentially
and (b) does not allow interaction terms. Type III (a) does not assign variance
sequentially and (b) allows interaction terms. For an easy-to-understand blog
post, see~\cite{sumsofsquaresBlog}.}. By default, R's \texttt{stats} [T10] core
package [T10] uses Type I, \texttt{statsmodels} [T14] uses Type II, and
\texttt{SPSS} [T16] uses Type III. The three different sums of squares
procedures lead to different F-statistic values and p-values, which may lead
analysts to different conclusions, but more importantly, the procedures encode
slightly different conceptual hypotheses about how strongly independent
variables relate to the dependent variable relative to each other. For instance,
Type I, which is sensitive to the order in which independent variables are
specified, assumes that the independent variables that are listed closer to the
beginning of the list have stronger relationships to the dependent variable.
Thus, when analysts have no theoretical knowledge or conceptual hypotheses about
how important independent variables are compared to each other, Type I sums of
squares is misaligned with their conceptual hypothesis. However, the R
\texttt{stats} ANOVA function will use Type I sums of squares and silently
change the conceptual hypotheses analysts are testing. 

It is important to note that syntactic similarity between packages may lower the
barrier to trying and adopting new statistical approaches. For example,
\texttt{brms} [T2] uses the same syntax as \texttt{lme4} [T6], smoothing the
transition between linear modeling and Bayesian linear modeling for analysts.
However, syntactic similarities may lead to erroneous conclusions about what the
results of a model say about a conceptual hypothesis. In \texttt{brms}, the
model intercept is the mean of the posterior when all the independent variables
are at \textit{their means}, but in \texttt{lme4}, the intercept if the mean of
the model when all the independent variables are \textit{at 0}.

The conceptual impact ``invisible'' model implementation choices tools make
reinforces the interplay between conceptual and model implementation concerns
during hypothesis formalization. Tools should highlight the conceptual
assumptions and consequences of modeling choices beyond listing ways to change
defaults in their documentation manuals or Q\&A sites. Doing this would give
analysts more control over their analysis. Analysts could revise and refine
their hypotheses in response to statistical modeling constraints or change the
statistical models and tools they use to address their hypotheses. 

\end{comment}