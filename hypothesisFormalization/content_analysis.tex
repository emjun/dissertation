\section{Formative content analysis} \label{sec:contentAnalysisHypoForm} 
To complement our in-depth synthesis of prior work, we conducted a formative
content analysis of 50 peer-reviewed publications from five different domains. 

\subsubsection{Methods}
We randomly sampled ten papers published in 2019 from each of the following
venues: (1) the Proceedings of the National Academy of Sciences (PNAS), (2)
Nature, (3) Psychological Science (PS), (4) Journal of Financial Economics
(JFE), and (5) the ACM Conference on Human Factors in Computing Systems (CHI).
We sampled papers that used statistical analyses as either primary or secondary
methodologies. Our sample represents a plurality of domains and recent
practices.~\footnote{Google Scholar listed the venues among the top three in
their respective areas in 2018. Venues were often clustered in the rankings
without an obvious top-one, so we chose among the top three based on ease of
access to publications (e.g., open access or access through our institution).
Some papers were accepted and published before 2019, but the journals had
included them in 2019 issues.}

% We analyzed published papers because researchers are
% not only likely but required to report significant operationalization choices in
% their publications. 

The first two authors iteratively developed a \codebook to code papers at the
paragraph-level. The \codebook contained five broad categories: (i) research
goals, (ii) data sample information, (iii) statistical analysis, (iv) results
reporting, and (v) computation. Each category had more specific codes to capture
more nuanced differences between papers. This tiered coding scheme enabled us to
see general content patterns across papers and nuanced steps within papers. The
first two authors reached substantial agreement (IRR = .69 - .72) even before
resolving disagreements. The first three authors then (i) read and coded all
sections of papers except the figures, tables, and auxiliary materials that did
not pertain to methodology\footnote{PNAS and Nature papers included a materials
and methods section after references that were distinct from extended tables,
figures, and other auxiliary material. We coded the materials and methods
sections in the appendices and included them in the content
analysis.~\autoref{appendix:contentAnalysisProcedure} describes our process in
greater detail.}; (ii) discussed and summarized the papers' goals and main
findings to ensure comprehension and identify contribution types; and (iii)
visualized each paper as a ``reorderable matrix''~\cite{bertin2011graphics}. 

\figureExampleReorderableMatrix

We adapted Bertin's ``reorderable matrix''~\cite{bertin2011graphics}, an
interactive visualization technique for data exploration, in our analysis. We
visualized each paper in our sample as a matrix where each row represented a
code in our \codebook and each column represented a coded paragraph. We fixed the
order of paragraphs to match the paper's progression. We colored codes (rows)
according to their categories in our \codebook, repeatedly reordered the rows
representing codes, and transposed the matrices to detect visual patterns in the
papers. Figure~\ref{figure:exampleReorderableMatrix} shows an example matrix. 
% and each column represented a code in our \codebook

% We visualized each paper as a reorderable matrix in order to more easily and
% systematically identify patterns in paper structure and content. 
The visual representation of papers' content and structure helped us notice
common patterns across papers and guided our follow-up analyses and discussions
about what steps (\rqSteps) and considerations (\rqProcess) researchers reported
having during hypothesis formalization. Across multiple papers, the matrices
showed how researchers typically start with broader research goals that they
decompose into specific hypotheses (i.e., hypothesis refinement) over the course
of a paper section, for example. Within a single paper, the matrices visually
showed patterns of how researchers motivated and pieced together multiple
experiments and interpreted statistical results in order to make a primary
scientific argument.~\autoref{chapter:appHypoForm} includes our \codebook with
definitions and examples as well as a summary, citation, and annotated matrix
for each paper.

% We were interested in (i) learning about the
% breadth of steps involved in hypothesis formalization rather than assessing how
% well papers fit a predetermined set of steps and (ii) detecting any
% co-occurrence or ordering among steps. 

% For example, research goals could be
% statements or questions about something unknown, or focused examinations of
% possible associations between constructs, among other codes. 

\subsubsection{Findings}

\noindent\textbf{Overview:} We coded a total of 2,989 paragraphs across 50 papers. Results were the most
commonly discussed topic. Approximately 31\% of the paragraphs (in 50 papers)
discussed interpretations of statistical results, and ~11\% (in 37 papers)
provided details about statistical results (e.g., parameter estimates).
Interpreted results often co-occurred with statistical results. ~21\% of
paragraphs (in 40 papers) described data collection design (e.g., how the
experiment was designed, how the data were collected, etc.). Specifications of
statistical models appeared in ~19\% of paragraphs (in 50 papers). ~11\% of
paragraphs (in 45 papers) discussed proxy variables, or measures to quantify
abstract constructs (e.g., music enjoyment). To our surprise, more papers
mentioned software than included equations. Researchers mentioned software used
for statistical analysis in 3\% of paragraphs (in 25 papers), sometimes even
specifying function names and parameters, a level of detail we did not expect to
find in publications. Only fifteen papers (JFE: 9, PS: 5, PNAS: 1) included
equations in a total of 71 paragraphs. This suggests that mathematical
equations, though part of the hypothesis formalization process, are less
important to researchers than their tool-specific implementations.

% Papers published in PNAS and Nature had noticeably different structures than the
% CHI, JFE, and PS papers. The PNAS and Nature papers decoupled research goals,
% data sample properties, and results (in the main paper body) from data
% collection design and statistical analysis (in the appended materials and
% methods section). For individual studies in the CHI, JFE, and PS papers, codes
% repeated in noticeably linear patterns from research goals to data collection
% and sample information to proxy variables and statistical analyses to results.
% Although researchers write about the hypothesis operationalization and
% statistical analysis process linearly, prior observational work describes data
% analysis as highly iterative~\cite{}. The lack of information about hypothesis
% operationalization and the contradiction between scientific narratives and
% processes further suggest the opacity of hypothesis formalization and the need to study it and motivate our lab study. 

We present more comprehensive tables and findings about paper structure, about paper
contributions and venue differences in~\autoref{chapter:appHypoForm}.

% \ej{Fill this in: Through open coding and discussion, we identified 41 papers
% that made empirical contributions showing or explaining a phenomenon, six
% contributed novel methodologies, eight developed and evaluated prototype tools
% (could be physical, biological, etc.), and nine made various other
% contributions (e.g., replications, finding new a species, etc.). Table~\ref{}
% gives an overview of contribution types across venues. -- Mention some papers
% were overlapped? Move this to supp material?}

\theme{Researchers deconstruct hypotheses into sub-goals that correspond to statistical analyses.}

In approximately 70\% of papers in the corpus, we found that researchers
deconstructed their motivating research questions and overarching hypotheses into
more tightly scoped objectives or relationships of interest that map to
specific statistical analyses. For example, in~\cite{N8}, the researchers asked how theories of macroevolution
varied across groups of species. The authors divided pre-existing hypotheses
into three classes of hypotheses and assessed each class in turn. For one class
of ``geometric'' hypotheses about insect egg size, the researchers
discriminated between two opposing hypotheses by examining ``the scaling
exponent of length and width (the slope of the regression of log-transformed
length and log-transformed width).'' As this example demonstrates, hypothesis
formalization involves an \emph{iterative hypothesis refinement process at the
conceptual level}. This refinement process distills hierarchies of hypotheses
and/or a single conceptual hypothesis into sub-hypotheses and formalizes these
sub-hypotheses in statistical model implementations. Researchers also relate sub-hypotheses to one other during this process, which implies their causal models about the motivating conceptual hypothesis (and domain).

\theme{Researchers concretize hypotheses using proxies that are based on theory or available data.}
Proxy variables further refine conceptual hypotheses by identifying how observable some
concepts are, measuring the observable ones, indirectly measuring the less
observable ones, and comparing measurement choices to other possible measures or
ideal scenarios. As such, proxy variable selection is an important
transition step between conceptual and data concerns during hypothesis
formalization.

When defining proxy variables, researchers (i) used previously validated
measures when available for theoretical and methodological soundness, such as
the Barcelona Music Reward Questionnaire (BMRQ) to measure music reward
(in~\cite{PS1}), or (ii) developed new measures as a research contribution. For
example, in~\cite{CHI0}, the authors develop an EEG-based measure for
``immersiveness'' in VR they demonstrated to be superior to previous measures
that required halting immersive VR experiences to ask users about immersion.
Researchers also sometimes justified choosing proxies based on available data.
For example, in~\cite{JFE5}, the researchers wanted to develop a proxy variable
for job rank based on titles and ``financial outcomes'' (e.g., compensation,
bonuses, etc.) to see if housing bankers were promoted or demoted after the 2008 stock market
crash. However, because the financial outcomes were not public, the researchers
relied on title only to compare bankers' ranks, which was sub-optimal because
job titles differ between companies. 

% the economists asked if and how
% housing brokers were disciplined after the 2008 stock market crash. The


% When developing new proxies, researchers validated their measures by comparing
% to other measures, which may themselves be based on theory or available data. 

Researchers consider their proxy choices as study limitations and consider
alternative proxies to ensure that their findings are robust. Validating
findings with multiple proxies suggests that hypothesis formalization can be a
\emph{recursive process}. Proxies lead to follow-up hypotheses about possible
latent measurement factors, for instance, which in turn lead to additional analyses that address
the same conceptual hypothesis. 

\theme{Data collection and sampling influence statistical analysis.}
Researchers often described their data sampling and study design as factors that
necessitated additional steps in their analysis process. In~\cite{PS0}
and~\cite{PS5}, researchers accounted for effects of task order in their study
protocol by running additional regressions or analyzing tasks separately.
Researchers also ran initial analyses to assess the effect of possibly
confounding variables in their study design, such as gender in~\cite{PS3} or
location of stimuli in~\cite{PS4}. Other times, researchers performed robustness
checks after their main analyses, such as in response to a gender imbalance
in~\cite{PS5} and possible sample selection biases due to database constraints
in~\cite{JFE1}.

Although data collection driven by statistical modeling plans was expected of
replication studies (e.g.,~\cite{PS8,PS5,PS0}) or papers that make
methodological contributions (e.g.,~\cite{JFE6, JFE7}), we found an instance
in~\cite{PS2}\textemdash neither replication nor methodological
contribution\textemdash where researchers explicitly reported selecting a
statistical model before designing their study. The researchers chose to use a
well-validated computational model, the linear ballistic accumulator (LBA), to
quantify aspects of human decision making. This model selection influenced the
way they designed their study protocol so that they could obtain a sample large
enough for accurate parameter estimation. 

Based on these observations, it seems that modeling choices more frequently
react to data collection processes and possible sample biases, following a
linear data collection-first process implied by prior work. However, there are
also instances where model implementation comes first and researchers' data
collection procedures must adhere to modeling needs, suggesting a previously
missing \emph{loop between statistical model implementation and data collection}
that is separate from any influences conceptual hypotheses have on data
collection. 

\subsection{Discussion}
The content analysis confirmed prior findings on (i) the connection between
hypotheses and causal models (e.g.,\cite{mcelreath2020statistical}), (ii) the
importance of proxies to quantify concepts, and (iii) the constraints that data
collection design and logistics place on modeling. Extending prior work, the
content analysis also (i) suggested that decomposing hypotheses into specific
objectives is a mechanism by which conceptual hypotheses relate to causal
models; (ii) crystallized the hypothesis refinement loop involving conceptual
hypotheses, causal models and proxies; and (iii) surfaced the dual-search nature
of hypothesis formalization by suggesting that model implementation may shape
data collection. 

% The content analysis confirmed prior findings on (i) the connection between
% hypotheses and causal models, (ii) the importance of proxies to quantify
% concepts, and (iii) the constraints that data collection design and logistics
% place on modeling. Extending prior work, the content analysis also (i) suggested
% that decomposing hypotheses into specific objectives is a mechanism by which
% conceptual hypotheses relate to causal models; (ii) crystallized the hypothesis
% refinement loop involving conceptual hypotheses, causal models and proxies; and (iii)
% surfaced the dual-search nature of hypothesis formalization by suggesting that
% model implementation may shape data collection. 

The content analysis also raised questions about how much the pressures to write
compelling scientific narratives~\cite{kerr1998harking} influence which aspects
of hypothesis formalization are described or omitted (e.g., in practice, model
implementations may constrain data collection more often than we found in our
dataset), how the steps are portrayed linearly even though the process may have been more iterative, how analysts determine
which tools to use, and how analysts without domain expertise may approach hypothesis formalization differently. These questions motivated us to design and conduct a lab
study to provide greater visibility into how analysts who are not necessarily
researchers approach the process with expectations of rigor but without pressure
of publication. 

\subsubsection{Limitations}
The major limitation of analyzing published papers is the disconnect between
actual and reported analytical practice. The pressures to write compelling
scientific narratives~\cite{kerr1998harking} likely influence which aspects of
hypothesis formalization are described or omitted. For instance, in practice, model
implementations may constrain data collection more often than we found in our
sample. Nevertheless, the lack of information in prior work and the content
analysis suggests that hypothesis formalization remains an opaque process
deserving of greater scrutiny. Hypothesis formalization may explain how analysts
determine which tools to use and how domain expertise may influence the analytical conclusions reached. 

\subsection{Takeaways: Expected Steps in Hypothesis Formalization}
% \figurePriorWork 
Towards our first two research questions about what actions analysts take to
formalize hypotheses (\rqSteps) and why (\rqProcess), prior work and our
formative content analysis suggest that hypothesis formalization involves steps
in three categories: conceptual, data-based, and statistical.
\textit{Conceptually,} analysts develop conceptual hypotheses and causal models
about their domain that guide their data analysis. With respect to
\textit{data}, analysts explore data and incorporate insights from exploration,
which can be top-down or bottom-up, into their process of formalizing
hypotheses. The \textit{statistical} concerns analysts must address involve
mathematical and computational concerns, such as identifying a statistical
approach (e.g., linear modeling), representing the problem mathematically (e.g.,
writing out a linear model equation), and then implementing those using
software. In our work, we find evidence to support separating statistical
considerations into concerns about mathematics, statistical specification in
tools, and model implementation using tools.

A key observation about prior work is that there is a tension between iterative
and linear workflows during hypothesis formalization. Although sensemaking
processes involve iteration, concerns about methodological soundness, as
evidenced in pre-registration efforts that require researchers to specify and
follow their steps without deviation, advocate for, or even impose, more linear
processes. More specifically, theories of sensemaking that draw on cognitive
science, in particular~\cite{russell1993cost,grolemund2014cognitive}, propose
larger iteration loops between conceptual and statistical considerations. Some
textbooks and research concerning statistical thinking and
practices~\cite{wild1999statisticalThinking,carver2016guidelines} appear less
committed to iteration while other researchers and practitioners in applied
statistics emphasize \textit{workflows} for iterating on statistical
models~\cite{yu2020veridical,lee2019robust,gelman2013bayesianTextbook}.
Workflows (e.g., model expansion) can help researchers start with simple models and build up to more
complex ones by incrementally testing and refining their
understanding of characteristics of the data, the model fitting algorithms, and
computational settings~\cite{betancourt2020bayesianWorkflow,gelman2020bayesianWorkflow,gabry2019visualization}. Moreover, empirical work in HCI on data analysis embraces
iteration during exploration and observes iteration during some phases of
confirmatory data analysis, such as statistical model choice, but not in others,
such as tool selection. In our work, we are sensitive to this tension and aim to
provide more granular insight into iterations and linear processes involved in
hypothesis formalization. We also anticipate that the steps identified in prior
work will recur in our lab study, but we do not limit our investigation to these
steps. 