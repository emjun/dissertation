\section{Discussion: Design Implications for Statistical Analysis Software} \label{sec:implications}
% \figureImplications
Our findings suggest three opportunities for tools to facilitate the dual-search
process and align conceptual hypotheses with statistical model implementations
at various stages of hypothesis formalization. 

% Figure~\ref{figure:implications} provides a
% simplified view of our model with the implications labeled. Any given tools may
% realize combinations of these implications. We recognize that, as Schunn and
% Klahr~\cite{schunn1996BeyondTwoSpace} note, the process of realizing these
% implications from the model may refine and revise our understanding of the
% spaces involved in hypothesis formalization.

\subsection{Meta-libraries: Connecting Model Implementations with Mathematical Equations}

% Additionally, even for tools that rely on symbolic
% formulae, how tool interfaces (formulae and graphical), model implementation,
% and computed outputs are connected is ambiguous. 

% Additionally, for tools that rely on symbolic formulae, how the formulae relate
% to model implementation and computed outputs is ambiguous. 

% find the appropriate
% library or libraries to execute such a model in its knowledge base; 

Specialized tools, although necessary for sophisticated statistical computation,
require a steep learning curve. \textit{Meta-libraries} could allow analysts to
specify their models in high-level code; execute the models using the appropriate
libraries in their knowledge bases; and then output library information,
functions invoked, any computational settings used, the mathematical model that
is approximated, and the model results. Libraries such as
Parsnip~\cite{parsnip} have begun to provide a unified higher-level interface
that allows analysts to specify a statistical model using more ``generically''
named functions, parameter names, and symbolic formulae (when necessary).
Parsnip then compiles and invokes various library-specific functions for the
same statistical model.

Probabilistic programming languages (PPLs), such as Pyro~\cite{pyro}, Stan~\cite{stan},
BUGS~\cite{bugs}, PyMC~\cite{pymc3}, already enable the development of
meta-libraries. PPLs support modular specification of data, probabilistic
models, and probabilistic hypotheses. Existing libraries, including
\texttt{brms}, provide higher-level APIs whose syntax uses symbolic formulae,
for instance, and compile to programs in a PPL (i.e., Stan in the case of
\texttt{brms}). 

As already seen in Parsnip and tools using PPLs, meta-libraries could
bring three benefits. First, they would provide simpler, less fragmented
interfaces to analysts while continuing to take advantage of tool
specialization. Second, meta-libraries that output complete mathematical
representations would more tightly couple mathematical representations with
implementations, providing an on-ramp for analysts to expand their statistical
knowledge. Third, meta-libraries that show the mathematical representations
alongside underlying libraries' function calls could show syntactical variation
in underlying libraries, indirectly teaching analysts how they might express
their statistical models in other tools, familiarizing analysts with new tools
and models, and even mend fragmented views of identical models (e.g., ANOVA and
regression). 

Future meta-libraries could consider providing a higher-level, declarative
interface that does not require analysts to write symbolic formulae. Designing
such declarative meta-libraries would require formative elicitation studies
(similar to natural programming studies such as~\cite{verou2018extending}) on
declarative primitives that are memorable, distinguishable, and reliably
understood. An additional challenge would lie in maintaining support for various
libraries executed under the hood, especially as libraries change their APIs,
which would strengthen the case for meta-libraries. Although meta-libraries
would not solve the problems involved in understanding how computational
settings affect model execution or conceptual hypotheses, they could
nevertheless provide scaffolding for analysts to more closely examine specific
libraries, especially if multiple libraries execute the same model but do not
all encounter the same computational bottlenecks. 

\subsection{High-level Libraries: Expressing Conceptual Hypotheses to Bootstrap Model Implementations}

The absence of tools for directly expressing conceptual hypotheses may be an
explanation for why data workers in the lab study dove into model
implementation details. High-level libraries could allow analysts to specify data
collection design (e.g., independent variables, dependent variables, controlled
effects, possible random effects); variable data types; expected or known
covariance relationships based on domain expertise; and hypothesized findings in
a library-specific grammar. High-level libraries could compile these conceptual
and data declarations into weighted constraints that represent the applicability
of various statistical approaches, in a fashion similar to
Tea~\cite{jun2019tea}, a domain-specific language for automatically selecting
appropriate statistical analyses for common hypothesis tests. Libraries could
then execute the appropriate statistical approaches, possibly by using a
meta-library as described above. 

In addition to questions of how to represent a robust taxonomy of statistical
approaches computationally, another key challenge for developing high-level
libraries is identifying a set of minimal yet complete primitives that are
useful and usable for analysts to express information that is usually expressed
at different levels of abstraction: conceptual hypotheses, study designs, and
possibly even partial statistical model specifications. For instance, even if a
conceptual hypothesis is expressible in a library, it may be impossible to
answer with a study design or partial statistical model that is expressed in the
same program. An approach may be to draw upon and integrate aspects from
existing high-level libraries and systems that aim to address separate steps of
the hypothesis formalization process, such as
Touchstone2~\cite{eiselmayer2019touchstone2} for study design and Tea and
Statsplorer~\cite{wacharamanotham2015statsplorer} for statistical analysis. 

% Finally, future high-level libraries could provide mathematical
% equations as part of the output to further strengthen the pathway from
% conceptual hypotheses to mathematical representations to model implementations
% while providing an abstraction that more directly maps to analysts' conceptual
% and data-based considerations. 

% identifying a minimal yet complete set of declarations for
% deducing appropriate statistical models and representing a robust taxonomy of
% statistical approaches computationally. 

\subsection{Bidirectional Conceptual Modeling: Co-authoring Conceptual Models and Model Implementations}
Conceptual, or causal, modeling was difficult for the analysts in the lab study.
Some even resisted conceptual modeling for fear of biasing their analyses. Yet,
implicit conceptual models were evident in the hypotheses analysts chose to
implement and the sub-hypotheses researchers articulated in the content
analysis. 

% Tools that automatically derive conceptual models from analysts' model implementations may assuage analysts's fears while also 
% Based on this insight, there is opportunity to increase transparent
% and complete reporting and increased engagement with bias in the analysis
% process through automatic authoring of conceptual models. 

Mixed-initiative systems that make explicit the connection between conceptual
models and statistical model implementations could facilitate hypothesis
formalization from either search process and allow analysts to reflect on their
analyses without fear of bias. For example, a mixed-initiative programming
environment could allow analysts to write an analysis script, detect data
variables in the analysis scripts, identify how groups of variables co-occur in
statistical models, and then visualize conceptual models as graphs where the
nodes represent variables and the edges represent relationships. The
automatically generated conceptual models would serve as templates that analysts
could then manipulate and update to better reflect their internal conceptual
models by specifying the kind of relationship between variables (e.g.,
correlation, linear model, etc.) and assigning any statistical model roles
(e.g., independent variable, dependent variable). As analysts update the visual
conceptual models, they could evaluate script changes the system proposes.
In this way, analysts could externally represent their causal models while
authoring analysis scripts and vice versa. 

Although bidirectional programming environments already exist for vector graphics
creation~\cite{hempel2019sketch}, they have yet to be realized in mainstream data analysis
tools. To realize bidirectional, automatic conceptual modeling, researchers would
need to address important questions about (i) the visual grammar, which would
likely borrow heavily from the causal modeling literature; (ii) program analysis
techniques for identifying variables and defining co-occurrences (e.g., line-based
vs. function-based) in a way that generalizes to multiple statistical libraries;
and (iii) adoption, as analysts who may benefit most from such tools (likely
domain non-experts) may be the most resistant to tools that limit the number of
``insights'' they take away from an analysis. 

% Implication for bidrectional causal modeling: 
% - Walker et al. on Design and Causal learning (PsySci 2020)
% - Burnston (CogSci 2013)

\begin{comment}

\subsection{Joint conceptual and statistical planning}
--- Could this explain OR capture why some papers had multiple sub-hypotheses??
RElate dto TOuchstone
Similarly, given a particular analysis, could suggest a data colelcito/study design that would allow that to happen

Support reasoning at the conceptual and statisitcal implementation levels -- conceptual --> implementation \&\& implementation --> conceptual

Provide a unified interface (syntax, semantics) for specifying specific classes of models
\end{comment}