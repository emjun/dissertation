\section{Exploratory Lab Study} \label{sec:inLabStudy}

To address the limitation of the content analysis, understand analysts'
considerations (\rqProcess) while formalizing their hypotheses (\rqSteps), and
examine the role of statistical software in this process (\rqTools), we designed
and conducted a virtual lab study with freelance data workers who approach the
hypothesis formalization and analysis process with expectations of rigor but
without the pressure of publication.

\subsection{Methods} 
\textbf{Data workers:} We recruited 24 data workers with experience in domains
ranging from marketing to physics to education through Upwork (22) and by
word of mouth (2).\footnote{We refer to our participants as data workers because they work with data but do not represent the entire population of data scientists, which may include statistical experts.}

Twelve data workers held occupations as scientists, freelance data scientists,
project managers, or software engineers. Six were currently enrolled in or had just
finished graduate programs that involved data analysis. Five identified as
current or recent undergraduates looking for jobs in data science. One was an
educator. Data workers self-reported having significant experience on a 10-point
scale adapted from a scale for programming experience~\cite{feigenspan2012measuring} (min=2, max=10, mean=6.4,
std=2.04) and would presumably have familiarity with hypothesis formalization.

The lab study enables us to contrast normative expert practices (found in 
prior work and our formative content analysis) to observed practices with data
workers who are not statistical experts but still work in real-world analysis
settings (i.e., research, marketing, consulting). A benefit of studying these
data workers is that they are likely to benefit most from new tools. 

% \tabledata workers

\noindent\textbf{Protocol:} %Based on our content analysis findings, 
We designed
and conducted a lab study with three parts. Parts 1 and 3 were recorded and automatically transcribed using Zoom. We compensated data workers \$45 for
their time. The first author conducted the study and took notes throughout.

\textit{Part 1: Structured Tasks.}  
Part 1 asked data workers to imagine they were leading a research team to answer
the following research question: ``What aspects of
an individual's background and demographics are associated with income after
they have graduated from high school?''\footnote{We chose the open-ended research
question about income after high school because we expected it to be widely
approachable and require no domain expertise to understand.}
We asked data workers to complete the following tasks:
\begin{itemize}
    \item \textit{Task 1: Hypothesis generation.} Imagining they had access to
    any kind of data thinkable, data workers brainstormed at least three
    hypotheses related to the research question.
    \item \textit{Task 2: Conceptual modeling.} Next, data workers saw a sample
    data schema and developed a conceptual model for one or more of their
    hypotheses. We used the term ``conceptual model'' instead of ``causal
    model'' to avoid (mis)leading data workers. We provided the following
    definition: ``A conceptual model summarizes the process by which some
    outcome occurs. A conceptual model specifies the factors you think influence
    an outcome, what factors you think do not influence an outcome, and how
    those factors might interact to give rise to the outcome.'' 
    \item \textit{Task 3: Statistical model specification.} Finally, we
    presented data workers with a sample dataset and instructed them to specify
    but not implement a statistical model to test one or more of their
    hypotheses. %\ej{Given that statistical software tools were not discussed with statistical specification.} 
\end{itemize}

After the three tasks, we conducted a semi-structured interview with
data workers about (i) their validity concerns\footnote{If data workers were
unfamiliar with the term ``validity,'' we rephrased the questions to be about
``soundness'' or ``reliability.''} and (ii) experiences. To help us
contextualize our observations and assess the generalizability of our findings,
we asked data workers to compare the study's structure and tasks to their
day-to-day data analysis practices.

\textit{Part 2: Take-home analysis.} After the first Zoom session, data workers
implemented their analyses using the previously shown dataset, shared any
analysis artifacts (e.g., scripts, output, visualizations, etc.), and completed
a survey about their implementation experience. Prior to Part 3, the first
author reviewed all submitted materials and developed participant-specific
questions for the final interview.

\textit{Part 3: Final Interview.} The first author asked data workers to give an
overview of their analysis process and describe the hypotheses they tested, how
their analysis impacted their conceptual model and understanding, why they made
certain implementation choices, what challenges they faced (if any), and any
additional concerns about validity.

\noindent\textbf{Materials:} The data schema and dataset used in the study came from a
publicly available dataset from the Pew Research Center~\cite{pewDataset}. Each
task was presented in a separate document. All study materials are included as
supplementary material.

\noindent\textbf{Analysis:} The first author reviewed the data workers' artifacts multiple
times to analyze their content and structure;
thematically analyzed notes and transcripts from data workers' Zoom sessions;
and regularly discussed observations with the other authors throughout analysis.

\subsection{Findings and Discussion} 

% \todo{To clarify, the math representations we expected participants to provide were equations and/or statistical test names (e.g., ANOVA). It is possible that our stimuli primed participants to respond how they would perform, rather than represent, the task even after we clarified the question’s intent. Another interpretation of findings is that data workers prefer to reason about and communicate their analyses procedurally even if they know the math. This would still suggest reviewing how tools support model expression, which we did. We will clarify our expectations about math equations and incorporate the alternative interpretation.]}

Eighteen of the 24 data workers we recruited completed all three parts of the study.
The other six data workers completed only the first Zoom session. In our analysis,
we incorporate data from all data workers for as far as they completed the study. 

We found that data workers had four major steps (\rqSteps) and considerations
(\rqProcess): (i) identifying or creating proxies, (ii) fitting their present
analysis to familiar approaches, (iii) using their tools to specify models
(\rqTools), and (iv) minimizing bias by relying on data. Data workers also faced
challenges acquiring and incorporating domain and statistical knowledge
(\rqProcess).

% Overall/In conclusion, we found that data workers do not reason about statistical
% models independently. Instead, they rely on their prior experiences, tools, and
% external information to identify and implement statistical models that answer their conceptual hypotheses

\theme{Data workers consider proxies and data collection while articulating hypotheses.}
We encouraged data workers to not consider the feasibility of collecting data
while brainstorming hypotheses. Yet, while brainstorming hypotheses,
data workers expressed concern with how to measure constructs [D2, D5, D8,
D12, D18, D22, D24] and how to obtain data [D2, D6, D8, D9, D11, D21, D24].

For instance, D18, a computer science student who had worked on more than five data
analysis projects, grappled with the idea of `privilege' and how to best
quantify it: \longquote{I'm trying to highlight the fact that those who will be
privileged before graduation...that experience will enable them to make again
more money after graduation. I won't say `privilege' because we need to quantify
and qualify for that...it's just an abstract term.} Eventually, D18 wrote two
separate hypotheses about `privilege,' operationalizing it as parental income:
(1) ``People with higher incomes pre graduating, end up having higher
differences between pre and post graduation incomes than those with lower
incomes pre graduation.'' and (2) ``People with parents with lower incomes tend
to have lower incomes pre graduation than those with parents with higher
incomes.'' 

D18 continued to deliberate `privilege' as measured by low and high income,
saying, \shortquote{...again you need to be careful with low and high because
these are just abstract terms. We need to quantify that. What does it mean to be
`low?' What does it mean to be `high?'}. Finally, D18 decided to
\shortquote{maybe use the American standards for low income and high income.}
Although an accepted ``American standard'' may not exist, D18 nevertheless
believed that cultural context was necessary to specify because it could provide
a normalizing scale to compare income during analysis, demonstrating how
data workers plan ahead for statistical modeling while brainstorming and refining
hypotheses. 

Similarly, D2, a freelance data scientist, was very specific about how to measure personality:
%in their hypothesis
``More extraverted individuals (extraversion measured using
the corresponding social network graph) are likely to
achieve higher yearly income later in life.'' 

In the presence of the data schema, more data workers were concerned with proxies
[D2, D5, D6, D7, D8, D9, D16, D18, D21]. Some even adapted their working
definitions to match the available data, similar to how researchers in the
content analysis determined proxies based on data. For instance, D8, who hypothesized that
``individuals interested in STEM fields tend to earn more post high school than
individuals interested in other fields,'' operationalized ``interest'' as
``Major'' --- a variable included in the data schema --- even though they had
previously brainstormed using other proxies such as club attendance in high school. 


% Data workers' focus on measurement and data collection corroborate our findings from
% the content analysis and show how conceptual hypotheses and data collection
% inform one another.

These data workers' closely related considerations of data and concept measurement
demonstrate how conceptual hypotheses and data collection may inform each other,
corroborating our findings from the content analysis.


\theme{Data workers consider implementation and tools when specifying statistical models.}
\figureLabStudyStatSpec When we asked data workers to specify their models
without considering implementation, we anticipated they would name specific
statistical tests (e.g., ``ANOVA''), approaches (e.g., ``linear regression'' or
``decision trees''), or write mathematical models (e.g., $Y = B_0 + B_1X_{age} +
B_2X_{gender}$) that they could then implement using their tools because (a) some
researchers in the literature survey did so in their papers and (b) several data
workers mentioned having years of analysis experience. However, despite the
explicit instruction to disregard implementation, 16 data workers provided to-do
lists or summaries of steps to perform a statistical analysis as their model
specifications [D1, D2, D3, D5, D7, D8, D9, D11, D12, D14, D16, D18, D20, D21,
D22, D23, D24]. Of these 16 data workers, eight also named specific statistical
tests in their descriptions [D3, D7, D8, D11, D12, D14, D18, D20]. 

For example, D8, a data science consultant with 7/10 analysis experience,
specified a list of steps that included creating new variables that aggregated
columns in the dataset, cleaning and wrangling the data, visualizing histograms,
performing chi-squared test, and interpreting the statistical results. Notably,
D8 also specified null and alternative hypotheses, which acted as an
intermediate artifact during hypothesis formalization.
Figure~\ref{figure:labStudyStatSpec} shows D8's statistical specification.

Only four data workers named specific statistical methods without describing
their steps [D4, D6, D15, D17]. Two data workers, D22, a neuroscientist by
training with 8/10 analysis experience, and D19, an educator with 6/10 analysis
experience, attempted to specify their models mathematically. D22 used the
familiar R syntax: ``Current Income \textasciitilde\xspace Educational attainment + Gender +
Interactions of those two.'' On the other hand, D19 gave up because although
they knew the general form of logistic regression, they did not know how to
represent the specific variables in the model they wanted to perform. 

The implementation and software details data workers discussed and included in
their specifications suggest that data workers prefer to skip over mathematical
equations and jump to specification and implementation in their tools. Although
it is possible that study instructions primed data workers to respond about how
they would perform, rather than represent, the task even after researcher
clarifications, this would not explain the level of implementation detail data
workers included. Nine data workers went so far as to mention specific
libraries, even functions, that they would use to program their analyses [D3,
D9, D12, D13, D14, D16, D19, D21, D23]. In their reflective interviews, data
workers also expressed that they often do not specify models outside of
implementing them, which D19 succinctly described: \longquote{I don't normally
write this down because all of this is in a [software] library.} 

% even though some papers included equations as an intermediate step in our content analysis.
% Another possible explanation for why data workers focused on implementation when
% specifying their models is that some statistical paradigms, especially
% machine-learning, are designed to automatically discover the statistical
% structure of data. In a machine-learning setting, for instance, data workers do
% not have to think about the functional form of statistical models but rather
% think about the machine learning architecture or approach for discovering this
% form. We were not able to test or verify this possibility with the data workers
% because we did not collect comprehensive information about their exposure to
% several statistical paradigms. Nevertheless, the inclination to defer functional
% formulation of statistical models to implementation underscores the importance of tools during statistical analysis.

Data workers' statistical knowledge appears to be situated in the programs they
write, and their knowledge of and familiarity with tools constrains the
statistical methods they explore and consider. As such, tools may be a key point
of intervention for guiding data workers toward statistical methods that may be
unfamiliar but are best suited for their conceptual hypotheses.

\theme{Data workers try to fit analyses to previous projects and familiar approaches.}
Data workers spent significant thought and time categorizing their analyses as
``prediction,'' ``classification,'' or ``correlation'' problems [D2, D3, D7,
D10, D11, D18, D19, D21, D22]. To categorize, data workers relied on their
previous projects. While reflecting on their typical analysis process, D21, a software engineer working in healthcare, said (emphasis added),
\longquote{I usually tend to jump...to look at data and \textbf{match [the
analysis problem] with similar patterns} I have seen in the past and start
implementing that or do some rough diagrams [for thinking about parameters, data
type, and implementation] on paper...and start implementing it.} 

Data workers also looked at variable data types (i.e., categorical or continuous) to
categorize. For example, D3, a freelance analyst, pivoted from thinking about \textbf{predicting}
income to \textbf{classifying} income groups (emphasis added) based on data type
information: \longquote{The income, the column, the target value here, is
categorical. I think maybe it wouldn't be a bad idea to see what
\textbf{classification} tasks, what we could do. So instead of trying to
\textbf{predict} because we're not trying to \textbf{predict an exact number},
it seems...like more of a \textbf{classification} problem...}

A provocative case of adhering to prior experiences was D6, a psychological research scientist. Although several
data workers were surprised and frustrated that income was ordinal in the dataset
with categories such as``Under \$10K,'' ``\$10K to \$20K,'' ``\$20K to \$30K,''
up to ''150K+'', none went so far as D6 to synthetically generate normally
distributed income data so that they could implement the linear regression
models they had specified despite saying they knew that income was not normally
distributed. 

When asked further about the importance of normal data, D6 described how they
plan analyses based on having normal data, strive to collect normally
distributed, and rely on domain knowledge to transform the data to be normal
when it may not be after collection: \longquote{...I feel like having non normal
data is something that's like hard for us to deal with. Like it just kind of
messes everything up like. And I know, I know it's not always assumption of all
the tasks, but just that we tend to try really hard to get our variables to be
normally distributed. So, you know, we might like transform it or, you know,
kind of clean it like clean outliers, maybe transform if needed...I mean, it
makes sense because like a lot of measures we do use are like depressive
symptoms or anxiety symptoms and kind of they're naturally normally
distributed...I can probably count on my hand the number of non parametric tests
I've like included in manuscripts.} D6's description of their day-to-day
analyses exemplifies the dual-search nature of hypothesis formalization: Data
workers (i) jump from hypothesis refinement to model specification or
implementation with specific proxies in mind and then (ii) collect and
manipulate their data to fit their model choices. 

We recognize that data workers may have taken shortcuts for the study they would
not typically make in real life. Nevertheless, the constraints we imposed by
using a real-world dataset are to be expected in real-world analyses. Therefore,
our observations still suggest that rather than consider the nature and
structure of their hypotheses and data to inform using new statistical
approaches, which statistical pedagogy and theory may suggest, data workers may
choose familiar statistical approaches and mold their new analyses after
previous ones. 

% Furthermore, D6's description of their day-to-day analyses
% exemplifies the dual-search nature of hypothesis formalization: Data workers (i)
% jump from hypothesis refinemnent to model specification or implementation with
% specific proxies in mind and then (ii) collect and manipulate their data to fit
% their model choices. 

\theme{Data workers try to minimize their biases by focusing on data.}
%  To mitigate their biases, several data workers mentioned wanting to
% research related literature and prior work \ej{[D6, 10, 15 -- not neccessarily
% to mitigate bias]}.
% data workers drew upon their lived experiences [D5, D10, D13, D15, D16, D20,
% D21, D24], media (e.g., books, podcasts, news, etc.) [D2, D3, D5, D6, D7,
% D13, D24], and previous coursework and research [D4, D6, D7] to develop their
% hypotheses. data workers also relied on personal experiences [D8, D12, D20,
% D24] to devise their conceptual models. At the same time, data workers
% recognized that their personal experiences biased their hypotheses and
% conceptual models [D12, D13, D14, D17]. 

Throughout the study, data workers expressed concern that they were biasing the
analysis process. Data workers drew upon their personal experiences to develop
hypotheses [D5, D10, D13, D15, D16, D20, D21, D24] and conceptual models [D8,
D12, D20, D24]. D12, a data analysis project manager, described how their personal experiences may subconsciously
bias their investigation by comparing a hypothetical physicist and social worker
answering the same research question: \longquote{Whereas a social worker by
design...they're meant to look at the humanity behind the numbers [unlike a
physicist]. So like, they may actually end up with different results...actually
sitting in front of this data, trying to model it.}

A few data workers even refused to specify conceptual models for fear of biasing the
statistical analyses [D10, D11, D19]. On the surface, data workers resisted
because they believed that some relationships, such as the effect of age on
income, were too ``obvious'' and did not warrant documentation [D10, D11].
However, relationships between variables that were ``obvious'' to some
data workers were not to others. For instance, D10, a business analyst, described how income would
plateau with age, but other data workers, such as D18, assumed income would monotonically increase with age.

When we probed further into why D10, D11, and D19 rejected a priori conceptual
models, they echoed D10's belief that conceptual models ``put blinders on you.''
Even the data workers who created conceptual models echoed similar concerns of
wanting to ``[l]et the model do the talking'' in their implementations [D3, D15,
D18, D19]. Instead of conceptual modeling, D10 chose to look at all n-ary
relationships in the dataset to determine which variables to keep in a final
statistical model, saying, \longquote{It's so easy to run individual tests...You
can run hypothesis tests faster than you can actually think of what the
hypothesis might be so there's no need to really presuppose what relationships
might exist [in a conceptual model].} Of course, one could start from the
same premise that statistical tests are so easy to execute and conclude that
conceptual modeling is all the more important to prioritize analyses
and prevent false discoveries. 

Similarly, data workers were split on whether they focused their implementation exclusively on their
hypotheses or examined other relationships
in the dataset opportunistically. Nine data workers stuck strictly to testing their hypotheses [D1,
D4, D5, D6, D7, D11, D13, D20, D24]. However, five data workers were more focused on
exploring relationships in the dataset and pushed their hypotheses aside [D2,
D3, D10, D16, D18], and an additional four data workers explored relationships among
variables not previously specified in their hypotheses in addition to their
hypotheses [D14, D15, D17, D21]. D18 justified their choice to ignore their
hypotheses and focus on emergent relationships in the data by saying that they
wanted to be \shortquote{open minded based on the data...open to possibilities.}

Data workers' concerns about bias and choice of which relationships to analyze
(hypothesis only vs. opportunistic) highlight the tension between the two
searches involved in hypothesis formalization: concept-first model
implementations and implementation-first conceptual understanding. Conceptual
models are intermediate artifacts that could reconcile the two search processes
and challenge data workers' ideas of what ``data-driven'' means. However, given some
data workers' resistance to prior conceptual modeling, workflows that help
data workers conceptually model as a way to reflect on their model implementations
and personal biases may be more promising than ones that require them before
implementation.

\theme{Data workers face challenges obtaining and integrating conceptual and statistical information.}
Based on data workers' information search behaviors and self-reports, we found that
data workers faced challenges obtaining and integrating both domain and statistical
knowledge.

Data workers consulted outside resources such as API documentation, Wikipedia, and
the \textit{Towards Data Science} blog throughout the study: one while
brainstorming hypotheses [D13]; three while conceptual modeling [D12, D13, D22];
six while specifying statistical models [D3, D6, D12, D13]. Six data workers
also mentioned consulting outside resources while implementing their analyses
[D1, D3, D11, D14, D15, D21]. By far, statistical help was the most common. 

Furthermore, when data workers reflected on their prior data analysis experiences,
they detailed how collaborators provided domain and statistical expertise that
are instrumental in formalizing hypotheses. Collaborators share data that help
domain experts generate hypotheses [D9], critique and revise conceptual models
and proxies [D4, D8], answer critical data quality questions [D10],
and ensure statistical methods are appropriate [D5, D6, D22].

In the survey participants completed after implementing their analyses, the three most
commonly reported challenges were (i) \textbf{formatting} the data [D1, D4,
D5, D6, D13, D16, D18, D20, D21, D24], (ii) \textbf{identifying} which
statistical analyses to perform with the data to test their hypotheses [D1,
D11, D14, D18, D20, D21], and (iii) \textbf{implementing and executing} analyses
using their tools [D1, D6, D7, D13, D20, D21]. Although we expected data workers
would have difficulty wrangling their data based on prior
work~\cite{kandel2012enterprise}, we were surprised that identifying and
executing statistical tests were also prevalent problems given that (a) data workers
were relatively experienced and (b) could choose their tools. These results, together with 
our observations that data workers rely on their prior experiences and tools, suggest
that data workers have difficulty adapting to new scenarios where new tools and
statistical approaches may be necessary. 

\subsection{Takeaways from the Lab Study}
After the first session, 13 out of the 24 data workers described all the tasks as
familiar, and 10 described most of the tasks and process as familiar. Data workers
commonly remarked that although the process was familiar, the order of the tasks
was ``opposite'' of their usual workflows. In practice, data workers may start with
model implementation before articulating conceptual hypotheses, which opposes
the direction of data analysis that the ASA
recommends~\cite{carver2016guidelines}. Nevertheless, our observations reinforce
the dual-search, non-linear nature of hypothesis formalization.

Moreover, one data worker, D24, a physics researcher who primarily conducted
simulation-based studies expressed that the study and its structure felt
foreign, especially because they had no control over data collection. Other data
workers in the study also described the importance of designing and conducting
data collection as part of their hypothesis formalization process [D4, D6, D9].
Designing data collection methods informs the statistical models data workers
plan to use and helps to refine their conceptual hypotheses by requiring data
workers to identify proxies and the feasibility of collecting the proxy
measures, reinforcing what we saw in the content analysis. The remarks also
suggest that disciplines practice variations of the hypothesis formalization
process we identify based on discipline-specific data collection norms and
constraints. For example, simulating data may sometimes take less time than
collecting human subjects data, so data workers working with simulations may
dive into modeling and data whereas others may need to plan experiments for a
longer period of time. 

% although the data workers in our lab study came from diverse
% domains, including medicine, psychology, and business, and had different data
% collection practices, 
% Furthermore, our sample is limited and may be biased. 
Approximately half of the data workers had either just finished or were enrolled in undergraduate or
graduate programs involving data analysis. As such, half of our sample likely has
limited professional experience outside of their studies and/or freelance work
on Upwork. Additionally, data work available on Upwork may be more narrowly
focused and less representative of end-to-end data analysis or research projects
expected of those with greater statistical expertise. Still, several data
workers in our study mentioned other employments where they gained professional
experience working on larger analysis and research projects. Despite the
limitations of recruiting participants from Upwork and word of mouth, our sample
represents data workers who have training in a diversity of disciplines (e.g.,
medicine, psychology, business), are familiar with a range of statistical
methods, and have experience using a broad range of statistical tools. As such,
the data workers in our study may be representative of analysts who are likely
to benefit most from new tools for supporting hypothesis formalization. 

Finally, we found that data workers relied on prior experiences and tools to specify
and formalize their hypotheses. Tools that scaffold the hypothesis formalization
process by suggesting statistical models that operationalize the conceptual
hypotheses, conceptual models, or partial specifications data workers create along
the way may (i) nudge data workers towards more
robust analyses that test their hypotheses, (ii) overcome limitations of data workers'
prior experiences, and (iii) even expand data workers' statistical knowledge. Thus, we
investigated how current tool designs serve (or under-serve) hypothesis
formalization.
% or similar models data workers start with

