%% Replace all of the orange portions with your personal info

\thispagestyle{empty}
\begin{centering}
\vspace{1in}
University of Washington \\
\vspace*{1.\baselineskip}
{\bf Abstract}\\
\vspace*{1\baselineskip}

{\thesisTitle}\\ %self-explanatory
\vspace*{1.\baselineskip}
{\authorName} \\ %self-explanatory
\vspace*{1.\baselineskip}


\ifdefined\secondAdvisor
    Co-chairs
    \else
    Chair
\fi
of the Supervisory Committee:\\ %change to co-chair if co-advised 
\advisorTitle~\advisor\\ \vspace{-.5em} \advisorDepartment \\
\ifdefined\secondAdvisor
    \secondAdvisorTitle~\secondAdvisor\\\vspace{-.5em}\secondAdvisorDepartment \\
\fi
\end{centering}
\vspace*{\baselineskip}

Data analysis is critical to science, public policy, and business. Despite their
importance, statistical analyses are difficult to author, especially for
researchers with expertise outside of statistics. Existing statistical tools,
prioritizing mathematical expressivity and computational control, are low-level
while researchers’ motivating questions and hypotheses are high-level.
Researchers need to translate their questions and hypotheses into low-level
statistical code in an error-prone process that involves grappling with their
domain knowledge, statistics, and programming. 

This dissertation introduces two tools that embody a new way of authoring
analyses: Tea and Tisane. Researchers directly express their domain knowledge
through higher level abstractions, and the tools will validate the data, select
a statistical analysis, and implement it, all while educating analysts about why
a statistical approach is valid. Tea helps analysts author statistical tests.
Tea’s key insight is that statistical test selection can be cast as a constraint
satisfaction problem. Tisane enables analysts to author generalized linear
models with or without mixed effects, which are difficult for even statistical
experts to author. Using Tisane, analysts can express their conceptual models
using a high-level domain specific language. Tisane translates these conceptual
models into causal DAGs and engages analysts in a disambiguation process to
arrive at an output statistical model. Real-world researchers have already used
these tools to conduct analyses in published research that push their own
disciplines forward. I will also introduce “hypothesis formalization,” a series
of cognitive and operational steps analysts take to translate their research
questions into statistical implementations. Hypothesis formalization
retrospectively explains why Tea improves statistical testing and directly
inspired the design of Tisane. 

Tea and Tisane serve as platforms for further research into computational
support for statistical analysis. This work also exemplifies how combining
human-computer interaction with other areas in and outside of computer science
leads to software tools that impact real-world users.
