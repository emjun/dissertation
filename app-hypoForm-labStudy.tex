Materials for the lab study that helped us define hypothesis formalization
(\autoref{sec:labStudyHypoForm}) are below. 

\section{Task 1: Hypothesis generation}

Scenario: You are a researcher leading a research team that can acquire any kind
of data that you can imagine collecting.\\

You are tasked to answer the following question: \textit{What aspects of an
individual’s background and demographics are associated with income after they
have graduated from high school?} \\

\textbf{Task: Your job is to brainstorm at least three hypotheses that you might
want to test to answer the above question.} You are encouraged to brainstorm
more!

Instructions: Add your hypotheses below the horizontal line. \\

\textbf{Please continually vocalize your thoughts aloud.} The researcher will remind you
to speak up if they hear you go silent at any point during the task. \\

\hrulefill

Feel free to express your hypotheses in any way you find helpful. You can draw,
write, etc. \\

To draw: You may do so below, use your own materials, or use this Google Draw
canvas.

\clearpage

\section{Task 2: Conceptual modeling}

Scenario: Now, imagine your research staff informs you that they are able to
collect data with the data schema on the next page. Considering the data schema
and looking at your hypotheses, consider how you might go about testing one or
more of your previously defined hypotheses. You may revise your hypotheses to
incorporate one or more of the factors in the data schema. \\

\textbf{Task: Develop a conceptual model for how to go about testing one or more of your
hypotheses.} \\

A conceptual model summarizes the process by which some outcome
occurs. A conceptual model specifies the factors you think influence an outcome,
what factors you think do not influence an outcome, and how those factors might
interact to give rise to the outcome. \\

For example, a conceptual model for plant growth might communicate the
following: \textit{Sunlight directly influences plant growth. Water directly influences
plant growth. Both sunlight and water are necessary for plant growth.
Plant-owner love indirectly influences plant growth because love affects plant
placement, which might affect sunlight.} \\

A conceptual model can be a diagram or a list of statements. \\

\textbf{In your conceptual model, please feel free to include any
other variables not included in the data schema.}\\

\hfill \\

Instructions: Add your conceptual model below the horizontal line (after the
data schema). Please just make sure it is clear which hypothesis/es you are
testing.\\


\textbf{As before, please continually vocalize your thoughts aloud.} The researcher will
remind you to speak up if they hear you go silent at any point during the task.\\

** For this section, please \textbf{share your screen} with me.**

\begin{itemize}
    \item \textbf{Participant:} numeric
    \item \textbf{Race:} categorical 
    \begin{itemize}
        \item  \{White, Black, etc.\}
    \end{itemize}
    \item \textbf{Sex:} categorical 
    \begin{itemize}
        \item  \{Male, Female\}
    \end{itemize}
    \item \textbf{Age:} numeric
    \item \textbf{State:} categorical 
    \begin{itemize}
        \item  \{Alabama, Arizona, etc.\}
    \end{itemize}
    \item \textbf{Community:} categorical 
    \begin{itemize}
        \item  \{Suburban, Urban, Rural\}
    \end{itemize}
    \item \textbf{Highest Edu Completed:} categorical 
    \begin{itemize}
        \item  \{High school, Some college, Two-year college, etc.\}
    \end{itemize}
    \item \textbf{Current Student Status:} categorical 
    \begin{itemize}
        \item  \{Yes, No\}
    \end{itemize}
    \item \textbf{Major:} categorical 
    \begin{itemize}
        \item  \{Economics, Education, etc.\}
    \end{itemize}
    \item \textbf{Employment:} categorical 
    \begin{itemize}
        \item  \{Disabled, Full-time, Not employed, etc.\}
    \end{itemize}
    \item \textbf{Marital Status:} categorical
    \begin{itemize}
        \item  \{Divorced, Living with a partner, etc.\}
    \end{itemize}
    \item \textbf{Housing:} categorical 
    \begin{itemize}
        \item  \{Live in a dorm, Live  with parents, etc.\}
    \end{itemize}
    \item \textbf{Annual Income:} categorical
    \begin{itemize}
        \item  \{Less than \$10K, \$10K to under \$20K, \$20K to under \$30K, etc.\}
    \end{itemize}
\end{itemize}

\hrulefill

Feel free to express your conceptual model in any way you find helpful. You can
draw, write, etc. \\

To draw: You may do so below or use this Google Draw canvas.	
\clearpage


\section{Task 3: Statisitcal model specification}
Scenario: Good news! Your research staff was able to collect the following
dataset, consistent with the data schema you saw before. \\

The dataset is here: <add URL here>\\

\hfill

\textbf{Task: Without considering implementation, precisely specify a statistical
model/analysis (e.g., model, parameters, etc.) that could generate the data and
help you test one or more of your hypotheses.} \\

Your specification should be as precise as possible and should include which
statistical model/s, evaluation metric/s, and any necessary interpretation
information you plan to use to interpret the statistical results. Your
specification should be precise enough that you could hand it to your research
staff and they would have no trouble implementing the analysis. \\

You are welcome to use anything (e.g., reference materials, visualizations,
etc.) to help you write your specification. \\

\hfill 

Instructions: \\
** For this section, please \textbf{share your screen} with me.** \\

Add your specification below the horizontal line. \\

\textbf{As before, please also continually vocalize your thoughts aloud.} The
researcher will remind you to speak up if they hear you go silent at any point
during the task. 

\hrulefill

Feel free to precisely specify the statistical analysis in any way you find
helpful. You can draw, write, etc. \\

To draw: You may do so below or use this Google Draw canvas.
\clearpage

\section{Take-home analysis instructions}
Scenario: Your research staff needs your help implementing the statistical
analysis you specified beforehand. Using your implementation, your research
staff will go out, collect a fresh dataset containing the same exact variables,
and then execute your statistical analysis on the fresh data. \\

The dataset from before is here: <URL>\\

\textbf{Task: Use the data and any software tool/s of your choice to implement your
statistical analysis.}

Once you are finished,
\begin{enumerate}
    \item please add any artifacts you create, including scripts, visualizations, statistical results, etc. to this folder, and 
    \item complete the survey <link to survey>. Your ID\# is P\#\#. 
\end{enumerate} 

\textit{Please make sure to complete the analysis before answering the survey.}

