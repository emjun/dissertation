% \ej{Remove "this paper"; Rephrase the language describing code and data releases.}
\begin{flushright}
    \begin{quote}
        \textit{
        The enormous variety of modern quantitative methods leaves researchers with the
        nontrivial task of matching analysis and design to the research question.} \\ 
        \vspace{-10pt}
        \begin{flushright}
            \textit{                                 - Ronald Fisher~\cite{fisher1937design}}
        \end{flushright}
    \end{quote}
\end{flushright}
    
Since the development of modern statistical methods (e.g., Student's t-test,
ANOVA, etc.), statisticians have acknowledged the difficulty of identifying
which statistical tests people should use to answer their specific research
questions. Almost a century later, choosing appropriate statistical tests for
evaluating a hypothesis remains a challenge. As a consequence, errors in
statistical analyses are common~\cite{kaptein2012rethinking}, especially given
that data analysis has become a common task for people with little to no
statistical expertise.

A wide variety of tools (such as SPSS~\cite{wiki:spss}, SAS~\cite{wiki:sas}, and
JMP~\cite{wiki:jmp}), programming languages (e.g., R~\cite{wiki:r-language}),
and libraries (including numpy~\cite{oliphant2006numpy}, scipy~\cite{scipy}, and
statsmodels~\cite{seabold2010statsmodels}), enable people to perform specific
statistical tests, but they do not address the fundamental problem that users
may not know which statistical test to perform and how to verify that specific
assumptions about their data hold. In fact, all of these tools place the burden
of valid, replicable statistical analyses on the user and demand deep knowledge
of statistics.

Users not only have to identify their research questions, hypotheses, and domain
assumptions, but also must select statistical tests for their hypotheses (e.g.,
Student's t-test or one-way ANOVA). For each statistical test, users must be
aware of the statistical assumptions each test makes about the data (e.g.,
normality or equal variance between groups) and how to check for them, which
requires additional statistical tests (e.g., Levene's test for equal variance),
which themselves may demand further assumptions about the data. This cognitively
demanding process requires significant knowledge about statistical tests
and their preconditions as well as the ability to perform the tests and verify their
preconditions. This process can easily lead to mistakes.

In response, we design and developed Tea\footnote{named after Fisher's ``Lady
Tasting Tea'' experiment~\cite{fisher1937design}}, a high-level declarative
language for automating statistical test selection and execution that abstracts
the details of statistical analysis from the users. Tea captures users'
hypotheses and domain knowledge, translates this information into a constraint
satisfaction problem, identifies all valid statistical tests to evaluate a
hypothesis, and executes the tests. Tea's higher-level, declarative nature aims
to lower the barrier to valid, replicable analyses.

Tea is easy to integrate directly into common data analysis workflows for users
who have minimal programming experience. Tea is implemented as an open-source
Python library, so programmers can use Tea wherever they use Python, including
within Python notebooks.

In addition, Tea is flexible. Its abstraction of the analysis process and use of
a constraint solver to select tests is designed to support its extension to
emerging statistical methods, such as Bayesian analysis. Currently, Tea supports
frequentist Null Hypothesis Significance Testing (NHST).

This work makes the following contributions:
\begin{itemize}
    \item a novel DSL for automatically selecting and executing statistical
    analyses based on users' hypotheses and domain knowledge
    (\autoref{sec:TeaPL}), 
    \item a runtime system that formulates statistical test selection as a maximum constraint satisfaction problem (\autoref{sec:TeaRS}), and
    \item an initial evaluation showing that Tea can express and execute common NHST statistical tests (\autoref{sec:eval}). 
\end{itemize}

After describing related work, the chapter describes a usage scenario, providing
an overview of Tea (\autoref{sec:usagescenario}). Then, we discuss the concerns
about statistics in the HCI community that shaped Tea's
design~(\autoref{sec:design}), the implementation of
\TeaPL~(\autoref{sec:TeaPL}), the implementation of
\TeaRS~(\autoref{sec:TeaRS}), and the evaluation of Tea as a
whole~(\autoref{sec:eval}). The chapter concludes with a discussion of Tea's
goals, limitations, and future work (\autoref{sec:discussionTea}) and a summary
of how \tea demonstrates my thesis(\autoref{sec:summaryTea})

% As we found in our analysis of tools (\autoref{sec:toolsAnalysis}), a wide
% variety of tools (such as SPSS~\cite{wiki:spss}, SAS~\cite{wiki:sas}, and
% JMP~\cite{wiki:jmp}), programming languages (e.g., R~\cite{wiki:r-language}),
% and libraries (including numpy~\cite{oliphant2006numpy}, scipy~\cite{scipy}, and
% statsmodels~\cite{statsmodelsPaper}), enable people to perform specific
% statistical tests, but they do not address the fundamental problem that users
% may not know which statistical test to perform and how to verify that specific
% assumptions about their data hold. 

% To address this overlooked need, we designed Tea\footnote{named after Fisher's
% ``Lady Tasting Tea'' experiment~\cite{fisher1937design}}, a high-level
% declarative language for automating statistical test selection and execution
% that abstracts the details of statistical analysis from the users. Tea captures
% users' hypotheses and domain knowledge (\higherLevel, \connectConceptualStats),
% reformulates these into a constraint satisfaction problem, identifies all valid
% statistical tests to evaluate a hypothesis, and executes the tests. 

\section{Background and related work} \label{sec:relatedWorkTea}
% Tea extends prior work on domain-specific languages for the data life cycle and
% constraint-based approaches in HCI. 
Domain-specific languages encapsulate key, routine ideas of domain (e.g.,
statistical analysis), making programs more concise to write for end-users,
providing interfaces to connect with other DSLs and systems, and shift the
burden of accurate processing from users to systems through specialized
reasoning. In the context of the data lifecycle, researchers have developed DSLs
that focus on supporting various stages of data exploration, experiment design,
and data cleaning. To support data exploration,
Vega-Lite~\cite{satyanarayan2017vega} is a high-level declarative language that
supports users in developing interactive data visualizations without writing
functional reactive components. PlanOut~\cite{bakshy2014planout} is a DSL for
expressing and coordinating online field experiments. More niche than PlanOut,
Touchstone2 provides the Touchstone Language for specifying condition
randomization in experiments (e.g., Latin
Squares)~\cite{eiselmayer2019touchstone2}.%Experimental design is also an essential aspect of the domain knowledge users encode in Tea programs. 
To support rapid data cleaning,  Wrangler~\cite{kandel2011wrangler} combines a
mixed-initiative interface with a declarative transformation language. Tea
provides a language to support another crucial step in the data life cycle:
statistical analysis. Tea can be integrated into data analysis workflows and
work in tandem with tools such as Wrangler that produce cleaned CSV files ready
for analysis.
%As a declarative language, Tea has a similar goal for statistical analysis. Tea users do not write any code that performs statistical procedures. They instead focuses on expressing their experimental designs, assumptions, and hypotheses with variables in their data. 

Furthermore, languages provide semantic structure and meaning that can be
reasoned about automatically. For domains with well defined goals, constraint
solvers can be a promising technique. Some of the previous constraint-based
systems in HCI have been Scout~\cite{swearngin2018scout}, a mixed-initiative
system for rapidly prototyping interface designs. Designers specify high-level
constraints based on design concepts (e.g., a profile picture should be more
emphasized than the name), and Scout synthesizes novel interfaces. Scout also
uses Z3's theories of booleans and integer linear arithmetic. More specific to
the data lifecycle are Draco~\cite{moritz2019formalizing} and
SetCoLa~\cite{hoffswell2018setcola}, which formalize visualization constraints
for graphs. Whereas SetCoLa is specifically focused on graph layout, Draco
formalizes visualization best practices as logical constraints to synthesize new
visualizations. The knowledge base can grow and support new design
recommendations with additional constraints. Similarly, Tea codifies tests and
their preconditions as constraints in a knowledge base. Tea aims to provide an
architecture that supports the growth of a statistical analysis knowledge base
as communities adopt new statistical best practices and methods. To our
knowledge, Tea is the first constraint-based system for statistical analysis. 

 %Tea currently uses booleans but could leverage integer arithmetic to increase the expressivity of constraints and statistical tests. 

% \addcontentsline{toc}{subsection}{Statistical scope}
\subsection{Statistical Scope}
Tea is designed for statistical tests common to Null Hypothesis Significance
Testing (NHST). While there are calls to incorporate other methods of
statistical analysis~\cite{kay2016researcher,kaptein2012rethinking}, Null
Hypothesis Significance Testing (NHST) remains the norm in HCI and other
disciplines. Therefore, Tea currently implements a module for NHST with the
tests found to be most common by Wacharamanotham et
al.~\cite{wacharamanotham2015statsplorer}. In particular, Tea supports four
classes of tests: correlation (parametric: \r, \pb; non-parametric: \ktau,
\srho), bivariate mean comparison (parametric: \student, \paired;
non-parametric: \mannu, \wilcox, \welch), multivariate mean comparison
(parametric: \f, \rm, \facANOVA, \twoANOVA; non-parametric: \kw, \friedman), and
comparison of proportions (\chiSq, \fisher). Tea also supports an implementation
of bootstrapping~\cite{efron1992bootstrap}.

\section{Usage Scenario}\label{usageScenarioTea}
\figureTeaProgram

This section describes how an analyst can use Tea to answer their research
questions. We use as an example analyst a historical criminologist who wants to
determine how imprisonment differed across regions of the US in
1960\footnote{The example is taken from Ehrlich~\cite{ehrlich1973participation}
and Vandaele~\cite{vandaele1987participation}. The data set comes as part of the
MASS package in R.}. Figure~\ref{fig:tea_program} shows the Tea code for this
example.

The analyst specifies the data file's path in Tea. Tea handles loading and
storing the data set for the duration of the analysis session. The analyst does
not have to worry about reformatting the data during the analysis process in any way.

The analyst asks if the probability of imprisonment was higher in
southern states than in non-southern states. The analyst identifies
two variables that could help them answer this question: the
probability of imprisonment (`Prob') and geographic location
(`So'). %The latter has already been coded as `1' for southern and `0' for non-southern.
Using Tea, the analyst defines the geographic
location as a dichotomous nominal variable where `1' indicates a
southern state and `0' indicates a non-southern state, and indicates that the
probability of imprisonment is a numeric data type (ratio) with a
range between 0 and 1. %The analyst can additionally identify any
%other variables they care about in the data set.

The analyst then specifies their study design, defining the study type
to be ``observational study'' (rather than ``experimental study'') and
defining the contributor (independent) variable to be the geographic location and
the outcome (dependent) variable to be the probability of
imprisonment.

Based on their prior research, the analyst knows that the probability
of imprisonment in southern and non-southern states is normally
distributed. The analyst provides an assumptions clause to Tea in
which they specify this domain knowledge. They also specify an
acceptable Type I error rate (probability of finding a false positive
result), more colloquially known as the `significance threshold'
($\alpha = .05$) that is acceptable in criminology. If the analyst
does not have assumptions or forgets to provide assumptions, Tea will
use the default of $\alpha = .05$.

% Should we switch the story where hypothesis turns out to be incorrect and alt
% metrics give evidence of that?
The analyst hypothesizes that southern states will have a higher
probability of imprisonment than non-southern states. The analyst
directly expresses this hypothesis in Tea.  \emph{Note that at no
  point does the analyst indicate which statistical tests should be
  performed.}

From this point on, Tea operates entirely automatically.  When the
analyst runs their Tea program, Tea checks properties of the data and
finds that the Student's t-test is appropriate. Tea executes the Student's
t-test and non-parametric alternatives, such as the Mann-Whitney U
test, which provide alternative, consistent results.

Tea generates a table of results from executing the tests, ordered by their
power (i.e., results from the parametric t-test will be listed first
given that it has higher power than the non-parametric
equivalent). Based on this output, the analyst concludes that their
hypothesis---that the probability of imprisonment was higher in
southern states than in non-southern states in 1960---is
supported. The results from alternative statistical tests support this
conclusion, so the analyst can be confident in their assessment.

%The analyst wants to conduct the same analysis with a data set from a different
%year. In Tea, they have to change only one line of code: the file path. As long as the variables exist in the new data set, Tea can conduct the same analysis without altering the rest of the code.\chasins{is this very different from prior approaches?  say python?  does it warrant this much emphasis?}

The analyst can now share their Tea program with colleagues.  Other
researchers can easily see what assumptions the analyst made and what
the intended hypothesis was (since these are explicitly stated in the
Tea program), and reproduce the exact results using Tea.

% If the analyst wanted to do the same analysis for multiple years,
% the analyst could use an outer Python loop to execute the same analysis for all data
% sets they had.

%Tea enables users who may not have statistical expertise to conduct valid, replicable analyses without having to write statistical functions \chasins{clarify}. Instead, users focus on expressing knowledge about their data source, variables of interest, assumptions, and hypotheses. Tea automates test selection, test precondition checking, and test execution, and it surfaces multiple valid statistical tests for testing a given hypothesis. Tea analyses can also be shared and re-run.


\otherSystems

\section{Design Considerations} \label{sec:design}

%The American Psychological Assocation (APA) initiated a Task Force on Statistical Inference~\cite{APATFSI} in the late 1990s to address concerns about statistical practices~\cite{wilkinson1999statistical}. 
% The guidelines and recommendations for
% study design, analysis, and reporting

%In 2007, Cairns outlined common statistical analysis problems in the HCI community that echo concerns articulated in~\cite{wilkinson1999statistical}: not checking or reporting the assumptions made by statistical tests, choosing the incorrect statistical tests to test hypotheses, conducting statistical tests multiple times (multiple comparison), and inconsistent reporting of results, including the omission of non-statistically significant results. 

In designing Tea's language and runtime system, we considered best practices for conducting statistical analyses and derived our own insights on improving the
interaction between users and statistical tools.

We identified five key recommendations for statistical analysis from Cairns' report on common
statistical errors in HCI~\cite{cairns2007hci}, which echoes many concerns articulated by Wilkinson~\cite{wilkinson1999statistical}, and from the American Psychological Association's
Task Force on Statistical Inference~\cite{APATFSI}: 
\begin{itemize}
    \item Users should make explicit their assumptions about the data~\cite{APATFSI}. 
    \item Users should verify and report the results from checking assumptions statistical tests make about the data
    and variables~\cite{cairns2007hci,APATFSI}.
    \item Users should account for multiple comparisons~\cite{cairns2007hci,APATFSI}.
    \item When possible, users should consider alternative analyses that test their hypothesis and select the simplest one~\cite{APATFSI}.
    \item Users should contextualize results from statistical tests using effect sizes and confidence intervals~\cite{APATFSI}.
\end{itemize}

An additional practice we wanted to simplify in Tea was \textit{reproducing analyses}. Table~\ref{tab:otherSystems} shows how Tea compares to current tools in supporting these best practices.

% The last four recommendations pertain to details that require statistical
% expertise many users may not have. Tea aims to lower the barrier to valid statistical analysis. 

Based on these guidelines, we identified two key interaction principles for Tea: 
\begin{enumerate}
    \item \textit{Users should be able to express their expertise, assumptions,
    and intentions for analysis.} Users have domain knowledge and goals
    that cannot be expressed with the low-level API calls to the specific
    statistical tests required by the majority of current tools. A higher level
    of abstraction that focuses on the goals and context of analysis is
    likely to appeal to users who may not have statistical expertise (\autoref{sec:TeaPL}).
    \item \textit{Users should not be burdened with statistical details to
    conduct valid analyses.} Currently, users must not only remember their hypotheses but
    also identify possibly appropriate tests and manually check the
    preconditions for all the tests. %best practices and steps to data analysis.
    Simplifying the user's procedure by automating the test selection process
    can help reduce cognitive demand (\autoref{sec:TeaRS}).
\end{enumerate}

While there are calls to incorporate other methods of statistical
analysis~\cite{kay2016researcher,kaptein2012rethinking}, Null
Hypothesis Significance Testing (NHST) remains the norm in HCI and
other disciplines. Therefore, Tea currently implements a module for
NHST with the tests found to be most common
by~\cite{wacharamanotham2015statsplorer} (see~\autoref{subsec:NHST} for a list of tests).

\section{System overview}
Tea consists of a high-level DSL and a runtime system. There are three key steps
to compiling a Tea program from user specifications to executing statistical
tests:

\begin{enumerate}
    \item \textbf{Check for completeness and syntax.} Tea first checks that a
    user's program specifies a data set, variable declarations, study design
    description, a set of assumptions, and hypotheses using the correct syntax.
    The data set can be empty (with only column names), which may be useful for
    pre-registration for instance. If there are any syntax errors or missing
    parts, Tea will issue an error and stop execution.
    \item \textbf{Check for consistent, well-formed hypotheses.} Using the
    variable declarations, Tea then checks that the hypotheses the user states
    are consistent with variable data types. For instance, Tea would issue an
    error and halt execution if a nominal variable was hypothesized to have a
    positive relationship with another nominal variable. If the nominal
    variables have categories given by numbers (e.g., a variable for education where `1' stands for `High School', `2'
    for `College', etc. ), a linear relationship would be possible to compute by
    treating the categories as raw continuous values. However, treating the numbers as
    values is incorrect and the results misleading because the numbers represent
    discrete categories, not continuous values. Tea avoids such mistakes.
    \item \textbf{Inspect data properties and infer valid statistical tests.}
    Once Tea's compiler verifies that a Tea program is complete, syntactically
    correct, and consistent, \TeaRS~inspects the data to verify properties
    about it and find a set of valid statistical tests. The
    higher-level Tea program is then compiled to logical constraints, which is
    further discussed in~\autoref{sec:TeaRS}.
\end{enumerate}

\subsection{Tea's Domain-Specific Language} \label{sec:TeaPL}
Tea is a DSL embedded in Python, implemented as a Python library\footnote{Tea is open-source and
available for download on \texttt{pip}, a common Python package manager.}. It takes
advantage of existing Python data structures (e.g., classes, dictionaries, and
enums). We chose Python because of its widespread adoption in data science.

A key challenge in designing \tea's DSL is determining the level of granularity
necessary to produce an accurate analysis.  In Tea programs, users describe
their studies in five ways: (1) providing a data set, (2) describing the
variables of interest in that \dataSet, (3) specifying their study design, (4)
stating their assumptions about the variables, and (5) formulating hypotheses
about the relationships between variables. Figure~\ref{fig:modes} shows an an
example Tea program and its output. 

\subsubsection{Data}
Data is required for executing statistical analyses. One challenge in managing
data for analysis is minimizing both duplicated data and user intervention.

To reduce the need for user intervention for data manipulation, Tea
requires the data to be a CSV in long format. CSVs are a common output
format for data storage and cleaning tools. Long format (sometimes
called ``tidy data''~\cite{wickham2014tidy}) is a denormalized format
that is widely used for collecting and storing data, especially for
within-subjects studies.

Unlike R and Python libraries such as numpy~\cite{oliphant2006numpy}, Tea only
requires one instance of the data. Users do not have to duplicate the data or
subsets of it for analyses that require the data to be in slightly different
forms. Minimizing data duplication or segmentation is also important to avoid
user confusion about where some data exist or which subsets of data pertain to
specific statistical tests.

Optionally, users can also indicate a column in the \dataSet that acts
as a relational (or primary) key, or an attribute that uniquely
identifies rows of data. For example, this key could be a participant
identification number in a behavioral experiment. A key is useful for
verifying a study design, described below. Without a key, Tea's default
is that all rows in the data set comprise independent observations (that is, all
variables are between subjects).

To use \tea for pre-registration prior to collecting data, a CSV with only
column names is necessary.

\subsubsection{Variables}
Variables represent columns of interest in the data set. Variables
have a name, a data type (\emph{nominal}, \emph{ordinal},
\emph{interval}, or \emph{ratio}), and, when appropriate, valid
categories.  Users (naturally) refer to variables through a Tea program using
their names. Only nominal and ordinal variables have a list of
possible categories. For ordinal variables, the categories are also
ordered from left to right.

Variables encapsulate queries. The queries represent the index of the
variable's column in the original data set and any filtering
operations applied to the variable. For instance, it is common to
filter by category for nominal variables.
% in statistical tests.

\subsubsection{Study Design}
Three aspects of study design are important for conducting statistical
analyses: (1) the type of study (observational study vs. randomized
experiment), (2) the independent and dependent variables, and (3) the
number of observations per participant (e.g., between-subjects
variables vs. within-subjects variables).

For semantic precision, Tea uses different terms for independent and
dependent variables for observational studies and experiments.  In
experiments, variables are described as either ``independent'' or
``dependent'' variables. In observational studies, variables are either
``contributor'' (independent) or ``outcome'' (dependent) variables. 
% If variables are neither independent nor dependent, they are treated as co-variates.

\subsubsection{Assumptions} \label{subsec:assumptions}
Users' assumptions based on domain knowledge are critical for
conducting and contextualizing studies and analyses. Often, users'
assumptions are particular to variables and specific properties (e.g.,
equal variances across different groups). Current tools generally do
not require that users encode these assumptions, leaving them implicit.

Tea takes the opposite approach to contextualize and increase the
transparency of analyses. It requires that users be explicit about
assumptions and statistical properties pertaining to the analysis as a
whole (e.g., acceptable Type I error rate/significance threshold) and
the data.


Tea supports two modes for treating user assumptions: \textit{strict} and
\textit{relaxed}. In both modes, Tea verifies all user assumptions and issues
warnings for assumptions that statistical testing does not verify. In the
\textit{strict} mode, Tea overrides user assumptions when selecting valid
statistical tests. In the \textit{relaxed} mode, Tea defers to user assumptions
and proceeds as if the assumptions verified even if they did not. The
\textit{strict} mode is the default, but users can specify the \textit{relaxed}
mode. Figure~\ref{fig:modes} shows the two modes and the different warnings and
output they generate.
\figureModes

If users also know that a data transformation (i.e., log transformation) applies
to a variable, they can express this as an assumption. Data transformations are
not properties to be verified but rather treatments of data that are applied
during assumption verification, statistical test selection, and test execution,
which is why they are included in the assumptions clause. The next section discusses the
verification process for assumptions in greater detail.

\subsubsection{Hypotheses}
% \teaHypotheses
Hypotheses drive the statistical analysis process. Users often have
hypotheses that are technically alternative hypotheses.

Tea focuses on capturing users' alternative hypotheses about the
relationship between two or more variables. Tea uses the alternate
hypothesis to conduct either a two-sided or one-sided statistical
test. By default, Tea uses the null hypothesis that there is no
relationship between variables.

% \todo{Add table of hypotheses}

% Figure~\ref{fig:teaHypotheses} exemplifies the range of hypotheses Tea supports.

\subsection{Tea's Constraint-based Runtime System} \label{sec:TeaRS}
Tea compiles programs into logical constraints about the data and
variables, which it resolves using a constraint solver. A significant
benefit of using a constraint solver is extensibility. Adding new
statistical tests does not require modifying the core of Tea's runtime
system. Instead, defining a new test requires expressing a single new
logical relationship between a test and its preconditions.

At runtime, Tea invokes a solver that operates on the logical
constraints it computes to produce a list of valid statistical tests
to conduct. This process presents three key technical challenges: (1)
incorporating statistical knowledge as constraints, (2) expressing
user assumptions as constraints, and (3) recursively selecting
statistical tests to verify preconditions of other statistical tests.

\subsubsection{SMT Solver}
As its constraint solver, Tea uses Z3~\cite{de2008z3}, a Satisfiability Modulo Theory (SMT) solver.

Satisfiability is the process of finding an assignment to variables that makes a
logical formula true. For example, given the logical rules $0 < x < 100$ and $y
< x$, \{$x = 1, y = 0$\}, \{$x = 10, y = 5$\}, and \{$x = 99, y = -100$\} would all be
valid assignments that satisfy the rules. SMT solvers determine the
satisfiability of logical formulas, which can encode boolean, integer, real
number, and uninterpreted function constraints over variables. SMT solvers can also
be used to encode constraint systems, as we use them here. A wide variety of 
applications ranging from the synthesis of novel interface
designs~\cite{swearngin2018scout}, the verification of website
accessibility~\cite{panchekha2018verifying}, and the synthesis of data
structures~\cite{loncaric2016cozy} employ SMT solvers. 

\subsubsection{Logical Encodings}
The first challenge of framing statistical test selection as a constraint satisfaction
problem is defining a logical formulation of statistical
knowledge.

Tea encodes the applicability of a statistical test based on its preconditions.
A statistical test is applicable if and only if all of its preconditions (which
are properties about variables) hold. We derived preconditions for tests
from an online HCI and statistics course~\cite{klemmerCoursera}, a statistics
textbook~\cite{field2012discoveringR}, and publicly available data science
resources from universities~\cite{ucla:whatstat, kent:tutorials}.

Tea represents each precondition for a statistical test as an uninterpreted
function representing a property over one or more variables. Each property is
assigned \texttt{true} if the property holds for the variable/s; similarly, if the
property does not hold, the property function is assigned \texttt{false}.

Tea also encodes statistical knowledge about variable types and properties that
are essential to statistical analysis as axioms, such as the constraint that only a
continuous variable can be normally distributed.

\subsubsection{Algorithm}
Tea frames the problem of finding a set of valid statistical tests as a maximum
satisfiability (MaxSAT) problem that is seeded with user assumptions.

First, Tea translates each user assumption about a data property into an axiom
about a property and variable. As described in~\autoref{subsec:assumptions}, user
assumptions about properties but not data transformations are always checked. In
the \textit{strict} mode, Tea overrides any user assumptions it does not find to
hold, creating an axiom that a property is \texttt{false}. In the \textit{relaxed} mode, Tea
    defers to user assumptions, creating axioms that a property is \texttt{true}. For
any user assumptions that do not pass statistical testing, Tea warns the user and explains
how it will proceed depending on the mode.

Then, for each new statistical test Tea tries to satisfy, Tea checks to see if
each precondition holds. For each precondition checked, Tea adds the property
and variable checked as an axiom to observe as future tests are checked. If any
property violates the axioms derived from users' assumptions, the property is
removed and the test is invalidated. Users' assumptions
always take precedence.

The constraint solver then prunes the search space. Tea does not compute all
properties for all variables, a significant optimization when
analyzing very large data sets.

At the end of this process, Tea finds a set of valid statistical tests
to execute. If this set is empty, Tea defaults to its implementation
of bootstrapping~\cite{efron1992bootstrap}. Otherwise, Tea proceeds
and executes all valid statistical tests. Tea returns a table of
results to users, applying multiple comparison corrections~\cite{holm1979simple} and
calculating effect sizes when appropriate.

\subsubsection{Optimization: Recursive Queries}
When Tea verifies a property holds for a variable, it often must invoke another
statistical test. For example, to check that two groups have equal variance,
Tea must execute Levene's test. The statistical test used for
verification may then itself have a precondition, such as a minimum sample size.

Such recursive queries are inefficient for SMT solvers like Z3 to reason
about. To eliminate recursion, Tea lifts some statistical tests to properties.
For instance, Tea does not encode the Levene's test as a statistical test.
Instead, Tea encodes the property of having equal variance between groups and
executes the Levene's test for two groups when verifying that property for particular variables.

\subsubsection{User Output}
The result of running a Tea program with data is a listing of the results of
executing valid statistical tests, as shown in Figure~\ref{fig:modes}. For each valid
statistical test executed, the output contains the properties of data that Tea
checked and used to determine that a statistical test applied, the test
statistic value, p-value (and an adjusted p-value, if applicable), effect sizes
(Cohen's $d$~\cite{cohen1988statistical} and Vargha Delaney
A12~\cite{vargha2000critique}), the alpha level the user specified in their
program, the precise null hypothesis the statistical test examined, an
interpretation of the results in APA format~\cite{american1983publication}, and text
recommending users to focus on effect size rather than the p-value for a
holistic view of their data. This output is intended to inform users of why Tea
selected specific statistical tests and how to interpret their results.

% \todo{This output is continuing to change and be updated to incorporate visualizations to give users more context and helpfufl outputs for inclusion in future manuscripts.}


\def\r{Pearson's r\xspace}
\def\ktau{Kendall's $\tau$\xspace}
\def\srho{Spearman's $\rho$\xspace}
\def\pb{Pointbiserial\xspace}
\def\student{Student's t-test\xspace}
\def\paired{Paired t-test\xspace}
\def\mannu{Mann-Whitney U\xspace}
\def\wilcox{Wilcoxon signed rank\xspace}
\def\welch{Welch's t-test\xspace}
\def\f{F-test\xspace}
\def\rm{Repeated measures one way ANOVA\xspace}
\def\kw{Kruskal Wallis\xspace}
\def\friedman{Friedman\xspace}
\def\facANOVA{Factorial ANOVA\xspace}
\def\twoANOVA{Two-way ANOVA\xspace}
\def\chiSq{Chi Square\xspace}
\def\fisher{Fisher's Exact\xspace}
\def\boot{Bootstrap\xspace}

\begin{table*}[htbp]
    \begin{center}
    \caption{\textbf{Results of applying Tea to 12 textbook tutorials.}\label{tab:results}}
\begin{minipage}{\linewidth}
\vspace*{-12pt}
\footnotesize{Tea is comparable to an expert selecting statistical tests. Tea can
prevent false positive and false negative results by suggesting only tests that
satisfy all assumptions. \textit{Tutorial} gives the test described in the
textbook; \textit{Candidate tests (p-value)} gives all tests a user could run on
the provided data with corresponding p-values; \textit{Assumptions} gives all
satisfied (lightly shaded) and violated (white)
assumptions; \textit{Tea suggests} indicates which tests Tea suggests based on
their preconditions (assumptions about the data). \textbf{Emphasized} p-values indicate instances where a
candidate test leads to a wrong conclusion about statistical significance.
Although this table focuses on p-values, Tea produces richer output that
provides a more holistic view of the statistical analysis results by including
effect sizes, for instance. Refer to Figure~\ref{fig:modes} for an
example of output from a Tea program.}
\end{minipage}
    \begin{tabularx}{\textwidth}{p{0.2\textwidth}p{0.4\textwidth}lc}
    \toprule
    \colH{Tutorial} & \colH{Candidate tests (p-value)} & \colH{Assumptions*} & \colH{Tea suggests} \\
    \midrule
      Pearson                   & \r    \hfill (6.96925e-06) & \valid{2} \valid{4} \invalid{5} & \no \\
      \cite{kabacoff2011action} & \ktau \hfill (2.04198e-05) & \valid{2} \valid{4}             & \yes \\
                                & \srho \hfill (2.83575e-05) & \valid{2} \valid{4}             & \yes \\
    \midrule
    \srho                       & \srho \hfill (.00172)  & \valid{2} \valid{4}           & \yes \\
    \cite{field2012discoveringR} & \r    \hfill (.01115)    &   \valid{2} \invalid{4}                   & \no \\ 
                                & \ktau \hfill (.00126) & \valid{2} \valid{4} & \yes \\
    \midrule
    \ktau                       & \ktau \hfill (.00126) & \valid{2} \valid{4}             & \yes \\
    \cite{field2012discoveringR} & \r   \hfill (.01115) & \valid{2} \invalid{4}             & \no \\
                                & \srho \hfill (.00172) & \valid{2} \valid{4}             & \yes \\
    \midrule
    \pb                         & \pb (\r)  \hfill (.00287) & \valid{2} \valid{4} \invalid{5}             & \no \\
    \cite{field2012discoveringR} & \srho \hfill (.00477) & \valid{2} \invalid{4}             & \no \\
                                 & \ktau \hfill (.00574) & \valid{2} \invalid{4}             & \no \\
                                 &\boot     \hfill (<0.05)                 &           & \yes \\
    \midrule
    \student                     & \student \hfill (.00012) & \valid{2} \valid{4} \valid{5} \valid{6} \valid{7} \valid{8} & \yes \\
    % \cite{kabacoff2011action}    & \paired  \hfill (N/A)                    & \valid{2} & \no \\
    \cite{kabacoff2011action}    & \mannu   \hfill (9.27319e-05)  & \valid{2} \valid{4} \valid{7} \valid{8} & \yes \\
                                %  & \wilcox  \hfill (N/A)                    & \valid{2} & \no \\
                                 & \welch   \hfill (.00065)  & \valid{2} \valid{4} \valid{5} \valid{7} \valid{8} & \yes \\
    \midrule
    \paired                      & \paired \hfill (.03098)    & \valid{2} \valid{4} \valid{5} \valid{7} \valid{8} & \yes \\
    \cite{field2012discoveringR} & \student \hfill (\textbf{.10684})    & \valid{2} \valid{4} \valid{5} \invalid{7} & \no \\
                                 & \mannu   \hfill (\textbf{.06861})    & \valid{2} \valid{4} \invalid{7} & \no \\
                                 & \wilcox  \hfill (.04586)   & \valid{2} \valid{4} \valid{7} \valid{8} & \yes \\
                                 & \welch   \hfill (\textbf{.10724})    & \valid{2} \invalid{7} & \no \\
    \midrule
    \wilcox                      & \wilcox  \hfill (.04657)    & \valid{2} \valid{4} \valid{7} \valid{8} & \yes \\
    \cite{field2012discoveringR} & \student \hfill (.02690)   & \valid{2} \valid{4} \invalid{7} & \no \\
                                 & \paired  \hfill (.01488)   & \valid{2} \valid{4} \invalid{5} \valid{7} \valid{8} & \no \\
                                 & \mannu   \hfill (.00560)   & \valid{2} \valid{4} \invalid{7} & \no \\
                                 & \welch   \hfill (.03572)   & \valid{2} \valid{4} \invalid{7} & \no \\
    \midrule 
    \f                            & \f      \hfill (9.81852e-13)            & \valid{2} \valid{4} \valid{5} \valid{6} \valid{9} & \yes \\
    \cite{field2012discoveringR}  & \kw     \hfill (2.23813e-07)  & \valid{2} \valid{4} \valid{9} & \yes \\ 
                                  & \friedman \hfill (8.66714e-07) & \valid{2} \invalid{7} & \no \\
                                %   & \rm     \hfill ()           & \valid{2}  & \no \\
                                  & \facANOVA \hfill (9.81852e-13)          & \valid{2} \valid{4} \valid{5} \valid{6} \valid{9} & \yes \\
    \midrule
    \kw                           & \kw      \hfill (.03419)    & \valid{2} \valid{4} \valid{9} & \yes \\
    \cite{field2012discoveringR}  & \f       \hfill (\textbf{.05578})              & \valid{2} \valid{4} \invalid{5} \valid{9} & \no \\
                                  & \friedman \hfill (3.02610e-08) & \valid{2} \invalid{7} & \no \\
                                %   & \rm     \hfill ()           & \valid{2} & \no \\
                                  & \facANOVA \hfill (\textbf{.05578})             & \valid{2} \valid{4} \invalid{5} \valid{9} & \no \\
    \midrule
    \rm                           & \rm     \hfill (.0000)           & \valid{2} \valid{4} \valid{5} \valid{6} \valid{7} \valid{9} & \yes \\
    \cite{field2012discoveringR}  & \kw     \hfill (4.51825e-06)  & \valid{2} \valid{4} \invalid{7} \valid{9} & \no \\
                                  & \f      \hfill (1.24278e-07)           & \valid{2} \valid{4} \valid{5} \valid{6} \invalid{7} \valid{9} & \no \\
                                  & \friedman \hfill (5.23589e-11) & \valid{2} \valid{4} \valid{7} \valid{9} & \yes \\ % friedman says 3+ groups
                                  & \facANOVA \hfill (1.24278e-07)          & \valid{2} \valid{4} \valid{5} \valid{6} \valid{9} & \yes \\
    \midrule
    \twoANOVA                     &\twoANOVA \hfill (3.70282e-17)           & \valid{2} \valid{4} \invalid{5} \valid{9} & \no \\
    \cite{field2012discoveringR}  &\boot     \hfill (<0.05)                 &           & \yes \\
    \midrule
    \chiSq                        & \chiSq \hfill (4.76743e-07) & \valid{2} \valid{4} \valid{9} & \yes \\
    \cite{field2012discoveringR}  & \fisher \hfill (4.76743e-07) & \valid{2} \valid{4} \valid{9} & \yes \\
    \bottomrule
    \multicolumn{4}{p{\linewidth}}{*\small{\assume{1} one variable,
                       \assume{2} two variables,
                       \assume{3} two or more variables,
                       \assume{4} continuous vs. categorical vs. ordinal data,
                       \assume{5} normality,
                       \assume{6} equal variance,
                       \assume{7} dependent vs. independent observations,
                       \assume{8} exactly two groups,
                       \assume{9} two or more groups}}

    \end{tabularx}
    \end{center}
\end{table*}

\section{Evaluation} \label{sec:evalTea}
We assessed the validity of Tea in two ways. First, we compared Tea's
suggestions of statistical tests to suggestions in textbook tutorials.
We use these tutorials as a proxy for expert test selection.
Second, for each tutorial, we compared the analysis results of the test(s)
suggested by Tea to those of the test suggested in the textbook as well as all
other candidate tests. We use the set of all candidate tests as as a proxy for
non-expert test selection.

We differentiate between \textit{candidate} and \textit{valid} tests. A candidate test can be
computed on the data, when ignoring any preconditions regarding the data types or
distributions. A valid test is a candidate test for which all preconditions are
satisfied.

\subsection{How does Tea compare to textbook tutorials?}
Our goal was to compare Tea to expert recommendations.

We sampled 12 data sets and examples from R tutorials (\cite{kabacoff2011action}
and~\cite{field2012discoveringR}). These included eight parametric tests, four
non-parametric tests, and one Chi-square test. We chose these tutorials because they
appeared in two of the top 20 statistical textbooks on Amazon and had publicly available
data sets, which did not require extensive data wrangling.
% and examples that tested the 16 statistical tests that Tea currently supports
%(see~\ref{tab:tea_tests}).

We translated all analyses into Tea and encoded any assumptions explicitly
stated in the tutorial. Tea selected tests based only on the data and the
assumptions expressed in the Tea program. Where Tea disagreed with the
tutorials, either (1) the tutorial authors chose the wrong analyses or (2) the tutorial authors
had implicit assumptions about the data that did not hold up to statistical testing. 

For nine out of the 12 tutorials, Tea suggested the same statistical test (see
Table~\ref{tab:results}). For three out of 12 tutorials, which used a parametric
test, Tea suggested using a non-parametric alternative instead. Tea's
recommendation of using a non-parametric test instead of a parametric one did
not change the statistical significance of the result at the $.05$ level. Tea
suggested non-parametric tests based on the Shapiro-Wilk test for normality. It
is possible that tutorial authors visualized the data to make implicit
assumptions about the data, but this practice or conclusion was not made
explicit in the tutorials.

For the two-way ANOVA tutorial from~\cite{field2012discoveringR}, which studied how gender
and drug usage of individuals affected their perception of attractiveness, a
precondition of the two-way ANOVA is that the dependent measure is normally
distributed in each category. This precondition was violated.  As a result, Tea
defaulted to bootstrapping the means for each group and reported the means and
confidence intervals. 
For the pointbiserial correlation tutorial from~\cite{field2012discoveringR},
Tea also defaulted to bootstrap for two reasons. First, the precondition of
normality is violated. Second, the data uses a dichotomous (nominal) variable,
which invalidates \srho and \ktau.

Tea generally agrees with expert recommendations and is more conservative
in the presence of non-normal data, minimizing the risk of false positive
findings.

\subsection{Does Tea avoid common mistakes made by non-expert users?}
Our goal was to assess whether any of the tests suggested by Tea (i.e., valid
candidate tests) or any of the invalid candidate tests would lead to a different
conclusion than the one drawn in the tutorial. Table~\ref{tab:results} shows the
results. Specifically, emphasized p-values indicate instances for which the
result of a test differs from the tutorial in terms of statistical significance
at the $.05$ level.

For all of the 12 tutorials, Tea's suggested tests led to the same conclusion
about statistical significance. For two out of the 12 tutorials, two or more
candidate tests led to a different conclusion. These candidate tests were
invalid due to violations of independence or normality.

To summarize, the evaluation shows us that (i) Tea can replicate and even improve
upon expert choices and (ii) Tea could help novices avoid common mistakes and
false conclusions.

\section{Discussion, Limitations, and Future Work} \label{sec:discussionTea} 

Our goal with Tea was to determine the feasibility of automating statistical
test selection based on high-level input from analysts. Automating statistical
test selection raises important concerns about the impact of such automation on
the reliability of statistical conclusions. In this regard, there are two chief
concerns pertaining to (i) selective inference and (ii) multiple testing, both
of which inflate the Type I Error Rate and can lead to more false discoveries. 

Tea relies on statistical tests (e.g., Shapiro-Wilk's test for normality) to
assess properties of data to determine which statistical tests (e.g., Student's
t-test) are used to assess the input hypothesis. Repeated property testing of
the data is a form of ``double-dipping''~\cite{}, or using the data to make
decisions about analyses on the data. Preventing this would be ideal to reduce
the false positive discovery rate. However, the statistics community is still
developing techniques to address this issue. A naive approach would be to only
use a sample of the data to determine the final statistical test and then use
another sample to make statistical inferences. While viable for large datasets,
this may not be possible for smaller datasets. A more recently proposed
technique, data fission~\cite{} overcomes, in theory, this dependence on dataset
size.  Data fission introduces noise to the data to make analysis decisions
(i.e., statistical test selection) and then stripping the noise to obtain final
results. Tea does not currently implement either of these approaches. In the
future, Tea should incorporate these and future recommendations from the
statistics community. 

Furthermore, there is an inherent tension between executing multiple statistical
tests (e.g., Student's t-test and Welch's t-test) to show analysts the
robustness, or sensitivity, of statistical results and increasing the number of
comparisons performed. In Tea, we believed that providing analysts with the
ability to compare statistical tests, make sensitivity judgments, and report the
results of a test most common in their disciplines was more important than
restricting the number of statistical tests, especially we have observed
analysts intentionally run multiple statistical tests in order to compare
results on their own. To more fully support sensitivity analyses and discourage
cherry-picking statistical tests and results, Tea should provide more explicit
support for interpreting, comparing, and contrasting statistical test results in
the future. This will be particularly important in scenarios where statistical
tests may disagree with one another. conflicting test conclusions. 

% Future work should take
% advantage of the multiple statistical test outputs as a check on the robustness
% of conclusions and help analysts make sense of the results holistically.
%This is well-aligned with the intention of multiverse analyses to improve robust data analysis. 

% This could mitigate concerns that Tea encourages p-hacking and cherry-picking
% results that are convenient for the analyst, given that Tea may output multiple
% possible statistical tests. 

% The output of executing a Tea program with data is one or more
% statistical test results. 

Finally, Tea's test selection is well suited for answering a class of relatively simple
research questions. At the same time, there are more complex research questions
that analysts want to ask about their domain using data that require more
complex statistical analyses. These are currently out of reach for \tea. For
instance, domain experts may not want to know that there is a difference between
treatment and control groups but also estimate the influence of the treatment on
an outcome in the presence of other variables that also influence treatment and
the outcome. Therefore, in order to support a larger class of research questions
and statistical models, we need to re-consider and extent Tea's abstractions and
constraint-based reasoning approach. 

\begin{comment}
Lessons learned

Why constraints? are they really necessary?

- inflated alpha
- inherent tension in executing multiple statistical tests vs. sensitivity


Multiple testing

Design 

Validity 
\end{comment}


\begin{comment}
To further prevent cherry picking and more holistic understanding of the robustness of conclusion

Reasoning: Constraints were well-suited for statistical test selection, for
which there is general consensus about the applicability of tests. However, that
is not always the case....

Interpretation of results -- tension between 

To further prevent cherry picking and more holistic understanding of the robustness of conclusions

At the time of Tea's release (circa late 2019), we believed that \tea's
abstraction and modularity would enable the incorporation of other statistical
analysis approaches, such as Bayesian inference, as they move into the
mainstream. Although we have yet to incorporate Bayesian inference into \tea, or
more generally provide tool support for authoring Bayesian analyses, our
experience generalizing \tea's approach to statistical modeling in \tisane has
shown us the importance of making even more explicit implicit connections
between variables (abstraction) and providing reasoning approaches that match
analysis needs.


Testing to estimation 

NHST to statistical modeling

How do we support this larger and more complex class of use cases? Can we
generalize our approach in Tea--making implicit assumptions explicit and
automatically reasoning about these assumptions to identify valid analyses--to
statistical models? To answer this question, we set out to develop a holistic
understanding of how analysts translate research questions into statistical
analyses. 

\subsection{Ongoing development}
With teams of undergrads, I have continued to improve Tea in two specific ways. 

First, recent development has focused on updating the outputs of Tea to include
(i) interactive visualizations and (ii) text for reporting the statistical
results in the American Psychological Association's recommended formats for each
valid statistical test. The interactive visualizations aim to illustrate what
the results of the statistical tests mean, such as scatterplots for correlations
and heatmaps for the Chi-squared test. We selected the visualizations for each
test based on recommendations from Franconeri et
al.~\cite{franconeriVisualizationChooser}, what existing tools such as
JMP~\cite{jmp} already use, and our own experiences using and trying to
communicate statistical results. Development and initial user testing is on
track to wrap up by the end of spring quarter. 

Second, a usability issue with Tea's current API is its reliance of ``magic
strings.'' We are currently refactoring the API to be more object-oriented by
extending Tisane's variables data classes. We hope this revision will be more
usable with ``free'' help from existing IDEs such as VSCode that provide API
suggestions inline when specifying parameters. 

Both features will be incorporated into a new release of Tea, which I have
currently scheduled for June, 2022. 
\end{comment}

\section{Summary of Contributions} \label{sec:summaryTea}

% Reminder: Thesis statement
% Domain-specific languages that provide abstractions for expressing conceptual
% knowledge, data collection procedures, and analysis intents instead of specific
% statistical modeling decisions coupled with automated reasoning to compile
% conceptual specifications into statistical analysis code help statistical
% non-experts more readily author valid analyses. 

A common approach to assessing support for conceptual hypotheses in data is to
use statistical tests (e.g., Student's t-test, Chi-Square test, ANOVA).
Statistical testing requires analysts to grapple with their conceptual
hypotheses, know a number of tests and when they are applicable (i.e., know the
preconditions for when these tests hold), assess the applicability of tests
(i.e., check preconditions), and pick and implement specific tests using
low-level APIs. 

Tea's key insight is that we can reformulate statistical test
selection as a constraint satisfaction problem. We designed and implemented a
higher-level DSL around this insight that takes an analyst's hypothesis and
assumptions about their data as input and provides the results of executing
valid statistical tests as output. In an evaluation, we found that Tea avoids
faulty test selection and conclusions that are easy to make using existing
tools. In this way, \tea improves statistical conclusion and internal validity~\cite{shadish2010campbell}. 

Tea demonstrates the feasibility and benefit of developing systems that
emphasize \textit{higher-level abstractions} and \textit{automated reasoning}
for statistical tests (\autoref{para:thesisStatement}). However, using
statistics to answer real-world questions requires going beyond statistical
testing to grappling with statistical modeling and effect estimation. Next, we
consider how our approach generalizes to a larger class of statistical analyses. 

\begin{comment}
% Replace the conclusion section with a summary section. Again, you should tie this chapter back to the main themes of the thesis.
4-5 sentences: 
1. Restate problem 
2. Articulate core contributions: problem/idea + technical
3. Key Evaluation results
4. 1 soundbite/takeaway
5. Transition to next chapter. 

1. Restate problem 

A common approach to assessing support for conceptual hypotheses in data is to
use statistical tests (e.g., Student's t-test, Chi-Square test, ANOVA).
Statistical testing requires analysts to grapple with their conceptual
hypotheses, know a number of tests and when they are applicable (i.e., know the
preconditions for when these tests hold), assess the applicability of tests
(i.e., check preconditions), and pick and implement specific tests using
low-level APIs. 

2. Articulate core contributions: problem/idea + technical

Tea's key insight is that we can reformulate statistical test selection as a
constraint satisfaction problem. We designed and implemented a higher-level DSL
around this insight that takes an analyst's hypothesis and assumptions about
their data as input and provides the results of executing valid statistical
tests as output. 

3. Key Evaluation results

In an evaluation, we found that Tea avoids faulty test
selection and conclusions that are easy to make using existing tools.

4. 1 soundbite/takeaway 

It is possible to design higher level language and automated reasoning to select
valid statistical tests that are widely used and even avoid faulty conclusions
that are easy to make using existing tools. The key is to make implicit
assumptions about the data explicit and reason about analyst assumptions and
computed data properties together in a logical constraint system. 

5. Transition to next chapter. 

However, an important limitation to overcome in the future chapters, most
notably in Tisane (~\autoref{chapter:tisane}) is how to generalize automated
support for more complex research questions and statistical analyses. 

Tea demonstrates the feasibility and benefit of developing systems that
emphasize \textit{higher-level abstractions} and \textit{automated reasoning}
for statistical tests (\autoref{para:thesisStatement}). Next, we consider how
this approach generalizes to a larger class of statistical analyses. could
\end{comment}

\textit{This work was done in collaboration with Maureen Daum, Jared Roesch, Sarah E.
Chasins, Emery Berger, \reneJust, and Katharina Reinecke. It was originally
published and presented at \uistConf{2019}~cite{}. Since publication, multiple
people, including most notably Shreyash Nigam, Reiden Chea, and Annie Denton,
have contributed to updating and improving Tea.}
