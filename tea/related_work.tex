\section{Background and Related work}
% \vspace{-7pt}
Tea extends prior work on domain-specific languages for the data life cycle, 
tools for statistical analysis, and constraint-based approaches in HCI. 

\vspace{-6pt}
\subsection{Domain-specific Languages for the Data Life Cycle}
Prior domain-specific languages (DSLs) have focused on several different stages
of data exploration, experiment design, and data cleaning to shift the burden of
accurate processing from users to systems. To support data exploration,
Vega-lite~\cite{satyanarayan2017vega} is a high-level declarative language that
supports users in developing interactive data visualizations without writing
functional reactive components. PlanOut~\cite{bakshy2014planout} is a DSL for
expressing and coordinating online field experiments. More niche than PlanOut,
Touchstone2 provides the Touchstone Language for specifying condition
randomization in experiments (e.g., Latin
Squares)~\cite{eiselmayer2019touchstone2}.%Experimental design is also an
essential aspect of the domain knowledge users encode in Tea programs. To
support rapid data cleaning,  Wrangler~\cite{kandel2011wrangler} combines a
mixed-initiative interface with a declarative transformation language. Tea can
be integrated with tools such as Wrangler that produce cleaned CSV files ready
for analysis.

In comparison to these previous DSLs, Tea provides a language to support another crucial step in the data life cycle: statistical analysis. 

%As a declarative language, Tea has a similar goal for statistical analysis. Tea users do not write any code that performs statistical procedures. They instead focuses on expressing their experimental designs, assumptions, and hypotheses with variables in their data. 

%Tea is a Python-based domain specific language designed to integrate into existing workflows that involve programming. Tea also captures users' domain knowledge, and analyses are focused on hypotheses. Tea currently provides a wider range of statistical tests than Statsplorer and can be extended to support emerging statistical practices. 
\vspace{-13pt}
\subsection{Constraint-based Systems in HCI}
%\chasins{is this subsection really necessary??}
Languages provide semantic structure and meaning that can be reasoned about
automatically. For domains with well defined goals, constraint solvers can be a
promising technique. Some of the previous constraint-based systems in HCI have
been Draco~\cite{moritz2019formalizing} and SetCoLa~\cite{hoffswell2018setcola},
which formalize visualization constraints for graphs. Whereas SetCoLa is
specifically focused on graph layout, Draco formalizes visualization best
practices as logical constraints to synthesize new visualizations. The knowledge
base can grow and support new design recommendations with
additional constraints.

%Tea is more similar to Draco. Tea codifies tests and their preconditions as constraints. Tea aims to provide an architecture that supports the growth of a statistical analysis knowledge base as communities adopt new statistical best practices and methods. 

Another constraint-based system is Scout~\cite{swearngin2018scout}, a mixed-initiative system that supports
interface designers in rapid prototyping. Designers specify high-level 
constraints based on design concepts (e.g., a profile picture should be more
emphasized than the name), and Scout synthesizes novel interfaces. Scout also uses
Z3's theories of booleans and integer linear arithmetic. %Tea currently uses booleans but could leverage integer arithmetic to increase the expressivity of constraints and statistical tests. 

We extend this prior work by providing the first constraint-based system for statistical analysis. 


\section{Statistical scope}
Tea is designed for statistical tests common to Null Hypothesis Significance
Testing (NHST). While there are calls to incorporate other methods of
statistical analysis~\cite{kay2016researcher,kaptein2012rethinking}, Null
Hypothesis Significance Testing (NHST) remains the norm in HCI and other
disciplines. Therefore, Tea currently implements a module for NHST with the
tests found to be most common by~\cite{wacharamanotham2015statsplorer}. In
particular, Tea supports four classes of tests: correlation (parametric: \r,
\pb; non-parametric: \ktau, \srho), bivariate mean comparison (parametric:
\student, \paired; non-parametric: \mannu, \wilcox, \welch), multivariate mean
comparison (parametric: \f, \rm, \facANOVA, \twoANOVA; non-parametric: \kw,
\friedman), and comparison of proportions (\chiSq, \fisher). Tea also supports
an implementation of bootstrapping~\cite{efron1992bootstrap}.
