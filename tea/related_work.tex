\section{Background and Related work} \label{sec:relatedWorkTea}
% Tea extends prior work on domain-specific languages for the data life cycle and
% constraint-based approaches in HCI. 
Domain-specific languages encapsulate key, routine ideas of domain (e.g.,
statistical analysis), making programs more concise to write for end-users,
providing interfaces to connect with other DSLs and systems, and shift the
burden of accurate processing from users to systems through specialized
reasoning. In the context of the data lifecycle, researchers have developed DSLs
that focus on supporting various stages of data exploration, experiment design,
and data cleaning. To support data exploration,
Vega-Lite~\cite{satyanarayan2017vega} is a high-level declarative language that
supports users in developing interactive data visualizations without writing
functional reactive components. PlanOut~\cite{bakshy2014planout} is a DSL for
expressing and coordinating online field experiments. More niche than PlanOut,
Touchstone2 provides the Touchstone Language for specifying condition
randomization in experiments (e.g., Latin
Squares)~\cite{eiselmayer2019touchstone2}.%Experimental design is also an essential aspect of the domain knowledge users encode in Tea programs. 
To support rapid data cleaning,  Wrangler~\cite{kandel2011wrangler} combines a
mixed-initiative interface with a declarative transformation language. Tea
provides a language to support another crucial step in the data life cycle:
statistical analysis. Tea can be integrated into data analysis workflows and
work in tandem with tools such as Wrangler that produce cleaned CSV files ready
for analysis.
%As a declarative language, Tea has a similar goal for statistical analysis. Tea users do not write any code that performs statistical procedures. They instead focuses on expressing their experimental designs, assumptions, and hypotheses with variables in their data. 

Furthermore, languages provide semantic structure and meaning that can be
reasoned about automatically. For domains with well defined goals, constraint
solvers can be a promising technique. Some of the previous constraint-based
systems in HCI have been Scout~\cite{swearngin2018scout}, a mixed-initiative
system for rapidly prototyping interface designs. Designers specify high-level
constraints based on design concepts (e.g., a profile picture should be more
emphasized than the name), and Scout synthesizes novel interfaces. Scout also
uses Z3's theories of booleans and integer linear arithmetic. More specific to
the data lifecycle are Draco~\cite{moritz2019formalizing} and
SetCoLa~\cite{hoffswell2018setcola}, which formalize visualization constraints
for graphs. Whereas SetCoLa is specifically focused on graph layout, Draco
formalizes visualization best practices as logical constraints to synthesize new
visualizations. The knowledge base can grow and support new design
recommendations with additional constraints. Similarly, Tea codifies tests and
their preconditions as constraints in a knowledge base. Tea aims to provide an
architecture that supports the growth of a statistical analysis knowledge base
as communities adopt new statistical best practices and methods. To our
knowledge, Tea is the first constraint-based system for statistical analysis. 

 %Tea currently uses booleans but could leverage integer arithmetic to increase the expressivity of constraints and statistical tests. 

\addcontentsline{toc}{subsection}{Statistical scope}
\section*{Statistical scope}
Tea is designed for statistical tests common to Null Hypothesis Significance
Testing (NHST). While there are calls to incorporate other methods of
statistical analysis~\cite{kay2016researcher,kaptein2012rethinking}, Null
Hypothesis Significance Testing (NHST) remains the norm in HCI and other
disciplines. Therefore, Tea currently implements a module for NHST with the
tests found to be most common by Wacharamanotham et
al.~\cite{wacharamanotham2015statsplorer}. In particular, Tea supports four
classes of tests: correlation (parametric: \r, \pb; non-parametric: \ktau,
\srho), bivariate mean comparison (parametric: \student, \paired;
non-parametric: \mannu, \wilcox, \welch), multivariate mean comparison
(parametric: \f, \rm, \facANOVA, \twoANOVA; non-parametric: \kw, \friedman), and
comparison of proportions (\chiSq, \fisher). Tea also supports an implementation
of bootstrapping~\cite{efron1992bootstrap}.