% \vspace{-10pt}
\section{Usage Scenario}\label{sec:usagescenario}
\figureTeaProgram
This section describes how an analyst who has no statistical
background can use Tea to answer their research questions. We use as
an example analyst a historical criminologist who wants to determine
how imprisonment differed across regions of the US in
1960\footnote{The example is taken
  from Ehrlich~\cite{ehrlich1973participation}
  and Vandaele~\cite{vandaele1987participation}. The data set comes as part of
  the MASS package in R.}. Figure~\ref{fig:tea_program} shows the Tea
code for this example.

The analyst specifies the data file's path in Tea. Tea handles loading and
storing the data set for the duration of the analysis session. The analyst does
not have to worry about reformatting the data during the analysis process in any way.

The analyst asks if the probability of imprisonment was higher in
southern states than in non-southern states. The analyst identifies
two variables that could help them answer this question: the
probability of imprisonment (`Prob') and geographic location
(`So'). %The latter has already been coded as `1' for southern and `0' for non-southern.
Using Tea, the analyst defines the geographic
location as a dichotomous nominal variable where `1' indicates a
southern state and `0' indicates a non-southern state, and indicates that the
probability of imprisonment is a numeric data type (ratio) with a
range between 0 and 1. %The analyst can additionally identify any
%other variables they care about in the data set.

The analyst then specifies their study design, defining the study type
to be ``observational study'' (rather than ``experimental study'') and
defining the contributor (independent) variable to be the geographic location and
the outcome (dependent) variable to be the probability of
imprisonment.

Based on their prior research, the analyst knows that the probability
of imprisonment in southern and non-southern states is normally
distributed. The analyst provides an assumptions clause to Tea in
which they specify this domain knowledge. They also specify an
acceptable Type I error rate (probability of finding a false positive
result), more colloquially known as the `significance threshold'
($\alpha = .05$) that is acceptable in criminology. If the analyst
does not have assumptions or forgets to provide assumptions, Tea will
use the default of $\alpha = .05$.

% Should we switch the story where hypothesis turns out to be incorrect and alt
% metrics give evidence of that?
The analyst hypothesizes that southern states will have a higher
probability of imprisonment than non-southern states. The analyst
directly expresses this hypothesis in Tea.  \emph{Note that at no
  point does the analyst indicate which statistical tests should be
  performed.}

From this point on, Tea operates entirely automatically.  When the
analyst runs their Tea program, Tea checks properties of the data and
finds that the Student's t-test is appropriate. Tea executes the Student's
t-test and non-parametric alternatives, such as the Mann-Whitney U
test, which provide alternative, consistent results.

Tea generates a table of results from executing the tests, ordered by their
power (i.e., results from the parametric t-test will be listed first
given that it has higher power than the non-parametric
equivalent). Based on this output, the analyst concludes that their
hypothesis---that the probability of imprisonment was higher in
southern states than in non-southern states in 1960---is
supported. The results from alternative statistical tests support this
conclusion, so the analyst can be confident in their assessment.

%The analyst wants to conduct the same analysis with a data set from a different
%year. In Tea, they have to change only one line of code: the file path. As long as the variables exist in the new data set, Tea can conduct the same analysis without altering the rest of the code.\chasins{is this very different from prior approaches?  say python?  does it warrant this much emphasis?}

The analyst can now share their Tea program with colleagues.  Other
researchers can easily see what assumptions the analyst made and what
the intended hypothesis was (since these are explicitly stated in the
Tea program), and reproduce the exact results using Tea.

% If the analyst wanted to do the same analysis for multiple years,
% the analyst could use an outer Python loop to execute the same analysis for all data
% sets they had.

%Tea enables users who may not have statistical expertise to conduct valid, replicable analyses without having to write statistical functions \chasins{clarify}. Instead, users focus on expressing knowledge about their data source, variables of interest, assumptions, and hypotheses. Tea automates test selection, test precondition checking, and test execution, and it surfaces multiple valid statistical tests for testing a given hypothesis. Tea analyses can also be shared and re-run.
