\def\r{Pearson's r\xspace}
\def\ktau{Kendall's $\tau$\xspace}
\def\srho{Spearman's $\rho$\xspace}
\def\pb{Pointbiserial\xspace}
\def\student{Student's t-test\xspace}
\def\paired{Paired t-test\xspace}
\def\mannu{Mann-Whitney U\xspace}
\def\wilcox{Wilcoxon signed rank\xspace}
\def\welch{Welch's t-test\xspace}
\def\f{F-test\xspace}
\def\rm{Repeated measures one way ANOVA\xspace}
\def\kw{Kruskal Wallis\xspace}
\def\friedman{Friedman\xspace}
\def\facANOVA{Factorial ANOVA\xspace}
\def\twoANOVA{Two-way ANOVA\xspace}
\def\chiSq{Chi Square\xspace}
\def\fisher{Fisher's Exact\xspace}
\def\boot{Bootstrap\xspace}

\begin{table*}[htbp]
    \begin{center}
    \caption{\textbf{Results of applying Tea to 12 textbook tutorials.}\label{tab:results}}
\begin{minipage}{\linewidth}
\vspace*{-12pt}
\footnotesize{Tea is comparable to an expert selecting statistical tests. Tea can
prevent false positive and false negative results by suggesting only tests that
satisfy all assumptions. \textit{Tutorial} gives the test described in the
textbook; \textit{Candidate tests (p-value)} gives all tests a user could run on
the provided data with corresponding p-values; \textit{Assumptions} gives all
satisfied (lightly shaded) and violated (white)
assumptions; \textit{Tea suggests} indicates which tests Tea suggests based on
their preconditions (assumptions about the data). \textbf{Emphasized} p-values indicate instances where a
candidate test leads to a wrong conclusion about statistical significance.
Although this table focuses on p-values, Tea produces richer output that
provides a more holistic view of the statistical analysis results by including
effect sizes, for instance. Refer to Figure~\ref{fig:modes} for an
example of output from a Tea program.}
\end{minipage}
    \begin{tabularx}{\textwidth}{p{0.2\textwidth}p{0.4\textwidth}lc}
    \toprule
    \colH{Tutorial} & \colH{Candidate tests (p-value)} & \colH{Assumptions*} & \colH{Tea suggests} \\
    \midrule
      Pearson                   & \r    \hfill (6.96925e-06) & \valid{2} \valid{4} \invalid{5} & \no \\
      \cite{kabacoff2011action} & \ktau \hfill (2.04198e-05) & \valid{2} \valid{4}             & \yes \\
                                & \srho \hfill (2.83575e-05) & \valid{2} \valid{4}             & \yes \\
    \midrule
    \srho                       & \srho \hfill (.00172)  & \valid{2} \valid{4}           & \yes \\
    \cite{field2012discoveringR} & \r    \hfill (.01115)    &   \valid{2} \invalid{4}                   & \no \\ 
                                & \ktau \hfill (.00126) & \valid{2} \valid{4} & \yes \\
    \midrule
    \ktau                       & \ktau \hfill (.00126) & \valid{2} \valid{4}             & \yes \\
    \cite{field2012discoveringR} & \r   \hfill (.01115) & \valid{2} \invalid{4}             & \no \\
                                & \srho \hfill (.00172) & \valid{2} \valid{4}             & \yes \\
    \midrule
    \pb                         & \pb (\r)  \hfill (.00287) & \valid{2} \valid{4} \invalid{5}             & \no \\
    \cite{field2012discoveringR} & \srho \hfill (.00477) & \valid{2} \invalid{4}             & \no \\
                                 & \ktau \hfill (.00574) & \valid{2} \invalid{4}             & \no \\
                                 &\boot     \hfill (<0.05)                 &           & \yes \\
    \midrule
    \student                     & \student \hfill (.00012) & \valid{2} \valid{4} \valid{5} \valid{6} \valid{7} \valid{8} & \yes \\
    % \cite{kabacoff2011action}    & \paired  \hfill (N/A)                    & \valid{2} & \no \\
    \cite{kabacoff2011action}    & \mannu   \hfill (9.27319e-05)  & \valid{2} \valid{4} \valid{7} \valid{8} & \yes \\
                                %  & \wilcox  \hfill (N/A)                    & \valid{2} & \no \\
                                 & \welch   \hfill (.00065)  & \valid{2} \valid{4} \valid{5} \valid{7} \valid{8} & \yes \\
    \midrule
    \paired                      & \paired \hfill (.03098)    & \valid{2} \valid{4} \valid{5} \valid{7} \valid{8} & \yes \\
    \cite{field2012discoveringR} & \student \hfill (\textbf{.10684})    & \valid{2} \valid{4} \valid{5} \invalid{7} & \no \\
                                 & \mannu   \hfill (\textbf{.06861})    & \valid{2} \valid{4} \invalid{7} & \no \\
                                 & \wilcox  \hfill (.04586)   & \valid{2} \valid{4} \valid{7} \valid{8} & \yes \\
                                 & \welch   \hfill (\textbf{.10724})    & \valid{2} \invalid{7} & \no \\
    \midrule
    \wilcox                      & \wilcox  \hfill (.04657)    & \valid{2} \valid{4} \valid{7} \valid{8} & \yes \\
    \cite{field2012discoveringR} & \student \hfill (.02690)   & \valid{2} \valid{4} \invalid{7} & \no \\
                                 & \paired  \hfill (.01488)   & \valid{2} \valid{4} \invalid{5} \valid{7} \valid{8} & \no \\
                                 & \mannu   \hfill (.00560)   & \valid{2} \valid{4} \invalid{7} & \no \\
                                 & \welch   \hfill (.03572)   & \valid{2} \valid{4} \invalid{7} & \no \\
    \midrule 
    \f                            & \f      \hfill (9.81852e-13)            & \valid{2} \valid{4} \valid{5} \valid{6} \valid{9} & \yes \\
    \cite{field2012discoveringR}  & \kw     \hfill (2.23813e-07)  & \valid{2} \valid{4} \valid{9} & \yes \\ 
                                  & \friedman \hfill (8.66714e-07) & \valid{2} \invalid{7} & \no \\
                                %   & \rm     \hfill ()           & \valid{2}  & \no \\
                                  & \facANOVA \hfill (9.81852e-13)          & \valid{2} \valid{4} \valid{5} \valid{6} \valid{9} & \yes \\
    \midrule
    \kw                           & \kw      \hfill (.03419)    & \valid{2} \valid{4} \valid{9} & \yes \\
    \cite{field2012discoveringR}  & \f       \hfill (\textbf{.05578})              & \valid{2} \valid{4} \invalid{5} \valid{9} & \no \\
                                  & \friedman \hfill (3.02610e-08) & \valid{2} \invalid{7} & \no \\
                                %   & \rm     \hfill ()           & \valid{2} & \no \\
                                  & \facANOVA \hfill (\textbf{.05578})             & \valid{2} \valid{4} \invalid{5} \valid{9} & \no \\
    \midrule
    \rm                           & \rm     \hfill (.0000)           & \valid{2} \valid{4} \valid{5} \valid{6} \valid{7} \valid{9} & \yes \\
    \cite{field2012discoveringR}  & \kw     \hfill (4.51825e-06)  & \valid{2} \valid{4} \invalid{7} \valid{9} & \no \\
                                  & \f      \hfill (1.24278e-07)           & \valid{2} \valid{4} \valid{5} \valid{6} \invalid{7} \valid{9} & \no \\
                                  & \friedman \hfill (5.23589e-11) & \valid{2} \valid{4} \valid{7} \valid{9} & \yes \\ % friedman says 3+ groups
                                  & \facANOVA \hfill (1.24278e-07)          & \valid{2} \valid{4} \valid{5} \valid{6} \valid{9} & \yes \\
    \midrule
    \twoANOVA                     &\twoANOVA \hfill (3.70282e-17)           & \valid{2} \valid{4} \invalid{5} \valid{9} & \no \\
    \cite{field2012discoveringR}  &\boot     \hfill (<0.05)                 &           & \yes \\
    \midrule
    \chiSq                        & \chiSq \hfill (4.76743e-07) & \valid{2} \valid{4} \valid{9} & \yes \\
    \cite{field2012discoveringR}  & \fisher \hfill (4.76743e-07) & \valid{2} \valid{4} \valid{9} & \yes \\
    \bottomrule
    \multicolumn{4}{p{\linewidth}}{*\small{\assume{1} one variable,
                       \assume{2} two variables,
                       \assume{3} two or more variables,
                       \assume{4} continuous vs. categorical vs. ordinal data,
                       \assume{5} normality,
                       \assume{6} equal variance,
                       \assume{7} dependent vs. independent observations,
                       \assume{8} exactly two groups,
                       \assume{9} two or more groups}}

    \end{tabularx}
    \end{center}
\end{table*}

\vspace{-5pt}
\subsubsection{Initial Evaluation} \label{sec:eval}
We assessed the validity of Tea in two ways. First, we compared Tea's
suggestions of statistical tests to suggestions in textbook tutorials.
We use these tutorials as a proxy for expert test selection.
Second, for each tutorial, we compared the analysis results of the test(s)
suggested by Tea to those of the test suggested in the textbook as well as all
other candidate tests. We use the set of all candidate tests as as a proxy for
non-expert test selection.

We differentiate between \textit{candidate} and \textit{valid} tests. A candidate test can be
computed on the data, when ignoring any preconditions regarding the data types or
distributions. A valid test is a candidate test for which all preconditions are
satisfied.

\vspace{-3pt}
\subsection{How does Tea compare to textbook tutorials?}
Our goal was to compare Tea to expert recommendations.

We sampled 12 data sets and examples from R tutorials (\cite{kabacoff2011action}
and~\cite{field2012discoveringR}). These included eight parametric tests, four
non-parametric tests, and one Chi-square test. We chose these tutorials because they
appeared in two of the top 20 statistical textbooks on Amazon and had publicly available
data sets, which did not require extensive data wrangling.
% and examples that tested the 16 statistical tests that Tea currently supports
%(see~\ref{tab:tea_tests}).

We translated all analyses into Tea and encoded any assumptions explicitly
stated in the tutorial. Tea selected tests based only on the data and the
assumptions expressed in the Tea program. Where Tea disagreed with the
tutorials, either (1) the tutorial authors chose the wrong analyses or (2) the tutorial authors
had implicit assumptions about the data that did not hold up to statistical testing. 

For nine out of the 12 tutorials, Tea suggested the same statistical test (see
Table~\ref{tab:results}). For three out of 12 tutorials, which used a parametric
test, Tea suggested using a non-parametric alternative instead. Tea's
recommendation of using a non-parametric test instead of a parametric one did
not change the statistical significance of the result at the $.05$ level. Tea
suggested non-parametric tests based on the Shapiro-Wilk test for normality. It
is possible that tutorial authors visualized the data to make implicit
assumptions about the data, but this practice or conclusion was not made
explicit in the tutorials.

For the two-way ANOVA tutorial from~\cite{field2012discoveringR}, which studied how gender
and drug usage of individuals affected their perception of attractiveness, a
precondition of the two-way ANOVA is that the dependent measure is normally
distributed in each category. This precondition was violated.  As a result, Tea
defaulted to bootstrapping the means for each group and reported the means and
confidence intervals. 
For the pointbiserial correlation tutorial from~\cite{field2012discoveringR},
Tea also defaulted to bootstrap for two reasons. First, the precondition of
normality is violated. Second, the data uses a dichotomous (nominal) variable,
which invalidates \srho and \ktau.

Tea generally agrees with expert recommendations and is more conservative
in the presence of non-normal data, minimizing the risk of false positive
findings.

\subsubsection{Does Tea avoid common mistakes made by non-expert users?}
Our goal was to assess whether any of the tests suggested by Tea (i.e., valid
candidate tests) or any of the invalid candidate tests would lead to a different
conclusion than the one drawn in the tutorial. Table~\ref{tab:results} shows the
results. Specifically, emphasized p-values indicate instances for which the
result of a test differs from the tutorial in terms of statistical significance
at the $.05$ level.

For all of the 12 tutorials, Tea's suggested tests led to the same conclusion
about statistical significance. For two out of the 12 tutorials, two or more
candidate tests led to a different conclusion. These candidate tests were
invalid due to violations of independence or normality.

To summarize, the evaluation shows us that (i) Tea can replicate and even improve
upon expert choices and (ii) Tea could help novices avoid common mistakes and
false conclusions.