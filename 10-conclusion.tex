% Reflection: As progress through PhD research, got and grappled with how to get
% closer to users and to statistical theory in tandem.

While statistical analysis has become more pervasive among end-users who are not
statistical experts, the tools for conducting analyses have continued to require
high statistical expertise. This dissertation examines how to design and develop
tools that not only lower the barriers for statistical non-experts but also
provide guarantees about the validity of authored analyses. We introduce two new
systems, \tea and \tisane. Both provide DSLs for expressing implicit conceptual
knowledge and then compile these high-level specifications into statistical
analyses, Null Hypothesis Significance Tests in \tea (\autoref{chapter:tea}) and
generalized linear models with or without mixed effects in \tisane
(\autoref{chapter:tisane}). 
% Analysts express their implicit knowledge about
% their domain and data--as assumptions and hypotheses in Tea and conceptual
% models in Tisane--and the DSLs compile them into statistical analysis code. 
Additionally, we develop a theory of hypothesis formalization that describes the
cognitive and operational steps involved in translating a conceptual research
question into a statistical analysis implementation in code. Our theory of
hypothesis formalization retrospectively validated our design in \tea and
directly inspired the design of \tisane. 

\polish{Relate this work to chasm/bridge that Amelia identified in her work}
Existing statistical analysis tools are either designed for students learning
basic statistics or statistical experts~\cite{mcnamara2015bridging}.
Tools do
not support statistical non-experts, such as researchers, through the process of
authoring accurate statistical models. 

\section{Discussion} \label{sec:discussionChallenges} 

This dissertation addresses three challenges central to the thesis that (i)
programming abstractions focused on capturing analysts' implicit conceptual
knowledge and (ii) formal representations and reasoning to determine statistical
analyses benefit statistical non-experts. We discuss each challenge and how the
projects in this dissertation address them below. 
% In developing programming abstractions, formal representations, and automated
% reasoning approaches for authoring statistical analyses in \tea and \tisane 

\subsection{Challenge 1: Designing the \textit{right} level of abstraction} 
With any programming language, end-users must learn and use a formalism. \tea
and \tisane provide high-level abstractions but the key to their design is that
they abstract the appropriate \textit{conceptual concerns} implicitly involved
in statistical analyses. In fact, the fact that an abstraction is high or low is
less relevant. Indeed, a key insight that guided our design of \rTisane
(\autoref{sec:rTisane}) was that analysts wanted low- and high-level conceptual
abstractions to express their domain knowledge with varying degrees of detail
that felt helpful and accurate to them (see~\autoref{sec:exploratoryStudy}).

When comparing the abstractions \tea and \tisane provide, it is easy to see that
the conceptual relationships between variables were still largely implicit in
\tea. An important takeaway from the theory of \hypoForm was the importance of
conceptual models, which are present for statistical testing and modeling alike.
Therefore, conceptual models should be a central concern in designing
programming abstractions for data analysis. 

\subsection{Challenge 2: Representing and reasoning about analysis decisions}
% abstractions that are approachable and helpful to users but **amenable to rigorous, formal reasoning**
% This dissertation argues that formal representations for reasoning about
% analysis choices is just as important as designing the appropriate programming
% abstractions. 
The abstractions that may be usable to statistical non-experts may not be
precise enough for formal reasoning (\autoref{sec:exploratoryStudy}). Therefore,
a key challenge in designing representations amenable to reasoning is in finding
a ``shared representation'' between analysts and computational techniques. Based
on \tea's key insight that statistical test selection can be reformulated as a
constraint satisfaction problem, we represented statistical tests using logical
constraints in a knowledge base. Using \tea's DSL, analysts specify additional
constraints about their hypothesis and data, which helps \tea's runtime system
solve a system of constraints to identify valid statistical tests. In \tisane,
the shared representation is the conceptual model, which \tisane represents as a
graph. This representation made reasoning about linear model formulations
straightforward by applying causal reasoning techniques on a part of the graph. 

In designing these shared representations, a temptation was to fit the DSL on
top of a reasoning approach that was straightforward. In this view, the DSL
would be a thin wrapper around the automated reasoning engine. For example, a
very early prototype of \tisane used Answer Set Programming (ASP) to define when
specific confounders should appear in a generalized linear model. In addition to
being a clunky way to represent linear model formulation rules when the
statistics community has converged on using graphs, this prototype required
analysts to incrementally refine their statistical models by interacting with
the UNSAT core. This interaction model, though interesting, did not allow us to
discovery and fully realize the real benefit of expressing conceptual models:
giving analysts an opportunity to reflect on their assumptions in an open-ended
way. 

\subsection{Challenge 3: Interaction as reflection}
As we saw in the case studies with \tisane, providing abstractions and
interactions with shared representations for formal reasoning increases
analysts' awareness of their implicit assumptions, data, and analysis practices.
By providing the appropriate abstractions, DSLs can make the specification
process a useful form of documentation. This may later be useful for sharing and
inspection. For instance, by stating their implicit conceptual and data
assumptions in \tea and \tisane, researchers can help improve scientific
replicability and reproducibility. 
% Through their involvement in interactive compilation, 
% reify the connection between the conceptual and
% statistical in our software tools. In this way, this dissertation brings to the
% domain of data analysis, classic principles from end-user software engineering

\begin{comment}
\section{Recent developments} \label{sec:recentDevelopments}
Mention: in LLMs impact the contributions of this dissertation, exciting opportunities to leverage them to realize the goals of this work

\subsection{What about in the face of LLMs?}

But how do people express their domain knowledge, make the process meaningful

Mention LLMs as a technology to use here?
\end{comment}

\section{Limitations and Future work} \label{sec:futureWork}
\todo{Add ideas from defense}

This dissertation scrutinizes how statistical non-experts author statistical
analyses. We elaborate one some of the limitations of this work and
opportunities for future research. 

\paragraph{Address additional aspects of validity}
This thesis focused on internal, external, and statistical conclusion validity. However, could reason about construct validity with LLMs.

Campbell's theory of validity -- encompassing statistical conclusion, internal,
external, and construct
validity~\cite{campbell2015quasiexperimentalDesigns,cook2002generalizedCausalInference}
-- has influenced disciplines widely (e.g.,~\cite{shadish2010campbell}),
including epidemiology (e.g.,~\cite{matthay2020causalDAGEpi}), software
engineering (e.g.,~\cite{neto2013conceptual}), and psychology
(e.g.,~\cite{campbell2015quasiexperimentalDesigns}). Viewed through the
Campbellian framework, Tisane helps analysts avoid four common threats to
statistical conclusion and external validity: (i) violation of statistical
method assumptions, (ii) fishing for statistical results, (iii) not accounting
for the influence of specific units, and (iv) overlooking the influence of data
collection procedures on outcomes~\cite{cook2002generalizedCausalInference}.

Most notable is construct validity -- which LLMs could help with? 

* Pull on thread that the specification process can be a meaningful one - finding in summative evaluation*


Correct analyses are more important than ever -- Mention stanford university prof?

Tisane fills a need to align analysts' conceptual models with the statistical
models they want to implement but find difficult to express with the current
tools available. By integrating conceptual, data, and statistical concerns,
Tisane facilitates the hypothesis
formalization~\cite{jun2021hypothesisFormalization} process, which can be an
error-prone and cognitively demanding process that existing tools do not yet support.

\subsection{Explicit support for statistical model revision and iteration}
In the future, we plan to develop additional strategies for enhancing the
validity of analyses authored with Tisane. As discussed in
Section~\ref{sec:disambiguation}, our current approach to family and link
functions is only an initial step. We look forward to developing and
comparing multiple strategies for scaffolding the family and link function
selection and revision process. For example, what if the Tisane GUI allowed
analysts to fit multiple models that varied in their family and link functions,
plotted each model's residuals against the predicted values, and gave analysts
visual guides for comparing models? To avoid false discovery rate inflation,
Tisane could partition analysts' data, fit models to only a subset, and output a
script for fitting a selected model using another subset. Although possible for
large datasets, this strategy would encounter limited statistical power for smaller
datasets. Alternatively, what if Tisane calculated Bayes factors for variables
in the
models~\cite{raftery1996approximate,gelfand1994bayesian,czado2006choosing} after
analysts tried multiple family and link combinations? Carefully balancing
statistical rigor and usefulness to domain researchers who may be statistical
non-experts deserves careful consideration.

We
look forward to extending Tisane to support model iteration, which presents two
technical challenges: (i) recognizing when conceptual revisions are necessary
and (ii) identifying and suggesting model changes that maintain conceptual
validity or, at the very least, quantify conceptual shifts. Furthermore, in the model revision process, analysts may consider
multiple alternatives. As R3 described, he preferred to run multiple variations
of a model and compare them, a workflow akin to a multiverse
analysis~\cite{steegen2016increasing}. Given that Tisane already generates a
combinatorial space of candidate statistical models, Tisane could generate a
multiverse script for Boba~\cite{liu2020boba} instead. A multiverse could help
check the robustness of findings, and Boba's visual analyzer could help analysts
further develop an understanding of their data and modeling choices. A
multiverse may also help analysts explore and compare family and link
combinations as well.

Tisane is one tool designed to enable analysts with limited statistical expertise to
author valid statistical models. Tisane enables future possibilities and raises
open research questions for creating an ecosystem of analysis tools that align
tool interfaces with analysts' conceptual goals.

--> Could be first step towards supporting more phases of the lifecycle?

\subsection{Support more phases of the data lifecycle}
This dissertation emphasizes the need for abstractions that capture analysts'
implicit domain knowledge. These abstractions enable valid analysis formulation
and promote reflective thinking. Building upon this, we can begin to ask how the
same ideas---abstractions and automated reasoning for conceptual knowledge often
implicit in statistical analyses---could apply to other phases of the data
lifecycle. Future work should explore how to elicit and track the evolution of
conceptual knowledge even before statistical analysis by developing new
elicitation techniques and representations of domain knowledge and ecosystem of
inter-operating tools to track and ensure validity throughout the data lifecycle. 

% How can software tools track and trace the
% meaningful evolution of conceptual knowledge? By tackling these questions, we
% can revolutionize data analysis and develop comprehensive tools that seamlessly
% integrate domain expertise.

\subsection{Support interpretation of statistical results}
While Tisane effectively addresses the gulf of execution by compiling conceptual
models into statistical models, it falls short of bridging the gulf of
evaluation. Tisane does not yet provide support for analysts to interpret the
results of their statistical models. Future research should focus on two related
challenges: (i) improving statistical reporting to enhance the understanding of
results and (ii) providing support for navigating the consequences of the
results, such as updates to analysts' conceptual models or the need to revise
statistical models.

Tisane's focus is on authoring GLMs/GLMMs, but accurate interpretation is also necessary. For instance, analysts may need
help interpreting what their statistical models and results mean in relation to
their input conceptual models. Do the results suggest their conceptual model is
correct? What kind of inferences should they make? Future work should address
these interpretation challenges, which may require eliciting hypotheses and expected results from analysts.}

Although researchers in our pilot or case studies did not presume
Tisane helped with formal causal analysis, the ability to express causal
relationships (\texttt{causes}) may lead some analysts to erroneously assume
that their models assess causality. Changing the name of the language construct
and/or building out support to interpret GLM/GLMM results may resolve this concern. One way to support accurate interpretation and reporting could be to
output a figure representing the input conceptual model along with visual
summaries of the data and/or statistical model for direct inclusion in publications. 
Tisane could also allow analysts to annotate their disambiguation decisions with
their own rationale and provide a richer log of selections than currently
supported. Tisane could even accept these augmented logs to save the state of the GUI in between analysis sessions.

\subsection{Connect statistical modeling and testing} 
A natural question arising from this dissertation is the choice between Tea and
Tisane for analysts. Tea focuses on statistical testing, determining evidence
for or against a specific claim, while Tisane emphasizes statistical modeling,
estimating variable influences in the presence of other variables (e.g.,
confounders, mediators, etc.) However, statistical testing and modeling are not
mutually exclusive. Analysts often want to conduct tests after building models
to answer substantive questions as well as assess model fit. A compelling future
direction is to enable analysts to ask follow-up questions about specific
estimates and effects from a statistical model. 

\subsection{Develop a grammar of study design}
\polish{Mention syntactic sugar from exploratory study}
There are relationships between Tisane's language constructs for specifying data
collection details, data schema specification, and experimental design. How
could we draw these connections out and formalize them? 

Tisane's graph IR is an entity-relationship (ER) model~\cite{chen1976ERDiagram}.
ER models, or diagrams, are used to describe data schema. ER models describe how
entities relate to other entities and attributes. In Tisane, a variable of Unit
type can be viewed as an entity. The \texttt{nests\_within} edge describes how
two units, or entities, relate to one another. Tisane's graph IR also relates
units (entities) to measures (attributes).

\subsection{Specialized domain-specific language constructs?}
\polish{Mention domain-specific language constructs?}


\subsection{Promote analytical best practices in science}
In the long term...
Tea and Tisane primarily follow a top-down authoring approach, where analysts
start with a research question and hypothesis. However, as observed in our lab
study to develop hypothesis formalization (\autoref{sec:labStudyHypoForm}),
analysts often develop and refine hypotheses based on statistical results.
Therefore, a future research direction would be to develop ways to incorporate
both data-driven and research question-driven approaches to model authoring and
refinement that do not promote cherry-picking. One possibility is to leverage
\tea's and \tisane's reasoning capabilities to reason in multiple other
directions, from statistical models and data to all possible conceptual models
or statistical models to data invariants that could inform study designs.

Moreover, one of the precautions integrated into the design of Tisane was to prevent
cherry-picking and p-hacking by using analysts' conceptual models to drive
statistical model formulation. While Tisane supports mapping one conceptual
model to a statistical model, an under-explored direction is to assess the
robustness of effects across multiple possible conceptual models, especially in
cases of ambiguity or debate in a discipline. On the other hand, multiverse
analyses~\cite{} embrace conceptual model uncertainty by considering all
possible statistical model formulations. Future research should look into how to
both be consistent with aspects of conceptual models researchers know and assess
evidence for other competing aspects of conceptual models.

Motivation: Recent news about Stanford University president

\begin{comment}
\subsection{Improving data science education}
1 sentence: Motivation: From conversations with students and my own experience taking statistics courses
throughout undergraduate and graduate education is that the connection between
statistical methods and the kinds of questions I want to ask is often unclear.
Furthermore, students feel they need to memorize a bunch of different methods,
at the cost of thinking through what their substantive questions are and what
they should care about in a statistical approach. 

2: Goal, Impact: Promote statistical thinking, an essential component of practicing data science. 

Especially, greatest potential: Especially to teach students to separate
multiple sets of concerns, specifically the conceptual from the statsitical 

3. Role of systems: Use \tea and \tisane to focus students to focus first on identifying and
articulating their motivation and intents for analysis 

Before introducing and teaching statistical details

4. Future work: Deploy \tea and \tisane 
- questions to answer: impact on statistical thinking, computational thinking 
- what additional tools are necessary to develop? 

A ramp from novice to expert
tools is missing in the current ecosystem~\cite{mcnamara2018keyAttributes}. Tea
and Tisane lay the foundation for a bridge between novice and expert tools by
providing abstractions that match those of statistical non-experts while also
giving experts control, flexibility, and compatibility with other expert tools
in Python and R. While pursuing the above research agenda, I look forward to
directly improving data science courses I teach by deploying my systems in the
classroom, discovering students' needs, and iterating on tools and curriculum.
In pursuing this goal, I want to ease
transitions between analysis paradigms (e.g., NHST and Bayesian inference). One
promising direction is to separate concerns about model specification from
interpretation and assessment. 


Get rid of?: In a small way, this separation also teaches students how to separate
specification from implementation, an essential perspective in computer science.
So that if students need to implement more complex analyses, they have some
awareness of how to organize their computational approach. 
\end{comment}

\section{Impact}
\ej{Include details of where this code is available, open source, etc.}
The most rewarding part of conducting the work in this dissertation has been to
see real-world use cases and adoption of tools. As of May 2023, \tea has been
downloaded 15K times, and the first release of \tisane has been downloaded 12K
times. Over the last few years, I have also enjoyed reading and answering a
flurry of emails where analysts, including scientists and social scientists,
share anecdotes of how they have used (and sometimes failed to use) \tea and
\tisane.

\ej{Add IHME anecdote/story/example?}

To more systematically capture and act on these kinds of anecdotes in the
future, I am excited to develop a web platform for \tea and \tisane, where users
can share their programs, data, and insights. Over time, I hope to collect a
gallery of examples to answer questions about challenges using the DSLs, the
learn which use cases are under-supported, and assess the practical impact of
using conceptual abstractions and automated reasoning. I would also like to see
if end-users repeatedly use \tea and \tisane or if these are one-off
engagements. I look forward to not only pursuing ideas and systems developed in
this dissertation but also fostering a community of users. 
% who can not only inform future research but also benefit from future systems. 

% Also future opportunities for deploying and testing a community of users 

\section{Closing Remarks}

designed and implemented two systems, Tea~\cite{jun2019tea} and
Tisane~\cite{jun2022tisane}, that leverage \textbf{domain-specific languages}
(DSLs) to capture analysts' implicit assumptions and conceptual knowledge. Users
\textbf{interactively compile} these high-level specifications into low-level
code. To infer valid statistical analyses, the systems \textbf{programmatically
represent and reason about core statistical authoring challenges} as constraints
and graphs.% (\autoref{fig:tools}).
As a result, my systems prevent common analysis
mistakes~\cite{jun2019tea,jun2022tisane}. 

This thesis furthers our understanding of what makes statistical analyses
difficult to author and then designs and implements two domain-specific
languages (DSLs) to addressing these issues. The DSLs illustrate that automating
aspects of statistical analysis benefits statistical non-experts. However, more
importantly, this dissertation illustrates how we can 
- designing abstractions that capture intent 
- reifying the connection between domain knowledge and statistical analysis 
- designing interfaces and interactions that increase analyst awareness of the impact of analysis, rationale. <-- change this based on rTisane eval? 
