% Reflection: As progress through PhD research, got and grappled with how to get
% closer to users and to statistical theory in tandem.

While statistical analysis has become more pervasive among analysts who are not
statistical experts, the tools for conducting analyses have continued to require
high statistical expertise. This dissertation examines how to design and develop
tools that not only lower the barriers for statistical non-experts but also
provide guarantees about the validity of authored analyses. We introduce three
new tools, \tea, \tisane, and \rTisane. All three provide DSLs for expressing
implicit conceptual knowledge. These high-level specifications are then compiled
to statistical analysis code for Null Hypothesis Significance Tests in \tea
(\autoref{chapter:tea}), generalized linear models with or without mixed effects
in \tisane (\autoref{chapter:tisane}), and generalized linear models without
mixed effects in \rTisane (\autoref{chapter:rTisane}).
% Analysts express their implicit knowledge about
% their domain and data--as assumptions and hypotheses in Tea and conceptual
% models in Tisane--and the DSLs compile them into statistical analysis code. 
Additionally, we develop a theory of hypothesis formalization that describes the
cognitive and operational steps involved in translating a conceptual research
question into a statistical analysis implementation. Our theory of hypothesis
formalization retrospectively validated our design in \tea and directly inspired
the design of \tisane and \rTisane. 

% \polish{Relate this work to chasm/bridge that Amelia identified in her work}
% Existing statistical analysis tools are either designed for students learning
% basic statistics or statistical experts~\cite{mcnamara2015bridging}.
% Tools do
% not support statistical non-experts, such as researchers, through the process of
% authoring accurate statistical models. 

\section{Discussion} \label{sec:discussionChallenges} 

% \polish{I first interpreted the two thesis points (i) and (ii) as being the "challenges" and wondered where the third bullet was before I realized I had misparsed the sentence. Consider revising for clarity.}
% This dissertation addresses three challenges central to the thesis that (i)
% programming abstractions focused on capturing analysts' implicit conceptual
% knowledge, data collection details, and analysis intents and (ii) automated
% reasoning to determine statistical analyses benefit statistical non-experts. We
% discuss each challenge and how the projects in this dissertation address them
% below. 
% In developing programming abstractions, formal representations, and automated
% reasoning approaches for authoring statistical analyses in \tea and \tisane 

This dissertation is centered on the thesis that two primary elements benefit
statistical non-experts: (i) programming abstractions designed to capture
analysts' implicit conceptual knowledge, data collection details, and analysis
intents, and (ii) the use of automated reasoning to determine statistical
analyses. Below, I discuss three key challenges related to this thesis and
how the projects in this dissertation address them.

\subsection{Challenge 1: Make implicit domain knowledge explicit} 
% What was key was Designing the \textit{right} level of abstraction
With any programming language or DSL, end-users must learn a formalism to use
it. \tea, \tisane, and \rTisane abstract the appropriate \textit{conceptual
concerns} implicitly involved in statistical analyses. The fact that an
abstraction is high or low is less relevant than if the abstraction is at the
\textit{right} level for end-users and their tasks. Indeed, a key insight that
guided our design of \rTisane (\autoref{chapter:rTisane}) was that analysts wanted
conceptual abstraction choices to express their domain knowledge
with varying degrees of detail that felt helpful and accurate to them
(see~\autoref{sec:exploratoryStudy}). %\polish{Rather "lower" and "higher" than some reference point might make more sense. Or, changing the descriptive language to avoid false or ambiguous contrasts.}

When comparing the abstractions \tea and \tisane/\rTisane provide, it is easy to see that
the conceptual relationships between variables were still largely implicit in
\tea. An important takeaway from the theory of \hypoForm was the importance of
conceptual models, which are present for statistical testing and modeling alike.
Therefore, conceptual models should be a central concern in designing
programming abstractions for data analysis. 

\subsection{Challenge 2: Represent and reason about key statistical analysis decisions}
A key tension when addressing the above challenge is balancing usability with
rigor. That is, abstractions that are usable to end-users may not be precise
enough for formal reasoning. Therefore, a key challenge in designing
representations amenable to reasoning is in finding a ``shared
representation''~\cite{heer2019agency} between analysts and computational
techniques. Based on \tea's key insight that statistical test selection can be
reformulated as constraint satisfaction, we represented statistical
tests using logical constraints in a knowledge base. Using \tea's DSL, analysts
specify additional constraints about their hypothesis and data, which helps
\tea's runtime system solve a system of constraints to identify valid
statistical tests. In \tisane and \rTisane, the shared representation is the
conceptual model, represented as a graph. This representation made
reasoning about linear model formulations straightforward by applying causal
reasoning techniques on a part of the graph. 

In designing these shared representations, a temptation was to fit the DSL on
top of a reasoning approach that was straightforward. In this view, the DSL
would be a thin wrapper around the automated reasoning engine. For example, a
very early prototype of \tisane used Answer Set Programming (ASP) to define when
specific confounders should appear in a generalized linear model. In addition to
being an inelegant way to represent linear model formulation rules when the
statistics community has converged on using graphs, this prototype required
analysts to incrementally refine their statistical models by interacting with
the UNSAT core. This form of interactivity, though interesting, obfuscated a
major benefit of expressing conceptual models and disambiguating their graph
representations: giving analysts an opportunity to reflect on their implicit
assumptions in an open-ended way.

\subsection{Challenge 3: Increase analysts' statistical understanding}
As we saw in the case studies with \tisane, providing abstractions and
interactions with shared representations for formal reasoning increases
analysts' awareness of their implicit assumptions, data, and analysis practices.
By providing the appropriate abstractions, DSLs can make the specification a
useful form of documentation and the process of specifying a reflective
activity. Both are especially important for statistical analysis.
Documentation can improve reproducibility. As we saw in the summative
evaluation (\autoref{sec:summativeEval}), reflection throughout analysis can also deepen domain thinking and
prevent drifting analysis goals that lead to questionable conclusions. 
% Through their involvement in interactive compilation, 
% reify the connection between the conceptual and
% statistical in our software tools. In this way, this dissertation brings to the
% domain of data analysis, classic principles from end-user software engineering

\begin{comment}
\section{Recent developments} \label{sec:recentDevelopments}
Mention: in LLMs impact the contributions of this dissertation, exciting opportunities to leverage them to realize the goals of this work

\subsection{What about in the face of LLMs?}

But how do people express their domain knowledge, make the process meaningful

Mention LLMs as a technology to use here?
\end{comment}

\section{Impact} \label{sec:impact}
% \ej{Include details of where this code is available, open source, etc.}
The most rewarding part of conducting the work in this dissertation has been to
see real-world tool usage. As of May 2023, \tea has been downloaded 15K
times, and \tisane has been downloaded 12K times. All code
is open source and available \href{https://github.com/tea-lang-org}{online}.
Over the last few years, I have also enjoyed reading and answering flurries of
emails where analysts, including many scientists and social scientists, share
anecdotes about how they have used (and sometimes failed to use) \tea and \tisane.
There are a couple success stories I want to highlight. First, an HCI researcher
used \tisane and caught an analysis bug prior to submitting and ultimately
publishing a research paper at ACM CHI~\cite{baughan2022dissociation}. Second, a
collaborator at the Institute for Health Metrics and Evaluation used \tisane to validate a model as part of a larger health
policy research project, which was published at a prestigious
journal~\cite{johnson2022varied}.
Real-world usage of \tea and \tisane thus far provides further evidence in support of my thesis.

\section{Limitations and Future work} \label{sec:futureWork}
% \todo{Add ideas from defense}
% \polish{Is it clear what the core challenges refer to?}
By addressing the core challenges---(i) explicating domain knowledge,
(ii) representing and reasoning about statistical analysis decisions, and (iii)
promoting statistical understanding---the DSLs and automated reasoning
approaches in this dissertation make it possible for statistical non-experts to
more readily author valid analyses. There are important limitations of this work
that suggest promising future directions for further lowering the barriers to
statistical analysis. 
% We elaborate one some of the
% limitations of this work and opportunities for future research. 

\subsection{Support Interpretation of Statistical Results}
% \polish{Mention: Did not fully address Challenge 3 fully, this is a fertile ground for future work}
While \tea, \tisane and \rTisane effectively address the gulf of execution by
compiling conceptual knowledge, data collection details, and intents for
analysis into statistical analyses, they fall short of bridging the gulf of
evaluation~\cite{norman2013doet}. \tisane and \rTisane do not yet provide
support for analysts to interpret the results of their statistical models.
Accurate interpretation of statistical results is critical for follow-up
analyses, such as model revisions, and communication. Future work should address
two key challenges: (i) statistical reporting designs to enhance the
understanding of results~\footnote{In unpublished work building on \tea, Reiden
Chea, Annie Denton, and I started to explore the first challenge by using a
combination of textual explanations and visualizations. A next step is to
evaluate the changes.} and (ii) support for navigating the consequences of the
results, such as updates to analysts' conceptual models or the need to revise
statistical models. A promising first step is to directly answer a query in
\rTisane by providing the estimate of the independent variable's influence on
the dependent variable, not just a statistical modeling script. Even this small
change would better address \thesisChallengeUnderstanding. 

\begin{comment}
Tisane's focus is on authoring GLMs/GLMMs, but accurate interpretation is also necessary. For instance, analysts may need
help interpreting what their statistical models and results mean in relation to
their input conceptual models. Do the results suggest their conceptual model is
correct? What kind of inferences should they make? Future work should address
these interpretation challenges, which may require eliciting hypotheses and expected results from analysts.

Although researchers in our pilot or case studies did not presume
Tisane helped with formal causal analysis, the ability to express causal
relationships (\texttt{causes}) may lead some analysts to erroneously assume
that their models assess causality. Changing the name of the language construct
and/or building out support to interpret GLM/GLMM results may resolve this concern. One way to support accurate interpretation and reporting could be to
output a figure representing the input conceptual model along with visual
summaries of the data and/or statistical model for direct inclusion in publications. 
Tisane could also allow analysts to annotate their disambiguation decisions with
their own rationale and provide a richer log of selections than currently
supported. Tisane could even accept these augmented logs to save the state of the GUI in between analysis sessions.
\end{comment}


\subsection{Guide Statistical Model Revision and Iteration}
Analysts in the summative evaluation of \rTisane (\autoref{sec:summativeEval})
started to iterate on \rTisanes output statistical model and sought more
support for model revision. There are two key challenges in supporting a more
iterative modeling workflow, such as a Bayesian
Workflow~\cite{gelman2020bayesianWorkflow}: (i) recognizing when conceptual
revisions are necessary and (ii) identifying and suggesting model changes that
maintain conceptual validity or, at the very least, quantify conceptual shifts.
Furthermore, in the model revision process, analysts may consider multiple
alternatives. Addressing these challenges will likely require reasoning from
statistical models to conceptual models and may benefit from the fact that
\tisane and \rTisane already generate a space of statistical models to seed
statistical model disambiguation. 
% By supporting statistical model iteration,
% \tisane and \rTisane in particular can become platforms for future research on
% how to support more phases of the data lifecycle, such as experimental design
% and data collection.

% In the future, we plan to develop additional strategies for enhancing the
% validity of analyses authored with Tisane. As discussed in
% Section~\ref{sec:disambiguation}, our current approach to family and link
% functions is only an initial step. We look forward to developing and
% comparing multiple strategies for scaffolding the family and link function
% selection and revision process. For example, what if the Tisane GUI allowed
% analysts to fit multiple models that varied in their family and link functions,
% plotted each model's residuals against the predicted values, and gave analysts
% visual guides for comparing models? To avoid false discovery rate inflation,
% Tisane could partition analysts' data, fit models to only a subset, and output a
% script for fitting a selected model using another subset. Although possible for
% large datasets, this strategy would encounter limited statistical power for smaller
% datasets. Alternatively, what if Tisane calculated Bayes factors for variables
% in the
% models~\cite{raftery1996approximate,gelfand1994bayesian,czado2006choosing} after
% analysts tried multiple family and link combinations? Carefully balancing
% statistical rigor and usefulness to domain researchers who may be statistical
% non-experts deserves careful consideration.

% Tisane is one tool designed to enable analysts with limited statistical expertise to
% author valid statistical models. Tisane enables future possibilities and raises
% open research questions for creating an ecosystem of analysis tools that align
% tool interfaces with analysts' conceptual goals.
\begin{comment}
\subsection{Support more phases of the data lifecycle}
This dissertation emphasizes the need for abstractions that capture analysts'
implicit domain knowledge. These abstractions enable valid analysis formulation
and promote reflective thinking. Building upon this, we can begin to ask how the
same ideas---abstractions and automated reasoning for conceptual knowledge often
implicit in statistical analyses---could apply to other phases of the data
lifecycle. Future work should explore how to elicit and track the evolution of
conceptual knowledge even before statistical analysis by developing new
elicitation techniques and representations of domain knowledge and ecosystem of
inter-operating tools to track and ensure validity throughout the data lifecycle. 

% How can software tools track and trace the
% meaningful evolution of conceptual knowledge? By tackling these questions, we
% can revolutionize data analysis and develop comprehensive tools that seamlessly
% integrate domain expertise.
\end{comment}

\subsection{Connect Statistical Testing and Modeling} 
A natural question arising from this dissertation is the choice between \tea and
\tisane/\rTisane for analysts. \tea focuses on statistical testing, determining
evidence for or against a specific claim, while \tisane and \rTisane emphasize
statistical modeling, estimating variable influences in the presence of other
variables (e.g., confounders, mediators, etc.) However, statistical testing and
modeling are not mutually exclusive. Analysts often want to conduct tests after
building models to answer substantive questions as well as assess model fit. A
compelling future direction is to enable analysts to ask follow-up questions
about specific estimates and effects from a statistical model, which may itself
prompt additional statistical modeling. 

\begin{comment}
\subsection{Develop a grammar of study design}
In \tea, \tisane, and \rTisane we explored different primitives for expressing
data collection details. \tea provides constructs for explicitly specifying a
study type and independent/dependent variables. \tisane and \rTisane focus on
capturing the frequency of measurement and any nesting of observations. The goal
of these constructs is to leverage a data schema


These
different approaches start to touch on different representations of data that
are worth exploring. Tisane's graph IR is an entity-relationship (ER)
model~\cite{chen1976ERDiagram}. ER models, or diagrams, are used to describe
data schema. 

Primitives in existing
libraries for experimental design and data collection influenced our language
constructs. However, there may be more usable ways to express data collection
details. For instance, participants in the exploratory lab study
(\autoref{sec:exploratoryStudy}) suggested syntactic sugar. However, 

, which start
to touch on different representations of data

% \polish{Mention syntactic sugar from exploratory study}
There are relationships between Tisane's language constructs for specifying data
collection details, data schema specification, and experimental design. How
could we draw these connections out and formalize them? 

Tisane's graph IR is an entity-relationship (ER) model~\cite{chen1976ERDiagram}.
ER models, or diagrams, are used to describe data schema. ER models describe how
entities relate to other entities and attributes. In Tisane, a variable of Unit
type can be viewed as an entity. The \texttt{nests\_within} edge describes how
two units, or entities, relate to one another. Tisane's graph IR also relates
units (entities) to measures (attributes).
\end{comment}

\begin{comment}
\subsection{Specialized domain-specific language constructs?}
% \polish{Mention domain-specific language constructs?}
\end{comment}

\subsection{Design for Additional Aspects of Validity During Statistical Analysis}
A goal of the tools in this thesis is to produce statistical analyses that are
valid-by-design based on an expressed analysis intent. We prioritized internal,
external, and statistical conclusion validity. For instance, viewed through the
Campbellian framework of
validity~\cite{campbell2015quasiexperimentalDesigns,cook2002generalizedCausalInference},
\tisane helps analysts avoid four common threats to external and statistical
conclusion validity: (i) violation of statistical method assumptions, (ii)
fishing for statistical results, (iii) not accounting for the influence of
specific units, and (iv) overlooking the influence of data collection procedures
on outcomes~\cite{cook2002generalizedCausalInference}. A major barrier to
designing for construct validity lies in having access to and incorporating a
knowledge base about any domain of analysis. Fortunately, recent advances in
language models, such as GPT-3~\cite{brown2020language} and GPT-4, now provide
one form of said knowledge base. Therefore, future work should explore ways to
leverage such technologies to help analysts assess the reasonableness of proxies
for constructs of interest, author conceptual models using proxies, compare
conceptual models that may exist in the wild, and validate the ``reasonableness''
of expressed conceptual models. These directions to support construct validity
will likely benefit analysts because they amplify an existing benefit of \rTisane: turning
the conceptual model specification process into a reflective activity. 
% marshall

% Campbell's theory of validity -- encompassing statistical conclusion, internal,
% external, and construct
% validity~\cite{campbell2015quasiexperimentalDesigns,cook2002generalizedCausalInference}
% -- has influenced disciplines widely (e.g.,~\cite{shadish2010campbell}),
% including epidemiology (e.g.,~\cite{matthay2020causalDAGEpi}), software
% engineering (e.g.,~\cite{neto2013conceptual}), and psychology
% (e.g.,~\cite{campbell2015quasiexperimentalDesigns}). 
% Most notable is construct validity -- which LLMs could help with? 

% Correct analyses are more important than ever -- Mention stanford university prof?

% Tisane fills a need to align analysts' conceptual models with the statistical
% models they want to implement but find difficult to express with the current
% tools available. By integrating conceptual, data, and statistical concerns,
% Tisane facilitates the hypothesis
% formalization~\cite{jun2021hypothesisFormalization} process, which can be an
% error-prone and cognitively demanding process that existing tools do not yet support.
\subsection{Support the Larger Data Lifecycle}
% Assume data collected or could be collected in ways expressed 

This thesis has focused on scenarios where analysts either have collected data
or can articulate how they will collect data. This simplification helped focus
us on tools for automated statistical analysis formulation. However, statistical
analysis occurs late in the data lifecycle and is rarely separate from other
steps. The ability to draw reliable conclusions based on statistical analyses
depends on the configuration and quality of data available to analyze. 

% Idea 1: Support study design, Incorporate ways of 

A limitation of this thesis is that it under-serves the bi-directional
connection between data collection and data analysis. While \tea, \tisane, and
\rTisane all use data collection details to determine valid statistical
analyses, they do not help analysts who may not know how to collect their data
but do know what kinds of conclusions they would like to reach. New tools are
needed to help analysts start with implications they would like to
draw from analyses and then work ``backwards'' to figure out what data their studies should
collect, what proxies to randomize, how to allocate observations, and how to
handle data collection constraints. These tools require new abstractions for
expressing analysis goals or partial study designs as well as representations
that reify the connection between statistical analyses and experimental designs.
Addressing these challenges will create a tighter loop between data collection
and analysis and promote improved planning and practice in science.
% Addressing these challenges requires combining techniques
% in HCI, experimental design, and statistics. 

% One overlooked dimension of analytical provenance that this thesis
% highlights is the evolving conceptual model. 
Another limitation of this thesis is that it neglects how prior steps of the
data lifecycle, such as visual analysis, inform an analyst's understanding of
the domain. How might interfaces to steps earlier in the data lifecycle elicit
conceptual knowledge? Or what might interaction traces tell us about the
conceptual model an analyst is constructing? For instance, in the context of
visual analysis, what if interfaces could show the implied conceptual model
based on a series of visualization recommendation selections and user-provided
annotations? Over time, interfaces and techniques for eliciting and tracking
conceptual models could help explain how highly skilled analysts arrive at
different, even contradictory, statistical conclusions from the same research
question and data~\cite{silberzahn2018manyAnalysts}. Furthermore, as analysts in
the summative evaluation of \rTisane noted, conceptual models could facilitate
communication between domain experts, statistical experts, and the public.
Investigating how to design interfaces for expressing, tracking, sharing, and
debating conceptual models could be one step towards improving reproducibility in science.

% given that multiple
% analysts, with varying degrees of statistical expertise, are often involved. As
% analysts  Therefore, interfaces for sharing and deabting
% Multiple analysts are often involved in the analysis and sensemking process. Use
% conceptual models as a boundary object for communication and collaboration.
% Explore interfaces for sharing and productively debating conceptual models,
% proxy choices, and analysis choices. 

% Idea 3: Likely require communication between collaborators. 
% Participants already start dreaming up. Could we do more, like not just improving the output but also generating diagrams


% Idea 2: Visualization + Statistical modeling; key connection is domain knowledge evolution? Analytical provenance? 
% This dissertation emphasizes the need for abstractions that capture analysts'
% implicit domain knowledge. These abstractions enable valid analysis formulation
% and promote reflective thinking. Building upon this, we can begin to ask how the
% same ideas---abstractions and automated reasoning for conceptual knowledge often
% implicit in statistical analyses---could apply to other phases of the data
% lifecycle. 


% By supporting statistical model iteration,
% \tisane and \rTisane in particular can become platforms for future research on
% how to support more phases of the data lifecycle, such as experimental design
% and data collection.

\begin{comment}
\subsection{Evaluating usability of DSLs/Methods for designing and evaluating DSLs with end-users}
Through combination of formative and summative evaluations that leveraged
content analyses, <name methods used in this dissertation>, we were able to
identify and assess design goals for enabling end-users who are not trained as
programmers author programs using DSLs. Recent attention in how to apply HCI
techniques (cite PLIERS). What distinguishes this work, however, is how grounded
the constructs were in exploration of existing end-user libraries and testing
with users throughout, even to the point of renaming moderates in \tisane to
interacts in \rTisane and changing its semantics. 

Relevant measures and objects of study 

Relevant measures -- comparing expression without and with
\end{comment}

\begin{comment}
\subsection{Promote analytical best practices in science}
In the long term...
Tea and Tisane primarily follow a top-down authoring approach, where analysts
start with a research question and hypothesis. However, as observed in our lab
study to develop hypothesis formalization (\autoref{sec:labStudyHypoForm}),
analysts often develop and refine hypotheses based on statistical results.
Therefore, a future research direction would be to develop ways to incorporate
both data-driven and research question-driven approaches to model authoring and
refinement that do not promote cherry-picking. One possibility is to leverage
\tea's and \tisane's reasoning capabilities to reason in multiple other
directions, from statistical models and data to all possible conceptual models
or statistical models to data invariants that could inform study designs.

Moreover, one of the precautions integrated into the design of Tisane was to prevent
cherry-picking and p-hacking by using analysts' conceptual models to drive
statistical model formulation. While Tisane supports mapping one conceptual
model to a statistical model, an under-explored direction is to assess the
robustness of effects across multiple possible conceptual models, especially in
cases of ambiguity or debate in a discipline. On the other hand, multiverse
analyses~\cite{} embrace conceptual model uncertainty by considering all
possible statistical model formulations. Future research should look into how to
both be consistent with aspects of conceptual models researchers know and assess
evidence for other competing aspects of conceptual models.

Motivation: Recent news about Stanford University president
\end{comment}

\begin{comment}
\subsection{Improving data science education}
1 sentence: Motivation: From conversations with students and my own experience taking statistics courses
throughout undergraduate and graduate education is that the connection between
statistical methods and the kinds of questions I want to ask is often unclear.
Furthermore, students feel they need to memorize a bunch of different methods,
at the cost of thinking through what their substantive questions are and what
they should care about in a statistical approach. 

2: Goal, Impact: Promote statistical thinking, an essential component of practicing data science. 

Especially, greatest potential: Especially to teach students to separate
multiple sets of concerns, specifically the conceptual from the statsitical 

3. Role of systems: Use \tea and \tisane to focus students to focus first on identifying and
articulating their motivation and intents for analysis 

Before introducing and teaching statistical details

4. Future work: Deploy \tea and \tisane 
- questions to answer: impact on statistical thinking, computational thinking 
- what additional tools are necessary to develop? 

A ramp from novice to expert
tools is missing in the current ecosystem~\cite{mcnamara2018keyAttributes}. Tea
and Tisane lay the foundation for a bridge between novice and expert tools by
providing abstractions that match those of statistical non-experts while also
giving experts control, flexibility, and compatibility with other expert tools
in Python and R. While pursuing the above research agenda, I look forward to
directly improving data science courses I teach by deploying my systems in the
classroom, discovering students' needs, and iterating on tools and curriculum.
In pursuing this goal, I want to ease
transitions between analysis paradigms (e.g., NHST and Bayesian inference). One
promising direction is to separate concerns about model specification from
interpretation and assessment. 


Get rid of?: In a small way, this separation also teaches students how to separate
specification from implementation, an essential perspective in computer science.
So that if students need to implement more complex analyses, they have some
awareness of how to organize their computational approach. 
\end{comment}

\begin{comment}
To more systematically capture and act on these kinds of anecdotes in the
future, I am excited to develop a web platform for \tea and \tisane, where users
can share their programs, data, and insights. Over time, I hope to collect a
gallery of examples to answer questions about challenges using the DSLs, the
learn which use cases are under-supported, and assess the practical impact of
using conceptual abstractions and automated reasoning. I would also like to see
if end-users repeatedly use \tea and \tisane or if these are one-off
engagements. I look forward to not only pursuing ideas and systems developed in
this dissertation but also fostering a community of users. 
% who can not only inform future research but also benefit from future systems. 
% Also future opportunities for deploying and testing a community of users 
\end{comment}

\section{Closing Remarks}
Statistical data analysis shapes science, policy, and business. Software tools
for authoring analyses are essential. However, these tools do not serve a long
tail of users who are not statistical experts. Statistical analysis software
must help these analysts (i) express what they know about their domain, the
data, and their intents for analysis and (ii) guide them towards valid
statistical analyses and conclusions. This dissertation aims to enable people,
regardless of statistical background, to reliably analyze data to guide discovery
and inform decision-making. 