% 1 paragraph summarizing key contributions
% Mention, weave back in challenges from intro
\ej{Fill in}

\begin{comment}
\section{Note on methodology and implications}
Overall: Integrate systems building with formative and summative evaluations and deeper dives into understanding end-users (hypothesis formalization)
- strength of this work is moving back and forth between and integrating empirical studies with systems building
- Even within empirical studies, explored and used many qualitative and quantitative approaches
- Within systems building: Use diversity of technologies
- Identify need for methodologies in the future for designing DSLs \textit{with} end-users
\subsection{Plurality of contribution types}
\end{comment}
A strength of this work is in how it integrates systems building with empirical
studies, both of which motivated initial methodological experiments/innovations.
We engaged in formative and summative evaluations to design, implement, iterate
on, and evaluate Tea and Tisane. The evaluations involved qualitative and
quantitative approaches. The empirical work helped
us make technical insights in how to computationally represent statistical
analysis--as constraints and as graphs. Furthermore, we took a deep dive into
triangulating the nuances involved in authoring statistical analyses through a
qualitative analysis, lab study, and tools assessment in order to develop our
theory of hypothesis formalization. 

In addition to the formal empirical studies we conducted in this dissertation,
we benefited from informal observations, reports from early users, and our
personal experiences throughout the design processes. For instance, Tea came
from years of personal experiences and informal observations of how computer
scientists author statistical analyses relying, at best, on charts and tables
describing when specific tests were applicable. 
% and a hope that there could be a more
% rigorous way to hypothesis testing.

% Reflecting back on design process: 
% - Tea started with numerous personal experiences and informal observations of how computer scientists author statistical analyses
% - Started DSL design with primitives we assumed would be approachable based on survey of introductory quant methods courses. 

\section{Impact}
\ej{Fill in}

\section{Discussion: Themes}
\subsection{The role of programs and the act of programming as a reflective practice}

What is the formalism end-users have to learn, where does it come from, how well does it align with what they want to do/say, etc. 
% Finding: interactive disambiguation not just necessary for refinement and
automated reasoning but *useful* to analysts for reflection

This work re-visits the literature on end-user software engineering for a new
domain: statistical analysis. This work is a re-interpretation of EUSE, viewing
programming tools as bicycles of the mind. Specificaly, this work asks how we
can make not only the programs themselves useful for accomplishing a task (i.e.,
running a statistical test or model) but also the programs themselves useful for
documentation and sharing as well as the process of programming reflective and
meaningful to end-users. In this way, a central theme that this work addresses
is how to reify the connection between the conceptual and statistical in our
software tools. 

We find that Tisane's interactive disambiguation is not just necessary for
conceptual model refinement and automated reasoning to derive statistical models
but useful to analysts for reflection. 

% \section{EUSE++}
% \section{Where all this is going / Why do we care about any of this?}
% - a reinterpretation of EUSE -- programming tools as bicycles of the mind

\subsection{Designing the \textit{right} levels of abstraction} 

Both Tea and Tisane provide higher levels of abstraction for analysts to express
their implicit assumptions to author statistical analyses. However, the key to
the systems was not that they were just higher level but rather that their
abstractions were at the appropriate conceptual and data collection details that
analysts could specify and was still amenable to rigorous reasoning. In fact, a
key insight that guided our re-design of Tisane
(see~\autoref{sec:exploratoryStudy}) was that analysts wanted low- and
high-level conceptual abstractions. Therefore, a key takeaway from this work is
that higher levels of abstraction are not always better. Rather, abstractions
that allow analysts to dig deeper into the parts they want to and can is what is
necessary and impactful. 

Furthermore, the conceptual relationships between variables were still implicit
in Tea, but we made them more prominent primitives in Tisane. Our work on
defining hypothesis formalization helped us to identify the
centrality/importance of grappling with conceptual relationships explicitly. 

% **Not just higher levels of abstraction but appropriate abstractions that allow analysts to dig deeper into the appropriate parts

This focus on the conceptual knowledge analysts can express and can guide
computational and statistical reasoning highlights/suggests a shift in
perspective our perspective on the design problem at hand with statistical
analysis. While much effort has been put toward making statistical computation
more precise and efficient and the mathematical abstractions expressive, the
real design barrier lies in the conceptualization of the problem of statistical
analysis. That is, statistical analysis is a means to an end for many analysts,
especially statistical non-experts. Analysts' primary goal is to understand
something about their domain. Therefore, statistical software should serve this
goal, by allowing analysts to think about their domains and goals for analysis
deeply while authoring analyses (e.g., by documenting their implicit assumptions
about their domains) and interpret the results of the analyses in light of their
conceptual domain knowledge. This view aligns with a familiar breakdown of
complex tasks into the gulfs of execution and evaluation, respectively. While
this thesis has focused on how to bridge the gulf of execution, there is much
important work on how to report and help analysts interpret the results of their
analyses in light of their conceptual assumptions and models. 

% \section{Re-orienting the task we are designing for}
% Shifts the design problem on its head to view statistical analysis as a means to
% the larger end of helping analysts understand their domain better.
% Design for the purpose that statistical analysis serves. -- Norman quote?

\section{Future work}
Has this dissertation lost its way? Further re-orienting towards what users *really* want: to understand their domain
Push further in directions this work orients us 
- more support for understanding results, especially when some questions may not be answerable with the data/how it was collected
- knowing how robust the results are --> why not just multiverse everything?
**how do we resolve and come out from under the tyranny of false positive rates fear

\subsection{Interpretation of results}

\subsection{Connecting modeling with testing} 

\subsection{Connecting statistical analysis with visual analysis}

\subsection{What about in the face of LLMs?}

\subsection{Methods for human-centered programming language design}
From Tea to Tisane, we changed how we designed DSL primitives. To identify
primitives in Tea's DSL, we surveyed two introductory quantitative methods
courses in human-computer interaction. For Tisane, we started by iterating on
primitives that made sense to us, as designers, and would be amenable to formal
reasoning. However, between the first and second releases, we sought START HERE

Finding the right balance between designer and end-user intervention/design 

need for human-centered methods to design DSLs

\subsection{The long view: Ecosystem of software tools and computational reasoning throughout the data lifecycle}
The long-view of this work...

Mention LLMs as a technology to use here?