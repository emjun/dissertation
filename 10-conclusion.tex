\section{Summary of contributions}
\section{Note on methodology and implications}
- strength of this work is moving back and forth between and integrating empirical studies with systems building
- Even within empirical studies, explored and used many qualitative and quantitative approaches
- Within systems building: Use diversity of technologies
- Identify need for methodologies in the future for designing DSLs \textit{with} end-users

\section{EUSE++}
\section{Where all this is going / Why do we care about any of this?}
- a reinterpretation of EUSE -- programming tools as bicycles of the mind

\section{Discussion: The role of programs and the act of programming as a reflective practice}
Finding: interactive disambiguation not just necessary for refinement and automated reasoning but *useful* to analysts for reflection

**Not just higher levels of abstraction but appropriate abstractions that allow analysts to dig deeper into the appropriate parts


\section{Re-orienting the task we are designing for}
Design for the purpose that statistical analysis serves. -- Norman quote?

\section{Future work}
Has this dissertation lost its way? Further re-orienting towards what users *really* want: to understand their domain
Push further in directions this work orients us 
- more support for understanding results, especially when some questions may not be answerable with the data/how it was collected
- knowing how robust the results are --> why not just multiverse everything?
**how do we resolve and come out from under the tyranny of false positive rates fear

\subsection{}
\subsection{What about in the face of LLMs?}
