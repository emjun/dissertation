\section{Second Release: rTisane} \label{sec:rTisane}

\rTisane provides a DSL for analysts to express their implicit conceptual
assumptions and interactively compiles this high-level specification into a
script to fit a statistical model. The key challenge in designing a language for
expressing conceptual models is in identifying the language primitives that
allows analysts to express their implicit, often ``fuzzy,'' assumptions as
accurately to their internalized conceptual models as possible yet is precise
enough to rigorously derive a statistical model. Our approach in \rTisane is to
prioritize expressivity in the DSL and precision during interactive compilation. 

% \rTisane supports a two-step process for analysts to specify conceptual models
% that are both expressive and precise. 

\rTisaneProgram

\def\Participant{\texttt{Participant}\xspace}
\def\Unit{\texttt{Unit}\xspace}
\def\Condition{\texttt{Condition}\xspace}
\def\Conditions{\texttt{Condition}s\xspace}

\subsection{Declaring variables}
Analysts can express two types of variables: Units and Measures. Units represent
observational or experimental units from which analysts collect data. %\polish{reference prog} 
A common
unit is a participant in a study, so \rTisane provides syntactic sugar for
constructing a \Participant variable directly. \Participant is a wrapper for
declaring a \Unit.

Measures are attributes of Units collected in a \dataSet. %\polish{For example, reference prog} 
Measures can be one of four types: continuous, unordered
categories (i.e., nominal), ordered categories (i.e., ordinal), and counts.
\rTisane provides syntactic sugar for declaring \Conditions as either unordered
or ordered categories. 

\begin{comment}
Units
syntactic sugar: `Participant'

Measures
syntactic sugar: `condition'
\end{comment}


\def\causes{\texttt{causes}\xspace}
\def\relates{\texttt{relates}\xspace}
\def\when{\texttt{when}\xspace}
\def\then{\texttt{then}\xspace}
\def\assume{\texttt{assume}\xspace}
\def\hypothesize{\texttt{hypothesize}\xspace}

\subsection{Expressing a conceptual model}
Analysts specify how the variables relate conceptually in their conceptual
model. These relationships are stored in an intermediate graph representation
where the variables are nodes and the relationships are edges. 

There are two types of relationships: \causes and \relates. \causes indicates a
unidirectional influence from a cause to an effect. %\polish{Ref prog}. 
Internally,
\causes introduces a directed edge from the cause node to the effect node. As we
found in the exploratory lab study, analysts want opportunities to provide more
detail describing conceptual relationship. Towards this design goal
(\optionalSpecificity), \rTisane allows analysts to optionally specify \when and
\then parameters in the \causes and \relates functions. %\polish{Ref prog}. 
There are four comparisons analysts can specify in \when and \then:
\texttt{increases} (for continuous, ordered categories, counts),
\texttt{decreases} (for continuous, ordered categories, counts), \texttt{equals}
(for any measure type), and \texttt{notEquals} (for any measure type). The
optional specificity is used when suggesting ways to resolve ambiguity in the
input program during disambiguation (\autoref{subsec:conceptualModelDisambig}).

Analysts in the exploratory lab study also distinguished between assumed, or
strongly held, and hypothesized, or more uncertain, relationships. \rTisane
requires analysts to make these explicit distinctions (\assumeHypothesize) when
adding conceptual relationships to a conceptual model. In addition to specifying
a relationship type, analysts must either \assume or \hypothesize a relationship.
%\polish{ref program}.

% While analysts are thinking through and specifying \causes and \relates
% relationships, 

Analysts can also specify interactions between two or more variables when
specifying their conceptual models. %Unlike binary relationships (i.e., \causes, \relates), 
Analysts can directly add interactions to conceptual models. Interactions do not
need to be assumed or hypothesized because they provide additional information
about existing relationships in the conceptual model. 

% Causes / relates (types of relationships)
% Optional specificity: when, then annotations
% Assume / Hypothesize (label relationships)
% Interactions as annotations 
% - default semantics: if labeled, interactions considered. Otherwise, not

\def\query{\texttt{query}\xspace}
\subsection{Querying for a statistical model}
Analysts \query \rTisane for a statistical model based on the input conceptual
model. The querying process initiates the interactive compilation process and
results in an \texttt{R} script specifying and fitting a generalized linear
model. During interactive compilation, analysts engage in two loops to
disambiguate their (i) conceptual model and (ii) output statistical model. 

\subsection{Conceptual Model Disambiguation} \label{subsec:conceptualModelDisambig}
The goal of conceptual model disambiguation is to make analysts' expressed
conceptual models more precise. This precision is necessary to derive
statistical models rigorously. Conceptual model disambiguation involves breaking
cycles in the conceptual model by (i) picking a direction for any \relates
relationships and (ii) removing edges. Cycles are necessary to break because
they imply multiple different data generating processes that could lead to
different statistical models. 

To disambiguate conceptual models, \rTisane uses a GUI that shows a graph
representing analysts' conceptual models. If there are any \relates
relationships, \rTisane suggests ways analysts could assume a direction of
influence. Additionally, \rTisane suggests ways to break any cycles in the
conceptual model. As analysts make changes, the visible graph updates. The GUI
explains why both these steps are necessary to derive a statistical model. 

Once analysts have disambiguated their conceptual models, \rTisane will derive a
space of possible statistical models using the updated conceptual model. To
narrow this space down to one output statistical model, \rTisane will ask
additional follow-up disambiguating questions. 

\begin{comment}
This problem is actually challenging because detecting all cycles in a graph is
an NP-hard problem. We adapt a version of Johnson's(?) algorithm.
\end{comment}

\subsection{Statistical model derivation and disambiguation}
In order to derive statistical models, \rTisane considers confounders and
interactions to include as covariates, interactions, and family and link
functions.

To determine confounders, \tisane relied on Vanderweele's recommendations for
confounder selection~\cite{vanderweele2019modifiedDisjunctiveCriterion}.
\rTisane uses more recent recommendations from Cinelli, Forney, and
Pearl~\cite{cinelli2020controls}. Cinelli et al.'s recommendations are based on
a meta-analysis of studies examining the impact of confounder selection based on
graphical structures on statistical modeling accuracy. By following Cinelli et
al.'s recommendations, \rTisane includes confounders that help assess the
average causal effect of the query's independent variable on the dependent
variable as accurately as possible. 

Furthermore, \rTisane includes any interactions analysts have specified in their
conceptual models. Otherwise, \rTisane does not consider any interactions. In
the interface, analysts have the option to remove any confounders or
interactions based on their domain knowledge. 

\rTisane determines family and link functions based on the variable data types.
Because \rTisane compiles down to statistical models fit using the \lme package
in \texttt{R}, \rTisane is limited to the family and link functions supported in
\lme. For instance, for queries involving continuous dependent variables,
\rTisane considers Gaussian, Inverse Gaussian, and Gamma families. For counts,
\rTisane considers Poisson and Negative Binomial families. For ordered
categories, \rTisane considers Binomial, Multinomial, Gaussian, Inverse
Gaussian, and Gamma family functions. For unordered categories, \rTisane
considers Binomial and Multinomial family functions. \rTisane considers any link
functions \lme supports for these family functions. 