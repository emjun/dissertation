\section{Second Release: rTisane} \label{sec:rTisane}

\rTisane consists of (i) a DSL for analysts to express their conceptual
models and (ii) interactive disambiguation steps to compile this high-level specification into a
script fitting a statistical model. 

So far, we have implemented \rTisane for GLMs. Given the breadth of findings
from the elicitation lab study, we narrowed the scope from \tisane in order to
really focus on designing and testing a set of language constructs core to
conceptual modeling. 
% Given the breadth of findings from the elicitation lab study, we decided to
% really focus on a set of language constructs core to conceptual modeling. 
% a DSL for expressing conceptual models
There are two key challenges to designing \rTisane: (i) ensuring the DSL's
constructs can express analysts' implicit conceptual models accurately and (ii) %identifying DSL primitives that 
balancing usability with rigor, allowing
analysts to express their often ``fuzzy'' conceptual assumptions without losing
precision to derive a statistical model.

% \rTisane supports a two-step process for analysts to specify conceptual models
% that are both expressive and precise. 

\rTisaneProgram
\polish{Listing 5.1 should ideally reside on a single page, not across pages.}
\polish{The comment says 5 categories for "Race", but the code has cardinality = 9

"sex <− categories(unit = person, "Sex",cardinality = 2)" ->
"sex <− categories(unit = person, "Sex", cardinality = 2)"} 
\polish{Code formatting: Why is hypothesize bolded but others not?}

\def\Participant{\texttt{Participant}\xspace}
\def\Unit{\texttt{Unit}\xspace}
\def\Condition{\texttt{Condition}\xspace}
\def\Conditions{\texttt{Condition}s\xspace}

\subsection{\rTisanes Domain-Specific Language}
Like \tisane, analysts express variables, a conceptual model, and a query for a
statistical model. \rTisanes DSL prioritizes expressivity and usability 

\subsubsection{Declaring variables}
Analysts can express two types of variables: Units and Measures. Units represent
observational or experimental units from which analysts collect data (see Line 5 in~\autoref{lst:rTisaneProgram}). 
A common unit is a participant in a study, so \rTisane provides syntactic sugar for
constructing a \Participant unit directly. \Participant is implemented as a wrapper for
declaring a \Unit.

Measures are attributes of Units collected in a \dataSet, so they are declared
through a Unit. Measures can be one of four
types: continuous, unordered categories (i.e., nominal), ordered categories
(i.e., ordinal), and counts (see Lines 7-18 in~\autoref{lst:rTisaneProgram}). \rTisane provides syntactic sugar for declaring
\Conditions as either unordered or ordered categories. Analysts declare
unordered and ordered categories through the \texttt{categories} function.
Analysts can specify a variable is ordered by passing a list to the
\texttt{order} parameter. Otherwise, the variable is considered unordered.
Analysts can use \texttt{continuous} and \texttt{count} functions to declare
continuous and count Measures. 
% We
% chose this design to reduce the number of unique functions and better match
% semantic similarity. 


\begin{comment}
Units
syntactic sugar: `Participant'

Measures
syntactic sugar: `condition'
\end{comment}


\def\causes{\texttt{causes}\xspace}
\def\relates{\texttt{relates}\xspace}
\def\when{\texttt{when}\xspace}
\def\then{\texttt{then}\xspace}
\def\assume{\texttt{assume}\xspace}
\def\hypothesize{\texttt{hypothesize}\xspace}

\subsubsection{Expressing a conceptual model}
Once analysts have constructed variables, they can specify how these variables
relate conceptually. To do so, they construct a \texttt{ConceptualModel} and add
variable relationships to it (Lines 21-31 in~\autoref{lst:rTisaneProgram}). The conceptual model %\texttt{ConceptualModel}
is represented as a graph where the variables are nodes and the relationships
are edges. 

There are two types of relationships: \causes and \relates. \causes indicates a
unidirectional influence from a cause to an effect. \causes
introduces a directed edge from the cause node to the effect node. \relates
indicates that two variables are related but exactly how is ambiguous because
the analyst is uncertain about the direction of influence. \relates introduces a
bi-directional edge  between two variables. During a disambiguation step,
\rTisane will walk analysts through possible graphical structures that a
bi-directional edge could represent (\considerPossibilities). To derive a
statistical model, \rTisane requires an analyst to assume a direction of
influence.


Towards the design goal of \optionalSpecificity, \rTisane allows analysts to
optionally specify \when and \then parameters in the \causes and \relates
functions. There are four comparisons analysts can specify in
\when and \then: \texttt{increases} (for continuous, ordered categories,
counts), \texttt{decreases} (for continuous, ordered categories, counts),
\texttt{equals} (for any measure type), and \texttt{notEquals} (for any measure
type). Supporting optional specificity is designed to (i) make the \rTisane
program an accurate document of analysts' implicit assumptions and (ii) suggest
ways to resolve conceptual ambiguity during disambiguation
(\considerPossibilities). 
% used when suggesting ways to resolve
% ambiguity in the input program during disambiguation
% (\autoref{subsec:conceptualModelDisambig}).

To add relationships to the conceptual model, analysts must assume or
hypothesize a relationship. This distinction supports how analysts distinguish
between assumed, or strongly held, and hypothesized, or more uncertain,
relationships. \rTisane requires analysts to make these explicit distinctions
(\assumeHypothesize) when adding conceptual relationships to a conceptual model.
In addition to specifying a relationship type, analysts must either \assume or
\hypothesize a relationship. 

% While analysts are thinking through and specifying \causes and \relates
% relationships, 

Analysts can also specify interactions between two or more variables by
declaring \texttt{interacts}. Interactions are annotations to conceptual models
and are added to the graph without \assume or \hypothesize. Interactions provide
additional information about existing relationships in the conceptual model
(\interactionAsPartialSpec). 
% Causes / relates (types of relationships)
% Optional specificity: when, then annotations
% Assume / Hypothesize (label relationships)
% Interactions as annotations 
% - default semantics: if labeled, interactions considered. Otherwise, not

\def\query{\texttt{query}\xspace}
\subsubsection{Querying for a statistical model}
Analysts \query \rTisane for a statistical model based on the input conceptual
model (Line 31 in~\autoref{lst:rTisaneProgram}). The query asks for a statistical model to accurately
estimate the average causal effect (ACE) of the independent variable on the
dependent variable. The querying process initiates the interactive compilation
process and results in an \texttt{R} script specifying and fitting a generalized
linear model. During interactive compilation, analysts engage in two loops to
disambiguate their (i) conceptual model and (ii) output statistical model. 

\subsection{Two-step Interactive Compilation}
There are two phases to interactively compiling a conceptual model to a
statistical model: (i) conceptual model disambiguation and (ii) statistical
model disambiguation. We added conceptual model disambiguation to address the
need to explore possible conceptual structures for resolving ambiguities
introduced by \relates (\considerPossibilities).

\subsubsection{Conceptual Model Disambiguation} \label{subsec:conceptualModelDisambig} 
\conceptualModelDisambiguation
The goal of conceptual model disambiguation is to make analysts' expressed
conceptual models precise enough to derive a statistical model, achieving
usability and rigor. Conceptual model disambiguation involves breaking cycles in
the conceptual model by (i) picking a direction for any \relates relationships
and/or (ii) removing edges. Cycles are necessary to break because they imply
multiple different data generating processes that could lead to different
statistical models. In this way, conceptual model disambiguation can help analysts
reflect on and clarify their implicit assumptions. 

To disambiguate conceptual models, \rTisane uses a GUI.~\autoref{fig:figureConceptualModelsDisambiguation} shows the conceptual model disambiguation interface for the input program in~\autoref{lst:rTisaneProgram}. The GUI shows a graph
representing analysts' conceptual models. If there are any \relates
relationships, \rTisane suggests ways analysts could assume a direction of
influence. Additionally, \rTisane suggests ways to break any cycles in the
conceptual model. As analysts make changes, the visible graph updates. The GUI
also explains why both these steps are necessary to derive a statistical model. 

Once analysts have disambiguated their conceptual models, \rTisane updates the
internal graph representation and derives a space of possible statistical
models. To narrow this space of possible statistical models down to one output
statistical model, \rTisane asks additional follow-up disambiguating questions. 

\begin{comment}
This problem is actually challenging because detecting all cycles in a graph is
an NP-hard problem. We adapt a version of Johnson's(?) algorithm.
\end{comment}

\subsubsection{Statistical model derivation and disambiguation}
\statisticalModelDisambiguation
To formulate possible statistical models, \rTisane considers potential
covariates to control for confounding, interactions, and family and link
functions.

To determine confounders, \rTisane uses more recent recommendations from
Cinelli, Forney, and Pearl~\cite{cinelli2020controls}\footnote{\tisane relied on
Vanderweele's recommendations for confounder
selection~\cite{vanderweele2019modifiedDisjunctiveCriterion}, but in \rTisane we
opted for more recent recommendations}. Cinelli et al.'s recommendations are
based on a meta-analysis of studies examining the impact of confounder selection
based on graphical structures on statistical modeling accuracy. By following
Cinelli et al.'s recommendations, \rTisane includes confounders that help assess
the average causal effect of the query's independent variable on the dependent
variable as accurately as possible. 

\rTisane searches for interactions analysts annotated in their conceptual models
and suggests any involving the query's dependent variable. Otherwise, \rTisane
does not consider any interactions. 

\rTisane determines family and link functions based on the query's dependent
variable data type. \polish{Explain the above sentence, see what do for \tisane} Because \rTisane compiles down to statistical models fit
using the \lme package in \texttt{R}, \rTisane is limited to the family and link
functions supported in \lme. For instance, for queries involving continuous
dependent variables, \rTisane considers Gaussian, Inverse Gaussian, and Gamma
families. For counts, \rTisane considers Poisson and Negative Binomial families.
For ordered categories, \rTisane considers Binomial, Multinomial, Gaussian,
Inverse Gaussian, and Gamma family functions. For unordered categories, \rTisane
considers Binomial and Multinomial family functions. \rTisane considers any link
functions \lme supports for these family functions. 

In the GUI, analysts have the option to remove any confounders or interactions
based on their domain knowledge. Based on prior experience or domain
recommendations, analysts can also pick a family and link function pair if
multiple possibilities could apply. 