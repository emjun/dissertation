\todo{Add papers that Rene mentioned, justify choice}
\subsection{Statistical scope}  \label{sec:GLM}
% Tisane supports two classes of models that are widely applicable to diverse
% domains and data collection
% settings~\cite{lo2015transform,barr2013random,bolker2009generalized}:
% Generalized Linear Models (GLMs) and Generalized Linear Mixed-effects Models
% (GLMMs).
GLMs and GLMMs are meaningful targets because they are commonly
used (e.g., in psychology~\cite{lo2015transform,cohen2013applied}, social
science~\cite{kreft1998introducing}, and
medicine~\cite{bolker2009generalized,barr2013random}) yet are easy to misspecify
for statistical experts and non-experts alike~\cite{barr2013random,
cohen2013applied}. We designed Tisane to support researchers who are domain
experts capable of supplying conceptual and data collection information but lack
the statistical expertise or confidence to author GLM/GLMMs accurately.
%capable of specifying accurate conceptual and
%data relationships.
Both GLMs and GLMMs consist of (i) a \textit{model effects structure},
which can include main and interaction effects and (ii) \textit{family} and
\textit{link} functions. The family function describes how the residuals of a
model are distributed. The link function transforms the predicted values of the
dependent variable. This allows modeling of linear and non-linear relationships
between the dependent variable and the predictors. In contrast to
transformations applied directly to the dependent variable, a link function does
not affect the error distributions around the predicted values. The key
difference between GLMs and GLMMs is that GLMMs contain random effects in their
model effects structure. Random effects describe how individuals (e.g., a study
participant) vary and are necessary in the presence of hierarchies, repeated
measures, and non-nesting composition
(\ref{sec:data-measurement-relationships})\footnote{Traditionally, the term
``mixed effects'' refers to the simultaneous presence of ``fixed'' and
``random'' effects in a single model. We try to avoid these terms as there are
many contradictory usages and definitions~\cite{gelmanFixedRandom}. When we do
use these terms, we use the definitions from Kreft and De
Leeuw~\cite{kreft1998introducing}.}.

Both GLMs and GLMMs assume that (i) the variables involved are linearly related,
(ii) there are no extreme outliers, and (iii) the family and link functions are
correctly specified. In addition, GLMs also assume that (iv) the observations
are independent. Tisane's interactive compilation process guides users through
specifying model effects structures, family and link functions to satisfy
assumption (iii), and random effects only when necessary to pick between GLMs
and GLMMs and satisfy assumption (iv).

In the scope of this thesis, GLM and GLMMs are an appropriate scope because they
encompass a large scope of statistical models such that our research
contributions are widely applicable and substantial. In addition, given that
NHSTs (in Tea) are mathematically related to GLMs and GLMMs, Tisane's focus
helps us to push the boundaries of the applicability of higher-level
abstractions for statistical analysis established/explored in Tea and lay the
groundwork to connect Tea and Tisane's programming and interaction models
(better design statistical computing tools accurately reflect statistical
taxonomies).

\section{Related work}
% Tisane assumes that analysts have chosen to use Tisane due to their knowledge of the linear relationship between variables and have removed outliers from data although Tisane does not require data. \jh{Confusing, so removing for now.}
\subsubsection{Causal Analysis}
One of our aims in Tisane was to connect conceptual models to statistical models
(\connectConceptualStats). Prior work in the causal reasoning literature shows
how linear models can be derived from causal graphs to make statistical
inferences and test the motivating causal
graph~\cite{spirtes1996using,spirtes1994conditional}. Recently, VanderWeele
proposed the ``modified disjunctive cause
criterion''~\cite{vanderweele2019modifiedDisjunctiveCriterion} as a new
heuristic for researchers without a clearly accepted formal causal model to
identify confounders to include in a linear model, for example. The criterion
identifies confounders in a graph based on expressed causal relationships.
Tisane applies the modified disjunctive cause criterion when suggesting
variables to include in a GLM or GLMM. Tisane does not automatically include
variables to the statistical models because substantive domain knowledge is
necessary to resolve issues of temporal dependence between variables, among
other considerations~\cite{vanderweele2019modifiedDisjunctiveCriterion}. To
guide analysts through the suggestions, Tisane provides analysts with
explanations to aid their decision making during disambiguation. Finally, GLMs
are not formal causal analyses. Tisane does not calculate average causal effect
or other causal estimands. Rather, Tisane only utilizes insights about the
connection between causal DAGs and linear models to guide analysts towards
including potentially relevant confounders in their GLMs grounded in domain
knowledge. 

There are multiple frameworks for
reasoning about causality~\cite{rubin2004teaching,pearl1995causal}. One
widespread approach is to use directed acyclic graphs (DAGs) to encode
conditional dependencies between
variables~\cite{pearl1995doCalculus,greenland1999causal,spirtes1994conditional,spirtes1996using}.
If analysts can specify a formal causal graph, Pearl's ``backdoor path
criterion''~\cite{pearl1995causal,pearl2000models} explains the set of variables
that control for confounding. However, in practice, specifying proper causal
DAGs is challenging and error-prone for domain experts who are not also experts
in causal analysis~\cite{suzuki2020causal} due to uncertainty of empirical
findings~\cite{suzuki2018mechanisms} and lack of guidance on which variables and
relationships to include~\cite{velentgas2013developing}. Accordingly, Tisane
does not expect analysts to specify a formal causal graph. Instead, analysts can
express causal relationships as well as ``looser'' association (not causal)
relationships between variables in the \SDSLlong.
