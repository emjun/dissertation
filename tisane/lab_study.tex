\section{Summative Evaluation: Controlled lab study} \label{sec:summativeEval}
% \highlight{We are in the process of running this lab study and collecting data.}

Two research questions motivated our evaluation of \rTisane:

\begin{itemize}
    \item \evalConceptualModels What is the impact of \rTisane on conceptual
    modeling?
    \item \evalStatisticalModels How does \rTisane impact the statistical models
    analysts implement? Specifically, do the covariates, family functions, and
    link functions analysts include/exclude differ when implementing statistical
    models on their own vs. using \rTisane? How well do statistical models
    analysts author on their own vs. using rTisane fit the data? \polish{Revise the specific questions}
    % \item \evalLearning Do analysts learn about their discipline or data
    % analysis as a result of using rTisane?
\end{itemize}

\subsection{Study design}
% \todo{Include a diagram summarizing study design?}
We conducted a within-subjects (Tool support: \rTisane vs. none) think-a-loud
lab study that consisted of four phases. We designed the study based on the
assumption that conceptual modeling is a helpful strategy when specifying
statistical models. As a result, all participants completed the phases in the
following order.

\begin{itemize}
    \item \textbf{Phase 1: Warm up.} We presented participants with the
    following open-ended research question: ``What aspects of an adult's
    background and demographics are associated with income?'' We asked
    participants to specify a conceptual model including variables they thought
    influenced income. This warm-up exercise helped to externalize and keep
    track of participants' pre-conceived notions and assumptions prior to seeing
    a more restricted data schema.
    \item \textbf{Phase 2: Express conceptual models} We presented participants
    with a data schema describing a dataset from the U.S. Census Bureau. We then
    asked participants to specify a conceptual model using only the available
    variables. At the end, we asked participants about their
    experiences specifying their conceptual models in a brief survey and semi-structured interview.
    \item \textbf{Phase 3: Implement statistical models} We asked participants
    to implement ``a statistical model that assesses the influence of variables
    [they] believe to be important (in the context of additional potentially
    influential factors) on income,'' relying on only their conceptual model. We
    then asked participants about their experiences implementing statistical
    models through a brief survey and semi-structured interview. 
    \item \textbf{Phase 4: Exit interview.} The study concluded with a survey
    and semi-structured interview where we asked participants about their
    experience in the study, reactions to using \rTisane, and connecting
    conceptual models to statistical models.
\end{itemize} 

In order to assess the effect of tooling on conceptual models and the quality of
statistical models, we counterbalanced the order of tool support, or if
participants completed each task with or without rTisane first. The order of
tool use was the same for Phase 2 and Phase 3. Within each of Phase 2 and Phase,
half the participants completed the task on their own (without \rTisane) then
with \rTisane. The other half started with \rTisane and then did the task on
their own. Prior to using \rTisane in Phases 2 and 3, participants followed a
tutorial introducing the relevant language
constructs.~\autoref{appendix:summativeEvaluation} contains all the study
materials.

\noindent \paragraph{Participants} We recruited 13 data analysts on Upwork. We
screened for participants who reported having experience with authoring
generalized linear models and using R. \polish{Update based on recruitment:On
average, participants reported having conducted N projects using generalized
linear models. Participants were familiar with a range of software tools,
including...} All studies were conducted over Zoom. Participants used \rTisane
on a remote controlled computer, so they did not have to install it on their
own. Each study lasted between two and three hours. Participant was compensated
\$25 per hour. We recorded participants' screens, video, and audio throughout
the study. We then transcribed the audio and used detailed researcher notes for
qualitative analyses.

% One dataset was on demographic factors and income in the U.S. in 2018 from the
% U.S. Census Bureau (\datasetIncome). The other dataset was about demographic
% factors and health conditions in the U.S. (\datasetHealth) from \ej{FILL IN}.
% More information on how the datasets were created are found in the
% supplemental material.

\subsection{Analysis Approach}
Our analysis procedure consisted of two parts: (i) a thematic analysis of lab
notes, transcripts, and open-ended survey questions and (ii) an artifact
analysis of conceptual models and statistical models analysts authored with and
without \rTisane. For the conceptual models, we compared their form and content
between tool support conditions. For the statistical models, we compared the
overall statistical approach, specific statistical model formulation, and
rationale for analysis decisions and conclusions. We also compared two goodness
of fit measures between statistical models: AIC and BIC. We iterated on the
thematic analysis and artifact analysis separately at first and then interpreted
emergent observations across the two analyses. 
% The artifact analysis helped to
% ground and explain aspects of the thematic analysis.

\begin{comment}
We collected and analyzed quantitative and qualitative measures to answer our
research questions. For each research question, we describe our analysis
approach and results below. 
% We qualitatively analyzed conceptual models, statistical models, audio
% recordings, and open-ended survey responses. Whenever possible, we also
% quantified the frequency of characteristics in conceptual models and
% statistical models across participants. For the statistical models
% participants authored, we compared the AIC and BIC scores for matched pairs of
% the independent variable of interest. 


\subsubsection{\evalConceptualModels}
** How does rTisane compare to on own? **
Main effect of rTisane: Initial CM is very ambiguous and not formal, rTisane is making it more formal. 
- distribute IR
- use IR to target other things (e.g., science diagrams that a Participant said)

On own, participants were not sure how to structure a conceptual model. 
Diversity of conceptual models [from conceptual model analysis] -- form, complexity (number of relationships levels of relationships)


API structure + ... 
START HERE: 
- Group API clusters into smaller clusters
- Write one sentence about each cluster

<Decouple conceptual model from data?>

In general, people able to express conceptual model with rTisane. No major missing concepts, minor syntactic sugar + more advanced constructs desirable;
Critical to mention: **"no perceived influence"** -- confirmed by no statistical differences in NASA-TLX scores between conditions
Takeaway: rTisane provides process but not suggest relationships (control up to end-user) - no agenda for how end-users build model

% Consistency within vs. across participants 
We qualitatively analyzed how consistent participants' conceptual models were
between conditions. We noted common challenges translating free-form conceptual
models into rTisane programs. We also thematically analyzed participant
transcripts and survey responses describing the influence of rTisane on their
conceptual modeling processes. 

% \subsubsection{\evalConceptualModelAuthoring}
% % To understand rTisane's influence on the conceptual modeling process, we
% % compared the ratings and thematically analyzed survey responses to questions
% % asking how participants decided what conceptual relationships to specify.
% At the end of Phases 2 and 3, we asked participants to rate and describe their
% experience specifying conceptual models without and with rTisane, respectively.
% In addition, we kept track of the conceptual modeling challenges participants
% vocalized and noted how participants overcame these challenges in-situ. We
% compared the survey responses, observed challenges, and observed approaches
% between Phases 2 and 3 to assess how rTisane influenced the conceptual modeling
% process. 

\subsubsection{\evalStatisticalModels}
We used AIC, BIC, and R-squared values to assess how well statistical models
authored with vs. without rTisane fit the data. We used rTisane to statistically
model and assess the influence of rTisane on AIC, BIC, and R-squared values.

We also thematically analyzed participants' reactions to the similarities,
differences, and surprises between statistical models. 

% \todo{Add statistical models executed}
% \todo{Add rTisane analysis script to supplemental material}
\end{comment}

\subsection{Findings}
\polish{1-2 sentences highlighting key findings.}
\subsubsection{\rTisane's impact on conceptual models}
\textbf{Key takeaway: For conceptual modeling, \rTisane imposed productive
constraints of expression and a structure to specification and disambiguation
that facilitated reflection of implicit domain assumptions, consideration of new
relationships, and accurate externalization of implicit assumptions.}

The conceptual models analysts expressed on their own were diverse in form,
meaning/content, and complexity. The majority [P2, P4, P5, P8, P11, P13] invoked
a graph-like structure. [P2, P4, P8 used \rTisane second; P5, P11, P13 used
\rTisane first]. \polish{Figure A} illustrates two example conceptual
models.\footnote{An example conceptual model given in the task instructions may
have biased analysts towards a graphical structure.} Participants also described
their conceptual models verbally [P10], in text [P6, P9], or as timeline [P12].
P12 developed a visual hierarchy to communicate how variables on the left were
variables that individuals had control over and would influence income
\polish{(see Figure B)}. P7, who used \rTisane first, even jumped to expressing
their conceptual model in a statistical model. 

Ten participants involved all five independent variables available in the task
description and data in their conceptual models [P2, P3, P4, P5, P7, P8, P9,
P11, P12, P13]. Two participants [P7, P13] even included interaction variables
in their conceptual models. P13 specified a relatively complex conceptual model
\polish{(see Figure C)} where age, race, and sex interacted to cause an
interaction between education and employment to determine an individual’s
income. \polish{GET A QUOTE.}

\theme{Without \rTisane, analysts found it more difficult to express nuances in their domain knowledge.}
In a survey and interview about their conceptual modeling experiences,
participants shared that they found it difficult to author conceptual models
without tool support due to concerns about mis-specifying relationships beyond
their domain knowledge [P5, P10] and doubts about how to communicate nuances in
relationships [P3, 13]. P13 explained, ``Identify how I may weigh certain
variables based on my general awareness and knowledge and overall weights of
each variable of how one may affect income more or less in various
circumstances.'' Similarly, P8 described specifying their conceptual model as a
general ``struggle'' because ``When doing it myself, there are so many
possibilities.''

% With rTisane 
\theme{\rTisane provided structure to express and inspect conceptual models.}
In contrast, participants appreciated how rTisane structured their
conceptual modeling process [P2,4,9,10,11,12,13]. Participants found the rTisane
DSL particularly helpful. For instance, P9 explained how rTisane``led [him] to
think about the relationships first, and then whether they were what [he] was
hypothesizing’’ and how this process was the ``reverse of the way [he] would
think about it normally.’’ Similarly, P4 explained how using the rTisane
language constructs required them to think through how different values of a
variable (e.g., different categories) could change income, so they observed that
their conceptual model with rTisane was ``more specific’’ than without. P4
further explained how rTisane’s DSL "encouraged me to think about interactions,
which I hadn't thought about before using rTisane.” P13 also found that rTisane
encouraged them to consider new relationships. Four participants even said that
rTisane made it easier for them to specify their conceptual models [P4, P8, P10,
P12]. P4 and P10 even believed that rTisane’s “formal structure made [conceptual
modeling] more rigorous” [P4] and ``more disciplined'' [P10].

Participants relied on the conceptual disambiguation step to verify that what
they expressed in code accurately represented their implicit assumptions [P2,
P8, P12]. P2 said ``The idea of a graph made me think much differently about the
model. I am not sure if this improved my model but it definitely made me
approach it differently.’’ Interestingly, P2 had drawn a conceptual model as a
graph on his own prior to using rTisane. However, it seems that moving back and
forth between code and a graph helped him to think about his assumptions in
different ways because he also expressed, “The interactive process was a good
way to check that the graph came out the same way I was picturing it. It was
helpful because it is easier to look at than code'' [P2]. 

\theme{\rTisane helped analysts think about their domains more deeply.}
In addition to organizing their conceptual modeling process, rTisane’s DSL
deepened participants’ thinking [P3, 4, 7, 8, 12, 13], giving them, as P12
described, a structure to explore the “boundaries of their domain knowledge.” P3
explained how even after specifying conceptual models on their own, rTisane’s
four composable relationships (assume, hypothesize x causes, relates)
facilitated a deeper consideration of each relationship and what they knew:
"Having to think about specifics like 'do we know the direction of the
relationship' or 'what happens when a category increases/decreases' actually
helped me put my thoughts out more clearly. I was able to think about more
possible scenarios that could conflict with my current assumption. (which i was
probably not doing when i was initially making generic diagram looking at the
broad picture.) In conclusion, I want to say that looking at 4 possible ways to
write a relationship made me think more about each one of them." [P3] P4 further
explained how the DSL’s support for optional specificity “encouraged [them] to
think about the directionality of my hypothesized relationships and for
categorical variables to think about the effect of each individual category."
rTisane’s language constructs also helped participants more easily make explicit
what they had in mind. P10 explained, "My thinking was that before I didn’t have
much idea about how can I link my variable with the output [variable], and how
this can interact. And so it may need some trial and error... using this API,
there are predefined functions, they are translated in R language, cause or
relates, it made my task easier–this translation was not on me anymore."

\theme{\rTisane ...}
Importantly, rTisane scaffolded the conceptual modeling process without
compromising the integrity of the expressed conceptual models to what
participants truly believed. Participants reported that rTisane had no perceived
influence on their conceptual models [P3, 4,5,6,11]. These findings give
evidence for not only the expressivity of rTisane’s DSL. Moreover, three
participants expressed identical conceptual models with and without rTisane [P9,
P11, P12]. Three participants included different variables [P13, P6] and
relationships [P10] between conditions. For six participants, the conceptual
models they authored with rTisane were subgraphs of conceptual models authored
without rTisane. For P2, P3, P4, and P8, who were in the Scaffold-Tisane
condition, rTisane seemed to have helped focus them on a set of variables and
relationships to analyze. P3 explained, "As I started working with rTisane, my
first instinct was still to go back to the canvas and do a brainstorming. The
process of listing down the categories and the generic relationship between the
variables (which was biased to my personal opinion) was still the same (with or
without rTisane)." For P5 and P7, who were in the rTisane-Scaffold condition,
rTisane provided a starting point for them to then expand upon on their own. For
example, P7 authored a statistical model involving an interaction between
variables in their rTisane conceptual model when asked to specify a conceptual
model on their own. Both these scenarios suggest that rTisane’s DSL for
expressing conceptual models can support both convergent and divergent creative
thinking about a domain, even to the point of helping an analyst, like P7,
arrive at a statistical model on their own. 

% Key takeaway: rTisane was expressive, as seen in how particpiants felt free and
% able to express their diverse conceptual models using rTisane, rTisane did not
% impose the content. Rather, the key benefit of rTisane was in providing
% structure to / facilitating this process more.

\subsubsection{\rTisane's impact on statistical models}
\textbf{Key takeaway:} For statistical modeling, \rTisane
focused participants on their analysis goals over low-level details that bogged
them down without tool support, with the exception of picking family and link
functions. rTisane can improve statistical model authoring, resulting models,
and communication about statistical analyses. In the future, rTisane could
improve explanations for analysis decisions and alternatives. 

On their own, six participants successfully authored linear regression models
[P2, P4, P7, P8, P9, P10]. A seventh participant, P6, started to author a
logistic regression model with Race and Income but stopped before binarizing
either variable. Two participants, both of whom had just finished authoring
statistical models with rTisane, implemented GLMs [P11, P13]. P11 started from
the rTisane’s output model script. After observing the model had “poor fit”
\polish{(quote)}, they log-transformed the income variable. Finally, three participants
were not able to author a statistical model due to unfamiliarity with
statistical methods [P3], lack of time [P5], and reliance on visual analyses
(ie.g., heatmaps, scatterplots) [P12]. 

\theme{Statistical models authored with \rTisane fit the data just as well or better in most cases.}
Of the eight participants who successfully authored linear regression or
generalized linear models on their own, three implemented identical models with
or without rTisane [P7, P9, P13]. Notably, all had authored a statistical model
with rTisane before on their own. For three participants [P4, P8, P10], their
statistical models with rTisane had lower AIC and BIC scores than the
statistical models without rTisane. In other words, rTisane models fit the data
better or equally well for six out of eight participants. For P11, the
statistical model they authored without rTisane dropped some observations, so
the models are directly comparable. For P2, the rTisane statistical model fit
worse than his own statistical model in part due to a change in his motivation
for analysis. 

\theme{...}
P2’s experience authoring statistical models without then with rTisane
illustrates the impact rTisane had on participants. P2 was curious about the
influence of occupation, or employment, on income. His conceptual model in
rTisane had the variables Education, Age, Race, and Sex causing Occupation,
which in turn, causes Income \polish{(see Figure X)}. On his own, he first authored an
ANOVA with Occupation as the IV and Income as the DV. Once he saw that
Occupation had a statistically significant influence on Income, he wanted to see
if the variables causing Occupation would “be able to predict which
occupation…And then…the income from the occupation just because that’s how I
like structured it [in the conceptual model] initially.” However, P2 did not
know how to author a model with Occupation as the outcome variable because it
was categorical, saying, “But the way I structured it in like the diagram. I'm
not sure exactly how to do that, because occupations like categorical. Um, so
I'm not sure like exactly…how to model that.” This roadblock led P2 to consider
an alternative ``regression model with income as like the output and then…all
[the IVs] as terms and then just include the interactions between Occupation and
the terms that were pointing into it, and that would just be one model.'' In
other words, P2 was trying to represent his conceptual model in a statistical
model, but he was not aware that, given his conceptual model, he did not need to
account for the other variables to estimate the causal influence of Occupation
on Income (i.e., through a blocked path). What he really wanted to do, however,
was determine if he had evidence in support for his entire conceptual model,
which led him to include interactions which he falsely presumed would capture
the dependencies in the conceptual model. What P2 really needed was a structural
equation model. 

P2’s analysis experience illustrates a key observation: Without rTisane,
participants adopted a more exploratory approach. Trying out different models
and seeing how each one assessed their entire conceptual models. From one
perspective, analysts subtly changed their implicit analysis goal from assessing
the evidence of the influence a variable or set of variables on Income to the
seeing how the data supported their implicit assumptions. While statistical
models can suggest evidence (or lack thereof) for supporting a conceptual model,
to rigorously assess an entire conceptual model, analysts would really need to
author structural equation models, which we did not observe at all. Therefore,
we observe two things: First, rTisane keeps analysts focused on considering a
set of variables based on their domain knowledge whereas without, analysts
either struggle to author statistical models, try to justify analysis decisions
based on results, or author poor fitting statistical models. 

Second, to prevent faulty interpretation of statistical modeling results as
support for an entire conceptual model, rTisane should support analysts in
expressing a query to author SEMs to assess a conceptual model. 

% Participants thought their statistical models gave support for their conceptual
% models based on the statistical significance of variables in the authored
% models.

\theme{Without \rTisane, analysts find statistical model formulation challenging.}
Participants reported formulating and evaluating statistical models [2, 3, 5, 8,
12], programming [6, 13], and preparing data [7, 14] as the major challenges to
authoring statistical models without rTisane. 

For example, P3 explained how "There are a number of statistical tests and it
gets confusing if I don't practice it frequently. This is what happened today, I
haven't worked on a hypothesis testing problem recently and while I knew what
libraries to go to, I was not sure which test to implement."

For example, P5 explained how "A good understanding of all stats models to
choose from is required.” Similarly, discussing the nuanced of which covariates
to include in a statistical model given a conceptual model, P8 explained how he
was uncertain about which “upstream relationships” (indirect causes) to include
in a statistical model, so believed rTisane to “better” [look up quote]: “It
immediately feels harder doing it directly [without rTisane] like this” [P8].

\theme{Could combine with the above? \rTisane focused analysts on their motivation for analysis.}
% rTisane’s main impact was that it turned analysts’ focus away from details that
% can distract from formulating a statistical model towards their motivation for
% analysis. As a result, 
In contrast, participants reported that rTisane guided them to think
about their domains more [P2, P12], lightened their burden in authoring
statistical models [P10], and even promoted research transparency [P5] and
reproducibility [P4]. Furthermore, rTisane reinforced prior knowledge about statistical methods [P6,
P11] and helped participants learn more about GLMs [P4, 6, 7, 13]. P6, <who had
tried to author regression models on their own> explained how she could take
what she learned from using rTisane to future analyses: “I like that a
multivariate linear regression was used, because this will inform any future
data analysis approach that I may be able to work on.”

Nevertheless, participants expected \rTisane to do more automatically.
Participants were expected rTisane’s output statistical model to include more
IVs [P2, P5], have interaction terms [P5, P6], and have coefficient values that
were similar to ones from their own statistical models without rTisane [P10].
Yet, when asked if anything surprised them about rTisane’s output statistical
model, six participants said the models were as they expected [P4, 6, 10, 11,
12, 13]. 

% When using rTisane, P2 got as output a GLM with Occupation as the only IV and Income as the DV. While P2 wanted more of an explanation for why additional variables were not needed, he found that using rTisane made him think ``more about how things were related...more about relationships, what’s proven beforehand and what you’re looking to prove’’ [P2]. 
% This suggesting opportunities to not only explain the result of an analysis but also why the tool did the SM. 

\theme{}
When asked who analysts imagined would benefit from using rTisane, analysts
identified two groups. First, participants described how experienced and novice
analysts alike would benefit from using statistical models [P2, 4, 9, 10, 12].
Second, participants mentioned how conceptual models written using rTisane could
be used as boundary objects in collaboration with less technical stakeholders
[P8, P9]. P8 imagined how a conceptual model written using rTisane could be a
communication tool, saying how the “visual representation would play a role in a
dialogue with the PI.” P8 went on to imagine how he would like to use rTisane’s
conceptual model to generate process diagrams in scientific papers, suggesting
how rTisane’s conceptual model could serve as an intermediate representation for
multiple kinds of outputs, not just statistical models. 

\theme{\rTisane needs to provide more support for selecting family and link functions.}
Despite benefiting from rTisane, many participants had difficulty picking family
and link functions in rTisane [P2, 4, 5, 9, 10, 11]. Triangulating from how
participants repeatedly described rTisane as facilitating a higher-level of
thinking, we attribute the difficulty of selecting family and link functions to
the stark contrast between rTisane’s relatively high-level conceptual modeling
abstractions and the low-level family and link selection options. In the future,
it may be more usable for rTisane to suggest a specific pair and explain its
suggestion rather than require the analyst to pick.

Given that asking about family and link functions is a low-level statistical
implementation detail, the contrast between rTisane’s conceptual modeling
abstractions and the family and link selection options was particularly stark.
For example, P4 explained, ``I didn't understand the benefit or tradeoffs
between different specifications. It wasn't obvious to me how to create a linear
OLS regression, or why I would want to use a specification besides linear OLS.''
In the future, it may be more usable for rTisane to suggest a specific pair and
explain its suggestion rather than require the analyst to pick.

\theme{Analysts want to use \rTisane iteratively.}
Participants described their typical analysis approach without rTisane as
iterative. During the study, this looked like visualizing the data [P6, 10,
11, 12], assessing correlations between variables [P6, P10] to pick variables
for an initial statistical model, or starting with ``a full model first and
then trim down and compare'' [P11]. In the scaffold condition, analysts engaged
with the connection between their conceptual and statistical models [P2, 5,
12]. For example, P5 refined her conceptual model prior to implementing a
statistical model: ``Conceptual model matters because it gives something to
start with...how all of them have impacts...after have done analysis, look at
R-sq and p-value to interpret significance of these predictors.'' P5 also
grappled with how to interpret the statistical modeling results in light of her
conceptual model, especially since she PICK Background or Prediction as reason.
viewed the purpose of analysis to be able to predict income: ``Really don’t see
how statistical analysis helps us with why…'' She also explained how even though
she had a conceptual model, she did not feel she had the appropriate
background:``[if she had] more of a social science...background knowledge that
we have to dig deeper.'' While P5 could have interpreted her statistical models
in light of the conceptual model she authored, her observation about background
is nevertheless indicative of….


\subsection{Discussion}
rTisane allowed less-experienced analysts to author (check SM output: valid?)
statistical analyses that they would not have been able to author on their own
or would have taken them a lot more time. Two ingredients were critical: the
formalism provided in rTisane’s DSL and the conceptual model disambiguation
interface.

Benefit of formalism: Usability - communication tension Formalism as scaffolding
and reflection While rTisane does not address the challenges of domain
knowledge, it does focus analysts on aspects they can have control over.
[Address how rTisane does not address all the problems users face] While
unbounded expression (as tempting with ChatGPT) is enticing, but a key benefit
of rTisane is that it reduces the potentially infinite space of conceptual
relationships into a finite set that is expressible in the API. The API is good
enough to capture what people want to say. The formalism has the added benefit
that it provides constraints/scaffolds for end-users to explore the boundaries
of their domain knowledge. A map for reflection. 

The CM disambiguation interface enabled participants to ensure that their
expressed conceptual model is what they intended and think more deeply about
their domain. Participants emphasized using these conceptual models as boundary
objects with collaborators, especially those with less statistical expertise.
One participant, P8, even equated conceptual models to the process diagrams in
scientific papers. Put another way, conceptual models are useful intermediate
representation for not just statistical analysis but also communication, a
direction we are excited to pursue in the future. 

**important**Interpreted using the theory of hypothesis formalization, we find
that analysts author statistical models following a pattern similar to
hypothesis formalization. rTisane embodies hypothesis formalization, a core
activity in statistical authoring + skill in developing statistical expertise.
Therefore, we gain evidence of the benefits of supporting hypothesis
formalization in tools. 

Larger takeaway: ** Not about just the scaffolded steps but about the tool
support for executing each of these steps**

\subsubsection{Key takeaways}
\textbf{Larger takeaway: ** Not about just the scaffolded steps but about the tool support for executing each of these steps**}

\subsection{Limitations and Future Work on \rTisane}
While participants found rTisane helpful, they suggested three major areas of
improvement: (i) statistical model interpretation, (ii) iterative model
revision, (iii) system usability. 

Once analysts execute the output statistical model from rTisane, they find the
output results too low-level. Because rTisane uses lme4 under the hood, the
outputs are the default model outputs from lme4. However, given that rTisane’s
input language is at the conceptual level, analysts expected the outputs to at
least relate back to the conceptual model they input. In other words, the input
and output levels of abstraction should be commensurate. This support would
facilitate what analysts already try to do with statistical analyses they author
on their own without rTisane. P8 found the output from lme4 overwhelming,
saying, “Looking at the summary() in R was too much to look at.” He suggested a
simple way to tie the results back to his input conceptual model: “Would be nice
if you could have the same visual representation with p-values/coefficients!”

{While an output table with coefficient values might be appropriate for a
low-level library that takes as input low-level statistical model syntax, the
output should be commensurately high level with the input language Participants
wanted to know what the results meant and their implications for their
conceptual models. They wanted tighter/further connection between conceptual and
statistical models. For instance, quote…..}

Furthermore, while participants could iterate on their conceptual models by
adding or removing variables and relationships, they could not engage in a
larger iteration loop with their output statistical model from rTisane.
Improving statistical result interpretation would help with model iteration. In
addition, participants also sought more direct support. For instance, P11
described the rTisane output statistical model as “an initial or baseline model
but follow-up evaluation of the model is needed.” They wanted to “go back and
tweak things a bit” about their statistical model. This kind of model iteration
is not only typical of the participants’ workflows but also even a best practice
recommendation from the statistics community~\cite{bayesian workflow}.
Supporting novice and more experienced analysts revise models will likely
require different levels of abstraction and automation. 

Finally, participants found going back and forth between code and an interface
outside their IDE complicated and ``clunky.'' While part of this may have been
in part due to the fact that participants were using rTisane on a remote
desktop, embedding rTisane in a notebook seems likely to reduce major usability
issues. Additionally, participants gave suggestions for syntactic sugar for
specifying conceptual models. For example, instead of specifying multiple causes
and relates statements, they wished they could batch specify and add them to the
conceptual model. Ways to reduce the specification burden for analysts by
providing syntactic sugar or even removing the need to program at all are
interesting avenues to explore. 
