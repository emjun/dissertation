\section{Summative Evaluation: Controlled lab study}
% \highlight{We are in the process of running this lab study and collecting data.}

We had three research questions:

\begin{itemize}
    \item \evalConceptualModels What is the influence of rTisane on conceptual
    modeling?
    \item \evalStatisticalModels How does rTisane impact the statistical models
    analysts implement? Specifically, do the covariates, family functions, and
    link functions analysts include/exclude differ when implementing statistical
    models on their own vs. using rTisane? How well do statistical models
    analysts author on their own vs. using rTisane fit the data?  
    \item \evalLearning Do analysts learn about their discipline or data
    analysis as a result of using rTisane?
\end{itemize}

\subsection{Study design}
% \todo{Include a diagram summarizing study design?}
We conducted a within-subjects think-a-loud lab study that consisted of four phases: 

\begin{itemize}
    \item \textbf{Phase 1: Warm up.} We presented participants with the
    following open-ended research question: ``What aspects of an adult’s
    background and demographics are associated with income?'' We asked
    participants to specify a conceptual model including variables they thought
    influenced income. This warm-up exercise helped to externalize and keep
    track of participants' pre-conceived notions and assumptions prior to seeing
    a more restricted data schema.
    \item \textbf{Phase 2: Express conceptual models} We presented participants
    with a data schema describing a dataset from the U.S. Census Bureau. We then
    asked participants to specify a conceptual model using only the available
    variables. At the end, we asked participants about their
    experiences specifying their conceptual models in a brief survey and semi-structured interview.
    \item \textbf{Phase 3: Implement statistical models} We asked participants
    to implement ``a statistical model that assesses the influence of variables
    [they] believe to be important (in the context of additional potentially
    influential factors) on income,'' relying on only their conceptual model. We
    then asked participants about their experiences implementing statistical
    models through a brief survey and semi-structured interview. 
    \item \textbf{Phase 4: Exit interview.} The study concluded with a survey
    and semi-structured interview where we asked participants to reflect on the
    process of explicitly connecting conceptual models to statistical models.
\end{itemize} 

We designed the study based on the assumption that conceptual modeling is a
helpful strategy when specifying statistical models. As a result, all
participants completed the phases in the above order. In order to assess the
effect of tooling on conceptual models and the quality of statistical models, we
counterbalanced the order in which participants specified conceptual and
statistical models. Half the participants specified their conceptual and
statistical models on their own (without rTisane) first. The other started with
rTisane.

\noindent \paragraph{Participants} We recruited \ej{24?} data analysts on
Upwork. We screened for participants who reported having experience with
authoring generalized linear models and using R. \ej{On average, participants
reported having conducted N projects using generalized linear models.
Participants were familiar with a range of software tools, including...} All
studies were conducted over Zoom. Each participant was compensated \ej{\$50} for
120 minutes of their time. We recorded participants' screens, video, and audio
throughout the study. We then transcribed the audio for qualitative analyses.

% One dataset was on demographic factors and income in the U.S. in 2018 from the
% U.S. Census Bureau (\datasetIncome). The other dataset was about demographic
% factors and health conditions in the U.S. (\datasetHealth) from \ej{FILL IN}.
% More information on how the datasets were created are found in the
% supplemental material.

\subsection{Analysis and Results}
\highlight{Results will be added once we have collected and analyzed the data over Summer 2023.}
We collected and analyzed quantitative and qualitative measures to answer our
research questions. For each research question, we describe our analysis
approach and results below. 
% We qualitatively analyzed conceptual models, statistical models, audio
% recordings, and open-ended survey responses. Whenever possible, we also
% quantified the frequency of characteristics in conceptual models and
% statistical models across participants. For the statistical models
% participants authored, we compared the AIC and BIC scores for matched pairs of
% the independent variable of interest. 


\subsubsection{\evalConceptualModels}
% Consistency within vs. across participants 
We qualitatively analyzed how consistent participants' conceptual models were
between conditions. We noted common challenges translating free-form conceptual
models into rTisane programs. We also thematically analyzed participant
transcripts and survey responses describing the influence of rTisane on their
conceptual modeling processes. 

% \subsubsection{\evalConceptualModelAuthoring}
% % To understand rTisane's influence on the conceptual modeling process, we
% % compared the ratings and thematically analyzed survey responses to questions
% % asking how participants decided what conceptual relationships to specify.
% At the end of Phases 2 and 3, we asked participants to rate and describe their
% experience specifying conceptual models without and with rTisane, respectively.
% In addition, we kept track of the conceptual modeling challenges participants
% vocalized and noted how participants overcame these challenges in-situ. We
% compared the survey responses, observed challenges, and observed approaches
% between Phases 2 and 3 to assess how rTisane influenced the conceptual modeling
% process. 

\subsubsection{\evalStatisticalModels}
We used AIC, BIC, and R-squared values to assess how well statistical models
authored with vs. without rTisane fit the data. We used rTisane to statistically
model and assess the influence of rTisane on AIC, BIC, and R-squared values.

We also thematically analyzed participants' reactions to the similarities,
differences, and surprises between statistical models. 

% \todo{Add statistical models executed}
% \todo{Add rTisane analysis script to supplemental material}

\subsubsection{\evalLearning}
Finally, we gauged participants' learning based on a thematic analysis of
open-ended survey interview answers at the end of the study. 

\subsection{Limitations}
To limit the number of language constructs in \rTisane introduced, we only
assessed language constructs for specifying a GLM. Given that the summative
evaluation was really focused on the core of \rTisane--the impact of conceptual
modeling on statistical modeling--we expect the results to apply. In fact, for
more complex data collection procedures that require mixed-effects, \rTisane may
have an even larger effect on statistical models and learning. 