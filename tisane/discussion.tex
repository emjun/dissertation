\section{Discussion}
Iteratively designing, developing, and evaluating \tisane's key insight---to
compile statistical models from conceptual models---led to \rTisane. A key step
in designing \rTisane was the exploratory lab study that used \tisane as a
probe. The exploratory lab study suggested the need to allow analysts to express
their conceptual models more granularly. Although obvious in hindsight, this
finding was \textit{counterintuitive} at the time. 
% \polish{I agree they are more granular, not as clear that they are "lower level". Can you elabore?}

% \polish{Tone down: A widely held belief, especially within the HCI community, is that the higher
% the level of abstraction for a task, the better for end-users. However, we saw
% the opposite.} 
Statistical non-experts engaged deeply with conceptual models
about their domain and wanted to be more detailed and specific when describing
their conceptual models. In other words, while the focus on the abstraction
should be at the conceptual level, within that, analysts want to fluidly move between levels of detail. Indeed, in the summative evaluation of \rTisane,
we saw that analysts made use of all language constructs and reported finding
them instrumental to deepening their consideration of domain knowledge.
Arguably, the conceptual modeling language constructs also benefited the quality
of statistical models output from \rTisane. Using \rTisane, analysts authored
statistical models that fit the data better than their own or authored identical
statistical models after using \rTisane.

Based on my experience with \tisane and \rTisane, I speculate that abstractions
can achieve usability and rigor by matching the content-focus of end-users while
giving them opportunities to get into the details of specifications.
This gives end-users the agency to express themselves more fully, transforming
the programming task from strictly a means to an end to a meaningful, reflective
activity in itself. 

% Second, Tisane's specification process could be more tiered, and disambiguation
% could leverage ambiguity in analysts' specification as opportunities for more
% numerous intelligent suggestions and guidance. 
% \todo{Mention that the focus needs to be on conceptual level, but within conceptual level, there should be opportunity to move up and down the ladder of conceptual abstraction}

\begin{comment}
    **important**Interpreted using the theory of hypothesis formalization, we find
    that analysts author statistical models following a pattern similar to
    hypothesis formalization. rTisane embodies hypothesis formalization, a core
    activity in statistical authoring + skill in developing statistical expertise.
    Therefore, we gain evidence of the benefits of supporting hypothesis
    formalization in tools. 
    Larger takeaway: ** Not about just the scaffolded steps but about the tool
    support for executing each of these steps**
    
    \textbf{Ideally, this section discusses key findings from Tisane/case studies + rTisane/summative evaluation}
    
    \textbf{Expand support from \rTisane to GLMMs:} To limit the number of language constructs in \rTisane introduced, we only
    assessed language constructs for specifying a GLM. Given that the summative
    evaluation was really focused on the core of \rTisane--the impact of conceptual
    modeling on statistical modeling--we expect the results to apply. In fact, for
    more complex data collection procedures that require mixed-effects, \rTisane may
    have an even larger effect on statistical models and learning. 
\end{comment}

\begin{comment}
In designing \rTisane, a key challenge was in finding the right point to bring
in lower-level statistical modeling details. Concretely, in \rTisane analysts at some point must grapple with graphical and
mathematical representations in the disambiguation phases. This is because it
was not possible to remove all complexity from statistical modeling without the
risk of losing the analyst's sense of control or understanding. Thus, our focus
has been to strip away unnecessary complexity and help analysts navigate through
necessary complexity by designing informative abstraction lowering
disambiguating steps. It may be possible to avoid any interactive disambiguation
by executing all possible statistical models given an input, likely ambiguous,
conceptual model. Although this approach would accomplish a different objective
than our goal of compiling a specific conceptual model into a specific
statistical model, this approach may give greater insight into if analysts
really want, need, or benefit from disambiguation. 

%\polish{Design pattern for balancing usability and rigor}
Our approach in \rTisane is to
prioritize expressivity in the DSL and precision during interactive compilation. 

%\polish{Future work: Suggest unobserved variables, more complex causal structures, perhaps based on similarities from a field/other analysts/other sources}

%\polish{Discussion of rTisane results}
Novel for non-expert audience: formalizing a porcess that is innate among statistical experts


Despite the goal to lower the
barriers to statistical specification, at some point, 

Another approach to explore in the
future may be to eliminate the need to engage analysts in disambiguation and
instead execute all possible statistical models given an input, likely
ambiguous, conceptual model. Although this approach would accomplish a different
objective than the goal here of compiling a specific conceptual model into a
specific statistical model, this approach may give greater insight into the need 

% Our answer was to have informative conceptual and
% statistical model disambiguation phases. 
robustnes of a particular effect in light of many possible conceptual models and
explanations.

% push on the idea that analysts care about conceptual ramifications and avoid asking them to 

In the future, there could be additional exploration into authoring a multiverse
of all possible statistical models given a specific ambiguous conceptual model.
This would accomplish a different objective than the goal of \tisane (and
\rTisane), which is to compile a specific conceptual model into a specific
statistical model for/with the end-user. The multiverse would help assess the
robustnes of a particular effect in light of many possible conceptual models and
explanations.

Concretely, it
took us several iterations to answer the question: What is the right point to
introduce cycle breaking and modeling to the end-user? 

**not remove all complexity but rather focus end-users on necessary complexity
and guide their thinking/help them navigate that complexity. 

\end{comment}