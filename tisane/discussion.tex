\section{Discussion, Limitations, and Future Work}
The exploratory lab study suggested the need to allow analysts to express their
conceptual models using more granular, low-level functions. Although obvious in
hindsight, this finding was \textit{counterintuitive} at the time. A widely held
belief, especially within the HCI community, is that the higher the level of
abstraction for a task, the better for end-users. However, we saw the opposite.
Statistical non-experts engaged deeply with conceptual models about their domain
and wanted to be more detailed and specific. In other words, while the focus on
the abstraction should be at the conceptual level, within that, analysts want to
move up and down the ladder of abstraction. More generally, our iterative
language design work with \tisane and \rTisane suggests that as long as
abstractions match the content-focus of end-users, there should be opportunities
to get low-level within those abstractions. This gives end-users the agency to
express themselves more fully, transforming the programming task from strictly a
means to an end to specification as a meaningful activity in itself. 

% Second, Tisane's specification process could be more tiered, and disambiguation
% could leverage ambiguity in analysts' specification as opportunities for more
% numerous intelligent suggestions and guidance. 
% \todo{Mention that the focus needs to be on conceptual level, but within conceptual level, there should be opportunity to move up and down the ladder of conceptual abstraction}

In designing \rTisane, a key challenge was in finding the right point to bring
in lower-level statistical modeling details. Concretely, in \rTisane analysts at some point must grapple with graphical and
mathematical representations in the disambiguation phases. This is because it
was not possible to remove all complexity from statistical modeling without the
risk of losing the analyst's sense of control or understanding. Thus, our focus
has been to strip away unnecessary complexity and help analysts navigate through
necessary complexity by designing informative abstraction lowering
disambiguating steps. It may be possible to avoid any interactive disambiguation
by executing all possible statistical models given an input, likely ambiguous,
conceptual model. Although this approach would accomplish a different objective
than our goal of compiling a specific conceptual model into a specific
statistical model, this approach may give greater insight into if analysts
really want, need, or benefit from disambiguation. 


\polish{Discussion of rTisane results}
Novel for non-expert audience: formalizing a porcess that is innate among statistical experts

\begin{comment}

Despite the goal to lower the
barriers to statistical specification, at some point, 

Another approach to explore in the
future may be to eliminate the need to engage analysts in disambiguation and
instead execute all possible statistical models given an input, likely
ambiguous, conceptual model. Although this approach would accomplish a different
objective than the goal here of compiling a specific conceptual model into a
specific statistical model, this approach may give greater insight into the need 

% Our answer was to have informative conceptual and
% statistical model disambiguation phases. 
robustnes of a particular effect in light of many possible conceptual models and
explanations.

% push on the idea that analysts care about conceptual ramifications and avoid asking them to 

In the future, there could be additional exploration into authoring a multiverse
of all possible statistical models given a specific ambiguous conceptual model.
This would accomplish a different objective than the goal of \tisane (and
\rTisane), which is to compile a specific conceptual model into a specific
statistical model for/with the end-user. The multiverse would help assess the
robustnes of a particular effect in light of many possible conceptual models and
explanations.

Concretely, it
took us several iterations to answer the question: What is the right point to
introduce cycle breaking and modeling to the end-user? 

**not remove all complexity but rather focus end-users on necessary complexity
and guide their thinking/help them navigate that complexity. 

\end{comment}