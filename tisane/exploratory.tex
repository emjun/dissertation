\def\unit{\texttt{Unit}\xspace}
\def\measure{\texttt{Measure}\xspace}
\def\setup{\texttt{SetUp}\xspace}

\section{Elicitation lab study} \label{sec:exploratoryStudy}
% Motivation
% Our motivating question was, ``What implicit semantics do analysts use to construct conceptual models of how variables relate?''. 
Our aim was to increase the expressivity of rTisane to represent analysts'
implicit domain knowledge. We used the first release of
\tisane~\cite{jun2022tisane} to probe analysts' internal processes and derive
design goals~\autoref{sec:rtisane_design_implications} for re-designing \tisane.

\subsection{Method}
%Participants
We recruited participants through a graduate-level quantitative research methods
course as a convenience sample to control recent exposure to
statistical concepts. Five computer science PhD students participated.
% All participants self-reported that they
% conducted research in programming languages or software engineering and planned
% to perform statistical analyses in future research projects. 
% \ej{Add: Approaches, mindsets of participants as they were completing the study}

% On a scale from
% one (not at all relatable) to five
% (extremely relatable), four participants indicated that the paper was relatable
% with a score of 3 or higher.
The study consisted of two parts: (i) a take-home assignment and (ii) an in-lab
session. The take-home assignment asked participants to read a recently
published CHI paper~\cite{winters2021heartbeat}\footnote{We chose the specific
paper because we believed its topic (i.e., empathetic biosignals) would be
broadly approachable and the statistical methods used (i.e., generalized linear
models) would be familiar with students enrolled in the research methods
course.} and describe the paper's research questions and hypotheses, the
authors' conceptual models, the study's design, and ways to analyze the data to
answer the research questions. The assignment was designed to ensure that
participants engaged with the paper's key ideas before coming into the lab. The
researcher reviewed each submission to prepare participant-specific questions
for a semi-structured, think-a-loud lab session. 

% used Tisane~\cite{jun2022tisane}, an open-source package for authoring
% generalized linear models with or without mixed effects, as a probe to
% understand how participants thought about and wanted to express their conceptual
% models and study designs. We 

At the start of the lab session, participants reviewed their homework submission
to remind themselves of the paper. The paper and participants' homework
responses remained available for reference throughout the study. Then,
participants completed three tasks: (i) declaring variables, (ii) expressing
conceptual models, and (iii) specifying study designs. For each task,
participants started with \tisane's language constructs to express their intent
and discussed their confusions, how they understood each presented construct,
and what they wanted to specify but could not (if applicable).\footnote{For the
lab study, we re-implemented \tisane (originally in Python) in R since (i) R is
a widely used programming language for data science and (ii) the research
methods course taught and used R.} The researcher repeatedly reminded
participants that the constructs presented were prototype possibilities and that
expressing their intentions was more important than using the constructs or
getting the syntax correct. Throughout, the researcher paid particular attention
to where \tisane broke down for participants and asked follow-up questions to
probe deeper into why. The researcher considered such breakdowns as openings
into semantic mismatches between the end-user and the DSL.

% Analysis 
We iteratively coded homework submissions, audio transcripts from the lab study
sessions, and lab study artifacts. We also consulted the
researcher's detailed notes from the lab sessions. 

% Based on an iteraWe conducted a content analysis of the produced code suggestions (output
% artifacts) along with a thematic analysis of the audio transcripts from each lab
% session. 

\subsection{Key Observations}
\polish{Parallel order!}
All participants demonstrated a working knowledge of the assigned paper's
motivating research questions, study design, and general study procedure. 
% Among
% the submitted conceptual models, four were a list of conceptual relationships,
% and one was a diagram. 
We made four key observations about what and how statistical non-experts wanted
to express their conceptual models: using varying degrees of specification,
separating moderation from bivariate relationships, distinguishing between known
and hypothesized relationships, and considering alternative conceptual models.
Participants also suggested syntactic sugar options to improve the DSL's
usability. Based on these observations, we derived design goals for re-designing
\tisane (see~\autoref{sec:rTisane}).

\subsubsection{Participants express conceptual knowledge with varying details.}
Contrary to the popular belief that higher levels of abstraction are better for
end-users, we found that statistical non-experts want to
move up and down the ladder of abstraction when expressing conceptual models. 

When defining ``causes,'' P2 described ``[Causes] is...like when we teach
logic...it's like implication, right?....So I'm saying if we are observing an
emotion and...emotion observed can lead to a change in emotional perspective.''
P0, P1, and P3 contrasted a bidirectional relationship between variables,
formerly encapsulated in the \texttt{associates\_with} construct in \tisane, to
their implicit understanding of ``causes.'' For instance, P1 stated ``the most
like, utilitarian definition by if A causes B, then by changing A, I can change
B whereas \texttt{associates\_with} means that...if I can turn dial A, B might
not change.'' In addition to differentiating between causal and associative
relationships, three participants [P0, P1, P3] provided statements of
\textit{specifically how} a variable influenced another in the conceptual models
submitted as homework. For example, P0 wrote, ``Hearing a heartbeat that seems
to be aligned with visual cues makes someone feel \textit{more} strongly what
another person is feeling'' (emphasis added), specifying a positive influence of
``hearting a heartbeat'' on empathy. These observations suggest that analysts
have an intuitive understanding of causality but bluntly stating that a variable causes
another does not capture the richness or nuance of their implicit domain knowledge. 
Additional annotations about how a variable influences another are necessary.

%   in addition to vaguer statements without a clear
% positive/negative direction, such as ``Visual cues directly influence a person's
% perception of another's emotions'' [P0]. 
% The homework assignment and lab study materials may have primed participants to
% think about ``direct'' and ``indirect'' relationships and commutativity, so
% analysts' descriptions of causal relationships as ``unidirectional'' and
% associative relationships as ``bidirectional'' were not surprising [P0, P2, P3].

% how a bidirectional relationship between two variables 
% indicative of proxies measuring the same latent construct (``...for two things
% to be bi-directional and for it to be a really, really direct relationship. Like
% that just never happens in the real world without them turning out to be like
% proxies for the same exact thing.'') [P3]; and that 

\subsubsection{Participants find moderation difficult to separate from bivariate relationships.}
Participants consistently found \tisane's \texttt{moderates}
construct difficult to understand [P0, P1, P2, P3]. Participants expressed
confusion about what moderation implied about the relationship between
two variables. For example, P3 grappled with if ``moderates'' was shorthand for
expressing associative relationships between each independent variable and the
dependent variable, how moderation implies causal relationships, and if
statistical and conceptual definitions of moderation differed from each other:
``[L]et's say there's two independent variables and one dependent variable. And
each of the [independent] variables individually is not correlated with the
outcome. But if you put them together, then the correlation appears....I mean,
it's sort of a philosophical question of whether, like each of the ones
individually causes [the dependent variable] in that case. But thinking from
a...statistical perspective, I think that's a situation where you might be able
to express...language and experience level together cause lines of code but
individually they don't because no individual correlation would appear there.''
Therefore, a clear delineation between bivariate relationships and partial statistical specifications of interaction terms is necessary. 
% that (i) the semantics of moderation is unclear in
% Tisane and (ii) moderation may be easier to reason about when decomposed into
% what happens to the dependent variable when two or more independent variables
% have specific values. 
% \paragraph{Design implication: Too high level --> Remove moderates}

\subsubsection{Participants distinguish between known and suspected relationships.}
% [P0, P1, P2, P3, P4]
Participants described relationships established in prior work as
``assumptions'' or ``assertions'' to check separately from the key research
questions that tested ``suspected'' relationships. P0 described how ``maybe we
have to differentiate as to like the \textit{known} [relationships] are kind of
the things you're \textit{assuming} there's relationships between these things
whereas the \textit{suspected}...[are] the things kind of like your research
questions are saying like, `We \textit{think} there's this relationship
but...it's what we're testing for'' (emphasis added). Similarly, P4 suggested
that Tisane should warn end-users when assumptions about known relationships are
violated in a given data set: ``I would also say that it would be very handy to
be able to say, kind of \textit{assert} that language has no effect on the line
of code. And be warned if it's not the case, like if your \textit{assertion} is
not...verified automatically with the DSL, but warned...that while your
\textit{assumption} is not holding there is actually an effect, which could be
very handy on your study.'' (emphasis added). The inability to indicate
relationships that are either known or suspected in \tisane may explain why
analysts repeatedly preferred less technical verbs, such as ``influences'' [P0]
or ``leads to'' [P3]. For instance, P0 explained how she preferred
``influences'' over ``causes'' because ``I guess it's like \textit{a level of
sureness} in it in which, like, `cause' feels more confident in your answers
than `influences''' (emphasis added). Providing a way to label conceptual
relationships as assumptions or the focus of the present analysis could make
\texttt{causes} and \texttt{associates_with}, the bivariate relationships in
\tisane, more approachable. 
% \paragraph{Design implication: Provide constructs for distinguishing betwen known and suspected relationships -->  assume/hypothesize}

% However, Tisane does not differentiate between known
% and suspected relationships, and as P4 noted, ``I don't know if everything that
% is not either specified to moderate, associate, or causes is by default,
% asserting that there is no effect,'' suggesting that the Tisane's DSL semantics
% could be more self-evident. In fact, 

\subsubsection{Participants consider alternative conceptual structures in the face of ambiguity.}
% In addition to detailing how variables influence one another, participants also
Participants grappled with what specific structures in a conceptual model meant. P1 and P3
described how a bidirectional relationship between two variables were really due
to hidden, confounding variables causing both variables. P3 described how ``in
the real world, when, when these bidirectional things happen, it means there's
sort of this middleman complex system. Or some like underlying process of which
[two variables are] both components...'' Another participant, P2, wondered aloud
about how even what appears like a direct relationship, may actually be a chain
of indirect or mediated relationships at a lower granularity: ``It's like Google
Maps. If you zoom out enough, that arrow becomes a direct arrow.'' These
observations suggest that while participants can deeply reflect on what could be
happening between variables conceptually, they need help exploring and
figuring out which of these structures matches their implicit understanding.
% P3's definition of associative relationships
% is consistent with an associative structure called a ``fork'' in causal DAGs.
% \paragraph{Need help exploring and picking among possible structures.}


\begin{comment}
\subsubsection{More expressivity for specifying study designs/experimental design}
**keep short**
Additional observations about expressing study design
More future work for how to express study designs

TODO: Participant as a separate construct
\end{comment}

\subsubsection{Participants expected more syntactic sugar for specifying data collection details.}
While our focus was on improving the support for conceptual modeling, we made a
few observations about challenges analysts faced when specifying data collection
details. First, analysts expected \textit{experimental conditions to be
standalone concepts}. In Tisane, experimental conditions can be specified as a
\measure of a \unit. Instead, P0 and P4 had separate conceptual categories for
conditions and measures in their mental models of study designs. P4 preferred a
separate condition data type currently unavailable in Tisane because the term
``Measure'' did not create a ``bucket'' appropriate for conditions. Second,
participants were interested in specifying trials, stimuli, and responses
elicited during each trial alongside participants: ``I want to have a trial
unit that is nested within trials, which is nested within or maybe I could just
have trial nested within Participant, but I'm not seeing a way to clearly
delineate or like to denote that'' [P1]. Future work should more closely examine
and iterate on language constructs and idioms for representing data collection
procedures. 

% In Tisane, nesting a trial within a
% participant would mean that multiple trials exist in each participant (e.g.,
% akin to how multiple children exist in a family), which even in natural
% language, does not capture what P1 wanted to convey, which was that each
% participant sees and provides responses for each trial. 
% \paragraph{Participant construct - syntactic sugar for Unit}

% P4 resorted to declaring
% condition as a \setup variable, violating the intended semantics of \setup
% variables in Tisane: ``Like it doesn't feel like the condition are measured.
% Right? I'm measuring the proxy. I'm not measuring the condition. So to me,
% that's why I'm putting the visual on the four condition as a...\setup
% variable.'' Similar to P4, when P0 saw the ``SetUp'' class, she thought it
% described conditions: ``[I]f I heard the word `setup,' I would think of more
% like conditions like, like, this person uses this IDE or this person is given
% this IDE.'' 

\subsection{DSL Re-design Goals} \label{sec:rtisane_design_implications} 

Based on our lab study observations, we derived four design goals for
re-designing Tisane's DSL to more accurately capture analysts' implicit
conceptual models: 
%  about how
% participants internally represent conceptual models and study designs have three
% implications for improving the design of Tisane.

\def\optionalSpecificity{\textbf{DG1 - Optional specificity}\xspace}
\def\interactionAsPartialSpec{\textbf{DG2 - Interactions as partial specifications}\xspace}
\def\considerPossibilities{\textbf{DG3 - Consideration of possibilities}\xspace}
\def\assumeHypothesize{\textbf{DG4 - Distinction between assumed and hypothesized}\xspace}
\begin{itemize}
    \item \optionalSpecificity: Analysts should be able to provide optional
    details about how variables change in relation to each other (e.g., positive
    or negative changes in values) when describing conceptual relationships.
    \item \interactionAsPartialSpec: Analysts should annotate conceptual models with interaction terms they want to include in an output statistical model. 
    \item \item \considerPossibilities: When expressing ambiguous relationships, analysts should have support
    in considering and picking among multiple possible conceptual structures.
    \item \assumeHypothesize: Analysts should be able to distinguish between assumed and hypothesized relationships in their conceptual models. 
\end{itemize}

In the second release of \tisane, we addressed these goals through new language
constructs. We also supported syntactic sugar to more accurately capture
study design details. 

\begin{comment}
\textbf{Deconstruct statistical constructs for clarity.}
Unsurprisingly, we found that class and function names such as ``Measure,''
``SetUp,'' and ``moderates'' confused participants. These names either suggested
informal conceptual categories or were too close to statistical jargon. As a
result, connecting a Tisane program's semantics to a mental model of study
design was challenging. Particularly insightful was participants' consternation
with the function ``moderates.'' Participants could reason about the details of
moderation relationships but did not know what the term meant or how to use it
to clearly communicate their conceptual models in detail. Using
specific,granular language constructs (e.g., pairwise relationships) and
allowing for system-gudiance to consider more complex statistical structures
(e.g., interaction effects for moderation) may help analysts more accurately
understand and use Tisane.

\textbf{Allow for conceptual ambiguity, make specificity optional.}
Despite being statistical non-experts, participants were noticeably focused on
low-level details about conceptual relationships. All participants
differentiated between known and suspected relationships and generally agreed
that known relationships should be checked for prior to assessing the suspected,
or ambiguous, ones. At the same time, when participants were confident about
some relationships, they were specific about if the relationships were
positive/negative. Welcoming ambiguity and specificity could help analysts write
and use conceptual models throughout data analysis and interpretation.

Both design implications suggest the need to re-structure Tisane's programming
and interaction models. First, Tisane's specification language needs to be more
specific and low-level. This finding is \textit{counterintuitive} because a
well-documented strategy for making programming tasks easier for non-experts has
been to raise the level of abstraction for a programming
domain~\cite{chasins2018rousillon,satyanarayan2017vega}. Yet, we have evidence
that suggests the opposite for specifying a conceptual model and study design.
Second, Tisane's specification process could be more tiered, and disambiguation
could leverage ambiguity in analysts' specification as opportunities for more
numerous intelligent suggestions and guidance. 

\todo{Mention that the focus needs to be on conceptual level, but within conceptual level, there should be opportunity to move up and down the ladder of conceptual abstraction}

%hand-offs to a reasoning engine to suggest possible analysis paths, for example. 

% the conceptual modeling and study design specification process to be more tiered

\end{comment}