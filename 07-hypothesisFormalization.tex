If you are copying and pasting material from one of your papers, then remember to:
\begin{itemize}
    % \item Remove the abstract and instead add a little overview of the chapter and how it ties in to the rest of the thesis. You should also mention the original paper's source like: ``This chapter includes materials originally published in $\backslash$citet\{myownppr\}''
    % \item Make sure the formatting still works -- this is single column now!
    \item Consider rephrasing conference-paper-style language:
    \begin{itemize}
        \item Find every place you mention some variation of ``in this paper'' and say ``in this chapter'' instead.
        \item Remove or rephrase the parts where you talk about ``our main contributions''.
        \item Rephrase the language describing code and data releases.
    \end{itemize}
    % \item Replace the conclusion section with a summary section. Again, you should tie this chapter back to the main themes of the thesis.
\end{itemize}

{\color{orange} Write an overview of chapter, how it ties to the rest of the thesis.}

{\color{orange} ``This chapter includes materials originally published in $\backslash$citet\{myownppr\}''}

Although prior work has observed \textit{that} data analysis is
iterative~\cite{liu2019paths,grolemund2014cognitive} and involves multiple
levels of considerations~\cite{liu2019understanding}, \textit{how} analysts move
between these cognitive, statistical, and computational realities iteratively
has remained under-scrutinized. To fill this gap, we conducted a content analysis of 50 published
empirical publications from diverse disciplines and a lab study with 24 data
scientists authoring analyses to identify the steps and challenges involved in
authoring analyses. 

Based on our content analysis and lab study findings, we coin and define
\textit{hypothesis formalization} as a dual-search process~\cite{klahr1988dual}
that involves developing and integrating cognitive representations from two
different perspectives---conceptual hypotheses and concrete model
implementations. Analysts move back and forth between these two perspectives
during formalization while balancing conceptual, data-driven, statistical, and
programming implementation constraints. Analysts iterate over conceptual steps
to refine their hypothesis in a \textit{hypothesis refinement loop}. Analysts
also iterate over computational and implementation steps in a \textit{model
implementation loop}. Data collection and data properties may also prompt
conceptual revisions and influence statistical model implementation.

\section{Background and Related Work} \label{sec:relatedWorkHypoForm}
Our work integrates and builds up existing theories of statistical thinking in
cognitive psychology and statistics. We also situate hypothesis formalization in
the larger context of scientific discovery. 
% frameworks of scientific
% discovery, theories of sensemaking, statistical practices, and empirical studies
% of data analysts.

\subsection{Statistical Thinking} 
Statistical thinking and practice require differentiating between
\textit{domain} and \textit{statistical} questions. The American Statistical
Association (ASA), a professional body representing statisticians, recommends
that universities teach this fundamental principle in introductory courses (see
Goal 2 in~\cite{carver2016guidelines}). Similarly, researchers Wild and Pfannkuch emphasize the importance of
differentiating between and integrating statistical knowledge and context (or
domain) knowledge when thinking
statistically~\cite{pfannkuch1997statistical,pfannkuch2000statistical,wild1999statisticalThinking}.
They propose a four step model for operationalizing ideas (``inklings'') into
plans for collecting data, which are eventually statistically analyzed. In their
model, analysts must transform ``inklings'' into broad questions and then into
precise questions that are then finally turned into a plan for data collection
(see Figure 2 in~\cite{wild1999statisticalThinking}). Statistical and domain
knowledge inform all four stages. However, it is unknown what kinds of statistical and domain
knowledge are helpful, how they are used and weighed against each other, and
when certain kinds of knowledge are helpful to operationalize inklings. Our
work in defining hypothesis formalization provides more granular insight into Wild and Pfannkuch's
proposed model of operationalization and aims to answer when, how, and what
kinds of statistical and domain knowledge are used during statistical data
analysis. 

More recently, in \textit{Statistical
Rethinking}~\cite{mcelreath2020statistical}, McElreath proposes that
there are three key representational phases involved in data analysis:
conceptual hypotheses, causal models underlying hypotheses (which McElreath
calls ``process models''), and statistical models. McElreath, like the ASA and
Wild and Pfannkuch, separates domain and statistical ideas and discusses the use
of causal models as an intermediate representation to connect
the two. McElreath emphasizes that conceptual hypotheses may correspond to
multiple causal and statistical models, and that the same statistical
model may provide evidence for multiple, even contradictory, causal models and
hypotheses. McElreath's framework does not directly address how analysts navigate
these relationships or how computation plays a role, both of which we take up in
this chapter. 

Overall, our work provides empirical evidence for prior frameworks but also (i)
provides more granular insight into \textit{how} and \textit{why} transitions between
representations occur and (ii) scrutinizes the role of
\textit{software and computation} through close observation of analyst workflows
in the lab as well as through a follow-up analysis of statistical software. Based on
these observations, we also speculate on how tools might better support hypothesis
formalization.

\subsection{Statistical data analysis as part of scientific discovery}
Klahr and Simon characterized scientific discovery as a dual-search process
involving the development and evaluation of hypotheses and
experiments~\cite{klahr1988dual}. They posited that scientific discovery
involved tasks specific to hypotheses (e.g., revising hypotheses) and to
experiments (e.g., analyzing data collected from experiments), which they
separated into two different ``spaces,'' and tasks moving between them, which is
where we place hypothesis formalization. Extending Klahr and Simon's two-space
model, Schunn and Klahr proposed a more granular four-space model involving data
representation, hypothesis, paradigm, and experiment
spaces~\cite{schunn1995FourSpace,schunn1996BeyondTwoSpace}. In the four-space
model, conceptual hypothesizing still lies in the hypothesis space, and
hypothesis testing and statistical modeling lies in the paradigm space. As such,
hypothesis formalization is a process connecting the hypothesis and paradigm
spaces. In Schunn and Klahr's four-space model, information flows
unidirectionally from the hypothesis space to the paradigm space. We extend this
prior research with evidence that the path from hypothesis and paradigm spaces
is actually bidirectional (see~\autoref{figure:hypoFormOverview}).

Figure~\ref{figure:priorWork} augments Schunn and Klahr's
original diagram (Figure 1 in~\cite{schunn1995FourSpace}) with
annotations depicting how our content analysis of research papers and lab study
triangulate a tighter dual-space search between hypothesis and
paradigm spaces with a focus on hypothesis formalization. Our mixed-methods
approach follows the precedent and recommendations of Klahr and
Simon's~\cite{klahr1999studies} study of scientific discovery activities.

% Klahr and Simon characterized scientific discovery as a dual-search process
% involving the development and evaluation of hypotheses and
% experiments~\cite{klahr1988dual}. They posited that scientific
% discovery involved tasks specific to hypotheses (e.g., revising hypotheses) and
% to experiments (e.g., analyzing data collected from experiments), which they
% separated into two different ``spaces,'' and tasks moving between them, which is
% where we place hypothesis formalization.

% exploratory data analysis would be an
% activity that impacts how analysts view, or represent, their data mentally, and
% Extending Klahr and Simon's two-space model, Schunn and Klahr proposed a more
% granular four-space model involving data representation, hypothesis, paradigm,
% and experiment spaces~\cite{schunn1995FourSpace,schunn1996BeyondTwoSpace}. In the four-space model, conceptual hypothesizing still lies in the
% hypothesis space, and hypothesis testing and statistical modeling lies in the
% paradigm space. As such, hypothesis formalization is a process connecting
% the hypothesis and paradigm spaces. In Schunn and Klahr's four-space model,
% information flows unidirectionally from the hypothesis space to the paradigm space.
% Here we extend this prior research with
% evidence that hypothesis formalization involves both
% concept-to-implementation and implementation-to-concept processes. (see
% Figure~\ref{figure:overview}).
%  Therefore, we not only draw upon but also extend prior research on scientific discovery.
% \figureMethods 

% using multiple methods. 

\figurePriorWorkCombined

\begin{comment}
\subsection{Theories of Sensemaking}
Human beings engage in \textit{sensemaking} to acquire new knowledge. Several
theories of
sensemaking~\cite{pirolli2005sensemaking,russell1993cost,klein2007dataFrame}
describe how and when human beings seek and integrate new data (e.g.,
observations, experiences, etc.) to develop their mental models about the world.

Russell et al.~\cite{russell1993cost} emphasize the importance of building up
and evaluating external representations of mental models, and define sensemaking
as ``the process of searching for a representation and encoding data in that
representation to answer task-specific questions.'' External representations are
critical because they influence the quality of conclusions reached at the end of
the sensemaking process and affect how much time and effort is required in the process. Some representations
may lead to insights more quickly. Russell et al. describe the iterative
process of searching for and refining external representations in a ``learning
loop complex'' that involves transitioning back and forth between (i) searching
for and (ii) instantiating representations. 
 
Grolemund and Wickham argued for statistical data analysis as a sensemaking
activity~\cite{grolemund2014cognitive}. They emphasize the (1)
bidirectional nature of updating mental models of the world and hypotheses based
on data and collecting data based on hypotheses and (2) the process of
identifying and reconciling discrepancies between hypotheses and data. Their
depiction of the analysis process parallels Klahr and Simon's framework of
scientific discovery.

and proposed a theory of data
analysis that includes a back and forth between an analyst's ``schema'' of how a
phenomenon occurs in the world, a statistical model, and data. Similar to
Russell et al., Grolemund and Wickham's model demonstrates the importance of
representing and re-representing conceptual knowledge in schema and statistical
models that are updated with more data. Analysts' domain expertise influence
their schemas, which represent conceptual knowledge about known and unknown
causal mechanisms, for example. Analysts' conceptual schema directly inform
their hypotheses, which are statistical predictions represented in statistical
models. These statistical models are then compared to collected data, and any
discrepancies between the data and hypothesis require analysts to re-examine and
possibly update their statistical model, schema, or both. 

In this paper, we consider hypothesis formalization to be a learning loop~\cite{russell1993cost} where
the conceptual hypothesis is an external representation of a set of assumptions
analysts may have about the world (e.g., an implicit causal model), that ultimately
affects which models are specified and which results are
obtained. We found that that there are smaller learning loops as analysts search
for and revise intermediate representations, such as explicit causal models,
mathematical equations, or partially specified models. The
hypothesis and model refinement loops can themselves be smaller learning loops
embedded in the larger loop of hypothesis formalization. 

Extending Grolemund and Wickham's model, our work on
 hypothesis formalization differentiates between conceptual and statistical
 hypotheses and probes the phases an analyst must go through to encode a
 conceptual hypothesis into a statistical model.

In summary, our work differs in (i) scope and (ii) method from prior work in HCI
on data analysis practices. Whereas hypothesis formalization has remained
implicit in prior descriptions of data analysis, we explicate
this specific process. While previous researchers have relied primarily on
post-analysis interviews with analysts, our lab study (\autoref{sec:labStudyHypoForm}) enables us to observe
decision making during hypothesis formalization in-situ.
\end{comment}

\section{Formative content analysis} \label{sec:contentAnalysisHypoForm} 
To complement our in-depth synthesis of prior work, we conducted a formative
content analysis of 50 peer-reviewed publications from five different domains. 

\subsubsection{Methods}
We randomly sampled ten papers published in 2019 from each of the following
venues: (1) the Proceedings of the National Academy of Sciences (PNAS), (2)
Nature, (3) Psychological Science (PS), (4) Journal of Financial Economics
(JFE), and (5) the ACM Conference on Human Factors in Computing Systems (CHI).
We sampled papers that used statistical analyses as either primary or secondary
methodologies. Our sample represents a plurality of domains and recent
practices.~\footnote{Google Scholar listed the venues among the top three in
their respective areas in 2018. Venues were often clustered in the rankings
without an obvious top-one, so we chose among the top three based on ease of
access to publications (e.g., open access or access through our institution).
Some papers were accepted and published before 2019, but the journals had
included them in 2019 issues.}

% We analyzed published papers because researchers are
% not only likely but required to report significant operationalization choices in
% their publications. 

The first two authors iteratively developed a codebook to code papers at the
paragraph-level. The codebook contained five broad categories: (i) research
goals, (ii) data sample information, (iii) statistical analysis, (iv) results
reporting, and (v) computation. Each category had more specific codes to capture
more nuanced differences between papers. This tiered coding scheme enabled us to
see general content patterns across papers and nuanced steps within papers. The
first two authors reached substantial agreement (IRR = .69 - .72) even before
resolving disagreements. The first three authors then (i) read and coded all
sections of papers except the figures, tables, and auxiliary materials that did
not pertain to methodology\footnote{PNAS and Nature papers included a materials
and methods section after references that were distinct from extended tables,
figures, and other auxiliary material. We coded the materials and methods
sections in the appendices and included them in the content analysis. The appendix describes our process in greater detail.}; (ii) discussed
and summarized the papers' goals and main findings to ensure comprehension and
identify contribution types; and (iii) visualized each paper as a ``reorderable
matrix''~\cite{bertin2011graphics}. 

\figureExampleReorderableMatrix

We adapted Bertin's ``reorderable matrix''~\cite{bertin2011graphics}, an
interactive visualization technique for data exploration, in our analysis. We
visualized each paper in our sample as a matrix where each row represented a
code in our codebook and each column represented a coded paragraph. We fixed the
order of paragraphs to match the paper's progression. We colored codes (rows)
according to their categories in our codebook, repeatedly reordered the rows
representing codes, and transposed the matrices to detect visual patterns in the
papers. Figure~\ref{figure:exampleReorderableMatrix} shows an example matrix. 
% and each column represented a code in our codebook

% We visualized each paper as a reorderable matrix in order to more easily and
% systematically identify patterns in paper structure and content. 
The visual representation of papers' content and structure helped us notice
common patterns across papers and guided our follow-up analyses and discussions about what steps
(\rqSteps) and considerations (\rqProcess) researchers reported having during
hypothesis formalization. Across multiple papers, the matrices showed how
researchers typically start with broader research goals that they decompose into
specific hypotheses (i.e., hypothesis refinement) over the course of a paper
section, for example. Within a single paper, the matrices visually showed patterns of how
researchers motivated and pieced together multiple experiments and interpreted
statistical results in order to make a primary scientific argument. The appendix include our codebook with definitions and examples as
well as a summary, citation, and annotated matrix for each paper.

% We were interested in (i) learning about the
% breadth of steps involved in hypothesis formalization rather than assessing how
% well papers fit a predetermined set of steps and (ii) detecting any
% co-occurrence or ordering among steps. 

% For example, research goals could be
% statements or questions about something unknown, or focused examinations of
% possible associations between constructs, among other codes. 

\subsubsection{Findings}
\begin{comment}
\noindent\textbf{Overview:} We coded a total of 2,989 paragraphs across 50 papers. Results were the most
commonly discussed topic. Approximately 31\% of the paragraphs (in 50 papers)
discussed interpretations of statistical results, and ~11\% (in 37 papers)
provided details about statistical results (e.g., parameter estimates).
Interpreted results often co-occurred with statistical results. ~21\% of
paragraphs (in 40 papers) described data collection design (e.g., how the
experiment was designed, how the data were collected, etc.). Specifications of
statistical models appeared in ~19\% of paragraphs (in 50 papers). ~11\% of
paragraphs (in 45 papers) discussed proxy variables, or measures to quantify
abstract constructs (e.g., music enjoyment). To our surprise, more papers
mentioned software than included equations. Researchers mentioned software used
for statistical analysis in 3\% of paragraphs (in 25 papers), sometimes even
specifying function names and parameters, a level of detail we did not expect to
find in publications. Only fifteen papers (JFE: 9, PS: 5, PNAS: 1) included
equations in a total of 71 paragraphs. This suggests that mathematical
equations, though part of the hypothesis formalization process, are less
important to researchers than their tool-specific implementations.

Papers published in PNAS and Nature had noticeably different structures than the
CHI, JFE, and PS papers. The PNAS and Nature papers decoupled research goals,
data sample properties, and results (in the main paper body) from data
collection design and statistical analysis (in the appended materials and
methods section). For individual studies in the CHI, JFE, and PS papers, codes
repeated in noticeably linear patterns from research goals to data collection
and sample information to proxy variables and statistical analyses to results.
Although researchers write about the hypothesis operationalization and
statistical analysis process linearly, prior observational work describes data
analysis as highly iterative~\cite{}. The lack of information about hypothesis
operationalization and the contradiction between scientific narratives and
processes further suggest the opacity of hypothesis formalization and the need to study it and motivate our lab study. 

We present more comprehensive tables and findings about paper structure, about paper
contributions and venue differences in the appendix.

% \ej{Fill this in: Through open coding and discussion, we identified 41 papers
% that made empirical contributions showing or explaining a phenomenon, six
% contributed novel methodologies, eight developed and evaluated prototype tools
% (could be physical, biological, etc.), and nine made various other
% contributions (e.g., replications, finding new a species, etc.). Table~\ref{}
% gives an overview of contribution types across venues. -- Mention some papers
% were overlapped? Move this to supp material?}

\end{comment}

The content analysis confirmed prior findings on (i) the connection between
hypotheses and causal models (e.g.,\cite{mcelreath2020statistical}), (ii) the
importance of proxies to quantify concepts, and (iii) the constraints that data
collection design and logistics place on modeling. Extending prior work, the
content analysis also (i) suggested that decomposing hypotheses into specific
objectives is a mechanism by which conceptual hypotheses relate to causal
models; (ii) crystallized the hypothesis refinement loop involving conceptual
hypotheses, causal models and proxies; and (iii) surfaced the dual-search nature
of hypothesis formalization by suggesting that model implementation may shape
data collection. 

\begin{comment}
Researchers decompose hypotheses into sub-goals that correspond to statistical
analyses. In approximately 70\% of papers in the corpus, we found that
researchers deconstructed their motivating research questions and overarching
hypotheses into more tightly scoped objectives or relationships of interest that
map to specific statistical analyses. For example, in~\cite{N8}, the researchers
asked how theories of macroevolution varied across groups of species. The
authors divided pre-existing hypotheses into three classes of hypotheses and
assessed each class in turn. For one class of ``geometric'' hypotheses about
insect egg size, the researchers discriminated between two opposing hypotheses
by examining ``the scaling exponent of length and width (the slope of the
regression of log-transformed length and log-transformed width).'' As this
example demonstrates, hypothesis formalization involves an \emph{iterative
hypothesis refinement process at the conceptual level}. This refinement process
distills hierarchies of hypotheses and/or a single conceptual hypothesis into
sub-hypotheses and formalizes these sub-hypotheses in statistical model
implementations. Researchers also relate sub-hypotheses to one other during this
process, which implies their causal models about the motivating conceptual
hypothesis (and domain).

% The refinement process
% implies researchers' causal models about the motivating conceptual hypothesis
% (and domain) that is observable in how they relate sub-hypotheses to one
% another. 

Researchers concretize hypotheses using proxies that are based on theory or
available data. Proxy variables further refine conceptual hypotheses by
identifying how observable some concepts are, measuring the observable ones,
indirectly measuring the less observable ones, and comparing measurement choices
to other possible measures or ideal scenarios. As such, proxy variable selection
is an important transition step between conceptual and data concerns during
hypothesis formalization.


% When defining proxy variables, researchers (i) used previously validated
% measures when available for theoretical and methodological soundness, such as
% the Barcelona Music Reward Questionnaire (BMRQ) to measure music reward
% (in~\cite{PS1}), or (ii) developed new measures as a research contribution. For
% example, in~\cite{CHI0}, the authors develop an EEG-based measure for
% ``immersiveness'' in VR they demonstrated to be superior to previous measures
% that required halting immersive VR experiences to ask users about immersion.
% Researchers also sometimes justified choosing proxies based on available data.
% For example, in~\cite{JFE5}, the researchers wanted to develop a proxy variable
% for job rank based on titles and ``financial outcomes'' (e.g., compensation,
% bonuses, etc.) to see if housing bankers were promoted or demoted after the 2008 stock market
% crash. However, because the financial outcomes were not public, the researchers
% relied on title only to compare bankers' ranks, which was sub-optimal because
% job titles differ between companies. 

Researchers consider their proxy choices as study limitations and consider
alternative proxies to ensure that their findings are robust. Validating
findings with multiple proxies suggests that hypothesis formalization can be a
\emph{recursive process}. Proxies lead to follow-up hypotheses about possible
latent measurement factors, for instance, which in turn lead to additional analyses that address
the same conceptual hypothesis. 

% In doing so, the researchers hypothesize that another latent variable or
% measurement artifact is responsible for their findings.

Data collection and sampling influence statistical analysis. Researchers often
described their data sampling and study design as factors that necessitated
additional steps in their analysis process. In~\cite{PS0} and~\cite{PS5},
researchers accounted for effects of task order in their study protocol by
running additional regressions or analyzing tasks separately. Researchers also
ran initial analyses to assess the effect of possibly confounding variables in
their study design, such as gender in~\cite{PS3} or location of stimuli
in~\cite{PS4}. Other times, researchers performed robustness checks after their
main analyses, such as in response to a gender imbalance in~\cite{PS5} and
possible sample selection biases due to database constraints in~\cite{JFE1}.

Although data collection driven by statistical modeling plans was expected of
replication studies (e.g.,~\cite{PS8,PS5,PS0}) or papers that make
methodological contributions (e.g.,~\cite{JFE6, JFE7}), we found an instance
in~\cite{PS2}\textemdash neither replication nor methodological
contribution\textemdash where researchers explicitly reported selecting a
statistical model before designing their study. The researchers chose to use a
well-validated computational model, the linear ballistic accumulator (LBA), to
quantify aspects of human decision making. This model selection influenced the
way they designed their study protocol so that they could obtain a sample large
enough for accurate parameter estimation. 
%  the human information-processing system that may underlie decision making.

% specifically to collect a data sample that would lead to accurate parameter estimation with the LBA.

% we found an instance of an analysis in~\cite{PS2} that neither replicated prior work nor methodological contirbution. 

% Interestingly, there was one instance of an analysis from a paper that neither replicated prior work nor contributed a novel methodology where researchers
% explicitly stated that they collected data with a specific statistical model in mind. 
% had selected a statistical model for analysis before collecting data. In~\cite{PS2}, the researchers chose to use a well-validated computational
% model, the linear ballistic accumulator (LBA), to quantify aspects of the human
% information-processing system that may underlie decision making. This model
% selection influenced their study protocol, which they designed specifically to
% collect a data sample that would lead to accurate parameter estimation with the
% LBA. 

% For instance, in~\ej{\cite{}}, after the researchers found bankers involved in
% the housing market were not punished with reduced employment or retention
% prospects after the 2008 financial crisis, the researchers asked a follow-up
% question about how employment may have differed between RMBS and non-RMBS
% bankers. The researchers use the non-RMBS banker group as a counterfactual.
% For the purpose of detecting firm-based fixed effects in their statistical
% analysis and controlling for other factors, the researchers justify only
% including non-RMBS with matched criteria and data from the top 18 underwriting
% firms. Although the researchers write about their initial data selection
% process without mentioning their analyses, it is clear that specific analyses
% may require small adjustments to or new forms of collecting and sampling data.
% Especially important here, is also that previously defined and tested
% sub-hypotheses and their results underlie this process.

% Based on prior work, we expected researchers to describe how their
% experimental protocols and sampling procedures would affect statistical
% models. We were suprised to find only \ej{one} example of modeling imposing
% data collection constraints. 

Based on these observations, it seems that modeling choices more frequently
react to data collection processes and possible sample biases, following a
linear data collection-first process implied by prior work. However, there are
also instances where model implementation comes first and researchers' data
collection procedures must adhere to modeling needs, suggesting a previously
missing \emph{loop between statistical model implementation and data collection}
that is separate from any influences conceptual hypotheses have on data
collection. 

% Pre-registration may require both of these processes although only the data
% collection first linear process may be the one included in pre-registration
% reports and publications. 
\end{comment}

\begin{comment}
Although researchers discussed the structure of their analyses, we found that
they did Although researchers more explicitly justified the structure of their
models, they did not justify why the particular model was the best one or better
than others they may have considered. There may be some trade-off analysis
researchers calculated but did not write in the paper, which is a limitation of
the literature survey that we hoped to address in the lab study. R


Sometimes, researchers implicitly communicated why their model fit their design
by discussing the two in tandem. For instance, no explicit rationale but
basically translated into procedure for CHI Sq. test. 

PNAS\_6 - page 19204 statistical consideraitons 

Related to: Sample info - Statistical Analysis? 



Other times, researchers were more explicit in how their statistical models took
(i) the study design or (ii) latent structures in the data. For instance, <find
PS example>. Find JFE example for structures. 

\end{comment}


\subsubsection{Limitations}
The major limitation of analyzing published papers is the disconnect between
actual and reported analytical practice. The pressures to write compelling
scientific narratives~\cite{kerr1998harking} likely influence which aspects of
hypothesis formalization are described or omitted. For instance, in practice, model
implementations may constrain data collection more often than we found in our
sample. Nevertheless, the lack of information in prior work and the content
analysis suggests that hypothesis formalization remains an opaque process
deserving of greater scrutiny. Hypothesis formalization may explain how analysts
determine which tools to use and how domain expertise may influence the analytical conclusions reached. 

\subsection{Expected Steps in Hypothesis Formalization}
% \figurePriorWork 
Towards our first two research questions about what actions analysts take to
formalize hypotheses (\rqSteps) and why (\rqProcess), prior work and our
formative content analysis suggest that hypothesis formalization involves steps
in three categories: conceptual, data-based, and statistical.
\textit{Conceptually,} analysts develop conceptual hypotheses and causal models
about their domain that guide their data analysis. With respect to
\textit{data}, analysts explore data and incorporate insights from exploration,
which can be top-down or bottom-up, into their process of formalizing
hypotheses. The \textit{statistical} concerns analysts must address involve
mathematical and computational concerns, such as identifying a statistical
approach (e.g., linear modeling), representing the problem mathematically (e.g.,
writing out a linear model equation), and then implementing those using
software. In our work, we find evidence to support separating statistical
considerations into concerns about mathematics, statistical specification in
tools, and model implementation using tools.

A key observation about prior work is that there is a tension between iterative
and linear workflows during hypothesis formalization. Although sensemaking
processes involve iteration, concerns about methodological soundness, as
evidenced in pre-registration efforts that require researchers to specify and
follow their steps without deviation, advocate for, or even impose, more linear
processes. More specifically, theories of sensemaking that draw on cognitive
science, in particular~\cite{russell1993cost,grolemund2014cognitive}, propose
larger iteration loops between conceptual and statistical considerations. Some
textbooks and research concerning statistical thinking and
practices~\cite{wild1999statisticalThinking,carver2016guidelines} appear less
committed to iteration while other researchers and practitioners in applied
statistics emphasize \textit{workflows} for iterating on statistical
models~\cite{yu2020veridical,lee2019robust,gelman2013bayesianTextbook}.
Workflows (e.g., model expansion) can help researchers start with simple models and build up to more
complex ones by incrementally testing and refining their
understanding of characterstics of the data, the model fitting algorithms, and
computational settings~\cite{betancourt2020bayesianWorkflow,gelman2020bayesianWorkflow,gabry2019visualization}. Moreover, empirical work in HCI on data analysis embraces
iteration during exploration and observes iteration during some phases of
confirmatory data analysis, such as statistical model choice, but not in others,
such as tool selection. In our work, we are sensitive to this tension and aim to
provide more granular insight into iterations and linear processes involved in
hypothesis formalization. We also anticipate that the steps identified in prior
work will recur in our lab study, but we do not limit our investigation to these
steps. 

\section{Exploratory Lab Study} \label{sec:labStudyHypoForm}

To address the limitation of the content analysis, understand analysts'
considerations (\rqProcess) while formalizing their hypotheses (\rqSteps), and
examine the role of statistical software in this process (\rqTools), we designed
and conducted a virtual lab study with freelance data workers who approach the
hypothesis formalization and analysis process with expectations of rigor but
without the pressure of publication.

\subsection{Methods} 
\textbf{Data workers:} We recruited 24 data workers with experience in domains
ranging from marketing to physics to education through Upwork (22) and by
word of mouth (2).\footnote{We refer to our participants as data workers because they work with data but do not represent the entire population of data scientists, which may include statistical experts.}

Twelve data workers held occupations as scientists, freelance data scientists,
project managers, or software engineers. Six were currently enrolled in or had just
finished graduate programs that involved data analysis. Five identified as
current or recent undergraduates looking for jobs in data science. One was an
educator. Data workers self-reported having significant experience on a 10-point
scale adapted from a scale for programming experience~\cite{feigenspan2012measuring} (min=2, max=10, mean=6.4,
std=2.04) and would presumably have familiarity with hypothesis formalization.

The lab study enables us to contrast normative expert practices (found in 
prior work and our formative content analysis) to observed practices with data
workers who are not statistical experts but still work in real-world analysis
settings (i.e., research, marketing, consulting). A benefit of studying these
data workers is that they are likely to benefit most from new tools. 

% \tabledata workers

\noindent\textbf{Protocol:} %Based on our content analysis findings, 
We designed
and conducted a lab study with three parts. Parts 1 and 3 were recorded and automatically transcribed using Zoom. We compensated data workers \$45 for
their time. The first author conducted the study and took notes throughout.

\textit{Part 1: Structured Tasks.}  
Part 1 asked data workers to imagine they were leading a research team to answer
the following research question: ``What aspects of
an individual's background and demographics are associated with income after
they have graduated from high school?''\footnote{We chose the open-ended research
question about income after high school because we expected it to be widely
approachable and require no domain expertise to understand.}
We asked data workers to complete the following tasks:
\begin{itemize}
    \item \textit{Task 1: Hypothesis generation.} Imagining they had access to
    any kind of data thinkable, data workers brainstormed at least three
    hypotheses related to the research question.
    \item \textit{Task 2: Conceptual modeling.} Next, data workers saw a sample
    data schema and developed a conceptual model for one or more of their
    hypotheses. We used the term ``conceptual model'' instead of ``causal
    model'' to avoid (mis)leading data workers. We provided the following
    definition: ``A conceptual model summarizes the process by which some
    outcome occurs. A conceptual model specifies the factors you think influence
    an outcome, what factors you think do not influence an outcome, and how
    those factors might interact to give rise to the outcome.'' 
    \item \textit{Task 3: Statistical model specification.} Finally, we
    presented data workers with a sample dataset and instructed them to specify
    but not implement a statistical model to test one or more of their
    hypotheses. %\ej{Given that statistical software tools were not discussed with statistical specification.} 
\end{itemize}

After the three tasks, we conducted a semi-structured interview with
data workers about (i) their validity concerns\footnote{If data workers were
unfamiliar with the term ``validity,'' we rephrased the questions to be about
``soundness'' or ``reliability.''} and (ii) experiences. To help us
contextualize our observations and assess the generalizability of our findings,
we asked data workers to compare the study's structure and tasks to their
day-to-day data analysis practices.

\textit{Part 2: Take-home analysis.} After the first Zoom session, data workers
implemented their analyses using the previously shown dataset, shared any
analysis artifacts (e.g., scripts, output, visualizations, etc.), and completed
a survey about their implementation experience. Prior to Part 3, the first
author reviewed all submitted materials and developed participant-specific
questions for the final interview.

\textit{Part 3: Final Interview.} The first author asked data workers to give an
overview of their analysis process and describe the hypotheses they tested, how
their analysis impacted their conceptual model and understanding, why they made
certain implementation choices, what challenges they faced (if any), and any
additional concerns about validity.

\noindent\textbf{Materials:} The data schema and dataset used in the study came from a
publicly available dataset from the Pew Research Center~\cite{pewDataset}. Each
task was presented in a separate document. All study materials are included in the appendix.

\noindent\textbf{Analysis:} The first author reviewed the data workers' artifacts multiple
times to analyze their content and structure;
thematically analyzed notes and transcripts from data workers' Zoom sessions;
and regularly discussed observations with the other authors throughout analysis.

\subsection{Findings and Discussion} 

% \todo{To clarify, the math representations we expected participants to provide were equations and/or statistical test names (e.g., ANOVA). It is possible that our stimuli primed participants to respond how they would perform, rather than represent, the task even after we clarified the question's intent. Another interpretation of findings is that data workers prefer to reason about and communicate their analyses procedurally even if they know the math. This would still suggest reviewing how tools support model expression, which we did. We will clarify our expectations about math equations and incorporate the alternative interpretation.]}

Eighteen of the 24 data workers we recruited completed all three parts of the study.
The other six data workers completed only the first Zoom session. In our analysis,
we incorporate data from all data workers for as far as they completed the study. 

We found that data workers had four major steps (\rqSteps) and considerations
(\rqProcess): (i) identifying or creating proxies, (ii) fitting their present
analysis to familiar approaches, (iii) using their tools to specify models
(\rqTools), and (iv) minimizing bias by relying on data. Data workers also faced
challenges acquiring and incorporating domain and statistical knowledge
(\rqProcess).

% Overall/In conclusion, we found that data workers do not reason about statistical
% models independently. Instead, they rely on their prior experiences, tools, and
% external information to identify and implement statistical models that answer their conceptual hypotheses

\theme{Data workers consider proxies and data collection while articulating hypotheses.}
We encouraged data workers to not consider the feasibility of collecting data
while brainstorming hypotheses. Yet, while brainstorming hypotheses,
data workers expressed concern with how to measure constructs [D2, D5, D8,
D12, D18, D22, D24] and how to obtain data [D2, D6, D8, D9, D11, D21, D24].

For instance, D18, a computer science student who had worked on more than five data
analysis projects, grappled with the idea of `privilege' and how to best
quantify it: \longquote{I'm trying to highlight the fact that those who will be
privileged before graduation...that experience will enable them to make again
more money after graduation. I won't say `privilege' because we need to quantify
and qualify for that...it's just an abstract term.} Eventually, D18 wrote two
separate hypotheses about `privilege,' operationalizing it as parental income:
(1) ``People with higher incomes pre graduating, end up having higher
differences between pre and post graduation incomes than those with lower
incomes pre graduation.'' and (2) ``People with parents with lower incomes tend
to have lower incomes pre graduation than those with parents with higher
incomes.'' 

D18 continued to deliberate `privilege' as measured by low and high income,
saying, \shortquote{...again you need to be careful with low and high because
these are just abstract terms. We need to quantify that. What does it mean to be
`low?' What does it mean to be `high?'}. Finally, D18 decided to
\shortquote{maybe use the American standards for low income and high income.}
Although an accepted ``American standard'' may not exist, D18 nevertheless
believed that cultural context was necessary to specify because it could provide
a normalizing scale to compare income during analysis, demonstrating how
data workers plan ahead for statistical modeling while brainstorming and refining
hypotheses. 

Similarly, D2, a freelance data scientist, was very specific about how to measure personality:
%in their hypothesis
``More extraverted individuals (extraversion measured using
the corresponding social network graph) are likely to
achieve higher yearly income later in life.'' 

In the presence of the data schema, more data workers were concerned with proxies
[D2, D5, D6, D7, D8, D9, D16, D18, D21]. Some even adapted their working
definitions to match the available data, similar to how researchers in the
content analysis determined proxies based on data. For instance, D8, who hypothesized that
``individuals interested in STEM fields tend to earn more post high school than
individuals interested in other fields,'' operationalized ``interest'' as
``Major'' --- a variable included in the data schema --- even though they had
previously brainstormed using other proxies such as club attendance in high school. 


% Data workers' focus on measurement and data collection corroborate our findings from
% the content analysis and show how conceptual hypotheses and data collection
% inform one another.

These data workers' closely related considerations of data and concept measurement
demonstrate how conceptual hypotheses and data collection may inform each other,
corroborating our findings from the content analysis.


\theme{Data workers consider implementation and tools when specifying statistical models.}
\figureLabStudyStatSpec When we asked data workers to specify their models
without considering implementation, we anticipated they would name specific
statistical tests (e.g., ``ANOVA''), approaches (e.g., ``linear regression'' or
``decision trees''), or write mathematical models (e.g., $Y = B_0 + B_1X_{age} +
B_2X_{gender}$) that they could then implement using their tools because (a) some
researchers in the literature survey did so in their papers and (b) several data
workers mentioned having years of analysis experience. However, despite the
explicit instruction to disregard implementation, 16 data workers provided to-do
lists or summaries of steps to perform a statistical analysis as their model
specifications [D1, D2, D3, D5, D7, D8, D9, D11, D12, D14, D16, D18, D20, D21,
D22, D23, D24]. Of these 16 data workers, eight also named specific statistical
tests in their descriptions [D3, D7, D8, D11, D12, D14, D18, D20]. 

For example, D8, a data science consultant with 7/10 analysis experience,
specified a list of steps that included creating new variables that aggregated
columns in the dataset, cleaning and wrangling the data, visualizing histograms,
performing chi-squared test, and interpreting the statistical results. Notably,
D8 also specified null and alternative hypotheses, which acted as an
intermediate artifact during hypothesis formalization.
Figure~\ref{figure:labStudyStatSpec} shows D8's statistical specification.

Only four data workers named specific statistical methods without describing
their steps [D4, D6, D15, D17]. Two data workers, D22, a neuroscientist by
training with 8/10 analysis experience, and D19, an educator with 6/10 analysis
experience, attempted to specify their models mathematically. D22 used the
familiar R syntax: ``Current Income \textasciitilde\xspace Educational attainment + Gender +
Interactions of those two.'' On the other hand, D19 gave up because although
they knew the general form of logistic regression, they did not know how to
represent the specific variables in the model they wanted to perform. 

The implementation and software details data workers discussed and included in
their specifications suggest that data workers prefer to skip over mathematical
equations and jump to specification and implementation in their tools. Although
it is possible that study instructions primed data workers to respond about how
they would perform, rather than represent, the task even after researcher
clarifications, this would not explain the level of implementation detail data
workers included. Nine data workers went so far as to mention specific
libraries, even functions, that they would use to program their analyses [D3,
D9, D12, D13, D14, D16, D19, D21, D23]. In their reflective interviews, data
workers also expressed that they often do not specify models outside of
implementing them, which D19 succinctly described: \longquote{I don't normally
write this down because all of this is in a [software] library.} 

% even though some papers included equations as an intermediate step in our content analysis.
% Another possible explanation for why data workers focused on implementation when
% specifying their models is that some statistical paradigms, especially
% machine-learning, are designed to automatically discover the statistical
% structure of data. In a machine-learning setting, for instance, data workers do
% not have to think about the functional form of statistical models but rather
% think about the machine learning architecture or approach for discovering this
% form. We were not able to test or verify this possibility with the data workers
% because we did not collect comprehensive information about their exposure to
% several statistical paradigms. Nevertheless, the inclination to defer functional
% formulation of statistical models to implementation underscores the importance of tools during statistical analysis.

Data workers' statistical knowledge appears to be situated in the programs they
write, and their knowledge of and familiarity with tools constrains the
statistical methods they explore and consider. As such, tools may be a key point
of intervention for guiding data workers toward statistical methods that may be
unfamiliar but are best suited for their conceptual hypotheses.

\theme{Data workers try to fit analyses to previous projects and familiar approaches.}
Data workers spent significant thought and time categorizing their analyses as
``prediction,'' ``classification,'' or ``correlation'' problems [D2, D3, D7,
D10, D11, D18, D19, D21, D22]. To categorize, data workers relied on their
previous projects. While reflecting on their typical analysis process, D21, a software engineer working in healthcare, said (emphasis added),
\longquote{I usually tend to jump...to look at data and \textbf{match [the
analysis problem] with similar patterns} I have seen in the past and start
implementing that or do some rough diagrams [for thinking about parameters, data
type, and implementation] on paper...and start implementing it.} 

Data workers also looked at variable data types (i.e., categorical or continuous) to
categorize. For example, D3, a freelance analyst, pivoted from thinking about \textbf{predicting}
income to \textbf{classifying} income groups (emphasis added) based on data type
information: \longquote{The income, the column, the target value here, is
categorical. I think maybe it wouldn't be a bad idea to see what
\textbf{classification} tasks, what we could do. So instead of trying to
\textbf{predict} because we're not trying to \textbf{predict an exact number},
it seems...like more of a \textbf{classification} problem...}

A provocative case of adhering to prior experiences was D6, a psychological research scientist. Although several
data workers were surprised and frustrated that income was ordinal in the dataset
with categories such as``Under \$10K,'' ``\$10K to \$20K,'' ``\$20K to \$30K,''
up to ''150K+'', none went so far as D6 to synthetically generate normally
distributed income data so that they could implement the linear regression
models they had specified despite saying they knew that income was not normally
distributed. 

When asked further about the importance of normal data, D6 described how they
plan analyses based on having normal data, strive to collect normally
distributed, and rely on domain knowledge to transform the data to be normal
when it may not be after collection: \longquote{...I feel like having non normal
data is something that's like hard for us to deal with. Like it just kind of
messes everything up like. And I know, I know it's not always assumption of all
the tasks, but just that we tend to try really hard to get our variables to be
normally distributed. So, you know, we might like transform it or, you know,
kind of clean it like clean outliers, maybe transform if needed...I mean, it
makes sense because like a lot of measures we do use are like depressive
symptoms or anxiety symptoms and kind of they're naturally normally
distributed...I can probably count on my hand the number of non parametric tests
I've like included in manuscripts.} D6's description of their day-to-day
analyses exemplifies the dual-search nature of hypothesis formalization: Data
workers (i) jump from hypothesis refinement to model specification or
implementation with specific proxies in mind and then (ii) collect and
manipulate their data to fit their model choices. 

We recognize that data workers may have taken shortcuts for the study they would
not typically make in real life. Nevertheless, the constraints we imposed by
using a real-world dataset are to be expected in real-world analyses. Therefore,
our observations still suggest that rather than consider the nature and
structure of their hypotheses and data to inform using new statistical
approaches, which statistical pedagogy and theory may suggest, data workers may
choose familiar statistical approaches and mold their new analyses after
previous ones. 

% Furthermore, D6's description of their day-to-day analyses
% exemplifies the dual-search nature of hypothesis formalization: Data workers (i)
% jump from hypothesis refinemnent to model specification or implementation with
% specific proxies in mind and then (ii) collect and manipulate their data to fit
% their model choices. 

\theme{Data workers try to minimize their biases by focusing on data.}
%  To mitigate their biases, several data workers mentioned wanting to
% research related literature and prior work \ej{[D6, 10, 15 -- not neccessarily
% to mitigate bias]}.
% data workers drew upon their lived experiences [D5, D10, D13, D15, D16, D20,
% D21, D24], media (e.g., books, podcasts, news, etc.) [D2, D3, D5, D6, D7,
% D13, D24], and previous coursework and research [D4, D6, D7] to develop their
% hypotheses. data workers also relied on personal experiences [D8, D12, D20,
% D24] to devise their conceptual models. At the same time, data workers
% recognized that their personal experiences biased their hypotheses and
% conceptual models [D12, D13, D14, D17]. 

Throughout the study, data workers expressed concern that they were biasing the
analysis process. Data workers drew upon their personal experiences to develop
hypotheses [D5, D10, D13, D15, D16, D20, D21, D24] and conceptual models [D8,
D12, D20, D24]. D12, a data analysis project manager, described how their personal experiences may subconsciously
bias their investigation by comparing a hypothetical physicist and social worker
answering the same research question: \longquote{Whereas a social worker by
design...they're meant to look at the humanity behind the numbers [unlike a
physicist]. So like, they may actually end up with different results...actually
sitting in front of this data, trying to model it.}

A few data workers even refused to specify conceptual models for fear of biasing the
statistical analyses [D10, D11, D19]. On the surface, data workers resisted
because they believed that some relationships, such as the effect of age on
income, were too ``obvious'' and did not warrant documentation [D10, D11].
However, relationships between variables that were ``obvious'' to some
data workers were not to others. For instance, D10, a business analyst, described how income would
plateau with age, but other data workers, such as D18, assumed income would monotonically increase with age.

When we probed further into why D10, D11, and D19 rejected a priori conceptual
models, they echoed D10's belief that conceptual models ``put blinders on you.''
Even the data workers who created conceptual models echoed similar concerns of
wanting to ``[l]et the model do the talking'' in their implementations [D3, D15,
D18, D19]. Instead of conceptual modeling, D10 chose to look at all n-ary
relationships in the dataset to determine which variables to keep in a final
statistical model, saying, \longquote{It's so easy to run individual tests...You
can run hypothesis tests faster than you can actually think of what the
hypothesis might be so there's no need to really presuppose what relationships
might exist [in a conceptual model].} Of course, one could start from the
same premise that statistical tests are so easy to execute and conclude that
conceptual modeling is all the more important to prioritize analyses
and prevent false discoveries. 

Similarly, data workers were split on whether they focused their implementation exclusively on their
hypotheses or examined other relationships
in the dataset opportunistically. Nine data workers stuck strictly to testing their hypotheses [D1,
D4, D5, D6, D7, D11, D13, D20, D24]. However, five data workers were more focused on
exploring relationships in the dataset and pushed their hypotheses aside [D2,
D3, D10, D16, D18], and an additional four data workers explored relationships among
variables not previously specified in their hypotheses in addition to their
hypotheses [D14, D15, D17, D21]. D18 justified their choice to ignore their
hypotheses and focus on emergent relationships in the data by saying that they
wanted to be \shortquote{open minded based on the data...open to possibilities.}

Data workers' concerns about bias and choice of which relationships to analyze
(hypothesis only vs. opportunistic) highlight the tension between the two
searches involved in hypothesis formalization: concept-first model
implementations and implementation-first conceptual understanding. Conceptual
models are intermediate artifacts that could reconcile the two search processes
and challenge data workers' ideas of what ``data-driven'' means. However, given some
data workers' resistance to prior conceptual modeling, workflows that help
data workers conceptually model as a way to reflect on their model implementations
and personal biases may be more promising than ones that require them before
implementation.

\theme{Data workers face challenges obtaining and integrating conceptual and statistical information.}
Based on data workers' information search behaviors and self-reports, we found that
data workers faced challenges obtaining and integrating both domain and statistical
knowledge.

Data workers consulted outside resources such as API documentation, Wikipedia, and
the \textit{Towards Data Science} blog throughout the study: one while
brainstorming hypotheses [D13]; three while conceptual modeling [D12, D13, D22];
six while specifying statistical models [D3, D6, D12, D13]. Six data workers
also mentioned consulting outside resources while implementing their analyses
[D1, D3, D11, D14, D15, D21]. By far, statistical help was the most common. 

Furthermore, when data workers reflected on their prior data analysis experiences,
they detailed how collaborators provided domain and statistical expertise that
are instrumental in formalizing hypotheses. Collaborators share data that help
domain experts generate hypotheses [D9], critique and revise conceptual models
and proxies [D4, D8], answer critical data quality questions [D10],
and ensure statistical methods are appropriate [D5, D6, D22].

In the survey participants completed after implementing their analyses, the three most
commonly reported challenges were (i) \textbf{formatting} the data [D1, D4,
D5, D6, D13, D16, D18, D20, D21, D24], (ii) \textbf{identifying} which
statistical analyses to perform with the data to test their hypotheses [D1,
D11, D14, D18, D20, D21], and (iii) \textbf{implementing and executing} analyses
using their tools [D1, D6, D7, D13, D20, D21]. Although we expected data workers
would have difficulty wrangling their data based on prior
work~\cite{kandel2012enterprise}, we were surprised that identifying and
executing statistical tests were also prevalent problems given that (a) data workers
were relatively experienced and (b) could choose their tools. These results, together with 
our observations that data workers rely on their prior experiences and tools, suggest
that data workers have difficulty adapting to new scenarios where new tools and
statistical approaches may be necessary. 

\subsection{Takeaways from the Lab Study}
After the first session, 13 out of the 24 data workers described all the tasks as
familiar, and 10 described most of the tasks and process as familiar. Data workers
commonly remarked that although the process was familiar, the order of the tasks
was ``opposite'' of their usual workflows. In practice, data workers may start with
model implementation before articulating conceptual hypotheses, which opposes
the direction of data analysis that the ASA
recommends~\cite{carver2016guidelines}. Nevertheless, our observations reinforce
the dual-search, non-linear nature of hypothesis formalization.

Moreover, one data worker, D24, a physics researcher who primarily conducted
simulation-based studies expressed that the study and its structure felt
foreign, especially because they had no control over data collection. Other data
workers in the study also described the importance of designing and conducting
data collection as part of their hypothesis formalization process [D4, D6, D9].
Designing data collection methods informs the statistical models data workers
plan to use and helps to refine their conceptual hypotheses by requiring data
workers to identify proxies and the feasibility of collecting the proxy
measures, reinforcing what we saw in the content analysis. The remarks also
suggest that disciplines practice variations of the hypothesis formalization
process we identify based on discipline-specific data collection norms and
constraints. For example, simulating data may sometimes take less time than
collecting human subjects data, so data workers working with simulations may
dive into modeling and data whereas others may need to plan experiments for a
longer period of time. 

% although the data workers in our lab study came from diverse
% domains, including medicine, psychology, and business, and had different data
% collection practices, 
% Furthermore, our sample is limited and may be biased. 
Approximately half of the data workers had either just finished or were enrolled in undergraduate or
graduate programs involving data analysis. As such, half of our sample likely has
limited professional experience outside of their studies and/or freelance work
on Upwork. Additionally, data work available on Upwork may be more narrowly
focused and less representative of end-to-end data analysis or research projects
expected of those with greater statistical expertise. Still, several data
workers in our study mentioned other employments where they gained professional
experience working on larger analysis and research projects. Despite the
limitations of recruiting participants from Upwork and word of mouth, our sample
represents data workers who have training in a diversity of disciplines (e.g.,
medicine, psychology, business), are familiar with a range of statistical
methods, and have experience using a broad range of statistical tools. As such,
the data workers in our study may be representative of analysts who are likely
to benefit most from new tools for supporting hypothesis formalization. 

Finally, we found that data workers relied on prior experiences and tools to specify
and formalize their hypotheses. Tools that scaffold the hypothesis formalization
process by suggesting statistical models that operationalize the conceptual
hypotheses, conceptual models, or partial specifications data workers create along
the way may (i) nudge data workers towards more
robust analyses that test their hypotheses, (ii) overcome limitations of data workers'
prior experiences, and (iii) even expand data workers' statistical knowledge. Thus, we
investigated how current tool designs serve (or under-serve) hypothesis
formalization.
% or similar models data workers start with



\section{Analysis of Software Tools} \label{sec:toolsAnalysis}

To understand how the design of statistical computing tools may support or
hinder hypothesis formalization (\rqTools), we analyzed widely used software
packages and suites. Throughout, we use the term ``package'' to refer to a set
of programs that must be invoked through code, such as \texttt{lme4},
\texttt{scipy}, and \texttt{statsmodels}. We use the term ``suite'' to refer to
a collection of packages that end-users can access either through code or
graphical user interfaces (GUIs), such as SPSS, SAS, and JMP. We use the term
``tool'' to refer to both. Software packages were a unit of analysis because
they are necessary for model implementation regardless of medium (e.g.,
computational notebook, CoLab, RStudio). As such, our findings apply to tools
that provide wrappers around packages included in our sample.

\vspace{-2mm}
\subsection{Method}

\textbf{Sample:} Our sampling procedure involved two phases: (i)
identifying software packages and suites for model implementation (not visual analysis
tools like Tableau) mentioned more than once across the content analysis and lab
study and (ii) adding recommended packages and suites from online data science communities
our lab participants mentioned or used (e.g., \textit{Towards Data Science}). To
identify these additional tools, we consulted online data analysis
fora~\cite{grolemund2019:recommendedR, bobriakov2017:top15Python,
bobriakov2018:top20Python, prabhu2019:topPython}. The final sample included 20
statistical tools: 14 packages (R: 10, Python: 4); three suites that support
in-tool programming; and three suites that do not support programming.
Table~\ref{tableAnalysisOfTools} contains an overview of our sample and results.

\noindent\textbf{Analysis:} Four specific questions guided our analysis:
\begin{itemize}
    \item \textbf{Specialization:} Data workers in the lab study eagerly named
    specific statistical tools they would use and looked up tool documentation
    during the tasks. This prompted us to ask, \textit{How specialized are the
    tools, and how might specialization (or lack thereof) affect how end-users
    discover and use them to formalize hypotheses?}
    \item \textbf{Statistical Taxonomies:} Data workers in the lab study tried to
    mold their analyses to prior experiences and their taxonomies of statistical
    methods. We wondered what role tools play in this: \textit{How do tools
    organize and group statistical models? How might tool organization and
    end-users' taxonomies interplay during hypothesis formalization?}
    \item \textbf{Model Expression:} Data workers in the lab study jumped to model
    implementation throughout the tasks. Only half provided names of statistical
    methods. We wondered if this was due to how tools enable end-users to express
    their models: \textit{What notation must end-users use to express models in
    the tools?}
    \item \textbf{Computational Issues:} Data workers in the lab study described
    their statistical models using specific function calls. Similarly, although
    it was uncommon for researchers in the content analysis to specify the
    software tools they used, when they did, researchers specified the
    functions, parameters, and settings used. This prompted us to wonder about
    the importance of computational settings: \textit{What specific kinds of
    computational control do tools provide end-users and how might that impact
    hypothesis formalization?}
\end{itemize}

To answer the four questions for each statistical tool, the first author read
and took notes on published articles about tools' designs and implementations,
API documentation and reference manuals, and available source code; followed
online tutorials; consulted question-and-answer sites (e.g., StackExchange) when
necessary; and analyzed sample data with the tools. The first author paid
particular attention to tool organization, programming idioms, functions and
their parameters, and tool failure cases. Table~\ref{tableAnalysisOfTools}
contains citations for resources consulted in the analysis. The iterative
analysis process involved discussions among the co-authors about how to evaluate
the properties of tools from our perspectives as both tool designers/maintainers
and end-users. Here, we focus on end-user (hereafter referred to as analyst)
perspectives informed by our lab study and make callouts to details relevant for
tool designers.

\subsection{Findings and Discussion}
\tableSoftwareAnalysis
We discuss our findings in light of our characterization of hypothesis
formalization in Figure~\ref{figure:hypoFormOverview}. We refer to specific steps and
transitions in Figure~\ref{figure:hypoFormOverview} in \textbf{boldface}.

% The ecosystem of available statistical tools, in theory, support a wide range of
% statistical models and therefore conceptual hypotheses. However, the lack of
% support for navigating specialized libraries, low-level interfaces, and
% computational differences add barriers to the hypothesis formalization process
% and ultimately push analysts towards relying on their own statistical knowledge,
% which may be limited, and understanding low-level computational details, which
% may not be necessary for non-experts.

% \theme{Tool specialization pushes computational concerns higher up the hypothesis formalization process.}
\theme{Specialization.}
Half the tools [T2, T3, T4, T5, T6, T7, T8, T9, T11, T12] in our sample are
specialized in the scope of statistical analysis methods they support (e.g.,
\texttt{brms} supports Bayesian generalized linear multilevel modeling).
\texttt{edgeR} [T3] provides multiple modeling methods but is specialized to the
context of biological count data. Such specialized tools are vital to creating a
widely adopted statistical computing ecosystem, such as R. 

Despite its importance, tool specialization pushes computational concerns higher
up the hypothesis formalization process. Specialized tools require
analysts to consider computational settings while picking a statistical tool
and, possibly, even while mathematically relating their variables. They fuse the
last two steps of hypothesis formalization (\textbf{Statistical Specification}
and \textbf{Model Implementation}). Ultimately, specialization requires analysts
to have more (i) computational knowledge and (ii) foresight about their model
implementations at the cost of focusing on conceptual or data-related concerns
early in hypothesis formalization. 

% For example, to pick between \texttt{MCMCglmm} and
% \texttt{glmmTMB}, which both support generalized linear mixed-effects modeling, analysts would have to know that the two 
% employ different model fitting procedures, namely MCMC routines and maximum likelihood
% estimation using the `Template Model Builder' library, respectively. 

% Analysts must
% not only compare and contrast which optimizers tools support prematurely but
% also consider the ramification these computational choices have on their
% mathematical and conceptual representations of their hypothesis.

One way tool designers minimize the requisite computational knowledge and
foresight while providing the benefits of specialized packages --- which may be
optimal for specific statistical models or data analysis tasks --- is to provide
micro-ecosystems of packages. For example, R's
\texttt{tidymodels}~\cite{tidymodels} and \texttt{tidyverse}~\cite{tidyverse}
create micro-ecosystems that use consistent API syntax and semantics across
interoperable packages. They also push analysts towards what the tool designers
believe to be best practices, such as the use of the tidy data format~\cite{wickham2014tidy}. Tools
that aim to support hypothesis formalization may consider fitting into or
creating micro-ecosystems that provide tool support all along the process,
focusing analysts on concepts, data, or model implementation at various points. 

% \theme{Tool taxonomies introduce challenges that detract from hypothesis
% formalization.} 
\theme{Statistical taxonomies.}
A consequence of tool specialization is the fragmented view of statistical
approaches. For example, we observed analysts in the lab study who viewed the
analysis as a classification task gravitate towards machine learning-focused
libraries, such as \texttt{RandomForest} [T9], \texttt{Keras} [T11], and
\texttt{scikit-learn} [T12]. Because classification can be implemented as
logistic regression, any tool that supports logistic regression, such as the
core \texttt{stats} library in R [T10], provides equally valid, alternative
perspectives on the same analysis and hypothesis. However, tools obfuscate these
connections and do not aid analysts in considering reasonable statistical models
that may be unfamiliar or outside their personal taxonomy. This may explain why
analysts adhered to their personal taxonomies during the lab study.

This problem carries over to tools that support numerous statistical methods.
Ten tools in our sample intend to provide more comprehensive statistical support
[T1, T10, T13, T14, T15, T16, T17, T18, T19, T20]. These tools group statistical
approaches using brittle and inconsistent taxonomies based on data types [T17];
analysis classes that are both highly specific (e.g., ``Item
Response Theory'') and vague (e.g., ``Multivariate analyses'') [T15, T16, T17,
T18, T19, T20]; and disciplines or applications (e.g., ``Epidemiology and
related,'' ``Direct Marketing'') [T16, T17, T20]. Although well-intended to
simplify statistical method selection, tools' taxonomies are at times
misleading. For instance, JMP combines various linear models into a ``Fit
Model'' option that is separate from ``Predictive Modeling'' and ``Specialized
Modeling,'' which are also distinct from the more general ``Multivariate
Methods.'' Once analysts select the ``Fit Model'' option, they can specify the
``Personality'' of their model as ``Generalized Regression,'' ``Generalized
Linear Model,'' or ``Partial Least Squares,'' among many others. This JMP menu
structure implies that (i) a Partial Least Squares model
is distinct from a regression model when it is in fact a type of regression
model and (ii) regression is not useful for prediction, which is not the case. 

In these ways, tools add a ``Navigate taxonomies'' step before the
\textbf{Statistical Specification} step, requiring analysts to match their
conceptual hypotheses with the tools' taxonomies, which may misalign with their personal taxonomies. One reason for this issue may be that tools do
not leverage analysts' intermediate artifacts or understanding during hypothesis
formalization. By the time analysts transition to \textbf{Statistical
Specification}, they have refined their conceptual hypotheses, developed causal
models, and made observations about data. However, tools' taxonomies require
analysts to set these aside and consider another set of decisions imposed by
tool-specific groupings of statistical methods. In this way, tool taxonomies may introduce challenges that detract from hypothesis
formalization.

% \theme{Syntactic and semantic mismatches create a rift between model implementations and conceptual hypotheses.}
\theme{Model expression: Syntax and semantics}
% ~\footnote{Tools also provided syntactic idiosyncrancies for including and
% exclduing intercepts and defining crossed and nested effects structures. Tanaka
% et al.~\cite{tanaka2019symbolic} compare in detail the syntactical trade-offs
% between \texttt{lme4} [T6] and asmerl, a tool not in our sample.}. 

Fifteen tools in our sample provide analysts with interfaces that use
mathematical notation to express statistical models [T1, T2, T3, T4, T6, T7, T8,
T9, T10, T14, T16, T17, T18, T19, T20]. R and Python packages use symbolic
mathematical syntax, and SPSS and Stata use natural language-like syntax.
Expressing a linear model with Sex, Race, and their interaction as predictors of
Annual Income involves the formula \texttt{AnnualIncome $\sim$ Sex + Race +
Sex*Race} in \texttt{lme4} and \texttt{AnnualIncome BY Sex Race Sex*Race} in
SPSS.  In a linear execution of steps involved in hypothesis formalization where
analysts relate variables mathematically (\textbf{Mathematical Equation}) before
specifying and implementing models using tools (\textbf{Statistical
Specification}, \textbf{Model Implementation}), the mathematical interfaces
match analysts' progression. However, in the lab study, analysts did not specify
their models mathematically even when given the opportunity, suggesting that
mathematical syntax may not adequately capture analysts' conceptual or
statistical considerations. 

% and opportunities for higher-level libraries that do
% not require mathematics, which we discuss in the next section.

Syntactic similarity between packages may lower the barrier to trying and
adopting new statistical approaches that more directly test hypotheses and therefore benefit hypothesis formalization. At the same time, syntactic similarity may also introduce unmet expectations of
semantic similarity. For example, \texttt{brms} [T2] uses the same formula
syntax as \texttt{lme4} [T6], smoothing the transition between linear modeling
and Bayesian linear modeling for analysts. However, based on syntactic
similarity, analysts may incorrectly assume statistical equivalence in computed
model values. For example, in \texttt{brms}, the model intercept is the mean of
the posterior when all the independent variables are at \textit{their means},
but in \texttt{lme4}, the intercept is the mean of the model when all the
independent variables are at \textit{zero}. 

% This mismatch between syntax and
% statistical meaning may be especially significant for hypotheses

Conversely, tools introduce syntactic differences between statistical approaches
that are for the most part semantically equivalent, which may lead to additional
challenges in hypothesis formalization. For instance, an ANOVA with repeated
measures and a linear mixed effects model are similar in intent but require two
different function calls, one without a formula (e.g., \texttt{AnovaRM} in
\texttt{statsmodels} [T14]) and another with (e.g., \texttt{mixedlm} in
\texttt{statsmodels} [T14]). Even when considering only ANOVA, tools may provide
similar syntax but implement different sums of squares procedures for
partitioning variance (i.e., Type I, Type II, or Type III).\footnote{Type I is
(a) sensitive to the order in which independent variables are specified because
it assigns variance sequentially and (b) allows interaction terms. Type II (a)
does not assign variance sequentially and (b) does not allow interaction terms.
Type III (a) does not assign variance sequentially and (b) allows interaction
terms. For an easy-to-understand blog post, see~\cite{sumsofsquaresBlog}.} By
default, R's \texttt{stats} core package [T10] uses Type I, \texttt{statsmodels}
[T14] uses Type II, and \texttt{SPSS} [T16] uses Type III. The three different
sum of squares procedures lead to different F-statistics and p-values, which may
lead analysts to different conclusions. More importantly, the procedures encode
different conceptual hypotheses. If analysts have theoretical knowledge or
conceptual hypotheses about the order of independent variables, tools defaulting
to Type I (e.g., R's \texttt{stats} core library) align the model implementation
with the conceptual hypotheses. However, if analysts do not have such conceptual
hypotheses, tools' default behavior would execute (without error) and silently
respond to a conceptual hypothesis different from the one the analyst seeks to
test. In this way, syntactic and semantic mismatches can create a rift between
model implementations and conceptual hypotheses. Furthermore, the impact of tools' ``invisible'' model implementation choices
reinforces the interplay between conceptual and model implementation concerns
during hypothesis formalization. 


\begin{comment}
Tools should highlight the conceptual
assumptions and consequences of modeling choices beyond listing ways to change
defaults in their documentation manuals or Q\&A sites. Doing this would give
analysts more control over and insight into their analysis. Analysts could
revise and refine their hypotheses in response to statistical modeling
constraints or change the statistical models and tools they use to address their
hypotheses. 
\end{comment}

\vspace{-2mm}
% \theme{Fine-grained computational control may require conceptual hypothesis revisions.}
\theme{Computational issues.}
Tools provide end-users with options for optimizers and solvers used to fit
statistical models [T1, T2, T4, T6, T7, T8, T10, T11, T13, T16, T18],
convergence criteria used for fitting
models [T3, T6, T16, T18], and memory and CPU allocation [T2, T5, T12, T15], among more specific
customizations. % (e.g., for matrix multiplication).
For instance, \texttt{lme4} [T6] allows analysts to specify the nonlinear
optimizer and its settings (e.g., the number of iterations, convergence
criteria, etc.) used to fit models. In \texttt{brms} [T2], analysts can also
specify the number of CPUs to dedicate to fitting their models. Some
computational settings are akin to performance optimizations, affecting computer
utilization but not the results. However, not all computational changes are so
well-isolated.

For example, the failure of a model's inference algorithm to converge (in \textbf{Model Implementation})
may prompt mathematical re-formulation (\textbf{Mathematical Equation}), which
may cast \textbf{Observations about Data} in a new light, prompting
\textbf{Causal Model} and \textbf{Conceptual Hypothesis} revision. In other
words, computational failures and decisions may bubble up to conceptual hypothesis
revision and refinement, which may then trickle back down to model
implementation iteration, and so on. In this way, computational
control can be another entry into the dual-search process of hypothesis formalization. 
% Fine-grained computational control may require conceptual hypothesis revisions.
% In
% fact, in the lab study, analysts described their ``data-driven'' analyses in a
% similar light.

In theory this low-level control could help analysts formalize
nuanced conceptual hypotheses in diverse computational environments. However, we found
that tools do not currently provide feedback on the ramifications of these
computational changes, introducing a gulf of evaluation~\cite{norman1986cognitive}. Analysts can
easily change parameters to fine-tune their computational settings, but how they
should interpret their model implementations and revisions conceptually is
unaddressed, suggesting opportunities for future tools to bridge the conceptual
and model implementation gap. 

\vspace{-2mm}
\subsection{Takeaways from the Analysis of Tools}
% Jeff: I think the ultimate question is not just how to more seamlessly transition between approaches 
% (though this could be quite valuable for multiverse specification), but 
% how to help people determine what are a "reasonable" set of appropriate approaches.

% Jeff:  how does the end user express their needs in a way that helps translate from conceptual or partially-formalized concerns to implementation choices? 
% I think our larger argument and research trajectory is that 
% ultimately this is missing and needed, 
% so it would be good to see that reflected more strongly in the tools section. 
% Perhaps this is a missing fifth category among our tool analysis questions?

Taken together, our analysis shows that tools can support a wide range of
statistical models but expect analysts to have more statistical expertise than
may be realistic. They provide limited guidance for analysts (i) to express and
translate their conceptual and partially-formalized concerns and (ii) identify
reasonable models. Tools also provide little-to-no feedback on the conceptual
ramifications of model implementation iterations. These gaps reveal a misalignment
between analysts' hypothesis formalization processes and tools' expectations and
design. Possible reasons for this mismatch may be that tools do not scaffold or
embody the dual-search nature of hypothesis formalization or leverage all the
intermediate artifacts analysts may create (e.g., refined conceptual hypotheses,
causal models, data observations, partial specifications, etc.) throughout the
process.

% JH: this can wait for the next section
% Focusing on
% current tools' designs as opportunities for improved alignment between analysts
% and tools, we derive three concrete implications for future tools.


\begin{comment}

\theme{Syntactic differences that elide semantic similarities between statistical models hinder model implementation search.}
Fifteen tools in our sample provided analysts with interfaces that used
mathematical notation to express statistical models [T1, T2, T3, T4, T6, T7, T8,
T9, T10, T14, T16, T17, T18, T19, T20]. R and Python packages used symbolic
mathematical syntax, and SPSS and Stata used natural language-like syntax.
Expressing a linear model with Sex, Race, and their interaction as predictors of
Annual Income would involve the formula \texttt{AnnualIncome ~ Sex + Race +
Sex*Race} in \texttt{lme4} and \texttt{AnnualIncome BY Sex Race Sex*Race} in
SPSS. Packages use idiosyncratic syntax for including and excluding intercepts
and defining crossed and nested effects structures. For instance, in R packages,
the intercept for linear models is included by default even if end-users do not
write it. Tanaka et al.~\cite{tanaka2019symbolic} compare in detail the
syntactical trade-offs \texttt{lme4} [T6] and asmerl, a tool not in our sample,
make.

Despite using mathematical notation, tools elide when different statistical
approaches and functions are equivalent in mathematical principles and
implementations. For instance, an ANOVA with repeated measures and a linear
mixed effects model are mathematically equivalent but require two different
function calls, one without a formula (ANOVA) and another with (linear model).
Knowing that the linear model is a mathematical representation of the ANOVA, could
encourage analysts to compare and use both approaches in the future. 

% tools' syntax elide mathematical details that affect model implementation. For
% example, when end-users include categorical variables in their models in any
% programmatic tool in our sample, they write one \textit{variable} name even
% though tools include one fewer than the number of \textit{categories} in a
% categorical variable to the computed statistical model. 

In a linear execution of steps involved in hypothesis formalization where
analysts relate the relationships between variables mathematically
(\textbf{Mathematical Equation}) before specifying and implementing models using
tools (\textbf{Statistical Specification}, \textbf{Model Implementation}), the
mathematical interfaces match analysts' progression in hypothesis formalization.
However, we observed in the lab study that analysts do not specify their models
mathematically even when given the opportunity, which may be due to syntactic
inconsistency across tools or the missed opportunity to use mathematical
notation to highlight semantic differences and similarities between statistical model implementations. 


\theme{Syntactic similarities that elide semantic differences create a rift between model implementations and conceptual hypotheses.}

Even for the same common statistical method and function, ANOVA, tools implement
different sums of squares procedures for partitioning variance (i.e., Type I,
Type II, or Type III)~\footnote{Type I is (a) sensitive to the order in which
independent variables are specified because it assigns variance sequentially and
(b) allows interaction terms. Type II (a) does not assign variance sequentially
and (b) does not allow interaction terms. Type III (a) does not assign variance
sequentially and (b) allows interaction terms. For an easy-to-understand blog
post, see~\cite{sumsofsquaresBlog}.}. By default, R's \texttt{stats} [T10] core
package [T10] uses Type I, \texttt{statsmodels} [T14] uses Type II, and
\texttt{SPSS} [T16] uses Type III. The three different sums of squares
procedures lead to different F-statistic values and p-values, which may lead
analysts to different conclusions, but more importantly, the procedures encode
slightly different conceptual hypotheses about how strongly independent
variables relate to the dependent variable relative to each other. For instance,
Type I, which is sensitive to the order in which independent variables are
specified, assumes that the independent variables that are listed closer to the
beginning of the list have stronger relationships to the dependent variable.
Thus, when analysts have no theoretical knowledge or conceptual hypotheses about
how important independent variables are compared to each other, Type I sums of
squares is misaligned with their conceptual hypothesis. However, the R
\texttt{stats} ANOVA function will use Type I sums of squares and silently
change the conceptual hypotheses analysts are testing. 

It is important to note that syntactic similarity between packages may lower the
barrier to trying and adopting new statistical approaches. For example,
\texttt{brms} [T2] uses the same syntax as \texttt{lme4} [T6], smoothing the
transition between linear modeling and Bayesian linear modeling for analysts.
However, syntactic similarities may lead to erroneous conclusions about what the
results of a model say about a conceptual hypothesis. In \texttt{brms}, the
model intercept is the mean of the posterior when all the independent variables
are at \textit{their means}, but in \texttt{lme4}, the intercept if the mean of
the model when all the independent variables are \textit{at 0}.

The conceptual impact ``invisible'' model implementation choices tools make
reinforces the interplay between conceptual and model implementation concerns
during hypothesis formalization. Tools should highlight the conceptual
assumptions and consequences of modeling choices beyond listing ways to change
defaults in their documentation manuals or Q\&A sites. Doing this would give
analysts more control over their analysis. Analysts could revise and refine
their hypotheses in response to statistical modeling constraints or change the
statistical models and tools they use to address their hypotheses. 

\end{comment}

\section{Discussion: Design Implications for Statistical Analysis Software} \label{sec:implications}
% \figureImplications
Our findings suggest three opportunities for tools to facilitate the dual-search
process and align conceptual hypotheses with statistical model implementations
at various stages of hypothesis formalization. 

% Figure~\ref{figure:implications} provides a
% simplified view of our model with the implications labeled. Any given tools may
% realize combinations of these implications. We recognize that, as Schunn and
% Klahr~\cite{schunn1996BeyondTwoSpace} note, the process of realizing these
% implications from the model may refine and revise our understanding of the
% spaces involved in hypothesis formalization.

\subsection{Meta-libraries: Connecting Model Implementations with Mathematical Equations}

% Additionally, even for tools that rely on symbolic
% formulae, how tool interfaces (formulae and graphical), model implementation,
% and computed outputs are connected is ambiguous. 

% Additionally, for tools that rely on symbolic formulae, how the formulae relate
% to model implementation and computed outputs is ambiguous. 

% find the appropriate
% library or libraries to execute such a model in its knowledge base; 

Specialized tools, although necessary for sophisticated statistical computation,
require a steep learning curve. \textit{Meta-libraries} could allow analysts to
specify their models in high-level code; execute the models using the appropriate
libraries in their knowledge bases; and then output library information,
functions invoked, any computational settings used, the mathematical model that
is approximated, and the model results. Libraries such as
Parsnip~\cite{parsnip} have begun to provide a unified higher-level interface
that allows analysts to specify a statistical model using more ``generically''
named functions, parameter names, and symbolic formulae (when necessary).
Parsnip then compiles and invokes various library-specific functions for the
same statistical model.

Probabilistic programming languages (PPLs), such as Pyro~\cite{pyro}, Stan~\cite{stan},
BUGS~\cite{bugs}, PyMC~\cite{pymc3}, already enable the development of
meta-libraries. PPLs support modular specification of data, probabilistic
models, and probabilistic hypotheses. Existing libraries, including
\texttt{brms}, provide higher-level APIs whose syntax uses symbolic formulae,
for instance, and compile to programs in a PPL (i.e., Stan in the case of
\texttt{brms}). 

As already seen in Parsnip and tools using PPLs, meta-libraries could
bring three benefits. First, they would provide simpler, less fragmented
interfaces to analysts while continuing to take advantage of tool
specialization. Second, meta-libraries that output complete mathematical
representations would more tightly couple mathematical representations with
implementations, providing an on-ramp for analysts to expand their statistical
knowledge. Third, meta-libraries that show the mathematical representations
alongside underlying libraries' function calls could show syntactical variation
in underlying libraries, indirectly teaching analysts how they might express
their statistical models in other tools, familiarizing analysts with new tools
and models, and even mend fragmented views of identical models (e.g., ANOVA and
regression). 

Future meta-libraries could consider providing a higher-level, declarative
interface that does not require analysts to write symbolic formulae. Designing
such declarative meta-libraries would require formative elicitation studies
(similar to natural programming studies such as~\cite{verou2018extending}) on
declarative primitives that are memorable, distinguishable, and reliably
understood. An additional challenge would lie in maintaining support for various
libraries executed under the hood, especially as libraries change their APIs,
which would strengthen the case for meta-libraries. Although meta-libraries
would not solve the problems involved in understanding how computational
settings affect model execution or conceptual hypotheses, they could
nevertheless provide scaffolding for analysts to more closely examine specific
libraries, especially if multiple libraries execute the same model but do not
all encounter the same computational bottlenecks. 

\subsection{High-level Libraries: Expressing Conceptual Hypotheses to Bootstrap Model Implementations}

The absence of tools for directly expressing conceptual hypotheses may be an
explanation for why data workers in the lab study dove into model
implementation details. High-level libraries could allow analysts to specify data
collection design (e.g., independent variables, dependent variables, controlled
effects, possible random effects); variable data types; expected or known
covariance relationships based on domain expertise; and hypothesized findings in
a library-specific grammar. High-level libraries could compile these conceptual
and data declarations into weighted constraints that represent the applicability
of various statistical approaches, in a fashion similar to
Tea~\cite{jun2019tea}, a domain-specific language for automatically selecting
appropriate statistical analyses for common hypothesis tests. Libraries could
then execute the appropriate statistical approaches, possibly by using a
meta-library as described above. 

In addition to questions of how to represent a robust taxonomy of statistical
approaches computationally, another key challenge for developing high-level
libraries is identifying a set of minimal yet complete primitives that are
useful and usable for analysts to express information that is usually expressed
at different levels of abstraction: conceptual hypotheses, study designs, and
possibly even partial statistical model specifications. For instance, even if a
conceptual hypothesis is expressible in a library, it may be impossible to
answer with a study design or partial statistical model that is expressed in the
same program. An approach may be to draw upon and integrate aspects from
existing high-level libraries and systems that aim to address separate steps of
the hypothesis formalization process, such as
Touchstone2~\cite{eiselmayer2019touchstone2} for study design and Tea and
Statsplorer~\cite{wacharamanotham2015statsplorer} for statistical analysis. 

% Finally, future high-level libraries could provide mathematical
% equations as part of the output to further strengthen the pathway from
% conceptual hypotheses to mathematical representations to model implementations
% while providing an abstraction that more directly maps to analysts' conceptual
% and data-based considerations. 

% identifying a minimal yet complete set of declarations for
% deducing appropriate statistical models and representing a robust taxonomy of
% statistical approaches computationally. 

\subsection{Bidirectional Conceptual Modeling: Co-authoring Conceptual Models and Model Implementations}
Conceptual, or causal, modeling was difficult for the analysts in the lab study.
Some even resisted conceptual modeling for fear of biasing their analyses. Yet,
implicit conceptual models were evident in the hypotheses analysts chose to
implement and the sub-hypotheses researchers articulated in the content
analysis. 

% Tools that automatically derive conceptual models from analysts' model implementations may assuage analysts's fears while also 
% Based on this insight, there is opportunity to increase transparent
% and complete reporting and increased engagement with bias in the analysis
% process through automatic authoring of conceptual models. 

Mixed-initiative systems that make explicit the connection between conceptual
models and statistical model implementations could facilitate hypothesis
formalization from either search process and allow analysts to reflect on their
analyses without fear of bias. For example, a mixed-initiative programming
environment could allow analysts to write an analysis script, detect data
variables in the analysis scripts, identify how groups of variables co-occur in
statistical models, and then visualize conceptual models as graphs where the
nodes represent variables and the edges represent relationships. The
automatically generated conceptual models would serve as templates that analysts
could then manipulate and update to better reflect their internal conceptual
models by specifying the kind of relationship between variables (e.g.,
correlation, linear model, etc.) and assigning any statistical model roles
(e.g., independent variable, dependent variable). As analysts update the visual
conceptual models, they could evaluate script changes the system proposes.
In this way, analysts could externally represent their causal models while
authoring analysis scripts and vice versa. 

Although bidirectional programming environments already exist for vector graphics
creation~\cite{hempel2019sketch}, they have yet to be realized in mainstream data analysis
tools. To realize bidirectional, automatic conceptual modeling, researchers would
need to address important questions about (i) the visual grammar, which would
likely borrow heavily from the causal modeling literature; (ii) program analysis
techniques for identifying variables and defining co-occurrences (e.g., line-based
vs. function-based) in a way that generalizes to multiple statistical libraries;
and (iii) adoption, as analysts who may benefit most from such tools (likely
domain non-experts) may be the most resistant to tools that limit the number of
``insights'' they take away from an analysis. 

% Implication for bidrectional causal modeling: 
% - Walker et al. on Design and Causal learning (PsySci 2020)
% - Burnston (CogSci 2013)

\begin{comment}

\subsection{Joint conceptual and statistical planning}
--- Could this explain OR capture why some papers had multiple sub-hypotheses??
RElate dto TOuchstone
Similarly, given a particular analysis, could suggest a data colelcito/study design that would allow that to happen

Support reasoning at the conceptual and statisitcal implementation levels -- conceptual --> implementation \&\& implementation --> conceptual

Provide a unified interface (syntax, semantics) for specifying specific classes of models
\end{comment}

% \todo{Limitation: testing implications with significant system development and folow-up user studies}
\vspace{-2mm}
\section{Discussion: Data analysis as problem solving}
% \section{Discussion, Limitations, and Future work}
% \todo{separate Discussion from Limitations and Future Work -- without sacrificing cohesion}
Hypothesis formalization is a dual-search process of translating conceptual
hypotheses into statistical model implementations. Due to constraints imposed by
domain expertise, data, and tool familiarity, the same conceptual hypothesis may
be formalized into different model implementations. A single model implementation may be useful for making multiple statistical inferences. The same model
implementation may also formalize two possibly opposing hypotheses. To navigate
these constraints, analysts use problem-solving strategies characteristic of the
larger scientific discovery process~\cite{klahr1988dual,schunn1995FourSpace}. As
such, hypothesis formalization exemplifies how data science is a design
practice. 
% Understanding
% hypothesis formalization enables us to identify when and why these differences
% arise, which may help diagnose reproducibility issues. 
 
At a conceptual level, hypothesis formalization involves \textit{hypothesis
refinement}, which, to use Schunn and Klahr's
language~\cite{schunn1995FourSpace}, is a \textit{scoping} process. In the
formative content analysis, we found that researchers \textit{decomposed} their
research goals and conceptual hypotheses into specific, testable sub-hypotheses
and \textit{concretized} constructs using proxies, born of theory or available
data. Also, we found that analysts in the lab study also quickly converged on
the need to specify established proxies or develop them based on the data schema
presented. In hypothesis formalization, scoping incorporates domain- and
data-specific observations to qualify the conceptual scope of researchers'
hypotheses. In other words, hypothesis refinement is an instance of
\textit{means-end analysis}~\cite{newell1972humanProblemSolving}, a
problem-solving strategy that aims to recursively change the current state of a
problem into sub-goals (i.e., increasingly specific objectives) in order to
apply a technique (i.e., a particular statistical model) to solve the problem
(i.e., test a hypothesis). 

At the other computational endpoint of hypothesis formalization, \textit{model implementation}
also involves iteration. Through our analysis of software tools, we
found that analysts must not only select tools among an array of specialized and
general choices but also navigate tool-specific taxonomies of statistical
approaches. These tool taxonomies may both differ from and inform analysts'
personal categorizations, potentially explaining why analysts in our lab study
relied on their personal taxonomies and tools. Based on their prior experience, analysts engage in
\textit{analogical reasoning}~\cite{holland1989induction}, finding parallels between the present analysis
problem's structure and previously encountered ones or ones that fit a tool's
design easily.

Upon selecting a statistical function, analysts may tune computational settings,
choose different statistical functions or approaches, which they may tune, and
so on. In this way, the model implementation loop in hypothesis
formalization captures the ``debugging cycles'' analysts encounter, such as the census
researcher in the introduction. The tool ecosystem as a
whole supports diverse model implementations, even for the same
mathematical equation. However, the tool interfaces provide low-level abstractions, such as
interfaces using mathematical formulae that, based on our observations in the
lab study, do not support the kind of higher-level conceptual reasoning required
of hypothesis formalization.

% Our investigation of hypothesis formalization focused on domain-general steps
% and considerations. 
% implementations because they are familiar with computational settings and their
% conceptual and statistical ramifications. 
% Additionally, in domains where pre-registration has become more mainstream, such
% as in psychology, future research could also examine how researchers arrive at
% the studies they pre-register. 

% information sources and collaboration are dimensions
% of hypothesis formalization that we did not investigate deeply.
\section{Future Work}
The steps, considerations, and strategies we have identified are domain-general.
Domain-specific expertise likely influences how quickly analysts switch between
steps and strategies during the dual-search process. Domain experts, including
researchers in our content analysis, may know which statistical model
implementations and computational settings to use a priori and design their
studies or specify their conceptual hypotheses in light of these expectations
--- incorporating means-end analysis and analogical reasoning strategies ---
more quickly. It may be these insights that analysts in our lab study sought
when they looked online for conceptual and statistical help. 

Future work could observe how domain experts perform hypothesis formalization
and characterize when and how analysts draw upon their own or collaborators'
expertise to circumvent iterations or justify early scoping decisions. These insights may also shed light on how pre-registration expectations and
practices could be made more effective. Given the level of detail required of
some pre-registration policies, researchers likely engage in a version of the
hypothesis formalization process we have identified prior to registering their
studies. Knowing how pre-registration fits into the hypothesis formalization
process could improve the design and adoption of pre-registration practices.

Future work could also explore how hypothesis formalization may differ in
machine learning settings. In this paper, our focus was on how analysts answer
domain questions and test hypotheses using statistical methods and their domain
knowledge. Our findings may not generalize to settings or methods where domain
knowledge is less important, such as deep learning and other machine
learning-based approaches. 

% Throughout this process, analysts seek additional, outside information, as we
% observed in the lab study. Based on our
% findings, we expect that (i) domain and data information is helpful for refining
% hypotheses, (ii) tool documentation is helpful during model implementation, and
% (iii) statistical information is helpful all throughout. Future work should
% investigate precisely how and when different kinds of information and
% collaborations help analysts problem solve during hypothesis formalization.

Finally, our findings suggest opportunities for future tools to bridge steps
involved in hypothesis formalization and guide analysts towards reasonable model
implementations. Our analysis of tools suggest possibilities for tools to
connect model implementations to their mathematical representations through
meta-libraries, provide higher-level abstractions for more directly expressing
conceptual hypotheses, and support automated conceptual modeling. Future system
development and user testing are necessary to validate these implications and
more readily support analysts translate their conceptual hypotheses into
statistical model implementations.

% Studying users in a lab or observational setting,
% especially over time, would likely reveal additional implications for improving
% existing tools and developing new ones. 

% Similar to how heuristic evaluations identify
% usability challenges, our heuristic analysis of software tools revealed
% misalignments between tool design and the hypothesis formalization process.





\section{Summary of Contributions}
% Replace the conclusion section with a summary section. Again, you should tie this chapter back to the main themes of the thesis.

Hypothesis formalization -- retrospecrive support for design in Tea, inspired design of Tisane

\ej{This work was originally published with...... at ....}